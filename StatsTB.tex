% Options for packages loaded elsewhere
\PassOptionsToPackage{unicode}{hyperref}
\PassOptionsToPackage{hyphens}{url}
%
\documentclass[
]{book}
\usepackage{amsmath,amssymb}
\usepackage{iftex}
\ifPDFTeX
  \usepackage[T1]{fontenc}
  \usepackage[utf8]{inputenc}
  \usepackage{textcomp} % provide euro and other symbols
\else % if luatex or xetex
  \usepackage{unicode-math} % this also loads fontspec
  \defaultfontfeatures{Scale=MatchLowercase}
  \defaultfontfeatures[\rmfamily]{Ligatures=TeX,Scale=1}
\fi
\usepackage{lmodern}
\ifPDFTeX\else
  % xetex/luatex font selection
\fi
% Use upquote if available, for straight quotes in verbatim environments
\IfFileExists{upquote.sty}{\usepackage{upquote}}{}
\IfFileExists{microtype.sty}{% use microtype if available
  \usepackage[]{microtype}
  \UseMicrotypeSet[protrusion]{basicmath} % disable protrusion for tt fonts
}{}
\makeatletter
\@ifundefined{KOMAClassName}{% if non-KOMA class
  \IfFileExists{parskip.sty}{%
    \usepackage{parskip}
  }{% else
    \setlength{\parindent}{0pt}
    \setlength{\parskip}{6pt plus 2pt minus 1pt}}
}{% if KOMA class
  \KOMAoptions{parskip=half}}
\makeatother
\usepackage{xcolor}
\usepackage{color}
\usepackage{fancyvrb}
\newcommand{\VerbBar}{|}
\newcommand{\VERB}{\Verb[commandchars=\\\{\}]}
\DefineVerbatimEnvironment{Highlighting}{Verbatim}{commandchars=\\\{\}}
% Add ',fontsize=\small' for more characters per line
\usepackage{framed}
\definecolor{shadecolor}{RGB}{248,248,248}
\newenvironment{Shaded}{\begin{snugshade}}{\end{snugshade}}
\newcommand{\AlertTok}[1]{\textcolor[rgb]{0.94,0.16,0.16}{#1}}
\newcommand{\AnnotationTok}[1]{\textcolor[rgb]{0.56,0.35,0.01}{\textbf{\textit{#1}}}}
\newcommand{\AttributeTok}[1]{\textcolor[rgb]{0.13,0.29,0.53}{#1}}
\newcommand{\BaseNTok}[1]{\textcolor[rgb]{0.00,0.00,0.81}{#1}}
\newcommand{\BuiltInTok}[1]{#1}
\newcommand{\CharTok}[1]{\textcolor[rgb]{0.31,0.60,0.02}{#1}}
\newcommand{\CommentTok}[1]{\textcolor[rgb]{0.56,0.35,0.01}{\textit{#1}}}
\newcommand{\CommentVarTok}[1]{\textcolor[rgb]{0.56,0.35,0.01}{\textbf{\textit{#1}}}}
\newcommand{\ConstantTok}[1]{\textcolor[rgb]{0.56,0.35,0.01}{#1}}
\newcommand{\ControlFlowTok}[1]{\textcolor[rgb]{0.13,0.29,0.53}{\textbf{#1}}}
\newcommand{\DataTypeTok}[1]{\textcolor[rgb]{0.13,0.29,0.53}{#1}}
\newcommand{\DecValTok}[1]{\textcolor[rgb]{0.00,0.00,0.81}{#1}}
\newcommand{\DocumentationTok}[1]{\textcolor[rgb]{0.56,0.35,0.01}{\textbf{\textit{#1}}}}
\newcommand{\ErrorTok}[1]{\textcolor[rgb]{0.64,0.00,0.00}{\textbf{#1}}}
\newcommand{\ExtensionTok}[1]{#1}
\newcommand{\FloatTok}[1]{\textcolor[rgb]{0.00,0.00,0.81}{#1}}
\newcommand{\FunctionTok}[1]{\textcolor[rgb]{0.13,0.29,0.53}{\textbf{#1}}}
\newcommand{\ImportTok}[1]{#1}
\newcommand{\InformationTok}[1]{\textcolor[rgb]{0.56,0.35,0.01}{\textbf{\textit{#1}}}}
\newcommand{\KeywordTok}[1]{\textcolor[rgb]{0.13,0.29,0.53}{\textbf{#1}}}
\newcommand{\NormalTok}[1]{#1}
\newcommand{\OperatorTok}[1]{\textcolor[rgb]{0.81,0.36,0.00}{\textbf{#1}}}
\newcommand{\OtherTok}[1]{\textcolor[rgb]{0.56,0.35,0.01}{#1}}
\newcommand{\PreprocessorTok}[1]{\textcolor[rgb]{0.56,0.35,0.01}{\textit{#1}}}
\newcommand{\RegionMarkerTok}[1]{#1}
\newcommand{\SpecialCharTok}[1]{\textcolor[rgb]{0.81,0.36,0.00}{\textbf{#1}}}
\newcommand{\SpecialStringTok}[1]{\textcolor[rgb]{0.31,0.60,0.02}{#1}}
\newcommand{\StringTok}[1]{\textcolor[rgb]{0.31,0.60,0.02}{#1}}
\newcommand{\VariableTok}[1]{\textcolor[rgb]{0.00,0.00,0.00}{#1}}
\newcommand{\VerbatimStringTok}[1]{\textcolor[rgb]{0.31,0.60,0.02}{#1}}
\newcommand{\WarningTok}[1]{\textcolor[rgb]{0.56,0.35,0.01}{\textbf{\textit{#1}}}}
\usepackage{longtable,booktabs,array}
\usepackage{calc} % for calculating minipage widths
% Correct order of tables after \paragraph or \subparagraph
\usepackage{etoolbox}
\makeatletter
\patchcmd\longtable{\par}{\if@noskipsec\mbox{}\fi\par}{}{}
\makeatother
% Allow footnotes in longtable head/foot
\IfFileExists{footnotehyper.sty}{\usepackage{footnotehyper}}{\usepackage{footnote}}
\makesavenoteenv{longtable}
\usepackage{graphicx}
\makeatletter
\def\maxwidth{\ifdim\Gin@nat@width>\linewidth\linewidth\else\Gin@nat@width\fi}
\def\maxheight{\ifdim\Gin@nat@height>\textheight\textheight\else\Gin@nat@height\fi}
\makeatother
% Scale images if necessary, so that they will not overflow the page
% margins by default, and it is still possible to overwrite the defaults
% using explicit options in \includegraphics[width, height, ...]{}
\setkeys{Gin}{width=\maxwidth,height=\maxheight,keepaspectratio}
% Set default figure placement to htbp
\makeatletter
\def\fps@figure{htbp}
\makeatother
\setlength{\emergencystretch}{3em} % prevent overfull lines
\providecommand{\tightlist}{%
  \setlength{\itemsep}{0pt}\setlength{\parskip}{0pt}}
\setcounter{secnumdepth}{5}
\usepackage{booktabs}
\usepackage{booktabs}
\usepackage{caption}
\usepackage{longtable}
\usepackage{colortbl}
\usepackage{array}
\usepackage{anyfontsize}
\usepackage{multirow}
\ifLuaTeX
  \usepackage{selnolig}  % disable illegal ligatures
\fi
\usepackage[]{natbib}
\bibliographystyle{plainnat}
\IfFileExists{bookmark.sty}{\usepackage{bookmark}}{\usepackage{hyperref}}
\IfFileExists{xurl.sty}{\usepackage{xurl}}{} % add URL line breaks if available
\urlstyle{same}
\hypersetup{
  pdftitle={Introduction to Statistics for Biomedical Sciences Using R},
  pdfauthor={Yuexi Dai, Tanvi Mane, Fatima Hashmi, Xinrui Yu, Philip He},
  hidelinks,
  pdfcreator={LaTeX via pandoc}}

\title{Introduction to Statistics for Biomedical Sciences Using R}
\author{Yuexi Dai, Tanvi Mane, Fatima Hashmi, Xinrui Yu, Philip He}
\date{June 26, 2025}

\begin{document}
\maketitle

{
\setcounter{tocdepth}{1}
\tableofcontents
}
\begin{verbatim}
install.packages("survival")
install.packages("ggsurvfit")
install.packages("condsurv")
install.packages("olsrr")
install.packages("leaps")
install.packages("MASS")
install.packages("sm")
install.packages("broom")
install.packages("broom.helpers")
install.packages("mlbench")
install.packages("ROCR")
install.packages("psych")
install.packages("epitools")
install.packages("DescTools")
install.packages("coin")
install.packages("multcomp")
install.packages("car")
install.packages("devtools")
devtools::install_github("phe9480/IntroStats")
install.packages("ggplot2")
install.packages("tibble")
install.packages("tidycmprsk")
install.packages("cardx")
install.packages("purrr")
install.packages("NHANES")
install.packages("lsr")
install.packages("effects")
install.packages("pROC")
install.packages("multimode")
\end{verbatim}

\begin{Shaded}
\begin{Highlighting}[]
\FunctionTok{library}\NormalTok{(tidyverse)}
\FunctionTok{library}\NormalTok{(dplyr)}
\FunctionTok{library}\NormalTok{(tidyr)}
\FunctionTok{library}\NormalTok{(knitr)}
\FunctionTok{library}\NormalTok{(lubridate)}
\FunctionTok{library}\NormalTok{(knitr)}
\FunctionTok{library}\NormalTok{(gtsummary)}
\FunctionTok{library}\NormalTok{(survival)}
\FunctionTok{library}\NormalTok{(ggsurvfit)}
\FunctionTok{library}\NormalTok{(condsurv) }\CommentTok{\# Your conditional survival tool!}
\FunctionTok{library}\NormalTok{(olsrr)}
\FunctionTok{library}\NormalTok{(leaps)}
\FunctionTok{library}\NormalTok{(MASS)}
\FunctionTok{library}\NormalTok{(sm)}
\FunctionTok{library}\NormalTok{(broom)}
\FunctionTok{library}\NormalTok{(broom.helpers)}
\FunctionTok{library}\NormalTok{(mlbench)}
\FunctionTok{library}\NormalTok{(ROCR)}
\FunctionTok{library}\NormalTok{(psych)}
\FunctionTok{library}\NormalTok{(epitools)}
\FunctionTok{library}\NormalTok{(DescTools)}
\FunctionTok{library}\NormalTok{(coin)}
\FunctionTok{library}\NormalTok{(multcomp)}
\FunctionTok{library}\NormalTok{(car)}
\FunctionTok{library}\NormalTok{(IntroStats) }\CommentTok{\# Your special IntroStats tool!}
\FunctionTok{library}\NormalTok{(ggplot2)}
\FunctionTok{library}\NormalTok{(tibble)}
\FunctionTok{library}\NormalTok{(devtools)}
\FunctionTok{library}\NormalTok{(tidycmprsk)}
\FunctionTok{library}\NormalTok{(cardx)}
\FunctionTok{library}\NormalTok{(purrr) }
\FunctionTok{library}\NormalTok{(NHANES)}
\FunctionTok{library}\NormalTok{(lsr)}
\FunctionTok{library}\NormalTok{(effects)}
\FunctionTok{library}\NormalTok{(pROC)}
\FunctionTok{library}\NormalTok{(multimode)}

\NormalTok{mycol }\OtherTok{=} \StringTok{"turquoise"}
\end{Highlighting}
\end{Shaded}

\begin{verbatim}
{r eval=FALSE}
bookdown::serve_boojjk()
\end{verbatim}

\hypertarget{introduction-to-r}{%
\chapter{Introduction to R}\label{introduction-to-r}}

\hypertarget{getting-started-with-r}{%
\section{Getting Started with R}\label{getting-started-with-r}}

In order to facilitate the journey of learning statistics, an abbreviated R primer is prepared to go through the essential elements in R programming. This primer is written based on \url{https://www.geeksforgeeks.org/r-tutorial/}. For further learning of R programming language, please refer to the website.

All class materials are written in R Markdown documents. Markdown is a simple formatting syntax for authoring HTML, PDF, and MS Word documents. For more details on using R Markdown see \url{http://rmarkdown.rstudio.com}.

When you click the \textbf{Knit} button a document will be generated that includes both content as well as the output of any embedded R code chunks within the document. You can embed an R code chunk like this:

\begin{Shaded}
\begin{Highlighting}[]
\CommentTok{\#Comment using the \# symbol}
\FunctionTok{print}\NormalTok{(}\StringTok{"Hello World!"}\NormalTok{)}
\end{Highlighting}
\end{Shaded}

\begin{verbatim}
## [1] "Hello World!"
\end{verbatim}

R is an interpreted programming language and it was developed by Ross Ihaka and Robert Gentleman at the University of Auckland, New Zealand. R is an open-source programming language and it is available on widely used platforms e.g.~Windows, Linux, and Mac.

R Studio is a very convenient editor for R. Follow the instructions below to install R and R studio.

R is a free programming language designed for statistical computing and data visualization. In biomedical research, R is widely used to explore and analyze patient data, clinical trials, and public health statistics.

\hypertarget{install-r-and-r-studio}{%
\section{Install R and R Studio}\label{install-r-and-r-studio}}

First, install R from \url{https://www.r-project.org/}.

\begin{figure}
\centering
\includegraphics{https://i.ibb.co/3pRvyYh/Download-R1.png}
\caption{Download R: Step 1}
\end{figure}

After download R, please install it immediately. Second, install R Studio from \url{https://posit.co/download/rstudio-desktop/}. It is important to maintain the order for R studio to automatically configure R.

\begin{figure}
\centering
\includegraphics{https://i.ibb.co/02DFmcF/Download-Rstudio.png}
\caption{Download R Studio}
\end{figure}

\hypertarget{r-coding-fundamentals}{%
\section{R Coding Fundamentals}\label{r-coding-fundamentals}}

\hypertarget{basic-calculations}{%
\subsection{Basic Calculations}\label{basic-calculations}}

\begin{Shaded}
\begin{Highlighting}[]
\DecValTok{3} \SpecialCharTok{+} \DecValTok{4}
\end{Highlighting}
\end{Shaded}

\begin{verbatim}
## [1] 7
\end{verbatim}

\begin{Shaded}
\begin{Highlighting}[]
\FunctionTok{log}\NormalTok{(}\DecValTok{10}\NormalTok{)}
\end{Highlighting}
\end{Shaded}

\begin{verbatim}
## [1] 2.302585
\end{verbatim}

\begin{Shaded}
\begin{Highlighting}[]
\FunctionTok{sqrt}\NormalTok{(}\DecValTok{16}\NormalTok{)}
\end{Highlighting}
\end{Shaded}

\begin{verbatim}
## [1] 4
\end{verbatim}

These commands perform basic arithmetic operations: addition, logarithm (natural base), and square root. These are often used in formula transformations and scaling.

\hypertarget{assignment}{%
\subsection{Assignment}\label{assignment}}

Use \texttt{=} for Simple Assignment. Use \texttt{\textless{}-} for Leftward Assignment

\begin{Shaded}
\begin{Highlighting}[]
\CommentTok{\# Use = for Simple Assignment}
\CommentTok{\# Use \textless{}{-} for Leftward Assignment}

\NormalTok{x }\OtherTok{=} \StringTok{"book"}
\NormalTok{y }\OtherTok{\textless{}{-}} \StringTok{"book"}

\FunctionTok{print}\NormalTok{ (x)}
\end{Highlighting}
\end{Shaded}

\begin{verbatim}
## [1] "book"
\end{verbatim}

\begin{Shaded}
\begin{Highlighting}[]
\FunctionTok{print}\NormalTok{ (y)}
\end{Highlighting}
\end{Shaded}

\begin{verbatim}
## [1] "book"
\end{verbatim}

\hypertarget{creating-variables}{%
\subsection{Creating Variables}\label{creating-variables}}

\begin{Shaded}
\begin{Highlighting}[]
\NormalTok{weight }\OtherTok{\textless{}{-}} \DecValTok{60}        \CommentTok{\# weight in kilograms}
\NormalTok{height }\OtherTok{\textless{}{-}} \FloatTok{1.65}      \CommentTok{\# height in meters}
\NormalTok{bmi }\OtherTok{\textless{}{-}}\NormalTok{ weight }\SpecialCharTok{/}\NormalTok{ (height}\SpecialCharTok{\^{}}\DecValTok{2}\NormalTok{)}
\NormalTok{bmi}
\end{Highlighting}
\end{Shaded}

\begin{verbatim}
## [1] 22.03857
\end{verbatim}

In biomedical contexts, Body Mass Index (BMI) is a key indicator of patient health. Here, we define weight and height, then calculate BMI using the formula: weight divided by height squared.

\hypertarget{creating-and-using-vectors}{%
\subsection{Creating and Using Vectors}\label{creating-and-using-vectors}}

\begin{Shaded}
\begin{Highlighting}[]
\NormalTok{ages }\OtherTok{\textless{}{-}} \FunctionTok{c}\NormalTok{(}\DecValTok{25}\NormalTok{, }\DecValTok{30}\NormalTok{, }\DecValTok{35}\NormalTok{, }\DecValTok{40}\NormalTok{, }\DecValTok{45}\NormalTok{)}
\FunctionTok{mean}\NormalTok{(ages)}
\end{Highlighting}
\end{Shaded}

\begin{verbatim}
## [1] 35
\end{verbatim}

The \texttt{c()} function creates a numeric vector. \texttt{mean()} computes the average age, a basic measure of central tendency.

\begin{Shaded}
\begin{Highlighting}[]
\CommentTok{\#Use c to combine values together to form an array}
\NormalTok{x }\OtherTok{\textless{}{-}} \FunctionTok{c}\NormalTok{(}\DecValTok{1}\NormalTok{, }\DecValTok{2}\NormalTok{, }\DecValTok{3}\NormalTok{, }\DecValTok{4}\NormalTok{)}
\FunctionTok{print}\NormalTok{(x)}
\end{Highlighting}
\end{Shaded}

\begin{verbatim}
## [1] 1 2 3 4
\end{verbatim}

\begin{Shaded}
\begin{Highlighting}[]
\CommentTok{\#Use rep() function, i.e., replicate}
\NormalTok{y }\OtherTok{\textless{}{-}} \FunctionTok{rep}\NormalTok{(}\DecValTok{1}\NormalTok{, }\DecValTok{3}\NormalTok{)}
\FunctionTok{print}\NormalTok{(y)}
\end{Highlighting}
\end{Shaded}

\begin{verbatim}
## [1] 1 1 1
\end{verbatim}

\begin{Shaded}
\begin{Highlighting}[]
\CommentTok{\#Use seq() function, i.e., sequence}
\NormalTok{z }\OtherTok{\textless{}{-}} \FunctionTok{seq}\NormalTok{(}\DecValTok{0}\NormalTok{, }\DecValTok{1}\NormalTok{, }\AttributeTok{by=}\FloatTok{0.1}\NormalTok{)}
\FunctionTok{print}\NormalTok{(z)}
\end{Highlighting}
\end{Shaded}

\begin{verbatim}
##  [1] 0.0 0.1 0.2 0.3 0.4 0.5 0.6 0.7 0.8 0.9 1.0
\end{verbatim}

\begin{Shaded}
\begin{Highlighting}[]
\CommentTok{\#Try mixed data types. The variable x1 becomes character type once one value is character.}
\NormalTok{x1 }\OtherTok{\textless{}{-}} \FunctionTok{c}\NormalTok{(}\DecValTok{1}\NormalTok{, }\DecValTok{2}\NormalTok{, }\DecValTok{3}\NormalTok{, }\StringTok{"text"}\NormalTok{)}
\FunctionTok{print}\NormalTok{(x1)}
\end{Highlighting}
\end{Shaded}

\begin{verbatim}
## [1] "1"    "2"    "3"    "text"
\end{verbatim}

\hypertarget{create-a-data-frame}{%
\subsection{Create a data frame}\label{create-a-data-frame}}

Data frame is a special format in R. It may consist of multiple variables. For example, a data frame to describe a set of students may contain 2 variables: Name and Age.

\begin{Shaded}
\begin{Highlighting}[]
\NormalTok{students }\OtherTok{\textless{}{-}} \FunctionTok{data.frame}\NormalTok{(}\AttributeTok{Name =} \FunctionTok{c}\NormalTok{(}\StringTok{"John Connor"}\NormalTok{, }\StringTok{"Will Smith"}\NormalTok{), }\AttributeTok{Age =} \FunctionTok{c}\NormalTok{(}\DecValTok{19}\NormalTok{, }\DecValTok{21}\NormalTok{))}
\NormalTok{students}
\end{Highlighting}
\end{Shaded}

\begin{verbatim}
##          Name Age
## 1 John Connor  19
## 2  Will Smith  21
\end{verbatim}

\hypertarget{a-variable-in-a-data-frame}{%
\subsection{A variable in a data frame}\label{a-variable-in-a-data-frame}}

\begin{Shaded}
\begin{Highlighting}[]
\NormalTok{students}\SpecialCharTok{$}\NormalTok{Name}
\end{Highlighting}
\end{Shaded}

\begin{verbatim}
## [1] "John Connor" "Will Smith"
\end{verbatim}

\begin{Shaded}
\begin{Highlighting}[]
\NormalTok{students}\SpecialCharTok{$}\NormalTok{Age}
\end{Highlighting}
\end{Shaded}

\begin{verbatim}
## [1] 19 21
\end{verbatim}

\hypertarget{logic-operations}{%
\subsection{Logic Operations}\label{logic-operations}}

Use \texttt{\&} for logical AND; \texttt{\textbar{}} for logical OR; use \texttt{!} for NOT.

\begin{Shaded}
\begin{Highlighting}[]
\NormalTok{a }\OtherTok{\textless{}{-}} \FunctionTok{c}\NormalTok{(}\DecValTok{2}\NormalTok{, }\DecValTok{3}\NormalTok{)}
\NormalTok{b }\OtherTok{\textless{}{-}} \FunctionTok{c}\NormalTok{(}\DecValTok{4}\NormalTok{, }\DecValTok{2}\NormalTok{)}

\FunctionTok{print}\NormalTok{(a }\SpecialCharTok{\textless{}}\NormalTok{ b)}
\end{Highlighting}
\end{Shaded}

\begin{verbatim}
## [1]  TRUE FALSE
\end{verbatim}

\begin{Shaded}
\begin{Highlighting}[]
\ControlFlowTok{if}\NormalTok{ (}\SpecialCharTok{!}\NormalTok{(a[}\DecValTok{1}\NormalTok{] }\SpecialCharTok{\textgreater{}}\NormalTok{ b[}\DecValTok{1}\NormalTok{])) \{}\FunctionTok{print}\NormalTok{ (}\StringTok{"a[1] \textless{}= b[1]"}\NormalTok{)\}}
\end{Highlighting}
\end{Shaded}

\begin{verbatim}
## [1] "a[1] <= b[1]"
\end{verbatim}

\hypertarget{special-reserved-values-in-r}{%
\subsection{Special reserved values in R}\label{special-reserved-values-in-r}}

As a good programming practice, do not name variables in the reserved variables. For example, Inf = \(\Infty\), NA = \(Not available\), NULL = Null value.

\begin{Shaded}
\begin{Highlighting}[]
\FunctionTok{print}\NormalTok{ (}\ConstantTok{Inf} \SpecialCharTok{\textgreater{}} \DecValTok{10000}\NormalTok{)}
\end{Highlighting}
\end{Shaded}

\begin{verbatim}
## [1] TRUE
\end{verbatim}

\begin{Shaded}
\begin{Highlighting}[]
\NormalTok{a }\OtherTok{=} \ConstantTok{NULL}
\CommentTok{\#The function is.null() is useful when initialize variables.}
\FunctionTok{print}\NormalTok{(}\FunctionTok{is.null}\NormalTok{(a))}
\end{Highlighting}
\end{Shaded}

\begin{verbatim}
## [1] TRUE
\end{verbatim}

\begin{Shaded}
\begin{Highlighting}[]
\CommentTok{\#NULL means empty, so it is ignored below.}
\NormalTok{a }\OtherTok{=} \FunctionTok{c}\NormalTok{(}\DecValTok{1}\NormalTok{, }\ConstantTok{NA}\NormalTok{, }\DecValTok{3}\NormalTok{, }\ConstantTok{NULL}\NormalTok{, }\ConstantTok{Inf}\NormalTok{)}

\NormalTok{a}
\end{Highlighting}
\end{Shaded}

\begin{verbatim}
## [1]   1  NA   3 Inf
\end{verbatim}

\hypertarget{logic-decisions}{%
\subsection{Logic Decisions}\label{logic-decisions}}

\begin{Shaded}
\begin{Highlighting}[]
\CommentTok{\# if{-}else statement}
\NormalTok{a }\OtherTok{\textless{}{-}} \DecValTok{76}
\NormalTok{b }\OtherTok{\textless{}{-}} \DecValTok{67}
 
\ControlFlowTok{if}\NormalTok{(a }\SpecialCharTok{\textgreater{}}\NormalTok{ b)\{}
\NormalTok{    c }\OtherTok{\textless{}{-}}\NormalTok{ a }\SpecialCharTok{{-}}\NormalTok{ b}
    \FunctionTok{print}\NormalTok{(}\StringTok{"condition a \textgreater{} b is TRUE"}\NormalTok{)}
    \FunctionTok{print}\NormalTok{(}\FunctionTok{paste}\NormalTok{(}\StringTok{"Difference between a, b is : "}\NormalTok{, c))}
\NormalTok{\} }\ControlFlowTok{else}\NormalTok{ \{}
\NormalTok{    c }\OtherTok{\textless{}{-}}\NormalTok{ a }\SpecialCharTok{{-}}\NormalTok{ b}
    \FunctionTok{print}\NormalTok{(}\StringTok{"condition a \textgreater{} b is FALSE"}\NormalTok{)}
    \FunctionTok{print}\NormalTok{(}\FunctionTok{paste}\NormalTok{(}\StringTok{"Difference between a, b is : "}\NormalTok{, c))}
\NormalTok{\}}
\end{Highlighting}
\end{Shaded}

\begin{verbatim}
## [1] "condition a > b is TRUE"
## [1] "Difference between a, b is :  9"
\end{verbatim}

\begin{Shaded}
\begin{Highlighting}[]
\CommentTok{\# R if{-}else{-}if{-}else ladder Example}
\NormalTok{a }\OtherTok{\textless{}{-}} \DecValTok{67}
\NormalTok{b }\OtherTok{\textless{}{-}} \DecValTok{76}
\NormalTok{c }\OtherTok{\textless{}{-}} \DecValTok{99}
 
\ControlFlowTok{if}\NormalTok{(a }\SpecialCharTok{\textgreater{}}\NormalTok{ b }\SpecialCharTok{\&\&}\NormalTok{ b }\SpecialCharTok{\textgreater{}}\NormalTok{ c)}
\NormalTok{\{}
    \FunctionTok{print}\NormalTok{(}\StringTok{"condition a \textgreater{} b \textgreater{} c is TRUE"}\NormalTok{)}
\NormalTok{\} }\ControlFlowTok{else} \ControlFlowTok{if}\NormalTok{(a }\SpecialCharTok{\textless{}}\NormalTok{ b }\SpecialCharTok{\&\&}\NormalTok{ b }\SpecialCharTok{\textgreater{}}\NormalTok{ c)}
\NormalTok{\{}
    \FunctionTok{print}\NormalTok{(}\StringTok{"condition a \textless{} b \textgreater{} c is TRUE"}\NormalTok{)}
\NormalTok{\} }\ControlFlowTok{else} \ControlFlowTok{if}\NormalTok{(a }\SpecialCharTok{\textless{}}\NormalTok{ b }\SpecialCharTok{\&\&}\NormalTok{ b }\SpecialCharTok{\textless{}}\NormalTok{ c)}
\NormalTok{\{}
    \FunctionTok{print}\NormalTok{(}\StringTok{"condition a \textless{} b \textless{} c  is TRUE"}\NormalTok{)}
\NormalTok{\} }\ControlFlowTok{else}\NormalTok{ \{}
  \FunctionTok{print}\NormalTok{(}\StringTok{"None of a \textgreater{} b \textgreater{} c, a \textless{} b \textgreater{} c and a \textless{} b \textless{} c is TRUE."}\NormalTok{)}
\NormalTok{\}}
\end{Highlighting}
\end{Shaded}

\begin{verbatim}
## [1] "condition a < b < c  is TRUE"
\end{verbatim}

\hypertarget{loop}{%
\subsection{Loop}\label{loop}}

\hypertarget{for-loop}{%
\subsubsection{For Loop}\label{for-loop}}

\begin{Shaded}
\begin{Highlighting}[]
\CommentTok{\# R Program to demonstrate}
\CommentTok{\# the use of for loop}
\ControlFlowTok{for}\NormalTok{ (i }\ControlFlowTok{in} \DecValTok{1}\SpecialCharTok{:} \DecValTok{4}\NormalTok{)}
\NormalTok{\{}
    \FunctionTok{print}\NormalTok{(i }\SpecialCharTok{\^{}} \DecValTok{2}\NormalTok{)}
\NormalTok{\}}
\end{Highlighting}
\end{Shaded}

\begin{verbatim}
## [1] 1
## [1] 4
## [1] 9
## [1] 16
\end{verbatim}

\begin{Shaded}
\begin{Highlighting}[]
\CommentTok{\# for loop along with concatenate}
\ControlFlowTok{for}\NormalTok{ (i }\ControlFlowTok{in} \FunctionTok{c}\NormalTok{(}\SpecialCharTok{{-}}\DecValTok{8}\NormalTok{, }\DecValTok{9}\NormalTok{, }\DecValTok{11}\NormalTok{, }\DecValTok{45}\NormalTok{))}
\NormalTok{\{}
    \FunctionTok{print}\NormalTok{(i)}
\NormalTok{\}}
\end{Highlighting}
\end{Shaded}

\begin{verbatim}
## [1] -8
## [1] 9
## [1] 11
## [1] 45
\end{verbatim}

\hypertarget{while-loop}{%
\subsubsection{While Loop}\label{while-loop}}

While loop in R programming language is used when the exact number of iterations of a loop is not known beforehand. It executes the same code again and again until a stop condition is met.

\begin{Shaded}
\begin{Highlighting}[]
\CommentTok{\#while loop}

\NormalTok{i }\OtherTok{=} \DecValTok{1}
\ControlFlowTok{while}\NormalTok{ (i }\SpecialCharTok{\textless{}} \DecValTok{10}\NormalTok{) \{}
   \FunctionTok{print}\NormalTok{(}\FunctionTok{paste}\NormalTok{(}\StringTok{"Current iteration:"}\NormalTok{,i))}
\NormalTok{   i }\OtherTok{=}\NormalTok{ i }\SpecialCharTok{+} \DecValTok{1}
\NormalTok{\}}
\end{Highlighting}
\end{Shaded}

\begin{verbatim}
## [1] "Current iteration: 1"
## [1] "Current iteration: 2"
## [1] "Current iteration: 3"
## [1] "Current iteration: 4"
## [1] "Current iteration: 5"
## [1] "Current iteration: 6"
## [1] "Current iteration: 7"
## [1] "Current iteration: 8"
## [1] "Current iteration: 9"
\end{verbatim}

\begin{Shaded}
\begin{Highlighting}[]
\NormalTok{i }\OtherTok{=} \DecValTok{1}
\ControlFlowTok{while}\NormalTok{ (i }\SpecialCharTok{\textless{}} \DecValTok{10}\NormalTok{) \{}
   \FunctionTok{print}\NormalTok{(}\FunctionTok{paste}\NormalTok{(}\StringTok{"Current iteration:"}\NormalTok{,i))}
   \ControlFlowTok{if}\NormalTok{ (i }\SpecialCharTok{==} \DecValTok{3}\NormalTok{) \{}\ControlFlowTok{break}\NormalTok{\}}
\NormalTok{   i }\OtherTok{=}\NormalTok{ i }\SpecialCharTok{+} \DecValTok{1}
\NormalTok{\}}
\end{Highlighting}
\end{Shaded}

\begin{verbatim}
## [1] "Current iteration: 1"
## [1] "Current iteration: 2"
## [1] "Current iteration: 3"
\end{verbatim}

\hypertarget{break-statement}{%
\subsubsection{Break Statement}\label{break-statement}}

A break statement is a jump statement that is used to terminate the loop at a particular iteration. The program then continues with the next statement outside the loop(if any). When break is encountered within a loop, the loop immediately ceases execution, and control is transferred to the statement immediately following the loop. In the case of nested loops, break only exits the innermost loop in which it is placed.

\begin{Shaded}
\begin{Highlighting}[]
\CommentTok{\# break for loop}
\ControlFlowTok{for}\NormalTok{ (i }\ControlFlowTok{in} \FunctionTok{c}\NormalTok{(}\DecValTok{3}\NormalTok{, }\DecValTok{6}\NormalTok{, }\DecValTok{23}\NormalTok{, }\DecValTok{19}\NormalTok{, }\DecValTok{0}\NormalTok{, }\DecValTok{21}\NormalTok{))\{}
    \ControlFlowTok{if}\NormalTok{ (i }\SpecialCharTok{==} \DecValTok{0}\NormalTok{) \{ }\ControlFlowTok{break}\NormalTok{ \} }
    \FunctionTok{print}\NormalTok{(i)}
\NormalTok{\}}
\end{Highlighting}
\end{Shaded}

\begin{verbatim}
## [1] 3
## [1] 6
## [1] 23
## [1] 19
\end{verbatim}

\begin{Shaded}
\begin{Highlighting}[]
\FunctionTok{print}\NormalTok{(}\StringTok{"Outside Loop"}\NormalTok{)}
\end{Highlighting}
\end{Shaded}

\begin{verbatim}
## [1] "Outside Loop"
\end{verbatim}

\hypertarget{stop-statement}{%
\subsubsection{Stop() statement}\label{stop-statement}}

This function is used to halt the execution of the entire R script or function and signals an error condition. When stop() is called, it generates an error message, and the program execution terminates at that point. This is typically used for handling critical errors or invalid conditions that prevent further meaningful execution. In summary:
break is a control flow statement for exiting loops. stop() is a function for signaling errors and terminating the entire execution.

\begin{Shaded}
\begin{Highlighting}[]
\NormalTok{stop\_example }\OtherTok{\textless{}{-}} \ControlFlowTok{function}\NormalTok{(x) \{}
      \ControlFlowTok{if}\NormalTok{ (x }\SpecialCharTok{\textless{}} \DecValTok{0}\NormalTok{) \{}
        \FunctionTok{stop}\NormalTok{(}\StringTok{"Input cannot be negative."}\NormalTok{) }\CommentTok{\# Halts execution and signals an error}
\NormalTok{      \}}
      \FunctionTok{return}\NormalTok{(}\FunctionTok{log}\NormalTok{(x))}
\NormalTok{    \}}

    \FunctionTok{stop\_example}\NormalTok{(}\DecValTok{2}\NormalTok{) }
\end{Highlighting}
\end{Shaded}

\begin{verbatim}
## [1] 0.6931472
\end{verbatim}

\hypertarget{next-statement}{%
\subsubsection{Next Statement}\label{next-statement}}

It discontinues a particular iteration and jumps to the next iteration. So when next is encountered, that iteration is discarded and the condition is checked again. If true, the next iteration is executed. Hence, the next statement is used to skip a particular iteration in the loop.

\begin{Shaded}
\begin{Highlighting}[]
\ControlFlowTok{for}\NormalTok{ (i }\ControlFlowTok{in} \FunctionTok{c}\NormalTok{(}\DecValTok{3}\NormalTok{, }\DecValTok{6}\NormalTok{, }\DecValTok{23}\NormalTok{, }\DecValTok{19}\NormalTok{, }\DecValTok{0}\NormalTok{, }\DecValTok{21}\NormalTok{)) \{}
    \ControlFlowTok{if}\NormalTok{ (i }\SpecialCharTok{==} \DecValTok{0}\NormalTok{) \{ }\ControlFlowTok{next}\NormalTok{ \}}
    \FunctionTok{print}\NormalTok{(i)}
\NormalTok{\}}
\end{Highlighting}
\end{Shaded}

\begin{verbatim}
## [1] 3
## [1] 6
## [1] 23
## [1] 19
## [1] 21
\end{verbatim}

\begin{Shaded}
\begin{Highlighting}[]
\FunctionTok{print}\NormalTok{(}\StringTok{\textquotesingle{}Outside Loop\textquotesingle{}}\NormalTok{)}
\end{Highlighting}
\end{Shaded}

\begin{verbatim}
## [1] "Outside Loop"
\end{verbatim}

\hypertarget{functions}{%
\subsection{Functions}\label{functions}}

\begin{Shaded}
\begin{Highlighting}[]
\CommentTok{\#function to add 2 numbers}
\NormalTok{add\_num }\OtherTok{\textless{}{-}} \ControlFlowTok{function}\NormalTok{(a,b)\{}
\NormalTok{  sum\_result }\OtherTok{\textless{}{-}}\NormalTok{ a}\SpecialCharTok{+}\NormalTok{b}
  \FunctionTok{return}\NormalTok{(sum\_result)}
\NormalTok{\}}

\FunctionTok{add\_num}\NormalTok{(}\DecValTok{2}\NormalTok{, }\DecValTok{3}\NormalTok{)}
\end{Highlighting}
\end{Shaded}

\begin{verbatim}
## [1] 5
\end{verbatim}

\begin{Shaded}
\begin{Highlighting}[]
\ControlFlowTok{for}\NormalTok{ (i }\ControlFlowTok{in} \DecValTok{1}\SpecialCharTok{:}\DecValTok{10}\NormalTok{)\{}
\NormalTok{  result }\OtherTok{=} \FunctionTok{add\_num}\NormalTok{(}\DecValTok{100}\NormalTok{,i)}
  \FunctionTok{print}\NormalTok{ (result)}
\NormalTok{\}}
\end{Highlighting}
\end{Shaded}

\begin{verbatim}
## [1] 101
## [1] 102
## [1] 103
## [1] 104
## [1] 105
## [1] 106
## [1] 107
## [1] 108
## [1] 109
## [1] 110
\end{verbatim}

\begin{Shaded}
\begin{Highlighting}[]
\NormalTok{area.rectangle }\OtherTok{=} \ControlFlowTok{function}\NormalTok{(}\AttributeTok{length=}\DecValTok{5}\NormalTok{, }\AttributeTok{width=}\DecValTok{4}\NormalTok{)\{}
\NormalTok{  area }\OtherTok{=}\NormalTok{ length }\SpecialCharTok{*}\NormalTok{ width}
  \FunctionTok{return}\NormalTok{(area)}
\NormalTok{\}}
 
\FunctionTok{area.rectangle}\NormalTok{(}\DecValTok{3}\NormalTok{, }\DecValTok{4}\NormalTok{)}
\end{Highlighting}
\end{Shaded}

\begin{verbatim}
## [1] 12
\end{verbatim}

\begin{Shaded}
\begin{Highlighting}[]
\FunctionTok{area.rectangle}\NormalTok{(}\AttributeTok{width =} \DecValTok{4}\NormalTok{, }\AttributeTok{length=}\DecValTok{3}\NormalTok{)}
\end{Highlighting}
\end{Shaded}

\begin{verbatim}
## [1] 12
\end{verbatim}

\begin{Shaded}
\begin{Highlighting}[]
\FunctionTok{area.rectangle}\NormalTok{(}\AttributeTok{length=}\DecValTok{3}\NormalTok{, }\AttributeTok{width =} \DecValTok{4}\NormalTok{)}
\end{Highlighting}
\end{Shaded}

\begin{verbatim}
## [1] 12
\end{verbatim}

\begin{Shaded}
\begin{Highlighting}[]
\FunctionTok{area.rectangle}\NormalTok{() }\CommentTok{\#default}
\end{Highlighting}
\end{Shaded}

\begin{verbatim}
## [1] 20
\end{verbatim}

\hypertarget{useful-conversion-functions-for-data-types}{%
\subsection{Useful Conversion Functions for data types}\label{useful-conversion-functions-for-data-types}}

\hypertarget{conversion-into-numeric-data-type}{%
\subsubsection{Conversion into numeric data type}\label{conversion-into-numeric-data-type}}

\begin{Shaded}
\begin{Highlighting}[]
\CommentTok{\# character data type into numeric data type }
\NormalTok{x}\OtherTok{\textless{}{-}}\FunctionTok{c}\NormalTok{(}\StringTok{\textquotesingle{}1\textquotesingle{}}\NormalTok{, }\StringTok{\textquotesingle{}2\textquotesingle{}}\NormalTok{, }\StringTok{\textquotesingle{}3\textquotesingle{}}\NormalTok{) }
  
\CommentTok{\# Print x }
\FunctionTok{print}\NormalTok{(x) }
\end{Highlighting}
\end{Shaded}

\begin{verbatim}
## [1] "1" "2" "3"
\end{verbatim}

\begin{Shaded}
\begin{Highlighting}[]
\CommentTok{\# Print the type of x }
\FunctionTok{print}\NormalTok{(}\FunctionTok{typeof}\NormalTok{(x)) }
\end{Highlighting}
\end{Shaded}

\begin{verbatim}
## [1] "character"
\end{verbatim}

\begin{Shaded}
\begin{Highlighting}[]
\CommentTok{\# Conversion into numeric data type }
\NormalTok{y}\OtherTok{\textless{}{-}}\FunctionTok{as.numeric}\NormalTok{(x) }
  
\CommentTok{\# print the type of y }
\FunctionTok{print}\NormalTok{(}\FunctionTok{typeof}\NormalTok{(y)) }
\end{Highlighting}
\end{Shaded}

\begin{verbatim}
## [1] "double"
\end{verbatim}

\hypertarget{conversion-into-integer-data-type}{%
\subsubsection{Conversion into integer data type}\label{conversion-into-integer-data-type}}

\begin{Shaded}
\begin{Highlighting}[]
\CommentTok{\# numeric data type into integer data type }
\NormalTok{x}\OtherTok{\textless{}{-}}\FunctionTok{c}\NormalTok{(}\FloatTok{1.3}\NormalTok{, }\FloatTok{5.6}\NormalTok{, }\FloatTok{55.6}\NormalTok{) }
  
\CommentTok{\# Print x }
\FunctionTok{print}\NormalTok{(x) }
\end{Highlighting}
\end{Shaded}

\begin{verbatim}
## [1]  1.3  5.6 55.6
\end{verbatim}

\begin{Shaded}
\begin{Highlighting}[]
\CommentTok{\# Print type of x }
\FunctionTok{print}\NormalTok{(}\FunctionTok{typeof}\NormalTok{(x)) }
\end{Highlighting}
\end{Shaded}

\begin{verbatim}
## [1] "double"
\end{verbatim}

\begin{Shaded}
\begin{Highlighting}[]
\CommentTok{\# Conversion into integer data type }
\NormalTok{y}\OtherTok{\textless{}{-}}\FunctionTok{as.integer}\NormalTok{(x) }
  
\CommentTok{\# Print y }
\FunctionTok{print}\NormalTok{(y) }
\end{Highlighting}
\end{Shaded}

\begin{verbatim}
## [1]  1  5 55
\end{verbatim}

\begin{Shaded}
\begin{Highlighting}[]
\CommentTok{\# Print type of y }
\FunctionTok{print}\NormalTok{(}\FunctionTok{typeof}\NormalTok{(y)) }
\end{Highlighting}
\end{Shaded}

\begin{verbatim}
## [1] "integer"
\end{verbatim}

\hypertarget{conversion-into-character-data-type}{%
\subsubsection{Conversion into character data type}\label{conversion-into-character-data-type}}

\begin{Shaded}
\begin{Highlighting}[]
\CommentTok{\# Conversion into character data type }
\NormalTok{x}\OtherTok{\textless{}{-}}\FunctionTok{c}\NormalTok{(}\FloatTok{1.3}\NormalTok{, }\FloatTok{5.6}\NormalTok{, }\FloatTok{55.6}\NormalTok{) }
  
\CommentTok{\# Print x }
\FunctionTok{print}\NormalTok{(x) }
\end{Highlighting}
\end{Shaded}

\begin{verbatim}
## [1]  1.3  5.6 55.6
\end{verbatim}

\begin{Shaded}
\begin{Highlighting}[]
\CommentTok{\# Print type of x }
\FunctionTok{print}\NormalTok{(}\FunctionTok{typeof}\NormalTok{(x)) }
\end{Highlighting}
\end{Shaded}

\begin{verbatim}
## [1] "double"
\end{verbatim}

\begin{Shaded}
\begin{Highlighting}[]
\NormalTok{y}\OtherTok{\textless{}{-}}\FunctionTok{as.character}\NormalTok{(x) }
  
\CommentTok{\# Print y }
\FunctionTok{print}\NormalTok{(y) }
\end{Highlighting}
\end{Shaded}

\begin{verbatim}
## [1] "1.3"  "5.6"  "55.6"
\end{verbatim}

\begin{Shaded}
\begin{Highlighting}[]
\CommentTok{\# Print type of y }
\FunctionTok{print}\NormalTok{(}\FunctionTok{typeof}\NormalTok{(y)) }
\end{Highlighting}
\end{Shaded}

\begin{verbatim}
## [1] "character"
\end{verbatim}

\hypertarget{conversion-in-to-logical-value}{%
\subsubsection{Conversion in to logical value}\label{conversion-in-to-logical-value}}

\begin{Shaded}
\begin{Highlighting}[]
\CommentTok{\# Conversion in to logical value }
\NormalTok{x }\OtherTok{=} \DecValTok{3}
\NormalTok{y }\OtherTok{=} \DecValTok{8}
  
\NormalTok{result}\OtherTok{\textless{}{-}}\FunctionTok{as.logical}\NormalTok{(x}\SpecialCharTok{\textgreater{}}\NormalTok{y) }
  
\CommentTok{\# Print result }
\FunctionTok{print}\NormalTok{(result) }
\end{Highlighting}
\end{Shaded}

\begin{verbatim}
## [1] FALSE
\end{verbatim}

\hypertarget{conversion-into-date-format}{%
\subsubsection{Conversion into date format}\label{conversion-into-date-format}}

\begin{Shaded}
\begin{Highlighting}[]
\NormalTok{dates }\OtherTok{\textless{}{-}} \FunctionTok{c}\NormalTok{(}\StringTok{"02/27/92"}\NormalTok{, }\StringTok{"02/27/92"}\NormalTok{,  }
           \StringTok{"01/14/92"}\NormalTok{, }\StringTok{"02/28/92"}\NormalTok{,  }
           \StringTok{"02/01/92"}\NormalTok{) }
  
\CommentTok{\# Conversion into date format }
\NormalTok{result}\OtherTok{\textless{}{-}}\FunctionTok{as.Date}\NormalTok{(dates, }\StringTok{"\%m/\%d/\%y"}\NormalTok{) }
  
\CommentTok{\# Print result }
\FunctionTok{print}\NormalTok{(result) }
\end{Highlighting}
\end{Shaded}

\begin{verbatim}
## [1] "1992-02-27" "1992-02-27" "1992-01-14" "1992-02-28" "1992-02-01"
\end{verbatim}

\hypertarget{conversion-in-to-data-frame}{%
\subsubsection{Conversion in to data frame}\label{conversion-in-to-data-frame}}

\begin{Shaded}
\begin{Highlighting}[]
\NormalTok{x }\OtherTok{\textless{}{-}} \DecValTok{1}\SpecialCharTok{:}\DecValTok{5}
\NormalTok{y }\OtherTok{\textless{}{-}} \DecValTok{2}\SpecialCharTok{*}\NormalTok{x}

\CommentTok{\# Conversion in to data frame }
\NormalTok{dat }\OtherTok{\textless{}{-}}\FunctionTok{as.data.frame}\NormalTok{(}\FunctionTok{cbind}\NormalTok{(x, y)) }
  
\NormalTok{dat}
\end{Highlighting}
\end{Shaded}

\begin{verbatim}
##   x  y
## 1 1  2
## 2 2  4
## 3 3  6
## 4 4  8
## 5 5 10
\end{verbatim}

\begin{Shaded}
\begin{Highlighting}[]
\NormalTok{dat}\SpecialCharTok{$}\NormalTok{x}
\end{Highlighting}
\end{Shaded}

\begin{verbatim}
## [1] 1 2 3 4 5
\end{verbatim}

\hypertarget{conversion-into-vector}{%
\subsubsection{Conversion into vector}\label{conversion-into-vector}}

\begin{Shaded}
\begin{Highlighting}[]
\NormalTok{x}\OtherTok{\textless{}{-}}\FunctionTok{c}\NormalTok{(}\AttributeTok{a=}\DecValTok{1}\NormalTok{, }\AttributeTok{b=}\DecValTok{2}\NormalTok{) }

\CommentTok{\# Print x }
\FunctionTok{print}\NormalTok{(x) }
\end{Highlighting}
\end{Shaded}

\begin{verbatim}
## a b 
## 1 2
\end{verbatim}

\begin{Shaded}
\begin{Highlighting}[]
\CommentTok{\# Conversion into vector }
\NormalTok{y}\OtherTok{\textless{}{-}}\FunctionTok{as.vector}\NormalTok{(x) }

\CommentTok{\# Print y }
\FunctionTok{print}\NormalTok{(y) }
\end{Highlighting}
\end{Shaded}

\begin{verbatim}
## [1] 1 2
\end{verbatim}

\hypertarget{conversion-to-matrix}{%
\subsubsection{Conversion to matrix}\label{conversion-to-matrix}}

\begin{Shaded}
\begin{Highlighting}[]
\NormalTok{x }\OtherTok{\textless{}{-}} \DecValTok{1}\SpecialCharTok{:}\DecValTok{21}
\NormalTok{x}
\end{Highlighting}
\end{Shaded}

\begin{verbatim}
##  [1]  1  2  3  4  5  6  7  8  9 10 11 12 13 14 15 16 17 18 19 20 21
\end{verbatim}

\begin{Shaded}
\begin{Highlighting}[]
\NormalTok{y }\OtherTok{\textless{}{-}} \FunctionTok{matrix}\NormalTok{(x, }\AttributeTok{ncol=}\DecValTok{3}\NormalTok{, }\AttributeTok{byrow=}\ConstantTok{FALSE}\NormalTok{)}
\NormalTok{y}
\end{Highlighting}
\end{Shaded}

\begin{verbatim}
##      [,1] [,2] [,3]
## [1,]    1    8   15
## [2,]    2    9   16
## [3,]    3   10   17
## [4,]    4   11   18
## [5,]    5   12   19
## [6,]    6   13   20
## [7,]    7   14   21
\end{verbatim}

\begin{Shaded}
\begin{Highlighting}[]
\NormalTok{y }\OtherTok{\textless{}{-}} \FunctionTok{matrix}\NormalTok{(x, }\AttributeTok{ncol=}\DecValTok{3}\NormalTok{, }\AttributeTok{byrow=}\ConstantTok{TRUE}\NormalTok{)}
\NormalTok{y}
\end{Highlighting}
\end{Shaded}

\begin{verbatim}
##      [,1] [,2] [,3]
## [1,]    1    2    3
## [2,]    4    5    6
## [3,]    7    8    9
## [4,]   10   11   12
## [5,]   13   14   15
## [6,]   16   17   18
## [7,]   19   20   21
\end{verbatim}

\begin{Shaded}
\begin{Highlighting}[]
\NormalTok{dat }\OtherTok{\textless{}{-}} \FunctionTok{as.data.frame}\NormalTok{(y)}
\NormalTok{dat}
\end{Highlighting}
\end{Shaded}

\begin{verbatim}
##   V1 V2 V3
## 1  1  2  3
## 2  4  5  6
## 3  7  8  9
## 4 10 11 12
## 5 13 14 15
## 6 16 17 18
## 7 19 20 21
\end{verbatim}

\begin{Shaded}
\begin{Highlighting}[]
\NormalTok{dat.matrix }\OtherTok{\textless{}{-}} \FunctionTok{as.matrix}\NormalTok{(dat)}
\NormalTok{dat.matrix}
\end{Highlighting}
\end{Shaded}

\begin{verbatim}
##      V1 V2 V3
## [1,]  1  2  3
## [2,]  4  5  6
## [3,]  7  8  9
## [4,] 10 11 12
## [5,] 13 14 15
## [6,] 16 17 18
## [7,] 19 20 21
\end{verbatim}

\hypertarget{plot-in-base-r}{%
\subsection{Plot in Base R}\label{plot-in-base-r}}

The \texttt{plot()} function in base R is a versatile tool that can be used for visualizing both raw data and mathematical functions. The following shows the common use for reference.

\hypertarget{line-plot-of-a-function}{%
\subsubsection{Line Plot of a Function}\label{line-plot-of-a-function}}

\begin{Shaded}
\begin{Highlighting}[]
\CommentTok{\# Plot y = sin(x) for x from 0 to 2π}
\NormalTok{x }\OtherTok{\textless{}{-}} \FunctionTok{seq}\NormalTok{(}\DecValTok{0}\NormalTok{, }\DecValTok{2}\SpecialCharTok{*}\NormalTok{pi, }\AttributeTok{length.out =} \DecValTok{100}\NormalTok{)}
\NormalTok{y }\OtherTok{\textless{}{-}} \FunctionTok{sin}\NormalTok{(x)}
\FunctionTok{plot}\NormalTok{(x, y, }\AttributeTok{type =} \StringTok{"l"}\NormalTok{, }\AttributeTok{col =} \StringTok{"blue"}\NormalTok{, }\AttributeTok{lwd =} \DecValTok{2}\NormalTok{, }\AttributeTok{lty =} \DecValTok{1}\NormalTok{,}
     \AttributeTok{main =} \StringTok{"Plot of y = sin(x)"}\NormalTok{, }\AttributeTok{xlab =} \StringTok{"x"}\NormalTok{, }\AttributeTok{ylab =} \StringTok{"sin(x)"}\NormalTok{)}
\end{Highlighting}
\end{Shaded}

\includegraphics{StatsTB_files/figure-latex/unnamed-chunk-8-1.pdf}

\begin{itemize}
\tightlist
\item
  \texttt{type\ =\ "l"}: draws a line instead of points
\item
  \texttt{col}: line color
\item
  \texttt{lwd}: line width
\item
  \texttt{lty}: line type (1 = solid, 2 = dashed, etc.)
\end{itemize}

\hypertarget{multiple-lines-in-one-plot}{%
\subsubsection{Multiple Lines in One Plot}\label{multiple-lines-in-one-plot}}

\begin{Shaded}
\begin{Highlighting}[]
\CommentTok{\# Compare sin(x) and cos(x)}
\NormalTok{y2 }\OtherTok{\textless{}{-}} \FunctionTok{cos}\NormalTok{(x)}
\FunctionTok{plot}\NormalTok{(x, y, }\AttributeTok{type =} \StringTok{"l"}\NormalTok{, }\AttributeTok{col =} \StringTok{"blue"}\NormalTok{, }\AttributeTok{lwd =} \DecValTok{2}\NormalTok{, }\AttributeTok{lty =} \DecValTok{1}\NormalTok{,}
     \AttributeTok{main =} \StringTok{"sin(x) vs cos(x)"}\NormalTok{, }\AttributeTok{xlab =} \StringTok{"x"}\NormalTok{, }\AttributeTok{ylab =} \StringTok{"Value"}\NormalTok{)}
\FunctionTok{lines}\NormalTok{(x, y2, }\AttributeTok{col =} \StringTok{"red"}\NormalTok{, }\AttributeTok{lwd =} \DecValTok{2}\NormalTok{, }\AttributeTok{lty =} \DecValTok{2}\NormalTok{)}
\FunctionTok{legend}\NormalTok{(}\StringTok{"topright"}\NormalTok{, }\AttributeTok{legend =} \FunctionTok{c}\NormalTok{(}\StringTok{"sin(x)"}\NormalTok{, }\StringTok{"cos(x)"}\NormalTok{),}
       \AttributeTok{col =} \FunctionTok{c}\NormalTok{(}\StringTok{"blue"}\NormalTok{, }\StringTok{"red"}\NormalTok{), }\AttributeTok{lty =} \DecValTok{1}\SpecialCharTok{:}\DecValTok{2}\NormalTok{, }\AttributeTok{lwd =} \DecValTok{2}\NormalTok{)}
\end{Highlighting}
\end{Shaded}

\includegraphics{StatsTB_files/figure-latex/unnamed-chunk-9-1.pdf}

\begin{itemize}
\tightlist
\item
  \texttt{lines()}: adds to existing plot
\item
  \texttt{legend()}: labels for different curves
\end{itemize}

\hypertarget{customizing-plot-appearance}{%
\subsubsection{Customizing Plot Appearance}\label{customizing-plot-appearance}}

\begin{Shaded}
\begin{Highlighting}[]
\CommentTok{\# Customize symbols and colors}
\FunctionTok{set.seed}\NormalTok{(}\DecValTok{123}\NormalTok{)}
\NormalTok{x }\OtherTok{\textless{}{-}} \FunctionTok{rnorm}\NormalTok{(}\DecValTok{50}\NormalTok{)}
\NormalTok{y }\OtherTok{\textless{}{-}} \DecValTok{2} \SpecialCharTok{*}\NormalTok{ x }\SpecialCharTok{+} \FunctionTok{rnorm}\NormalTok{(}\DecValTok{50}\NormalTok{)}
\FunctionTok{plot}\NormalTok{(x, y,}
     \AttributeTok{main =} \StringTok{"Customized Scatter Plot"}\NormalTok{,}
     \AttributeTok{xlab =} \StringTok{"x values"}\NormalTok{, }\AttributeTok{ylab =} \StringTok{"y values"}\NormalTok{,}
     \AttributeTok{col =} \StringTok{"darkgreen"}\NormalTok{, }\AttributeTok{pch =} \DecValTok{19}\NormalTok{)}
\end{Highlighting}
\end{Shaded}

\includegraphics{StatsTB_files/figure-latex/unnamed-chunk-10-1.pdf}

\begin{itemize}
\tightlist
\item
  \texttt{pch}: plot symbol (19 = solid circle)
\item
  \texttt{col}: color of points
\end{itemize}

\hypertarget{plot-area-under-a-curve-using-polygon}{%
\subsubsection{Plot Area Under a Curve Using polygon()}\label{plot-area-under-a-curve-using-polygon}}

To visualize an area under a curve (such as a normal distribution), we can use the \texttt{polygon()} function to shade the area.

\begin{itemize}
\tightlist
\item
  \texttt{dnorm(x)}: density of normal distribution
\item
  \texttt{polygon()}: plots filled area between curve and x-axis
\end{itemize}

This visualization helps demonstrate the empirical rule: \textasciitilde68\% of data lies within ±1 standard deviation.

\begin{Shaded}
\begin{Highlighting}[]
\CommentTok{\# Plot standard normal distribution and shade area between {-}1 and 1}
\NormalTok{x }\OtherTok{\textless{}{-}} \FunctionTok{seq}\NormalTok{(}\SpecialCharTok{{-}}\DecValTok{4}\NormalTok{, }\DecValTok{4}\NormalTok{, }\AttributeTok{length =} \DecValTok{200}\NormalTok{)}
\NormalTok{y }\OtherTok{\textless{}{-}} \FunctionTok{dnorm}\NormalTok{(x) }\CommentTok{\#standard normal curve}

\FunctionTok{plot}\NormalTok{(x, y, }\AttributeTok{type =} \StringTok{"l"}\NormalTok{, }\AttributeTok{lwd =} \DecValTok{2}\NormalTok{, }\AttributeTok{col =} \StringTok{"blue"}\NormalTok{,}
     \AttributeTok{main =} \StringTok{"Shaded Area Under Normal Curve"}\NormalTok{,}
     \AttributeTok{xlab =} \StringTok{"Z"}\NormalTok{, }\AttributeTok{ylab =} \StringTok{"Density"}\NormalTok{)}

\CommentTok{\# Define shaded region}
\NormalTok{x\_shade }\OtherTok{\textless{}{-}} \FunctionTok{seq}\NormalTok{(}\SpecialCharTok{{-}}\DecValTok{1}\NormalTok{, }\DecValTok{1}\NormalTok{, }\AttributeTok{length =} \DecValTok{200}\NormalTok{)}
\NormalTok{y\_shade }\OtherTok{\textless{}{-}} \FunctionTok{dnorm}\NormalTok{(x\_shade)}

\FunctionTok{polygon}\NormalTok{(}\FunctionTok{c}\NormalTok{(}\SpecialCharTok{{-}}\DecValTok{1}\NormalTok{, x\_shade, }\DecValTok{1}\NormalTok{),}
        \FunctionTok{c}\NormalTok{(}\DecValTok{0}\NormalTok{, y\_shade, }\DecValTok{0}\NormalTok{),}
        \AttributeTok{col =} \StringTok{"lightblue"}\NormalTok{, }\AttributeTok{border =} \ConstantTok{NA}\NormalTok{)}
\end{Highlighting}
\end{Shaded}

\includegraphics{StatsTB_files/figure-latex/unnamed-chunk-11-1.pdf}

\hypertarget{summary}{%
\section{Summary}\label{summary}}

In this chapter, we introduced:
- Installation of R and R Studio
- R coding fundamentals including
- Basic calculations
- Assignment
- Creating variables, vectors and data frames
- Logic operations
- Special reserved values in R
- Loop
- Functions
- Useful conversations between data types
- Plot in base R

\hypertarget{population-and-sample}{%
\chapter{Population and Sample}\label{population-and-sample}}

\begin{Shaded}
\begin{Highlighting}[]
\FunctionTok{library}\NormalTok{(IntroStats)}
\end{Highlighting}
\end{Shaded}

\hypertarget{definition-population-and-sample}{%
\section{Definition: Population and Sample}\label{definition-population-and-sample}}

\begin{itemize}
\item
  \textbf{Population:} The collection of all individuals or items under
  consideration in a statistical study.
\item
  \textbf{Sample:} That part of the population from which information is
  obtained.
\end{itemize}

\hypertarget{example}{%
\subsection{Example}\label{example}}

\begin{itemize}
\tightlist
\item
  How would you predict who will win the presidency? Trump or Harris?
\item
  What is the population to address the question? It is the entire US
  population, but it is unrealistic to gauge the sentiment of the population.
  But we would like to get information as representative as possible for the entire population.
\item
  A group of a few thousands people are interviewed and their voting
  are collected. This group is called a sample.
\end{itemize}

\begin{figure}
\centering
\includegraphics{https://i.ibb.co/bR3TN81f/population-sample.png}
\caption{Population and Sample}
\end{figure}

\hypertarget{example-1}{%
\subsection{Example}\label{example-1}}

\includegraphics{https://i.ibb.co/r8wP1vs/100pct.png}
What's Your Reaction?

The objective of this class is to address the key aspects when reading the scientific report with statistical perspective.

\hypertarget{a-typical-clinical-research-with-intent-of-statistical-inference}{%
\subsection{A typical clinical research with intent of statistical inference}\label{a-typical-clinical-research-with-intent-of-statistical-inference}}

\includegraphics{https://i.ibb.co/RGdv9vTH/BRAF.png}
\#\#\# Scientific Method

\begin{figure}
\centering
\includegraphics{https://i.ibb.co/zTpD6MLG/Scientific-Method.png}
\caption{Scientific Method}
\end{figure}

\hypertarget{westin-et-al-2023-study}{%
\subsection{Westin et al (2023) Study}\label{westin-et-al-2023-study}}

\begin{figure}
\centering
\includegraphics{https://i.ibb.co/PvD1nKpy/westin1.png}
\caption{Westin et al (2023) Study}
\end{figure}

\begin{figure}
\centering
\includegraphics{https://i.ibb.co/7JPZFgy3/westin2.png}
\caption{Westin et al (2023) Study}
\end{figure}

\begin{figure}
\centering
\includegraphics{https://i.ibb.co/CsnZ3B2P/westin3.png}
\caption{Westin et al (2023) Study}
\end{figure}

\textbf{Questions:}

\begin{enumerate}
\def\labelenumi{\arabic{enumi}.}
\tightlist
\item
  What is the population in this study?
\item
  What is the sample in the study?
\item
  What is the variable of interest?
\item
  What is the conclusion?
\item
  Is the conclusion intended for the patients included in the study or the designated population?
\item
  What is the limitation of this clinical study as described by the researchers?
\item
  Do you think the limitation describes the concerns of sampling? Why?
\item
  It's known that whether patients responded to the first line treatment (i.e., the previous treatment before coming to this study) may impact their outcome in this study. How would you randomize patients to account for the effect?
\end{enumerate}

\hypertarget{defintion-descriptive-and-inferential}{%
\section{Defintion: Descriptive and Inferential}\label{defintion-descriptive-and-inferential}}

\hypertarget{inferential-statistics}{%
\subsection{Inferential Statistics}\label{inferential-statistics}}

\begin{itemize}
\tightlist
\item
  \textbf{Inferential Statistics} consists of methods for drawing and
  measuring the reliability of conclusions about a \textbf{population}
  based on information obtained from a \textbf{sample} of the population.
\end{itemize}

\hypertarget{descriptive-study-and-inferential-study}{%
\subsection{Descriptive Study and inferential study}\label{descriptive-study-and-inferential-study}}

\begin{itemize}
\item
  \textbf{Descriptive Study:} If the purpose of the study is to examine and
  explore information for its own intrinsic interest only, the study
  is descriptive.
\item
  \textbf{Inferential Study:} If the information is obtained from a sample
  of a population and the purpose of the study is to use that
  information to draw conclusions about the population, the study is
  inferential.
\end{itemize}

\hypertarget{example.-is-the-study-descriptive-or-inferential}{%
\subsubsection{Example. Is the study descriptive or Inferential?}\label{example.-is-the-study-descriptive-or-inferential}}

\begin{itemize}
\tightlist
\item
  The 1948 presidential election results are displayed in table below.
\item
  The study is descriptive. It is a summary of votes and no inference
  is made.
\end{itemize}

\begin{longtable}[]{@{}lll@{}}
\toprule\noalign{}
Ticket & Votes & Percentage \\
\midrule\noalign{}
\endhead
\bottomrule\noalign{}
\endlastfoot
Truman--Barkley (Democratic) & 24,179,345 & 49.7\% \\
Dewey--Warren (Republican) & 21,991,291 & 45.2\% \\
Thurmond--Wright (States Rights) & 1,176,125 & 2.4\% \\
Wallace--Taylor (Progressive) & 1,157,326 & 2.4\% \\
Thomas--Smith (Socialist) & 139,572 & 0.3\% \\
\end{longtable}

\begin{Shaded}
\begin{Highlighting}[]
\NormalTok{df }\OtherTok{\textless{}{-}} \FunctionTok{data.frame}\NormalTok{(}
  \AttributeTok{Ticket =} \FunctionTok{c}\NormalTok{(}\StringTok{"Truman–Barkley (Democratic)"}\NormalTok{,}
             \StringTok{"Dewey–Warren (Republican)"}\NormalTok{,}
             \StringTok{"Thurmond–Wright (States Rights)"}\NormalTok{,}
             \StringTok{"Wallace–Taylor (Progressive)"}\NormalTok{,}
             \StringTok{"Thomas–Smith (Socialist)"}\NormalTok{),}
  \AttributeTok{Votes =} \FunctionTok{c}\NormalTok{(}\StringTok{"24,179,345"}\NormalTok{,}
            \StringTok{"21,991,291"}\NormalTok{,}
            \StringTok{"1,176,125"}\NormalTok{,}
            \StringTok{"1,157,326"}\NormalTok{,}
            \StringTok{"139,572"}\NormalTok{),}
  \AttributeTok{Percentage =} \FunctionTok{c}\NormalTok{(}\StringTok{"49.7\%"}\NormalTok{, }\StringTok{"45.2\%"}\NormalTok{, }\StringTok{"2.4\%"}\NormalTok{, }\StringTok{"2.4\%"}\NormalTok{, }\StringTok{"0.3\%"}\NormalTok{)}
\NormalTok{)}

\NormalTok{knitr}\SpecialCharTok{::}\FunctionTok{kable}\NormalTok{(df, }\AttributeTok{align =} \StringTok{"lrr"}\NormalTok{, }\AttributeTok{caption =} \StringTok{"1948 U.S. Presidential Election Results"}\NormalTok{)}
\end{Highlighting}
\end{Shaded}

\begin{table}

\caption{\label{tab:unnamed-chunk-13}1948 U.S. Presidential Election Results}
\centering
\begin{tabular}[t]{l|r|r}
\hline
Ticket & Votes & Percentage\\
\hline
Truman–Barkley (Democratic) & 24,179,345 & 49.7\%\\
\hline
Dewey–Warren (Republican) & 21,991,291 & 45.2\%\\
\hline
Thurmond–Wright (States Rights) & 1,176,125 & 2.4\%\\
\hline
Wallace–Taylor (Progressive) & 1,157,326 & 2.4\%\\
\hline
Thomas–Smith (Socialist) & 139,572 & 0.3\%\\
\hline
\end{tabular}
\end{table}

\hypertarget{example.-is-rawlings-baseball-livelier}{%
\subsubsection{Example. Is Rawlings baseball livelier?}\label{example.-is-rawlings-baseball-livelier}}

Spalding company had made baseballs for 101 years preceding 1977 for
major league baselines. Then the company stopped making major league
baseballs, and the league bought baseballs from the Rawling Company.

\begin{itemize}
\item
  Pictchers complained that the Rawling baselineballs are livelier.
  Sports Illustrated sponsored a study to address the livelier
  question.
\item
  In this study, an independent testing company randomly selected 85
  balls and measured bounce, weight, and hardness, and compared to
  baseballs in 1952, 1953, 1061, 1963, 1970, and 1973.
\item
  Is this study descriptive or inferential?
\end{itemize}

\hypertarget{example.-is-the-poll-an-inferential-or-descriptive-study}{%
\subsubsection{Example. Is the poll an inferential or descriptive study?}\label{example.-is-the-poll-an-inferential-or-descriptive-study}}

\begin{itemize}
\item
  1948 Presidential Election: Truman (D) vs Dewey (R).
\item
  The Gallup Poll predicted that Truman would win only 44.5\% of the
  vote and be defeated by Thomas Dewey. But the statistician predicted incorrectly.
\end{itemize}

\hypertarget{definition-observational-studies-vs-designed-experiments}{%
\subsection{Definition: Observational Studies vs Designed Experiments}\label{definition-observational-studies-vs-designed-experiments}}

\begin{itemize}
\item
  In an \textbf{observational study}, researchers simply observe
  characteristics and take measurements, as in a sample survey.
\item
  In a \textbf{designed experiment}, researchers impose treatments and/or
  controls, and then observe characteristics and take measurements.
\item
  Observational studies can reveal only association, whereas designed
  experiments can help establish causation.
\end{itemize}

\hypertarget{example.-vasectomies-and-prostate-cancer}{%
\subsubsection{Example. Vasectomies and Prostate Cancer}\label{example.-vasectomies-and-prostate-cancer}}

\begin{itemize}
\item
  Approximately 450, 000 vasectomies are performed each year in the
  US. It's a surgical procedure for contraception. The tube carrying
  sperm from the testicles is cut and tied.
\item
  A retrospective study was performed to understand the relationship
  between vasectomy and prostate cancer, published at J. Ame. Med.
  Assoc.
\item
  Results: 113 prostate cancers among 22,000 men who had a vasectomy,
  vs 70 prostate cancers among 22,000 men who didn't have vasectomy.
\item
  Study reveals 60\% increase of risk for men who had vasectomy,
  thereby an association.
\item
  \textbf{Having vasectomy causes an increased risk of prostate cancer?
  Why?}
\end{itemize}

\hypertarget{example.-folic-acid-and-birth-defects}{%
\subsubsection{Example. Folic Acid and Birth Defects}\label{example.-folic-acid-and-birth-defects}}

\begin{itemize}
\item
  Drs. Czeizel and Dudas published a study at New Eng. J. of Med.
  about birth defects.
\item
  4,753 women prior to conception enrolled to the study, and are
  divided randomly into 2 groups. One group took daily multivatimins
  containing 0.8 mg falic acid, whereas the other group received only
  trace elements.
\item
  13 per 1000 women had major birth defects who took folic acid vs 23
  per 1000 women had major birth defects who di not take folic acid.
\item
  \textbf{Do you think this study can be used to establish causality between
  folic acid and birth defects? Why?}
\end{itemize}

\hypertarget{simple-random-sampling}{%
\section{Simple Random Sampling}\label{simple-random-sampling}}

\hypertarget{definition-census-sampling-and-experimentation}{%
\subsection{Definition: census, sampling and experimentation}\label{definition-census-sampling-and-experimentation}}

\begin{itemize}
\item
  \textbf{Census}: Conducting a census to obtain information for the entire
  population of interest.
\item
  \textbf{Sampling and experimentation}: are common approaches to obtain
  information.
\item
  \textbf{Representative sample}: should reflect as closely as possible the
  relevant characteristics of the entire population.
\item
  \textbf{Probability Sampling}: a random device is used to decide which
  members of the population will constitute the sample rather than
  human judgement.
\item
  \textbf{Can human judgement / brain generate representative randomness?}
  Let's do a small experiment.
\end{itemize}

\hypertarget{definition-simple-random-sampling}{%
\subsection{Definition: Simple Random Sampling}\label{definition-simple-random-sampling}}

\begin{itemize}
\item
  \textbf{Simple Random Sampling:} A sampling procedure for which each
  possible sample of a given size is equally likely to be the one
  obtained.
\item
  \textbf{Simple random sample:} A sample obtained by simple random
  sampling.
\item
  There are two types of simple random sampling.

  \begin{itemize}
  \item
    One is \textbf{simple random sampling with replacement}, whereby a
    member of the population can be selected more than once;
  \item
    the other is \textbf{simple random sampling without replacement},
    whereby a member of the population can be selected at most once.
  \end{itemize}
\end{itemize}

\hypertarget{example.-sampling-27-nj-state-officials-from-governor-phil-murphys-cabinet}{%
\subsubsection{Example. Sampling 27 NJ State Officials from Governor Phil Murphy's cabinet}\label{example.-sampling-27-nj-state-officials-from-governor-phil-murphys-cabinet}}

\begin{Shaded}
\begin{Highlighting}[]
\NormalTok{nj.cabinet }\OtherTok{\textless{}{-}} \FunctionTok{data.frame}\NormalTok{(}
  \AttributeTok{cabinet =} \FunctionTok{c}\NormalTok{(}\StringTok{"Phil Murphy"}\NormalTok{, }
              \StringTok{"Tahesha Way"}\NormalTok{,}
             \StringTok{"Ed Wengryn"}\NormalTok{,}
             \StringTok{"Justin Zimmerman"}\NormalTok{,}
             \StringTok{"Christine Guhl{-}Sadovy"}\NormalTok{,}
             \StringTok{"Christine Norbut Beyer"}\NormalTok{,}
             \StringTok{"Allison Chris Myers"}\NormalTok{,}
             \StringTok{"Jacquelyn A. Suarez"}\NormalTok{,}
             \StringTok{"Kevin Walsh"}\NormalTok{,}
             \StringTok{"Victoria Kuhn"}\NormalTok{,}
             \StringTok{"Tim Sullivan"}\NormalTok{,}
             \StringTok{"Kevin Dehmer"}\NormalTok{,}
             \StringTok{"Shawn LaTourette"}\NormalTok{,}
             \StringTok{"Jeff Brown"}\NormalTok{,}
             \StringTok{"Dr. Brian Bridges"}\NormalTok{,}
             \StringTok{"Laurie Doran"}\NormalTok{,}
             \StringTok{"Sarah Adelman"}\NormalTok{,}
             \StringTok{"Christopher J. Rein"}\NormalTok{,}
             \StringTok{"Dave Cole"}\NormalTok{,}
             \StringTok{"Robert Asaro{-}Angelo"}\NormalTok{,}
             \StringTok{"Matt Platkin"}\NormalTok{,}
             \StringTok{"Lisa Asare"}\NormalTok{,}
             \StringTok{"Brigadier General Yvonne L. Mays"}\NormalTok{,}
             \StringTok{"Latrecia Littles{-}Floyd"}\NormalTok{,}
             \StringTok{"Colonel Patrick J. Callahan"}\NormalTok{,}
             \StringTok{"Elizabeth Maher Muoio"}\NormalTok{,}
             \StringTok{"Francis K. OConnor"}\NormalTok{),}
  \AttributeTok{Role =} \FunctionTok{c}\NormalTok{(}\StringTok{"Governor"}\NormalTok{, }
           \StringTok{"Lt. Governor and Secretary of State"}\NormalTok{, }
           \StringTok{"Secretary, Department of Agriculture"}\NormalTok{, }
           \StringTok{"Commissioner, Department of Banking and Insurance"}\NormalTok{, }
           \StringTok{"President, Board of Public Utilities"}\NormalTok{, }
           \StringTok{"Commissioner, Department of Children and Families"}\NormalTok{, }
           \StringTok{"Chair and Chief Executive Officer, }
\StringTok{           Civil Service Commission"}\NormalTok{, }
           \StringTok{"Commissioner, Department of Community Affairs"}\NormalTok{, }
           \StringTok{"Acting Comptroller, Office of the State Comptroller"}\NormalTok{, }
           \StringTok{"Commissioner, Department of Corrections"}\NormalTok{, }
           \StringTok{"Chief Executive Officer, Economic Development Authority"}\NormalTok{, }
           \StringTok{"Commissioner, Department of Education"}\NormalTok{, }
           \StringTok{"Commissioner, Department of Environmental Protection"}\NormalTok{, }
           \StringTok{"Commissioner, Department of Health"}\NormalTok{, }
           \StringTok{"Secretary, Office of the Secretary of Higher Education"}\NormalTok{, }
           \StringTok{"Director, Office of Homeland Security"}\NormalTok{, }
           \StringTok{"Commissioner, Department of Human Services"}\NormalTok{, }
           \StringTok{"Chief Technology Office, Office of Information Technology"}\NormalTok{, }
           \StringTok{"Chief Innovation Officer, Office of Innovation"}\NormalTok{, }
           \StringTok{"Commissioner, Department of Labor and Workforce Development"}\NormalTok{, }
           \StringTok{"Attorney General, Department of Law and Public Safety"}\NormalTok{, }
           \StringTok{"President and CEO, NJ Maternal and Infant Health Innovation Authority"}\NormalTok{, }
           \StringTok{"Adjutant General and Commissioner, Department of Military and Veteran Affairs"}\NormalTok{, }
           \StringTok{"Acting Chair and Chief Administrator, Motor Vehicle Commission"}\NormalTok{, }
           \StringTok{"Superintendent, New Jersey State Police"}\NormalTok{, }
           \StringTok{"State Treasurer, Department of the Treasury"}\NormalTok{, }
           \StringTok{"Commissioner, Department of Transportation"}\NormalTok{))}

\NormalTok{knitr}\SpecialCharTok{::}\FunctionTok{kable}\NormalTok{(nj.cabinet, }\AttributeTok{align =} \StringTok{"ll"}\NormalTok{, }\AttributeTok{caption =} \StringTok{"NJ Governor\textquotesingle{}s Cabinet, 2025"}\NormalTok{)}
\end{Highlighting}
\end{Shaded}

\begin{table}

\caption{\label{tab:unnamed-chunk-14}NJ Governor's Cabinet, 2025}
\centering
\begin{tabular}[t]{l|l}
\hline
cabinet & Role\\
\hline
Phil Murphy & Governor\\
\hline
Tahesha Way & Lt. Governor and Secretary of State\\
\hline
Ed Wengryn & Secretary, Department of Agriculture\\
\hline
Justin Zimmerman & Commissioner, Department of Banking and Insurance\\
\hline
Christine Guhl-Sadovy & President, Board of Public Utilities\\
\hline
Christine Norbut Beyer & Commissioner, Department of Children and Families\\
\hline
Allison Chris Myers & Chair and Chief Executive Officer, 
           Civil Service Commission\\
\hline
Jacquelyn A. Suarez & Commissioner, Department of Community Affairs\\
\hline
Kevin Walsh & Acting Comptroller, Office of the State Comptroller\\
\hline
Victoria Kuhn & Commissioner, Department of Corrections\\
\hline
Tim Sullivan & Chief Executive Officer, Economic Development Authority\\
\hline
Kevin Dehmer & Commissioner, Department of Education\\
\hline
Shawn LaTourette & Commissioner, Department of Environmental Protection\\
\hline
Jeff Brown & Commissioner, Department of Health\\
\hline
Dr. Brian Bridges & Secretary, Office of the Secretary of Higher Education\\
\hline
Laurie Doran & Director, Office of Homeland Security\\
\hline
Sarah Adelman & Commissioner, Department of Human Services\\
\hline
Christopher J. Rein & Chief Technology Office, Office of Information Technology\\
\hline
Dave Cole & Chief Innovation Officer, Office of Innovation\\
\hline
Robert Asaro-Angelo & Commissioner, Department of Labor and Workforce Development\\
\hline
Matt Platkin & Attorney General, Department of Law and Public Safety\\
\hline
Lisa Asare & President and CEO, NJ Maternal and Infant Health Innovation Authority\\
\hline
Brigadier General Yvonne L. Mays & Adjutant General and Commissioner, Department of Military and Veteran Affairs\\
\hline
Latrecia Littles-Floyd & Acting Chair and Chief Administrator, Motor Vehicle Commission\\
\hline
Colonel Patrick J. Callahan & Superintendent, New Jersey State Police\\
\hline
Elizabeth Maher Muoio & State Treasurer, Department of the Treasury\\
\hline
Francis K. OConnor & Commissioner, Department of Transportation\\
\hline
\end{tabular}
\end{table}

\begin{itemize}
\item
  List all possible samples without replacement of two officials. A total of \(2_C_{27} = 27*26/2 = 351\) options.
\item
  Answer: (Phil Murphy, Tahesha Way), (Phil Murphy, Ed Wengryn), \(\cdots\),
\item
  What is the probability of obtaining any particular sample of 2
  officials?
\item
  Answer: There are a total of 351 options, so the probability of obtaining one particular option is 1/351.
\end{itemize}

\hypertarget{random-sampling-in-r}{%
\subsection{Random Sampling in R}\label{random-sampling-in-r}}

\hypertarget{sampling-from-a-finite-population}{%
\subsubsection{Sampling from a finite population}\label{sampling-from-a-finite-population}}

Draw a random sample of 2 NJ state cabinet members. The population here is defined as the 27 NJ state cabinet members in 2025.

\begin{Shaded}
\begin{Highlighting}[]
\FunctionTok{sample}\NormalTok{(}\AttributeTok{x=}\NormalTok{nj.cabinet}\SpecialCharTok{$}\NormalTok{cabinet, }\AttributeTok{size=}\DecValTok{2}\NormalTok{, }\AttributeTok{replace=}\ConstantTok{FALSE}\NormalTok{)}
\end{Highlighting}
\end{Shaded}

\begin{verbatim}
## [1] "Robert Asaro-Angelo"         "Colonel Patrick J. Callahan"
\end{verbatim}

\hypertarget{sampling-from-a-distribution-from-an-infinite-population}{%
\subsubsection{Sampling from a distribution from an infinite population}\label{sampling-from-a-distribution-from-an-infinite-population}}

Random Number Generator is referring to a process that can be invoked to produce a sequence of random numbers.
- Type runif(10, min=0, max=1) in R will generate 10 random numbers between 0 and 1. These are called pseudo random numbers because the sequence is actually controlled by a seed.

\begin{Shaded}
\begin{Highlighting}[]
\CommentTok{\#try the following multiple times}
\FunctionTok{runif}\NormalTok{(}\DecValTok{10}\NormalTok{, }\AttributeTok{min=}\DecValTok{0}\NormalTok{, }\AttributeTok{max=}\DecValTok{1}\NormalTok{)}
\end{Highlighting}
\end{Shaded}

\begin{verbatim}
##  [1] 0.4025733 0.8802465 0.3640919 0.2882393 0.1706452 0.1721717 0.4820426
##  [8] 0.2529649 0.2162548 0.6743764
\end{verbatim}

The pseudo random numbers are very critical for investigating complex statistical modeling methods. To reproduce the random numbers, we can set a seed.

\begin{Shaded}
\begin{Highlighting}[]
\FunctionTok{set.seed}\NormalTok{(}\DecValTok{2024}\NormalTok{)}

\CommentTok{\#try the following multiple times}
\FunctionTok{runif}\NormalTok{(}\DecValTok{10}\NormalTok{, }\AttributeTok{min=}\DecValTok{0}\NormalTok{, }\AttributeTok{max=}\DecValTok{1}\NormalTok{)}
\end{Highlighting}
\end{Shaded}

\begin{verbatim}
##  [1] 0.8369425 0.3208675 0.6803633 0.6981731 0.4570092 0.7014203 0.4157110
##  [8] 0.3032091 0.8765881 0.1190482
\end{verbatim}

The same numbers are generated every time because the seed is set to the same number. This is important for generating reproducible results when using random numbers in statistics.

\hypertarget{stratified-sampling-with-proportional-allocation}{%
\section{Stratified Sampling with proportional allocation}\label{stratified-sampling-with-proportional-allocation}}

\begin{enumerate}
\def\labelenumi{\arabic{enumi}.}
\item
  Step 1. Divide the population into sub-populations (strata)
\item
  Step 2. From each stratum, obtain a simple random sample of size
  proportional to the size of the stratum.
\item
  Step 3. Use all the members obtained in Step 2 as the sample.
\end{enumerate}

\hypertarget{example-2}{%
\subsection{Example}\label{example-2}}

A population of 100 is divided into 3 strata of sizes 40, 40, and 20. Use stratified sampling method with proportional allocation to obtain a sample of size 10 from the population. \textbf{Solution:}

\begin{itemize}
\item
  Step 1. Divide population into 3 strata and number them as 1-40,
  41-80, 81-100.
\item
  Step 2. From each stratum, obtain a simple random sample of size
  proportional to the size of the stratum. \[\begin{aligned}
      \text{Stratum 1 Sample Size} &=& \text{Total Sample Size}\times \frac{\text{Stratum 1 size}}{\text{Population size}} \nonumber\\
      &=& 10\times \frac{40}{100} = 4.\nonumber
      \end{aligned}\]
\item
  Similarly, stratum 2 and 3 have sizes 4 and 2 respectively.
\item
  In summary, 4 numbers drawn from 1-40, 4 numbers drawn from 41-80, and 2 numbers drawn from 81-100.
\end{itemize}

\begin{Shaded}
\begin{Highlighting}[]
\NormalTok{s100 }\OtherTok{\textless{}{-}} \FunctionTok{data.frame}\NormalTok{(}\AttributeTok{x =} \DecValTok{1}\SpecialCharTok{:}\DecValTok{100}\NormalTok{, }\AttributeTok{strata =} \FunctionTok{c}\NormalTok{(}\FunctionTok{rep}\NormalTok{(}\DecValTok{1}\NormalTok{, }\DecValTok{40}\NormalTok{), }\FunctionTok{rep}\NormalTok{(}\DecValTok{2}\NormalTok{, }\DecValTok{40}\NormalTok{), }\FunctionTok{rep}\NormalTok{(}\DecValTok{3}\NormalTok{, }\DecValTok{20}\NormalTok{)))}

\NormalTok{knitr}\SpecialCharTok{::}\FunctionTok{kable}\NormalTok{(s100, }\AttributeTok{align =} \StringTok{"lr"}\NormalTok{, }\AttributeTok{caption =} \StringTok{"Numbers 1{-}100 with 3 strata"}\NormalTok{)}
\end{Highlighting}
\end{Shaded}

\begin{table}

\caption{\label{tab:unnamed-chunk-18}Numbers 1-100 with 3 strata}
\centering
\begin{tabular}[t]{l|r}
\hline
x & strata\\
\hline
1 & 1\\
\hline
2 & 1\\
\hline
3 & 1\\
\hline
4 & 1\\
\hline
5 & 1\\
\hline
6 & 1\\
\hline
7 & 1\\
\hline
8 & 1\\
\hline
9 & 1\\
\hline
10 & 1\\
\hline
11 & 1\\
\hline
12 & 1\\
\hline
13 & 1\\
\hline
14 & 1\\
\hline
15 & 1\\
\hline
16 & 1\\
\hline
17 & 1\\
\hline
18 & 1\\
\hline
19 & 1\\
\hline
20 & 1\\
\hline
21 & 1\\
\hline
22 & 1\\
\hline
23 & 1\\
\hline
24 & 1\\
\hline
25 & 1\\
\hline
26 & 1\\
\hline
27 & 1\\
\hline
28 & 1\\
\hline
29 & 1\\
\hline
30 & 1\\
\hline
31 & 1\\
\hline
32 & 1\\
\hline
33 & 1\\
\hline
34 & 1\\
\hline
35 & 1\\
\hline
36 & 1\\
\hline
37 & 1\\
\hline
38 & 1\\
\hline
39 & 1\\
\hline
40 & 1\\
\hline
41 & 2\\
\hline
42 & 2\\
\hline
43 & 2\\
\hline
44 & 2\\
\hline
45 & 2\\
\hline
46 & 2\\
\hline
47 & 2\\
\hline
48 & 2\\
\hline
49 & 2\\
\hline
50 & 2\\
\hline
51 & 2\\
\hline
52 & 2\\
\hline
53 & 2\\
\hline
54 & 2\\
\hline
55 & 2\\
\hline
56 & 2\\
\hline
57 & 2\\
\hline
58 & 2\\
\hline
59 & 2\\
\hline
60 & 2\\
\hline
61 & 2\\
\hline
62 & 2\\
\hline
63 & 2\\
\hline
64 & 2\\
\hline
65 & 2\\
\hline
66 & 2\\
\hline
67 & 2\\
\hline
68 & 2\\
\hline
69 & 2\\
\hline
70 & 2\\
\hline
71 & 2\\
\hline
72 & 2\\
\hline
73 & 2\\
\hline
74 & 2\\
\hline
75 & 2\\
\hline
76 & 2\\
\hline
77 & 2\\
\hline
78 & 2\\
\hline
79 & 2\\
\hline
80 & 2\\
\hline
81 & 3\\
\hline
82 & 3\\
\hline
83 & 3\\
\hline
84 & 3\\
\hline
85 & 3\\
\hline
86 & 3\\
\hline
87 & 3\\
\hline
88 & 3\\
\hline
89 & 3\\
\hline
90 & 3\\
\hline
91 & 3\\
\hline
92 & 3\\
\hline
93 & 3\\
\hline
94 & 3\\
\hline
95 & 3\\
\hline
96 & 3\\
\hline
97 & 3\\
\hline
98 & 3\\
\hline
99 & 3\\
\hline
100 & 3\\
\hline
\end{tabular}
\end{table}

The following code can perform the job. Please install the R packages devtools and IntroStats if not yet.

\begin{Shaded}
\begin{Highlighting}[]
\CommentTok{\#Install devtools if not installed yet}
\ControlFlowTok{if}\NormalTok{ (}\SpecialCharTok{!}\FunctionTok{require}\NormalTok{(}\StringTok{"devtools"}\NormalTok{)) }\FunctionTok{install.packages}\NormalTok{(}\StringTok{"devtools"}\NormalTok{)}

\CommentTok{\#Install IntroStats if not installed yet}
\ControlFlowTok{if}\NormalTok{ (}\SpecialCharTok{!}\FunctionTok{require}\NormalTok{(}\StringTok{"IntroStats"}\NormalTok{)) devtools}\SpecialCharTok{::}\FunctionTok{install\_github}\NormalTok{(}\StringTok{"phe9480/IntroStats"}\NormalTok{)}

\CommentTok{\#Draw a sample of 10 numbers according to the proportion of strata size.}

\NormalTok{IntroStats}\SpecialCharTok{::}\FunctionTok{strat.sample}\NormalTok{(}\AttributeTok{x=}\NormalTok{s100}\SpecialCharTok{$}\NormalTok{x, }\AttributeTok{size=} \DecValTok{10}\NormalTok{, }\AttributeTok{strata =}\NormalTok{ s100}\SpecialCharTok{$}\NormalTok{strata)}
\end{Highlighting}
\end{Shaded}

\begin{verbatim}
##  [1] 34 29 11 16 69 54 74 66 92 98
\end{verbatim}

\hypertarget{example-3}{%
\subsection{Example}\label{example-3}}

A town has 250 homeowners of which 25, 175, and 50 are upper income,
middle income, and low income, respectively. Explain how we can obtain a
sample of 20 homeowners, using stratified sampling with proportional
allocation, stratified by income group.

\begin{itemize}
\tightlist
\item
  Step 1. Divide 250 homeowners into 3 income classes: upper, middle, low.
\end{itemize}

\begin{Shaded}
\begin{Highlighting}[]
\NormalTok{homeowners }\OtherTok{=} \FunctionTok{data.frame}\NormalTok{(}\AttributeTok{x =} \DecValTok{1}\SpecialCharTok{:}\DecValTok{250}\NormalTok{, }\AttributeTok{strata =} \FunctionTok{c}\NormalTok{(}\FunctionTok{rep}\NormalTok{(}\StringTok{"Upper"}\NormalTok{, }\DecValTok{25}\NormalTok{), }\FunctionTok{rep}\NormalTok{(}\StringTok{"Middle"}\NormalTok{, }\DecValTok{175}\NormalTok{), }\FunctionTok{rep}\NormalTok{(}\StringTok{"Low"}\NormalTok{, }\DecValTok{50}\NormalTok{)))}
\NormalTok{homeowners}
\end{Highlighting}
\end{Shaded}

\begin{verbatim}
##       x strata
## 1     1  Upper
## 2     2  Upper
## 3     3  Upper
## 4     4  Upper
## 5     5  Upper
## 6     6  Upper
## 7     7  Upper
## 8     8  Upper
## 9     9  Upper
## 10   10  Upper
## 11   11  Upper
## 12   12  Upper
## 13   13  Upper
## 14   14  Upper
## 15   15  Upper
## 16   16  Upper
## 17   17  Upper
## 18   18  Upper
## 19   19  Upper
## 20   20  Upper
## 21   21  Upper
## 22   22  Upper
## 23   23  Upper
## 24   24  Upper
## 25   25  Upper
## 26   26 Middle
## 27   27 Middle
## 28   28 Middle
## 29   29 Middle
## 30   30 Middle
## 31   31 Middle
## 32   32 Middle
## 33   33 Middle
## 34   34 Middle
## 35   35 Middle
## 36   36 Middle
## 37   37 Middle
## 38   38 Middle
## 39   39 Middle
## 40   40 Middle
## 41   41 Middle
## 42   42 Middle
## 43   43 Middle
## 44   44 Middle
## 45   45 Middle
## 46   46 Middle
## 47   47 Middle
## 48   48 Middle
## 49   49 Middle
## 50   50 Middle
## 51   51 Middle
## 52   52 Middle
## 53   53 Middle
## 54   54 Middle
## 55   55 Middle
## 56   56 Middle
## 57   57 Middle
## 58   58 Middle
## 59   59 Middle
## 60   60 Middle
## 61   61 Middle
## 62   62 Middle
## 63   63 Middle
## 64   64 Middle
## 65   65 Middle
## 66   66 Middle
## 67   67 Middle
## 68   68 Middle
## 69   69 Middle
## 70   70 Middle
## 71   71 Middle
## 72   72 Middle
## 73   73 Middle
## 74   74 Middle
## 75   75 Middle
## 76   76 Middle
## 77   77 Middle
## 78   78 Middle
## 79   79 Middle
## 80   80 Middle
## 81   81 Middle
## 82   82 Middle
## 83   83 Middle
## 84   84 Middle
## 85   85 Middle
## 86   86 Middle
## 87   87 Middle
## 88   88 Middle
## 89   89 Middle
## 90   90 Middle
## 91   91 Middle
## 92   92 Middle
## 93   93 Middle
## 94   94 Middle
## 95   95 Middle
## 96   96 Middle
## 97   97 Middle
## 98   98 Middle
## 99   99 Middle
## 100 100 Middle
## 101 101 Middle
## 102 102 Middle
## 103 103 Middle
## 104 104 Middle
## 105 105 Middle
## 106 106 Middle
## 107 107 Middle
## 108 108 Middle
## 109 109 Middle
## 110 110 Middle
## 111 111 Middle
## 112 112 Middle
## 113 113 Middle
## 114 114 Middle
## 115 115 Middle
## 116 116 Middle
## 117 117 Middle
## 118 118 Middle
## 119 119 Middle
## 120 120 Middle
## 121 121 Middle
## 122 122 Middle
## 123 123 Middle
## 124 124 Middle
## 125 125 Middle
## 126 126 Middle
## 127 127 Middle
## 128 128 Middle
## 129 129 Middle
## 130 130 Middle
## 131 131 Middle
## 132 132 Middle
## 133 133 Middle
## 134 134 Middle
## 135 135 Middle
## 136 136 Middle
## 137 137 Middle
## 138 138 Middle
## 139 139 Middle
## 140 140 Middle
## 141 141 Middle
## 142 142 Middle
## 143 143 Middle
## 144 144 Middle
## 145 145 Middle
## 146 146 Middle
## 147 147 Middle
## 148 148 Middle
## 149 149 Middle
## 150 150 Middle
## 151 151 Middle
## 152 152 Middle
## 153 153 Middle
## 154 154 Middle
## 155 155 Middle
## 156 156 Middle
## 157 157 Middle
## 158 158 Middle
## 159 159 Middle
## 160 160 Middle
## 161 161 Middle
## 162 162 Middle
## 163 163 Middle
## 164 164 Middle
## 165 165 Middle
## 166 166 Middle
## 167 167 Middle
## 168 168 Middle
## 169 169 Middle
## 170 170 Middle
## 171 171 Middle
## 172 172 Middle
## 173 173 Middle
## 174 174 Middle
## 175 175 Middle
## 176 176 Middle
## 177 177 Middle
## 178 178 Middle
## 179 179 Middle
## 180 180 Middle
## 181 181 Middle
## 182 182 Middle
## 183 183 Middle
## 184 184 Middle
## 185 185 Middle
## 186 186 Middle
## 187 187 Middle
## 188 188 Middle
## 189 189 Middle
## 190 190 Middle
## 191 191 Middle
## 192 192 Middle
## 193 193 Middle
## 194 194 Middle
## 195 195 Middle
## 196 196 Middle
## 197 197 Middle
## 198 198 Middle
## 199 199 Middle
## 200 200 Middle
## 201 201    Low
## 202 202    Low
## 203 203    Low
## 204 204    Low
## 205 205    Low
## 206 206    Low
## 207 207    Low
## 208 208    Low
## 209 209    Low
## 210 210    Low
## 211 211    Low
## 212 212    Low
## 213 213    Low
## 214 214    Low
## 215 215    Low
## 216 216    Low
## 217 217    Low
## 218 218    Low
## 219 219    Low
## 220 220    Low
## 221 221    Low
## 222 222    Low
## 223 223    Low
## 224 224    Low
## 225 225    Low
## 226 226    Low
## 227 227    Low
## 228 228    Low
## 229 229    Low
## 230 230    Low
## 231 231    Low
## 232 232    Low
## 233 233    Low
## 234 234    Low
## 235 235    Low
## 236 236    Low
## 237 237    Low
## 238 238    Low
## 239 239    Low
## 240 240    Low
## 241 241    Low
## 242 242    Low
## 243 243    Low
## 244 244    Low
## 245 245    Low
## 246 246    Low
## 247 247    Low
## 248 248    Low
## 249 249    Low
## 250 250    Low
\end{verbatim}

\begin{itemize}
\item
  Step 2. From each stratum, obtain a simple random sample of size
  proportional to the size of the stratum. For a sample with a total size of 20, the sample size for 1st stratum according to the proportion is \(20*25/250 = 2\). Similarly, the 2nd stratum and 3rd stratum have sample sizes: \(20*175/250=14\) and \(20*50/250=4\).
\item
  Step 3. Combine all sample random samples together.
\end{itemize}

\begin{Shaded}
\begin{Highlighting}[]
\NormalTok{IntroStats}\SpecialCharTok{::}\FunctionTok{strat.sample}\NormalTok{(}\AttributeTok{x=}\NormalTok{homeowners}\SpecialCharTok{$}\NormalTok{x, }\AttributeTok{size=}\DecValTok{20}\NormalTok{, }\AttributeTok{strata=}\NormalTok{homeowners}\SpecialCharTok{$}\NormalTok{strata)}
\end{Highlighting}
\end{Shaded}

\begin{verbatim}
##  [1]  21   1  60 130 106  53  73 109 195 141  45 112 131 149 157 160 216 220 203
## [20] 225
\end{verbatim}

\textbf{Discussion}: Why is it useful to perform stratified sampling with
proportional allocation?

\hypertarget{summary-1}{%
\section{Summary}\label{summary-1}}

In this chapter, we discussed the fundamental statistical concepts including

\begin{itemize}
\tightlist
\item
  population and sample,
\item
  descriptive and inferential statistics,
\item
  simple random sampling,
\item
  stratified random sampling with allocation proportional to stratum size, and
\item
  their implementation in R.
\end{itemize}

\hypertarget{organizing-data}{%
\chapter{Organizing Data}\label{organizing-data}}

\begin{Shaded}
\begin{Highlighting}[]
\FunctionTok{library}\NormalTok{(IntroStats)}
\end{Highlighting}
\end{Shaded}

\hypertarget{variables-and-data}{%
\section{Variables and Data}\label{variables-and-data}}

\hypertarget{definition}{%
\subsection{Definition}\label{definition}}

\begin{itemize}
\tightlist
\item
  \textbf{Variable}: A characteristic that varies from one person or thing to another. For example, height, eye color.
\item
  \textbf{Quantitative variables}: Numerical variables. Height is a quantitative variable.
\item
  \textbf{Qualitative variables}: non-numerical variables, also called categorical variables. Eye color is a quanlitative variable.
\item
  \textbf{Discrete variable}: a quantitative variable whose possible values can be listed. For example, number of siblings.
\item
  \textbf{Continuous variable}: a quantitative variable whose possible values form some interval of numbers. For example, height of a person.
\end{itemize}

\begin{figure}
\centering
\includegraphics{https://i.ibb.co/xsXX1mx/variables.png}
\caption{Figure 1. Classification of Variables}
\end{figure}

\hypertarget{definition-1}{%
\subsection{Definition}\label{definition-1}}

\begin{itemize}
\tightlist
\item
  \textbf{Data}: Values of a variable
\item
  \textbf{Qualitative data}: Values of a qualitative variable
\item
  \textbf{Quantitative data}: Values of a quantitative variable.
\item
  \textbf{Discrete data}: Values of a discrete variable.
\item
  \textbf{Continuous data}: Values of a continuous variable.
\end{itemize}

\hypertarget{example-4}{%
\subsection{Example}\label{example-4}}

Human blood types: A, B, AB, O. What kind of data do you receive when you are told your blood type?

\textbf{Solution}: Blood type is a qualitative variable. Therefore your blood type is qualitative data.

\hypertarget{example-5}{%
\subsection{Example}\label{example-5}}

The U.S. Census Bureau collects data on household size and publishes the information in Current Population Reports. What kind of data is the number of people in your household?

\textbf{Solution}: Household size is a quantitative variable, which is also a discrete variable because its possible values are 1, 2, \ldots{}

\hypertarget{organizing-qualitative-data}{%
\section{Organizing Qualitative Data}\label{organizing-qualitative-data}}

\hypertarget{definition-frequency-distribution-of-qualitative-data}{%
\subsection{Definition: Frequency Distribution of Qualitative Data}\label{definition-frequency-distribution-of-qualitative-data}}

\textbf{A frequency distribution} of qualitative data is a listing of the distinct values and their frequencies.

\hypertarget{example-6}{%
\subsection{Example}\label{example-6}}

A class of students have the political party affiliation below. Order them and construct a frequency table.

\begin{Shaded}
\begin{Highlighting}[]
\NormalTok{x }\OtherTok{=} \FunctionTok{c}\NormalTok{(}\StringTok{"Democratic"}\NormalTok{, }\StringTok{"Democratic"}\NormalTok{,}\StringTok{"Republic"}\NormalTok{, }\StringTok{"Republic"}\NormalTok{, }\StringTok{"Republic"}\NormalTok{, }\StringTok{"Democratic"}\NormalTok{, }
      \StringTok{"Democratic"}\NormalTok{,}\StringTok{"Democratic"}\NormalTok{,}\StringTok{"Democratic"}\NormalTok{, }\StringTok{"Democratic"}\NormalTok{,}\StringTok{"Democratic"}\NormalTok{,}\StringTok{"Democratic"}\NormalTok{,}
      \StringTok{"Democratic"}\NormalTok{,}\StringTok{"Republic"}\NormalTok{, }\StringTok{"Republic"}\NormalTok{, }\StringTok{"Republic"}\NormalTok{, }\StringTok{"Democratic"}\NormalTok{,}\StringTok{"Democratic"}\NormalTok{,}
      \StringTok{"Republic"}\NormalTok{, }\StringTok{"Republic"}\NormalTok{, }\StringTok{"Republic"}\NormalTok{, }\StringTok{"Republic"}\NormalTok{, }\StringTok{"Republic"}\NormalTok{, }\StringTok{"Republic"}\NormalTok{, }
      \StringTok{"Republic"}\NormalTok{, }\StringTok{"Republic"}\NormalTok{, }\StringTok{"Democratic"}\NormalTok{, }\StringTok{"Republic"}\NormalTok{, }\StringTok{"Republic"}\NormalTok{,  }\StringTok{"Republic"}\NormalTok{, }
      \StringTok{"Democratic"}\NormalTok{,}\StringTok{"Democratic"}\NormalTok{, }\StringTok{"Republic"}\NormalTok{, }\StringTok{"Republic"}\NormalTok{, }\StringTok{"Republic"}\NormalTok{, }\StringTok{"Republic"}\NormalTok{, }
     \StringTok{"Republic"}\NormalTok{, }\StringTok{"Republic"}\NormalTok{, }\StringTok{"Other"}\NormalTok{, }\StringTok{"Other"}\NormalTok{, }\StringTok{"Other"}\NormalTok{, }\StringTok{"Other"}\NormalTok{, }\StringTok{"Other"}\NormalTok{)}
\end{Highlighting}
\end{Shaded}

Order the values:

\begin{Shaded}
\begin{Highlighting}[]
\FunctionTok{sort}\NormalTok{(x)}
\end{Highlighting}
\end{Shaded}

\begin{verbatim}
##  [1] "Democratic" "Democratic" "Democratic" "Democratic" "Democratic"
##  [6] "Democratic" "Democratic" "Democratic" "Democratic" "Democratic"
## [11] "Democratic" "Democratic" "Democratic" "Democratic" "Democratic"
## [16] "Other"      "Other"      "Other"      "Other"      "Other"     
## [21] "Republic"   "Republic"   "Republic"   "Republic"   "Republic"  
## [26] "Republic"   "Republic"   "Republic"   "Republic"   "Republic"  
## [31] "Republic"   "Republic"   "Republic"   "Republic"   "Republic"  
## [36] "Republic"   "Republic"   "Republic"   "Republic"   "Republic"  
## [41] "Republic"   "Republic"   "Republic"
\end{verbatim}

Count frequency:

\begin{Shaded}
\begin{Highlighting}[]
\FunctionTok{table}\NormalTok{(x)}
\end{Highlighting}
\end{Shaded}

\begin{verbatim}
## x
## Democratic      Other   Republic 
##         15          5         23
\end{verbatim}

\hypertarget{relative-frequency}{%
\subsection{Relative Frequency}\label{relative-frequency}}

\[
\text{Relative Frequency} = \frac{\text{Frequency}}{\text{Number of Observations}}
\]

\begin{Shaded}
\begin{Highlighting}[]
\NormalTok{freq }\OtherTok{=} \FunctionTok{table}\NormalTok{(x)}
\NormalTok{Relative\_frequency }\OtherTok{=}\NormalTok{ freq  }\SpecialCharTok{/} \FunctionTok{sum}\NormalTok{(freq)}

\NormalTok{Relative\_frequency}
\end{Highlighting}
\end{Shaded}

\begin{verbatim}
## x
## Democratic      Other   Republic 
##  0.3488372  0.1162791  0.5348837
\end{verbatim}

Sum of relative frequencies equals 1.

\hypertarget{pie-chart}{%
\subsection{Pie chart}\label{pie-chart}}

Construct a pie chart for political party affiliations. Use R function \texttt{pie()}. Either frequency or relative frequency can be used for drawing pie chart.

\begin{Shaded}
\begin{Highlighting}[]
\FunctionTok{pie}\NormalTok{(Relative\_frequency)}
\end{Highlighting}
\end{Shaded}

\includegraphics{StatsTB_files/figure-latex/unnamed-chunk-27-1.pdf}

Same pie chart here:

\begin{Shaded}
\begin{Highlighting}[]
\FunctionTok{pie}\NormalTok{(freq)}
\end{Highlighting}
\end{Shaded}

\includegraphics{StatsTB_files/figure-latex/unnamed-chunk-28-1.pdf}
All possible party affiliations in a pie chart must add up to 100\%.

\hypertarget{bar-charts}{%
\subsection{Bar charts}\label{bar-charts}}

Bar chart of frequency:

\begin{Shaded}
\begin{Highlighting}[]
\FunctionTok{barplot}\NormalTok{(freq, }\AttributeTok{col =} \FunctionTok{c}\NormalTok{(}\StringTok{"blue"}\NormalTok{, }\StringTok{"green"}\NormalTok{, }\StringTok{"red"}\NormalTok{), }\AttributeTok{ylab=}\StringTok{"Frequency"}\NormalTok{)}
\end{Highlighting}
\end{Shaded}

\includegraphics{StatsTB_files/figure-latex/unnamed-chunk-29-1.pdf}

Bar chart of frequency:

\begin{Shaded}
\begin{Highlighting}[]
\FunctionTok{barplot}\NormalTok{(Relative\_frequency, }\AttributeTok{col =} \FunctionTok{c}\NormalTok{(}\StringTok{"blue"}\NormalTok{, }\StringTok{"green"}\NormalTok{, }\StringTok{"red"}\NormalTok{), }\AttributeTok{ylab=}\StringTok{"Relative Frequency"}\NormalTok{)}
\end{Highlighting}
\end{Shaded}

\includegraphics{StatsTB_files/figure-latex/unnamed-chunk-30-1.pdf}

\hypertarget{organizing-quantitative-data}{%
\section{Organizing Quantitative Data}\label{organizing-quantitative-data}}

\hypertarget{single-value-grouping}{%
\subsection{Single-value Grouping}\label{single-value-grouping}}

Example. Number of TVs in each of 50 randomly selected households. Group the data by single values.

\begin{Shaded}
\begin{Highlighting}[]
\NormalTok{x }\OtherTok{=} \FunctionTok{c}\NormalTok{(}\DecValTok{1}\NormalTok{,}\DecValTok{1}\NormalTok{,}\DecValTok{1}\NormalTok{,}\DecValTok{2}\NormalTok{,}\DecValTok{3}\NormalTok{,}\DecValTok{5}\NormalTok{,}\DecValTok{2}\NormalTok{,}\DecValTok{0}\NormalTok{,}\DecValTok{4}\NormalTok{,}\DecValTok{5}\NormalTok{,}
      \DecValTok{3}\NormalTok{,}\DecValTok{2}\NormalTok{,}\DecValTok{3}\NormalTok{,}\DecValTok{2}\NormalTok{,}\DecValTok{3}\NormalTok{,}\DecValTok{2}\NormalTok{,}\DecValTok{2}\NormalTok{,}\DecValTok{1}\NormalTok{,}\DecValTok{1}\NormalTok{,}\DecValTok{1}\NormalTok{,}
      \DecValTok{4}\NormalTok{,}\DecValTok{4}\NormalTok{,}\DecValTok{3}\NormalTok{,}\DecValTok{3}\NormalTok{,}\DecValTok{4}\NormalTok{,}\DecValTok{2}\NormalTok{,}\DecValTok{2}\NormalTok{,}\DecValTok{2}\NormalTok{,}\DecValTok{1}\NormalTok{,}\DecValTok{1}\NormalTok{,}
      \DecValTok{6}\NormalTok{,}\DecValTok{4}\NormalTok{,}\DecValTok{3}\NormalTok{,}\DecValTok{2}\NormalTok{,}\DecValTok{2}\NormalTok{,}\DecValTok{3}\NormalTok{,}\DecValTok{4}\NormalTok{,}\DecValTok{5}\NormalTok{,}\DecValTok{3}\NormalTok{,}\DecValTok{2}\NormalTok{,}
      \DecValTok{4}\NormalTok{,}\DecValTok{3}\NormalTok{,}\DecValTok{3}\NormalTok{,}\DecValTok{3}\NormalTok{,}\DecValTok{2}\NormalTok{,}\DecValTok{2}\NormalTok{,}\DecValTok{2}\NormalTok{,}\DecValTok{1}\NormalTok{,}\DecValTok{1}\NormalTok{,}\DecValTok{1}\NormalTok{)}

\CommentTok{\#single{-}value grouping}
\FunctionTok{table}\NormalTok{(x)}
\end{Highlighting}
\end{Shaded}

\begin{verbatim}
## x
##  0  1  2  3  4  5  6 
##  1 11 15 12  7  3  1
\end{verbatim}

\hypertarget{limit-grouping}{%
\subsection{Limit Grouping}\label{limit-grouping}}

Multiple classes are created and each class consists of a range of values. Suppose we create categories 0-1, 2, 3, 4-6, then we can summarize as

\begin{itemize}
\tightlist
\item
  \textbf{0-1}: 12
\item
  \textbf{2}: 15
\item
  \textbf{3}: 12
\item
  \textbf{4-6}: 11
\end{itemize}

\hypertarget{cutpoint-grouping}{%
\subsection{Cutpoint Grouping}\label{cutpoint-grouping}}

\begin{itemize}
\tightlist
\item
  \textbf{\textless=2}: 27
\item
  \textbf{\textgreater2}: 23
\end{itemize}

\hypertarget{histogram}{%
\subsection{Histogram}\label{histogram}}

Histogram is to display the distribution of a sample with relative chance of occurrence. A peak point indicates a value the occurs the most frequently. It is often superimposed with a probability density curve, which indicates the same meaning.

\begin{Shaded}
\begin{Highlighting}[]
      \DocumentationTok{\#\#\#\#\#\#\#\#\#\#\#\#\#\#\#\#\#\#\#\#\#\#\#\#}
        \CommentTok{\#Histgram}
        \DocumentationTok{\#\#\#\#\#\#\#\#\#\#\#\#\#\#\#\#\#\#\#\#\#\#\#\#}
\NormalTok{        x }\OtherTok{=} \FunctionTok{rnorm}\NormalTok{(}\DecValTok{100}\NormalTok{)}
        \FunctionTok{hist}\NormalTok{(x) }
\end{Highlighting}
\end{Shaded}

\includegraphics{StatsTB_files/figure-latex/unnamed-chunk-32-1.pdf}

\begin{Shaded}
\begin{Highlighting}[]
    \FunctionTok{hist}\NormalTok{(x, }\AttributeTok{col =} \DecValTok{5}\NormalTok{, }\AttributeTok{main=}\StringTok{"Default Histogram"}\NormalTok{)}
\end{Highlighting}
\end{Shaded}

\includegraphics{StatsTB_files/figure-latex/unnamed-chunk-32-2.pdf}

\begin{Shaded}
\begin{Highlighting}[]
        \FunctionTok{hist}\NormalTok{(x, }\AttributeTok{col =} \DecValTok{5}\NormalTok{, }\AttributeTok{density =} \DecValTok{20}\NormalTok{, }\AttributeTok{main =} \StringTok{"Density = 20"}\NormalTok{)}
\end{Highlighting}
\end{Shaded}

\includegraphics{StatsTB_files/figure-latex/unnamed-chunk-32-3.pdf}

\begin{Shaded}
\begin{Highlighting}[]
        \FunctionTok{hist}\NormalTok{(x, }\AttributeTok{freq=}\ConstantTok{FALSE}\NormalTok{, }\AttributeTok{col =} \DecValTok{6}\NormalTok{, }\AttributeTok{main=}\StringTok{"Relative Frequency"}\NormalTok{)}
\end{Highlighting}
\end{Shaded}

\includegraphics{StatsTB_files/figure-latex/unnamed-chunk-32-4.pdf}

\begin{Shaded}
\begin{Highlighting}[]
        \FunctionTok{hist}\NormalTok{(x, }\AttributeTok{freq=}\ConstantTok{FALSE}\NormalTok{, }\AttributeTok{col =} \StringTok{"pink"}\NormalTok{, }\AttributeTok{main =} \StringTok{"Curve Overlay"}\NormalTok{)}
        \FunctionTok{lines}\NormalTok{(}\FunctionTok{density}\NormalTok{(x), }\AttributeTok{col =} \StringTok{"blue"}\NormalTok{, }\AttributeTok{lwd =} \DecValTok{3}\NormalTok{)}
\end{Highlighting}
\end{Shaded}

\includegraphics{StatsTB_files/figure-latex/unnamed-chunk-32-5.pdf}

A histogram displays the classes of the quantitative data on a horizontal axis and the frequencies (relative frequencies, percents) of those classes on a vertical axis. The frequency (relative frequency, percent) of each class is represented by a vertical bar whose height is equal to the frequency (relative frequency, percent) of that class. The bars should be positioned so that they touch each other.

To construct a histogram:

\begin{itemize}
\item
  \textbf{Step 1} Obtain a frequency (relative-frequency, percent) distribution of the data.
\item
  \textbf{Step 2} Draw a horizontal axis on which to place the bars and a vertical axis on which to display the frequencies (relative frequencies, percents).
\item
  \textbf{Step 3} For each class, construct a vertical bar whose height equals the frequency (relative frequency, percent) of that class.
\item
  \textbf{Step 4} Label the bars with the classes, as explained in Definition 2.9, the horizontal axis with the name of the variable, and the vertical axis with ``Frequency'' (``Relative frequency,'' ``Percent'').
\end{itemize}

\begin{Shaded}
\begin{Highlighting}[]
\FunctionTok{hist}\NormalTok{(x, }\AttributeTok{xlab=}\StringTok{"Number of TVs at Household"}\NormalTok{, }\AttributeTok{main=}\StringTok{"Histogram of Number of TVs at Household"}\NormalTok{, }\AttributeTok{col=}\StringTok{"maroon"}\NormalTok{)}
\end{Highlighting}
\end{Shaded}

\includegraphics{StatsTB_files/figure-latex/unnamed-chunk-33-1.pdf}

\begin{Shaded}
\begin{Highlighting}[]
\FunctionTok{hist}\NormalTok{(x, }\AttributeTok{xlab=}\StringTok{"Number of TVs at Household"}\NormalTok{, }\AttributeTok{main=}\StringTok{"Histogram of Number of TVs at Household"}\NormalTok{, }\AttributeTok{freq=}\ConstantTok{FALSE}\NormalTok{, }\AttributeTok{ylab=}\StringTok{"Relative Frequency"}\NormalTok{, }\AttributeTok{col=}\StringTok{"turquoise"}\NormalTok{)}
\end{Highlighting}
\end{Shaded}

\includegraphics{StatsTB_files/figure-latex/unnamed-chunk-34-1.pdf}

\hypertarget{dotplots}{%
\subsection{Dotplots}\label{dotplots}}

A dotplot is a graph in which each observation is plotted as a dot at an appropriate place above a horizontal axis. Observations having equal values are stacked vertically.

\begin{itemize}
\tightlist
\item
  \textbf{Step 1} Draw a horizontal axis that displays the possible values of the quantitative data.
\item
  \textbf{Step 2} Record each observation by placing a dot over the appropriate value on the horizontal axis.
\item
  \textbf{Step 3} Label the horizontal axis with the name of the variable.
\end{itemize}

\begin{Shaded}
\begin{Highlighting}[]
\FunctionTok{dotchart}\NormalTok{(x, }\AttributeTok{col=}\StringTok{"turquoise"}\NormalTok{, }\AttributeTok{xlab=}\StringTok{"Number of TVs at Household"}\NormalTok{, }
         \AttributeTok{main=}\StringTok{"Dot Chart of Number of TVs at Household"}\NormalTok{,)}
\end{Highlighting}
\end{Shaded}

\includegraphics{StatsTB_files/figure-latex/unnamed-chunk-35-1.pdf}

\hypertarget{stem-leaf-diagrams}{%
\subsection{Stem-leaf diagrams}\label{stem-leaf-diagrams}}

In a stem-and-leaf diagram (or stemplot), each observation Is separated into two parts, namely, a stem-consisting of all but the rightmost digit- and a leaf, the rightmost digit.

To Construct a stem-and-leaf diagram

\begin{itemize}
\tightlist
\item
  \textbf{Step 1} Think of each observation as a stem-consisting of all but the rightmost digit-and a leaf, the rightmost digit.
\item
  \textbf{Step 2} Write the stems from smallest to largest in a vertical column to the left of a vertical rule.
\item
  \textbf{Step 3} Write each leaf to the right of the vertical rule in the row that contains the appropriate stem.
\item
  \textbf{Step 4} Arrange the leaves in each row in ascending order.
\end{itemize}

\begin{Shaded}
\begin{Highlighting}[]
\NormalTok{x }\OtherTok{=} \FunctionTok{c}\NormalTok{(}\DecValTok{54}\NormalTok{, }\DecValTok{43}\NormalTok{, }\DecValTok{67}\NormalTok{, }\DecValTok{76}\NormalTok{, }\DecValTok{45}\NormalTok{, }\DecValTok{59}\NormalTok{, }\DecValTok{66}\NormalTok{, }\DecValTok{66}\NormalTok{, }\DecValTok{68}\NormalTok{, }\DecValTok{78}\NormalTok{, }\DecValTok{80}\NormalTok{, }\DecValTok{92}\NormalTok{)}

\FunctionTok{stem}\NormalTok{(x)}
\end{Highlighting}
\end{Shaded}

\begin{verbatim}
## 
##   The decimal point is 1 digit(s) to the right of the |
## 
##   4 | 35
##   5 | 49
##   6 | 6678
##   7 | 68
##   8 | 0
##   9 | 2
\end{verbatim}

\hypertarget{summary-2}{%
\section{Summary}\label{summary-2}}

In this chapter, we covered the methods for organizing data including

\begin{itemize}
\tightlist
\item
  \textbf{Variables and Data}: Different types of variables and data
\item
  \textbf{Organizing Qualitative Data}: How to organize qualitative data
\item
  \textbf{Organizing Quantitative Data}: How to organize quantitative data
\end{itemize}

\hypertarget{distribution-shapes}{%
\chapter{Distribution Shapes}\label{distribution-shapes}}

\begin{Shaded}
\begin{Highlighting}[]
\FunctionTok{library}\NormalTok{(IntroStats)}
\end{Highlighting}
\end{Shaded}

\hypertarget{distribution-of-a-dataset}{%
\section{Distribution of a dataset}\label{distribution-of-a-dataset}}

The \textbf{distribution} of a dataset describes the values and how often they occur. It can be shown as:
- Frequency or relative frequency tables
- Histograms for continuous data
- Density curves or probability distribution functions
- Cumulative distribution functions (CDF)
Understanding the distribution helps us see the center (mean/median), spread (variance/IQR), and shape (modality, symmetry, skewness, tails).

\hypertarget{example-7}{%
\subsection{Example}\label{example-7}}

A class of 12 students scored:
\texttt{58,\ 60,\ 61,\ 65,\ 66,\ 68,\ 70,\ 72,\ 75,\ 78,\ 80,\ 85}. Create a frequency table and histogram, and mark the mean and median.

\textbf{Solution:}

\begin{Shaded}
\begin{Highlighting}[]
\NormalTok{scores }\OtherTok{\textless{}{-}} \FunctionTok{c}\NormalTok{(}\DecValTok{58}\NormalTok{,}\DecValTok{60}\NormalTok{,}\DecValTok{61}\NormalTok{,}\DecValTok{65}\NormalTok{,}\DecValTok{66}\NormalTok{,}\DecValTok{68}\NormalTok{,}\DecValTok{70}\NormalTok{,}\DecValTok{72}\NormalTok{,}\DecValTok{75}\NormalTok{,}\DecValTok{78}\NormalTok{,}\DecValTok{80}\NormalTok{,}\DecValTok{85}\NormalTok{)}
\NormalTok{tab }\OtherTok{\textless{}{-}} \FunctionTok{table}\NormalTok{(}\FunctionTok{cut}\NormalTok{(scores, }\AttributeTok{breaks =} \FunctionTok{seq}\NormalTok{(}\DecValTok{55}\NormalTok{, }\DecValTok{90}\NormalTok{, }\AttributeTok{by=}\DecValTok{5}\NormalTok{), }\AttributeTok{right =} \ConstantTok{FALSE}\NormalTok{))}
\NormalTok{tab}
\end{Highlighting}
\end{Shaded}

\begin{verbatim}
## 
## [55,60) [60,65) [65,70) [70,75) [75,80) [80,85) [85,90) 
##       1       2       3       2       2       1       1
\end{verbatim}

\begin{Shaded}
\begin{Highlighting}[]
\FunctionTok{mean}\NormalTok{(scores); }\FunctionTok{median}\NormalTok{(scores)}
\end{Highlighting}
\end{Shaded}

\begin{verbatim}
## [1] 69.83333
\end{verbatim}

\begin{verbatim}
## [1] 69
\end{verbatim}

\begin{Shaded}
\begin{Highlighting}[]
\FunctionTok{hist}\NormalTok{(scores, }\AttributeTok{breaks =} \FunctionTok{seq}\NormalTok{(}\DecValTok{55}\NormalTok{, }\DecValTok{90}\NormalTok{, }\AttributeTok{by=}\DecValTok{5}\NormalTok{), }\AttributeTok{prob =} \ConstantTok{TRUE}\NormalTok{, }\AttributeTok{main =} \StringTok{"Scores Histogram"}\NormalTok{,}
     \AttributeTok{xlab =} \StringTok{"Score"}\NormalTok{)}
\FunctionTok{lines}\NormalTok{(}\FunctionTok{density}\NormalTok{(scores), }\AttributeTok{lwd=}\DecValTok{2}\NormalTok{)}
\FunctionTok{abline}\NormalTok{(}\AttributeTok{v =} \FunctionTok{mean}\NormalTok{(scores), }\AttributeTok{lwd=}\DecValTok{2}\NormalTok{, }\AttributeTok{lty=}\DecValTok{2}\NormalTok{)}
\FunctionTok{abline}\NormalTok{(}\AttributeTok{v =} \FunctionTok{median}\NormalTok{(scores), }\AttributeTok{lwd=}\DecValTok{2}\NormalTok{, }\AttributeTok{lty=}\DecValTok{3}\NormalTok{)}
\FunctionTok{legend}\NormalTok{(}\StringTok{"topright"}\NormalTok{, }\AttributeTok{legend=}\FunctionTok{c}\NormalTok{(}\StringTok{"Mean"}\NormalTok{,}\StringTok{"Median"}\NormalTok{), }\AttributeTok{lty=}\FunctionTok{c}\NormalTok{(}\DecValTok{2}\NormalTok{,}\DecValTok{3}\NormalTok{), }\AttributeTok{bty=}\StringTok{"n"}\NormalTok{)}
\end{Highlighting}
\end{Shaded}

\includegraphics{StatsTB_files/figure-latex/unnamed-chunk-38-1.pdf}
The mean and median are close, showing a relatively symmetric distribution.

\hypertarget{example-8}{%
\subsection{Example}\label{example-8}}

\begin{Shaded}
\begin{Highlighting}[]
\NormalTok{height }\OtherTok{=} \FunctionTok{rnorm}\NormalTok{(}\DecValTok{1000}\NormalTok{, }\AttributeTok{mean=}\DecValTok{64}\NormalTok{, }\AttributeTok{sd=}\DecValTok{3}\NormalTok{)}
\FunctionTok{hist}\NormalTok{(height, }\AttributeTok{prob =} \ConstantTok{TRUE}\NormalTok{, }\AttributeTok{col=}\StringTok{"purple"}\NormalTok{, }\AttributeTok{breaks =} \DecValTok{20}\NormalTok{)}
\FunctionTok{lines}\NormalTok{(}\FunctionTok{density}\NormalTok{(height), }\AttributeTok{col =} \DecValTok{4}\NormalTok{, }\AttributeTok{lwd =} \DecValTok{2}\NormalTok{)}
\end{Highlighting}
\end{Shaded}

\includegraphics{StatsTB_files/figure-latex/unnamed-chunk-39-1.pdf}

\hypertarget{modality}{%
\section{Modality}\label{modality}}

Examples of (a) unimodal, (b) bimodal, and (c) multimodal distributions, defined based on the number of peaks.

\begin{Shaded}
\begin{Highlighting}[]
\CommentTok{\#Unimodal}
\FunctionTok{draw.multi.modes}\NormalTok{(}\AttributeTok{peaks.loc =} \FunctionTok{c}\NormalTok{(}\DecValTok{5}\NormalTok{),  }\AttributeTok{width=}\FunctionTok{c}\NormalTok{(}\DecValTok{3}\NormalTok{), }
                 \AttributeTok{xlim=}\FunctionTok{c}\NormalTok{(}\SpecialCharTok{{-}}\DecValTok{5}\NormalTok{, }\DecValTok{15}\NormalTok{), }\AttributeTok{ylim=}\FunctionTok{c}\NormalTok{(}\DecValTok{0}\NormalTok{, }\FloatTok{0.42}\NormalTok{), }
                 \AttributeTok{main=}\StringTok{"Peak: 5, width: 3"}\NormalTok{)}
\end{Highlighting}
\end{Shaded}

\includegraphics{StatsTB_files/figure-latex/unnamed-chunk-40-1.pdf}

\begin{Shaded}
\begin{Highlighting}[]
\FunctionTok{draw.multi.modes}\NormalTok{(}\AttributeTok{peaks.loc =} \FunctionTok{c}\NormalTok{(}\DecValTok{5}\NormalTok{),  }\AttributeTok{width=}\FunctionTok{c}\NormalTok{(}\DecValTok{2}\NormalTok{), }
                 \AttributeTok{xlim=}\FunctionTok{c}\NormalTok{(}\SpecialCharTok{{-}}\DecValTok{5}\NormalTok{, }\DecValTok{15}\NormalTok{), }\AttributeTok{ylim=}\FunctionTok{c}\NormalTok{(}\DecValTok{0}\NormalTok{, }\FloatTok{0.42}\NormalTok{), }
                 \AttributeTok{main=}\StringTok{"Peak: 5, width: 2"}\NormalTok{)}
\end{Highlighting}
\end{Shaded}

\includegraphics{StatsTB_files/figure-latex/unnamed-chunk-40-2.pdf}

\begin{Shaded}
\begin{Highlighting}[]
\FunctionTok{draw.multi.modes}\NormalTok{(}\AttributeTok{peaks.loc =} \FunctionTok{c}\NormalTok{(}\DecValTok{5}\NormalTok{),  }\AttributeTok{width=}\FunctionTok{c}\NormalTok{(}\DecValTok{1}\NormalTok{), }
                 \AttributeTok{xlim=}\FunctionTok{c}\NormalTok{(}\SpecialCharTok{{-}}\DecValTok{5}\NormalTok{, }\DecValTok{15}\NormalTok{), }\AttributeTok{ylim=}\FunctionTok{c}\NormalTok{(}\DecValTok{0}\NormalTok{, }\FloatTok{0.42}\NormalTok{), }
                 \AttributeTok{main=}\StringTok{"Peak: 5, width: 1"}\NormalTok{)}
\end{Highlighting}
\end{Shaded}

\includegraphics{StatsTB_files/figure-latex/unnamed-chunk-40-3.pdf}

\begin{Shaded}
\begin{Highlighting}[]
\CommentTok{\#Bimodal}
\FunctionTok{draw.multi.modes}\NormalTok{(}\AttributeTok{peaks.loc =} \FunctionTok{c}\NormalTok{(}\DecValTok{5}\NormalTok{, }\DecValTok{15}\NormalTok{),  }\AttributeTok{width=}\FunctionTok{c}\NormalTok{(}\DecValTok{1}\NormalTok{,}\DecValTok{1}\NormalTok{),}
                 \AttributeTok{xlim=}\FunctionTok{c}\NormalTok{(}\SpecialCharTok{{-}}\DecValTok{5}\NormalTok{, }\DecValTok{25}\NormalTok{), }\AttributeTok{ylim=}\FunctionTok{c}\NormalTok{(}\DecValTok{0}\NormalTok{, }\FloatTok{0.25}\NormalTok{), }
                 \AttributeTok{main=}\StringTok{"Peak: 5 15, width: 1 1"}\NormalTok{)}
\end{Highlighting}
\end{Shaded}

\includegraphics{StatsTB_files/figure-latex/unnamed-chunk-41-1.pdf}

\begin{Shaded}
\begin{Highlighting}[]
\FunctionTok{draw.multi.modes}\NormalTok{(}\AttributeTok{peaks.loc =} \FunctionTok{c}\NormalTok{(}\DecValTok{5}\NormalTok{, }\DecValTok{15}\NormalTok{),  }\AttributeTok{width=}\FunctionTok{c}\NormalTok{(}\DecValTok{1}\NormalTok{,}\DecValTok{2}\NormalTok{),}
                 \AttributeTok{xlim=}\FunctionTok{c}\NormalTok{(}\SpecialCharTok{{-}}\DecValTok{5}\NormalTok{, }\DecValTok{25}\NormalTok{), }\AttributeTok{ylim=}\FunctionTok{c}\NormalTok{(}\DecValTok{0}\NormalTok{, }\FloatTok{0.25}\NormalTok{), }
                 \AttributeTok{main=}\StringTok{"Peak: 5 15, width: 1 2"}\NormalTok{)}
\end{Highlighting}
\end{Shaded}

\includegraphics{StatsTB_files/figure-latex/unnamed-chunk-41-2.pdf}

\begin{Shaded}
\begin{Highlighting}[]
\FunctionTok{draw.multi.modes}\NormalTok{(}\AttributeTok{peaks.loc =} \FunctionTok{c}\NormalTok{(}\DecValTok{5}\NormalTok{, }\DecValTok{15}\NormalTok{),  }\AttributeTok{width=}\FunctionTok{c}\NormalTok{(}\DecValTok{1}\NormalTok{,}\DecValTok{3}\NormalTok{),}
                 \AttributeTok{xlim=}\FunctionTok{c}\NormalTok{(}\SpecialCharTok{{-}}\DecValTok{5}\NormalTok{, }\DecValTok{25}\NormalTok{), }\AttributeTok{ylim=}\FunctionTok{c}\NormalTok{(}\DecValTok{0}\NormalTok{, }\FloatTok{0.25}\NormalTok{), }
                 \AttributeTok{main=}\StringTok{"Peak: 5 15, width: 1 3"}\NormalTok{)}
\end{Highlighting}
\end{Shaded}

\includegraphics{StatsTB_files/figure-latex/unnamed-chunk-41-3.pdf}

\begin{Shaded}
\begin{Highlighting}[]
\FunctionTok{draw.multi.modes}\NormalTok{(}\AttributeTok{peaks.loc =} \FunctionTok{c}\NormalTok{(}\DecValTok{5}\NormalTok{, }\DecValTok{15}\NormalTok{),  }\AttributeTok{width=}\FunctionTok{c}\NormalTok{(}\DecValTok{2}\NormalTok{,}\DecValTok{2}\NormalTok{),}
                 \AttributeTok{xlim=}\FunctionTok{c}\NormalTok{(}\SpecialCharTok{{-}}\DecValTok{5}\NormalTok{, }\DecValTok{25}\NormalTok{), }\AttributeTok{ylim=}\FunctionTok{c}\NormalTok{(}\DecValTok{0}\NormalTok{, }\FloatTok{0.25}\NormalTok{), }
                 \AttributeTok{main=}\StringTok{"Peak: 5 15, width: 2 2"}\NormalTok{)}
\end{Highlighting}
\end{Shaded}

\includegraphics{StatsTB_files/figure-latex/unnamed-chunk-41-4.pdf}

\begin{Shaded}
\begin{Highlighting}[]
\FunctionTok{draw.multi.modes}\NormalTok{(}\AttributeTok{peaks.loc =} \FunctionTok{c}\NormalTok{(}\DecValTok{5}\NormalTok{, }\DecValTok{15}\NormalTok{),  }\AttributeTok{width=}\FunctionTok{c}\NormalTok{(}\DecValTok{3}\NormalTok{,}\DecValTok{3}\NormalTok{),}
                 \AttributeTok{xlim=}\FunctionTok{c}\NormalTok{(}\SpecialCharTok{{-}}\DecValTok{5}\NormalTok{, }\DecValTok{25}\NormalTok{), }\AttributeTok{ylim=}\FunctionTok{c}\NormalTok{(}\DecValTok{0}\NormalTok{, }\FloatTok{0.42}\NormalTok{), }
                 \AttributeTok{main=}\StringTok{"Peak: 5 15, width: 3 3"}\NormalTok{)}
\end{Highlighting}
\end{Shaded}

\includegraphics{StatsTB_files/figure-latex/unnamed-chunk-41-5.pdf}

\begin{Shaded}
\begin{Highlighting}[]
\CommentTok{\#3{-}modal}
\FunctionTok{draw.multi.modes}\NormalTok{(}\AttributeTok{peaks.loc =} \FunctionTok{c}\NormalTok{(}\DecValTok{5}\NormalTok{, }\DecValTok{15}\NormalTok{, }\DecValTok{25}\NormalTok{),  }\AttributeTok{width=}\FunctionTok{c}\NormalTok{(}\DecValTok{3}\NormalTok{,}\DecValTok{3}\NormalTok{,}\DecValTok{3}\NormalTok{),}
                 \AttributeTok{xlim=}\FunctionTok{c}\NormalTok{(}\SpecialCharTok{{-}}\DecValTok{5}\NormalTok{, }\DecValTok{35}\NormalTok{), }\AttributeTok{ylim=}\FunctionTok{c}\NormalTok{(}\DecValTok{0}\NormalTok{, }\FloatTok{0.15}\NormalTok{), }
                 \AttributeTok{main=}\StringTok{"Peak: 5 15 25, width: 3 3 3"}\NormalTok{)}
\end{Highlighting}
\end{Shaded}

\includegraphics{StatsTB_files/figure-latex/unnamed-chunk-42-1.pdf}

\begin{Shaded}
\begin{Highlighting}[]
\FunctionTok{draw.multi.modes}\NormalTok{(}\AttributeTok{peaks.loc =} \FunctionTok{c}\NormalTok{(}\DecValTok{5}\NormalTok{, }\DecValTok{15}\NormalTok{, }\DecValTok{25}\NormalTok{),  }\AttributeTok{width=}\FunctionTok{c}\NormalTok{(}\DecValTok{1}\NormalTok{,}\DecValTok{2}\NormalTok{,}\DecValTok{3}\NormalTok{),}
                 \AttributeTok{xlim=}\FunctionTok{c}\NormalTok{(}\SpecialCharTok{{-}}\DecValTok{5}\NormalTok{, }\DecValTok{35}\NormalTok{), }\AttributeTok{ylim=}\FunctionTok{c}\NormalTok{(}\DecValTok{0}\NormalTok{, }\FloatTok{0.15}\NormalTok{), }
                 \AttributeTok{main=}\StringTok{"Peak: 5 15 25, width: 1 2 3"}\NormalTok{)}
\end{Highlighting}
\end{Shaded}

\includegraphics{StatsTB_files/figure-latex/unnamed-chunk-42-2.pdf}

\begin{Shaded}
\begin{Highlighting}[]
\FunctionTok{draw.multi.modes}\NormalTok{(}\AttributeTok{peaks.loc =} \FunctionTok{c}\NormalTok{(}\DecValTok{5}\NormalTok{, }\DecValTok{15}\NormalTok{, }\DecValTok{25}\NormalTok{),  }\AttributeTok{width=}\FunctionTok{c}\NormalTok{(}\DecValTok{3}\NormalTok{,}\DecValTok{2}\NormalTok{,}\DecValTok{3}\NormalTok{),}
                 \AttributeTok{xlim=}\FunctionTok{c}\NormalTok{(}\SpecialCharTok{{-}}\DecValTok{5}\NormalTok{, }\DecValTok{35}\NormalTok{), }\AttributeTok{ylim=}\FunctionTok{c}\NormalTok{(}\DecValTok{0}\NormalTok{, }\FloatTok{0.15}\NormalTok{), }
                 \AttributeTok{main=}\StringTok{"Peak: 5 15 25, width: 3 2 3"}\NormalTok{)}
\end{Highlighting}
\end{Shaded}

\includegraphics{StatsTB_files/figure-latex/unnamed-chunk-42-3.pdf}

\hypertarget{symmetry}{%
\section{Symmetry}\label{symmetry}}

Examples of symmetric distributions: (a) bell shaped, (b) triangular, and (c) uniform

\begin{Shaded}
\begin{Highlighting}[]
\CommentTok{\#bell shaped}
\FunctionTok{draw.bell.shape}\NormalTok{(}\AttributeTok{peak.loc=}\DecValTok{10}\NormalTok{, }\AttributeTok{width=}\DecValTok{2}\NormalTok{, }\AttributeTok{main=}\StringTok{"Bell Shaped Distribution"}\NormalTok{)}
\end{Highlighting}
\end{Shaded}

\includegraphics{StatsTB_files/figure-latex/unnamed-chunk-43-1.pdf}

\begin{Shaded}
\begin{Highlighting}[]
\CommentTok{\#Triangular shaped}
\FunctionTok{draw.triangle.shape}\NormalTok{(}\AttributeTok{peak.loc=}\DecValTok{10}\NormalTok{, }\AttributeTok{slope=}\FloatTok{0.25}\NormalTok{, }\AttributeTok{n=}\DecValTok{20}\NormalTok{,}
                    \AttributeTok{main=}\StringTok{"Triangle Shaped Distribution"}\NormalTok{,}
                    \AttributeTok{ylim=}\FunctionTok{c}\NormalTok{(}\DecValTok{0}\NormalTok{, }\FloatTok{0.5}\NormalTok{), }\AttributeTok{xlim=}\FunctionTok{c}\NormalTok{(}\DecValTok{7}\NormalTok{, }\DecValTok{13}\NormalTok{))}
\end{Highlighting}
\end{Shaded}

\includegraphics{StatsTB_files/figure-latex/unnamed-chunk-44-1.pdf}

\begin{Shaded}
\begin{Highlighting}[]
\FunctionTok{draw.triangle.shape}\NormalTok{(}\AttributeTok{peak.loc=}\DecValTok{10}\NormalTok{, }\AttributeTok{slope=}\DecValTok{1}\NormalTok{, }\AttributeTok{n=}\DecValTok{20}\NormalTok{,}
                    \AttributeTok{main=}\StringTok{"Triangle Shaped Distribution"}\NormalTok{,}
                    \AttributeTok{ylim=}\FunctionTok{c}\NormalTok{(}\DecValTok{0}\NormalTok{, }\FloatTok{0.5}\NormalTok{), }\AttributeTok{xlim=}\FunctionTok{c}\NormalTok{(}\DecValTok{7}\NormalTok{, }\DecValTok{13}\NormalTok{))}
\end{Highlighting}
\end{Shaded}

\includegraphics{StatsTB_files/figure-latex/unnamed-chunk-44-2.pdf}

\begin{Shaded}
\begin{Highlighting}[]
\FunctionTok{draw.triangle.shape}\NormalTok{(}\AttributeTok{peak.loc=}\DecValTok{10}\NormalTok{, }\AttributeTok{slope=}\DecValTok{4}\NormalTok{, }\AttributeTok{n=}\DecValTok{20}\NormalTok{,}
                    \AttributeTok{main=}\StringTok{"Triangle Shaped Distribution"}\NormalTok{,}
                    \AttributeTok{ylim=}\FunctionTok{c}\NormalTok{(}\DecValTok{0}\NormalTok{, }\FloatTok{0.5}\NormalTok{), }\AttributeTok{xlim=}\FunctionTok{c}\NormalTok{(}\DecValTok{7}\NormalTok{, }\DecValTok{13}\NormalTok{))}
\end{Highlighting}
\end{Shaded}

\includegraphics{StatsTB_files/figure-latex/unnamed-chunk-44-3.pdf}

\begin{Shaded}
\begin{Highlighting}[]
\CommentTok{\#Uniform shaped}
\FunctionTok{draw.uniform.shape}\NormalTok{(}\AttributeTok{range =} \FunctionTok{c}\NormalTok{(}\DecValTok{0}\NormalTok{, }\DecValTok{1}\NormalTok{), }\AttributeTok{main=}\StringTok{"Uniform Distribution"}\NormalTok{)}
\end{Highlighting}
\end{Shaded}

\includegraphics{StatsTB_files/figure-latex/unnamed-chunk-45-1.pdf}

\hypertarget{skewness}{%
\subsection{Skewness}\label{skewness}}

Skewed distributions: (a) right skewed (b) left skewed

\begin{Shaded}
\begin{Highlighting}[]
\FunctionTok{draw.skewed.shape}\NormalTok{(}\AttributeTok{peak.loc=}\DecValTok{10}\NormalTok{, }\AttributeTok{width=}\DecValTok{1}\NormalTok{, }\AttributeTok{skew =} \StringTok{"right"}\NormalTok{, }
                  \AttributeTok{main=}\StringTok{"Right Skewed Distribution"}\NormalTok{,}
                  \AttributeTok{xlim=}\FunctionTok{c}\NormalTok{(}\DecValTok{0}\NormalTok{, }\DecValTok{50}\NormalTok{), }\AttributeTok{ylim=}\FunctionTok{c}\NormalTok{(}\DecValTok{0}\NormalTok{, }\FloatTok{0.22}\NormalTok{))}
\end{Highlighting}
\end{Shaded}

\includegraphics{StatsTB_files/figure-latex/unnamed-chunk-46-1.pdf}

\begin{Shaded}
\begin{Highlighting}[]
\FunctionTok{draw.skewed.shape}\NormalTok{(}\AttributeTok{peak.loc=}\DecValTok{10}\NormalTok{, }\AttributeTok{width=}\DecValTok{2}\NormalTok{, }\AttributeTok{skew =} \StringTok{"right"}\NormalTok{, }
                  \AttributeTok{main=}\StringTok{"Right Skewed Distribution"}\NormalTok{,}
                  \AttributeTok{xlim=}\FunctionTok{c}\NormalTok{(}\DecValTok{0}\NormalTok{, }\DecValTok{50}\NormalTok{), }\AttributeTok{ylim=}\FunctionTok{c}\NormalTok{(}\DecValTok{0}\NormalTok{, }\FloatTok{0.22}\NormalTok{))}
\end{Highlighting}
\end{Shaded}

\includegraphics{StatsTB_files/figure-latex/unnamed-chunk-46-2.pdf}

\begin{Shaded}
\begin{Highlighting}[]
\FunctionTok{draw.skewed.shape}\NormalTok{(}\AttributeTok{peak.loc=}\DecValTok{10}\NormalTok{, }\AttributeTok{width=}\DecValTok{2}\NormalTok{, }\AttributeTok{skew =} \StringTok{"left"}\NormalTok{, }
                  \AttributeTok{main=}\StringTok{"Left Skewed Distribution"}\NormalTok{,}
                  \AttributeTok{xlim=}\FunctionTok{c}\NormalTok{(}\SpecialCharTok{{-}}\DecValTok{20}\NormalTok{, }\DecValTok{20}\NormalTok{), }\AttributeTok{ylim=}\FunctionTok{c}\NormalTok{(}\DecValTok{0}\NormalTok{, }\FloatTok{0.17}\NormalTok{))}
\end{Highlighting}
\end{Shaded}

\includegraphics{StatsTB_files/figure-latex/unnamed-chunk-46-3.pdf}

\begin{Shaded}
\begin{Highlighting}[]
\FunctionTok{draw.skewed.shape}\NormalTok{(}\AttributeTok{peak.loc=}\DecValTok{10}\NormalTok{, }\AttributeTok{width=}\DecValTok{3}\NormalTok{, }\AttributeTok{skew =} \StringTok{"left"}\NormalTok{, }
                  \AttributeTok{main=}\StringTok{"Left Skewed Distribution"}\NormalTok{,}
                  \AttributeTok{xlim=}\FunctionTok{c}\NormalTok{(}\SpecialCharTok{{-}}\DecValTok{20}\NormalTok{, }\DecValTok{20}\NormalTok{), }\AttributeTok{ylim=}\FunctionTok{c}\NormalTok{(}\DecValTok{0}\NormalTok{, }\FloatTok{0.17}\NormalTok{))}
\end{Highlighting}
\end{Shaded}

\includegraphics{StatsTB_files/figure-latex/unnamed-chunk-46-4.pdf}

\begin{Shaded}
\begin{Highlighting}[]
\FunctionTok{draw.skewed.shape}\NormalTok{(}\AttributeTok{peak.loc=}\DecValTok{10}\NormalTok{, }\AttributeTok{width=}\DecValTok{2}\NormalTok{, }\AttributeTok{skew =} \StringTok{"no"}\NormalTok{, }
                  \AttributeTok{main=}\StringTok{"Not Skewed (symmetric) Distribution"}\NormalTok{,}
                  \AttributeTok{xlim=}\FunctionTok{c}\NormalTok{(}\DecValTok{0}\NormalTok{, }\DecValTok{20}\NormalTok{), }\AttributeTok{ylim=}\FunctionTok{c}\NormalTok{(}\DecValTok{0}\NormalTok{, }\FloatTok{0.22}\NormalTok{))}
\end{Highlighting}
\end{Shaded}

\includegraphics{StatsTB_files/figure-latex/unnamed-chunk-46-5.pdf}

\begin{Shaded}
\begin{Highlighting}[]
\FunctionTok{draw.skewed.shape}\NormalTok{(}\AttributeTok{peak.loc=}\DecValTok{10}\NormalTok{, }\AttributeTok{width=}\DecValTok{3}\NormalTok{, }\AttributeTok{skew =} \StringTok{"no"}\NormalTok{, }
                  \AttributeTok{main=}\StringTok{"Not Skew (symmetric) Distribution"}\NormalTok{,}
                  \AttributeTok{xlim=}\FunctionTok{c}\NormalTok{(}\DecValTok{0}\NormalTok{, }\DecValTok{20}\NormalTok{), }\AttributeTok{ylim=}\FunctionTok{c}\NormalTok{(}\DecValTok{0}\NormalTok{, }\FloatTok{0.22}\NormalTok{))}
\end{Highlighting}
\end{Shaded}

\includegraphics{StatsTB_files/figure-latex/unnamed-chunk-46-6.pdf}

Another type of skewed distribution is reverse-J shaped.

\begin{Shaded}
\begin{Highlighting}[]
\FunctionTok{draw.reverse.Jshape}\NormalTok{ (}\AttributeTok{median.loc=}\DecValTok{10}\NormalTok{)}
\end{Highlighting}
\end{Shaded}

\includegraphics{StatsTB_files/figure-latex/unnamed-chunk-47-1.pdf}

\hypertarget{population-distribution-and-sample-distribution}{%
\chapter{Population distribution and sample distribution}\label{population-distribution-and-sample-distribution}}

\begin{Shaded}
\begin{Highlighting}[]
\FunctionTok{library}\NormalTok{(IntroStats)}
\end{Highlighting}
\end{Shaded}

\hypertarget{definition-2}{%
\section{Definition}\label{definition-2}}

A \textbf{population} is the entire set of individuals or items of interest, while a \textbf{sample} is a subset selected from the population. A \textbf{distribution} describes the values of a variable and how often they occur. The population distribution refers to the true distribution of the variable in the population; the sample distribution describes the values observed in a particular sample.

\hypertarget{example-9}{%
\section{Example}\label{example-9}}

Define a population of all 10,000 students' heights in a school, and take a sample to illustrate the concept. Suppose the height of this population has a mean of 170 cm and sd of 6 cm, and the following data represent the heights of all 10,000 students.

\begin{Shaded}
\begin{Highlighting}[]
\FunctionTok{set.seed}\NormalTok{(}\DecValTok{1}\NormalTok{)}
\NormalTok{height\_population }\OtherTok{\textless{}{-}} \FunctionTok{rnorm}\NormalTok{(}\DecValTok{10000}\NormalTok{, }\AttributeTok{mean=}\DecValTok{170}\NormalTok{, }\AttributeTok{sd=}\DecValTok{6}\NormalTok{)  }\CommentTok{\# cm}
\FunctionTok{hist}\NormalTok{(height\_population)}
\end{Highlighting}
\end{Shaded}

\includegraphics{StatsTB_files/figure-latex/unnamed-chunk-49-1.pdf}
Next, draw a sample of 30 students from the population. Run the following chunk multiple times. Each run represents a sample of 30 students. The sample statistics approximate the population statistics but are not identical.

\begin{Shaded}
\begin{Highlighting}[]
\NormalTok{height\_sample }\OtherTok{\textless{}{-}} \FunctionTok{sample}\NormalTok{(height\_population, }\DecValTok{30}\NormalTok{)}

\FunctionTok{mean}\NormalTok{(height\_sample); }
\end{Highlighting}
\end{Shaded}

\begin{verbatim}
## [1] 170.1488
\end{verbatim}

\begin{Shaded}
\begin{Highlighting}[]
\FunctionTok{sd}\NormalTok{(height\_sample)}
\end{Highlighting}
\end{Shaded}

\begin{verbatim}
## [1] 6.584451
\end{verbatim}

\begin{itemize}
\tightlist
\item
  \textbf{Population Data}: The values of a variable for the entire population
\item
  \textbf{Sample data}: The values of a variable for a sample of the population.
\end{itemize}

\hypertarget{relationship-between-a-population-distribution-and-sample-distribution}{%
\section{Relationship between a population distribution and sample distribution}\label{relationship-between-a-population-distribution-and-sample-distribution}}

The \textbf{population distribution} is fixed (though often unknown). A \textbf{sampling distribution} is the probability distribution of a statistic (like the mean) computed from samples of a given size. Repeated sampling from the same population will produce different samples, and thus different statistics. The sampling distribution reflects this variability.

\begin{itemize}
\tightlist
\item
  The distribution of population data is called the \textbf{population distribution}, or the distribution of the variable.
\item
  The distribution of sample data is called a \textbf{sample distribution}.
\end{itemize}

Consider the population is defined as all U.S. households. Then the size of households is distributed as follows according to U.S. census:

\begin{Shaded}
\begin{Highlighting}[]
\NormalTok{p }\OtherTok{=} \FunctionTok{c}\NormalTok{(}\FloatTok{0.27}\NormalTok{, }\FloatTok{0.34}\NormalTok{, }\FloatTok{0.15}\NormalTok{, }\FloatTok{0.14}\NormalTok{, }\FloatTok{0.06}\NormalTok{, }\FloatTok{0.03}\NormalTok{, }\FloatTok{0.01}\NormalTok{)}

\FunctionTok{barplot}\NormalTok{(p, }\AttributeTok{col=}\StringTok{"turquoise"}\NormalTok{, }\AttributeTok{space=}\DecValTok{0}\NormalTok{, }
        \AttributeTok{names.arg =} \FunctionTok{c}\NormalTok{(}\StringTok{"1"}\NormalTok{, }\StringTok{"2"}\NormalTok{, }\StringTok{"3"}\NormalTok{, }\StringTok{"4"}\NormalTok{, }\StringTok{"5"}\NormalTok{,}\StringTok{"6"}\NormalTok{,}\StringTok{"7+"}\NormalTok{), }
        \AttributeTok{ylim=}\FunctionTok{c}\NormalTok{(}\DecValTok{0}\NormalTok{, }\FloatTok{0.4}\NormalTok{), }
        \AttributeTok{main=}\StringTok{"Population Distribution (All US Households)"}\NormalTok{,}
        \AttributeTok{xlab =} \StringTok{"Household Size"}\NormalTok{)}
\end{Highlighting}
\end{Shaded}

\includegraphics{StatsTB_files/figure-latex/unnamed-chunk-51-1.pdf}

Now, suppose we randomly draw 6 samples of size 100 U.S. households. The distributions of the 6 samples are as follows.

\begin{Shaded}
\begin{Highlighting}[]
\CommentTok{\#draw 6 samples of size 100}
\FunctionTok{draw.US.households}\NormalTok{(}\AttributeTok{size=}\DecValTok{100}\NormalTok{, }\AttributeTok{main=}\StringTok{"Sample 1 Distribution (100 US Households)"}\NormalTok{)}
\end{Highlighting}
\end{Shaded}

\includegraphics{StatsTB_files/figure-latex/unnamed-chunk-52-1.pdf}

\begin{Shaded}
\begin{Highlighting}[]
\FunctionTok{draw.US.households}\NormalTok{(}\AttributeTok{size=}\DecValTok{100}\NormalTok{, }\AttributeTok{main=}\StringTok{"Sample 2 Distribution (100 US Households)"}\NormalTok{)}
\end{Highlighting}
\end{Shaded}

\includegraphics{StatsTB_files/figure-latex/unnamed-chunk-52-2.pdf}

\begin{Shaded}
\begin{Highlighting}[]
\FunctionTok{draw.US.households}\NormalTok{(}\AttributeTok{size=}\DecValTok{100}\NormalTok{, }\AttributeTok{main=}\StringTok{"Sample 3 Distribution (100 US Households)"}\NormalTok{)}
\end{Highlighting}
\end{Shaded}

\includegraphics{StatsTB_files/figure-latex/unnamed-chunk-52-3.pdf}

\begin{Shaded}
\begin{Highlighting}[]
\FunctionTok{draw.US.households}\NormalTok{(}\AttributeTok{size=}\DecValTok{100}\NormalTok{, }\AttributeTok{main=}\StringTok{"Sample 4 Distribution (100 US Households)"}\NormalTok{)}
\end{Highlighting}
\end{Shaded}

\includegraphics{StatsTB_files/figure-latex/unnamed-chunk-52-4.pdf}

\begin{Shaded}
\begin{Highlighting}[]
\FunctionTok{draw.US.households}\NormalTok{(}\AttributeTok{size=}\DecValTok{100}\NormalTok{, }\AttributeTok{main=}\StringTok{"Sample 5 Distribution (100 US Households)"}\NormalTok{)}
\end{Highlighting}
\end{Shaded}

\includegraphics{StatsTB_files/figure-latex/unnamed-chunk-52-5.pdf}

\begin{Shaded}
\begin{Highlighting}[]
\FunctionTok{draw.US.households}\NormalTok{(}\AttributeTok{size=}\DecValTok{100}\NormalTok{, }\AttributeTok{main=}\StringTok{"Sample 6 Distribution (100 US Households)"}\NormalTok{)}
\end{Highlighting}
\end{Shaded}

\includegraphics{StatsTB_files/figure-latex/unnamed-chunk-52-6.pdf}
When draw a large number of sample size, the sample distribution is approximately the population distribution. The larger the sample size, the better the approximation tends to be.

\begin{Shaded}
\begin{Highlighting}[]
\FunctionTok{draw.US.households}\NormalTok{(}\AttributeTok{size=}\DecValTok{1000}\NormalTok{, }\AttributeTok{main=}\StringTok{"Sample 7 Distribution (1000 US Households)"}\NormalTok{)}
\end{Highlighting}
\end{Shaded}

\includegraphics{StatsTB_files/figure-latex/unnamed-chunk-53-1.pdf}

\begin{Shaded}
\begin{Highlighting}[]
\FunctionTok{draw.US.households}\NormalTok{(}\AttributeTok{size=}\DecValTok{1000}\NormalTok{, }\AttributeTok{main=}\StringTok{"Sample 8 Distribution (1000 US Households)"}\NormalTok{)}
\end{Highlighting}
\end{Shaded}

\includegraphics{StatsTB_files/figure-latex/unnamed-chunk-53-2.pdf}

\begin{Shaded}
\begin{Highlighting}[]
\FunctionTok{draw.US.households}\NormalTok{(}\AttributeTok{size=}\DecValTok{100000}\NormalTok{, }\AttributeTok{main=}\StringTok{"Sample 9 Distribution (100,000 US Households)"}\NormalTok{)}
\end{Highlighting}
\end{Shaded}

\includegraphics{StatsTB_files/figure-latex/unnamed-chunk-53-3.pdf}

\begin{Shaded}
\begin{Highlighting}[]
\FunctionTok{draw.US.households}\NormalTok{(}\AttributeTok{size=}\DecValTok{100000}\NormalTok{, }\AttributeTok{main=}\StringTok{"Sample 10 Distribution (100,000 US Households)"}\NormalTok{)}
\end{Highlighting}
\end{Shaded}

\includegraphics{StatsTB_files/figure-latex/unnamed-chunk-53-4.pdf}

\hypertarget{descriptive-statistics}{%
\chapter{Descriptive Statistics}\label{descriptive-statistics}}

\begin{Shaded}
\begin{Highlighting}[]
\FunctionTok{library}\NormalTok{(IntroStats)}
\end{Highlighting}
\end{Shaded}

\textbf{Descriptive Statistics} consists of methods for organizing and summarizing information. Such as graphs, charts, tables and various descriptive measures: average, variation, percentiles.

\hypertarget{measures-of-central-tendency}{%
\section{Measures of Central Tendency}\label{measures-of-central-tendency}}

Descriptive statistics help us summarize and describe the main features of a data set. One of the most important types is the \textbf{measure of central tendency}, which describes where the center of the data lies. The most commonly used measures include:

\begin{itemize}
\tightlist
\item
  \textbf{Mean}: The arithmetic average of all values. It is sensitive to extreme values (outliers).
\item
  \textbf{Median}: The middle value when the data are sorted in ascending order. It is more robust to outliers.
\item
  \textbf{Mode}: The value that appears most frequently in a data set. It is useful for categorical data.
\end{itemize}

\hypertarget{population-mean-mu-and-sample-mean-barx}{%
\subsection{\texorpdfstring{Population mean \(\mu\) and sample mean \(\bar{x}\)}{Population mean \textbackslash mu and sample mean \textbackslash bar\{x\}}}\label{population-mean-mu-and-sample-mean-barx}}

The population mean is usually denoted as \(\mu\), which is estimated by a sample mean.

For a random variable \(X\), its mean is also called its expectation
\[\mu = E[X] = \int_{-\infty}^{\infty} xf(x)dx\],
where \(f(x)\) is the density function of \(X\). For any density function,

\[
\int_{-\infty}^\infty f(x) = 1
\]

In general, the expectation of \(g(X)\) for any function form \(g(\cdot)\) is \[E[g(X)] = \int_{-\infty}^{\infty} g(x)f(x)dx\]. For example, the second moment is the expectation of \(X^2\), i.e., \[E[X^2] = \int_{-\infty}^{\infty} x^2f(x)dx\].

The sample mean is denoted as \(\bar{x}\), which is calculated based on a sample data as
\[\bar{x} = \frac{\sum_{i=1}^n x_i}{n}\].

\hypertarget{example-10}{%
\subsubsection{Example}\label{example-10}}

Consider a population with a mean of 12 and sd of 2. Draw a sample of size 10 from the population below.

\begin{Shaded}
\begin{Highlighting}[]
      \DocumentationTok{\#\#\#\#\#\#\#\#\#\#\#\#\#\#\#\#\#\#\#\#\#\#\#\#}
        \CommentTok{\#Sample mean}
        \DocumentationTok{\#\#\#\#\#\#\#\#\#\#\#\#\#\#\#\#\#\#\#\#\#\#\#\#}
        \FunctionTok{set.seed}\NormalTok{(}\DecValTok{2023}\NormalTok{) }\CommentTok{\#set a fixed seed for random number generator}

\NormalTok{        x }\OtherTok{\textless{}{-}} \FunctionTok{rnorm}\NormalTok{(}\DecValTok{10}\NormalTok{, }\AttributeTok{mean=}\DecValTok{12}\NormalTok{, }\AttributeTok{sd=}\DecValTok{2}\NormalTok{) }\CommentTok{\#draw 10 random numbers from normal distribution}
\NormalTok{        x}
\end{Highlighting}
\end{Shaded}

\begin{verbatim}
##  [1] 11.832431 10.034113  8.249865 11.627711 10.733029 14.181595 10.172545
##  [8] 14.003279 11.201467 11.063754
\end{verbatim}

Then, the sample mean 11.30998 is the average of the 10 data points.

\begin{Shaded}
\begin{Highlighting}[]
\FunctionTok{mean}\NormalTok{(x) }\CommentTok{\#sample mean of x}
\end{Highlighting}
\end{Shaded}

\begin{verbatim}
## [1] 11.30998
\end{verbatim}

\hypertarget{population-mean-for-a-finite-population}{%
\subsection{Population Mean for a finite population}\label{population-mean-for-a-finite-population}}

For a variable \(x\), the mean of all possible observations for the entire population is called the population mean, denoted as \(\mu_x\) or simply \(\mu\) if no confusion.

If a population is finite, then the population mean is
\[
\mu = \frac{\sum x_i}{N}
\]
where \(N\) is the population size.

\textbf{Note:} For a population, there is ONLY 1 population mean. However, when we draw multiple samples from the population, there are multiple \textbf{sample means}. Recall, a sample mean is \[\bar{x} = \frac{\sum x_i}{n}\]

\hypertarget{example.}{%
\subsubsection{Example.}\label{example.}}

One golf course has 18 holes and the length of each hole (yd) is listed below.

\includegraphics{https://i.ibb.co/6Yj8g3r/141-Table-03-17.png}
What is the population mean for the length of holes?

\textbf{Solution.} The population is finite, so the population mean is calculated as
\[
\mu =  \frac{\sum x_i}{N} = \frac{445+575+\cdots+465}{18}=413.1
\]

\begin{Shaded}
\begin{Highlighting}[]
\NormalTok{golf }\OtherTok{=} \FunctionTok{c}\NormalTok{(}\DecValTok{445}\NormalTok{, }\DecValTok{575}\NormalTok{, }\DecValTok{350}\NormalTok{, }\DecValTok{240}\NormalTok{, }\DecValTok{455}\NormalTok{, }\DecValTok{180}\NormalTok{, }\DecValTok{450}\NormalTok{, }\DecValTok{570}\NormalTok{, }\DecValTok{460}\NormalTok{, }\DecValTok{495}\NormalTok{, }\DecValTok{505}\NormalTok{, }\DecValTok{155}\NormalTok{, }\DecValTok{510}\NormalTok{, }\DecValTok{440}\NormalTok{, }\DecValTok{530}\NormalTok{, }\DecValTok{170}\NormalTok{, }\DecValTok{440}\NormalTok{, }\DecValTok{465}\NormalTok{)}

\CommentTok{\#calcuate population mean}

\CommentTok{\#Population size}
\NormalTok{N }\OtherTok{\textless{}{-}} \FunctionTok{length}\NormalTok{(golf)}
\NormalTok{N}
\end{Highlighting}
\end{Shaded}

\begin{verbatim}
## [1] 18
\end{verbatim}

\begin{Shaded}
\begin{Highlighting}[]
\FunctionTok{sum}\NormalTok{(golf)}\SpecialCharTok{/}\NormalTok{N}
\end{Highlighting}
\end{Shaded}

\begin{verbatim}
## [1] 413.0556
\end{verbatim}

\begin{Shaded}
\begin{Highlighting}[]
\CommentTok{\#Use R function}
\FunctionTok{mean}\NormalTok{(golf)}
\end{Highlighting}
\end{Shaded}

\begin{verbatim}
## [1] 413.0556
\end{verbatim}

\hypertarget{example.-use-of-a-sample-mean}{%
\subsubsection{Example. Use of a sample mean}\label{example.-use-of-a-sample-mean}}

U.S. census bureau reports the mean income of U.S. households. Only 57,000 households were sampled from a total of more than 100 million.

-\textbf{Variable:} income
-\textbf{Population:} All U.S. households
-\textbf{Population data:} incomes of all U.S. households
-\textbf{Population mean:} mean income \(\mu\) of all U.S. households
-\textbf{Sample:} 57,000 U.S. households sampled by the census bureau
-\textbf{Sample data:} incomes of the 57,000 U.S. households sampled
-\textbf{Sample mean:} mean income \(\bar{x}\) of the 57,000 U.S. households sampled

The figure below illustrates the population vs sample.

\begin{figure}
\centering
\includegraphics{https://i.ibb.co/9TtXymk/141-Figure-03-17.png}
\caption{Population and sample}
\end{figure}

\hypertarget{population-median-and-sample-median}{%
\subsection{Population median and sample median}\label{population-median-and-sample-median}}

Median can be referring to a population or a sample. If \(m\) is the population median for a variable \(X\), this means that \(Pr(X \le m) \ge 0.5\) and \(Pr(X \ge m) \ge 0.5\). If \(X\) is a continuous variable with a density function \(f(x)\), then
\[ \int_{-\infty}^{m} f(x)dx =  \int_m^{\infty} f(x)dx = \frac{1}{2}\]

For sample median, first order the sample increasingly, and denoted as \(x_{(1)}<x_{(2)<\cdots<x_{(n)}}\), then the sample median is \(m = x_{(n+1)/2}\) if \(n\) is odd; and \(m = (x_{(n/2)}+x_{(n/2+1)})/2\).

Compared to the mean, median is not drastically affected by extreme values. It is commonly used in medical research.

\begin{Shaded}
\begin{Highlighting}[]
      \DocumentationTok{\#\#\#\#\#\#\#\#\#\#\#\#\#\#\#\#\#\#\#\#\#\#\#\#}
        \CommentTok{\#Sample median}
        \DocumentationTok{\#\#\#\#\#\#\#\#\#\#\#\#\#\#\#\#\#\#\#\#\#\#\#\#}
        
        \FunctionTok{median}\NormalTok{(x) }\CommentTok{\#sample median of x}
\end{Highlighting}
\end{Shaded}

\begin{verbatim}
## [1] 11.13261
\end{verbatim}

\hypertarget{example-11}{%
\subsubsection{Example}\label{example-11}}

For the above example, the population median is 12, and the sample median is 11.13261.

\hypertarget{population-mode-and-sample-mode}{%
\subsection{Population mode and sample mode}\label{population-mode-and-sample-mode}}

The mode can be referring to a population distribution or a sample. For population, the mode is the point with the maximum density, i.e., \(mode = \arg_{x} \max(f(x))\). For a sample, mode is the most frequent value. In the class age example above, the mode age is 23 because its frequency is the largest compared to any other age values.

\begin{Shaded}
\begin{Highlighting}[]
      \DocumentationTok{\#\#\#\#\#\#\#\#\#\#\#\#\#\#\#\#\#\#\#\#\#\#\#\#}
      \DocumentationTok{\#\#Sample mode}
        \DocumentationTok{\#\#\#\#\#\#\#\#\#\#\#\#\#\#\#\#\#\#\#\#\#\#\#\#}
        
\NormalTok{        IntroStats}\SpecialCharTok{::}\FunctionTok{getmode}\NormalTok{(x) }\CommentTok{\#sample mode of x}
\end{Highlighting}
\end{Shaded}

\begin{verbatim}
## NULL
\end{verbatim}

\hypertarget{example-12}{%
\subsubsection{Example}\label{example-12}}

For the above example, the population mode is 12, and there is no sample mode, because there is no data point occurred more than once in the sample.

\hypertarget{sample-mean-sample-median-and-sample-mode}{%
\subsection{Sample mean, sample median, and sample mode}\label{sample-mean-sample-median-and-sample-mode}}

In reality, the population mean, median, and mode are commonly not known. We draw a random sample, then calculate the sample mean, median and mode to estimate the population mean, population median, and population mode. The following examples show the calculation of sample mean, sample median and sample mode.

\hypertarget{relative-positions-of-the-mean-and-median}{%
\subsection{Relative positions of the mean and median}\label{relative-positions-of-the-mean-and-median}}

Relative positions of the mean and median for (a) right-skewed, (b) symmetric, and (c) left-skewed distributions.

\begin{figure}
\centering
\includegraphics{https://i.ibb.co/xLv8vrh/mean-median.png}
\caption{Figure 1. Relative positions of the mean and median}
\end{figure}

For a right-skewed distribution, the mean is toward the right because it is driven by the large numbers on the right. So mean is greater than median. Conversely, mean is less than median if the distribution is left skewed.

\hypertarget{example.-which-measure-is-appropriate}{%
\subsubsection{Example. Which measure is appropriate?}\label{example.-which-measure-is-appropriate}}

Example (a). A student takes 4 exams in a biology class and scored: 88, 75, 95, 100. Which measure of center is appropriate?

Example (b). National Association of REALTORS publishes data on resale prices of U.S. homes. Which measure of center is appropriate?

Example (c). The 2014 Boston Marathon had 2 categories of finishers: male and female. There are 19,579 and 16,092 respectively. Which measure of center should be used?

\hypertarget{three-sd-rule}{%
\subsection{Three-sd Rule}\label{three-sd-rule}}

\begin{figure}
\centering
\includegraphics{https://i.ibb.co/LhNpp65/107-Figure-03-04.png}
\caption{Sample standard deviation}
\end{figure}

Almost all the observations in any data set lie within three standard deviations to either side of the mean.

\hypertarget{example-13}{%
\subsubsection{Example}\label{example-13}}

\begin{figure}
\centering
\includegraphics{https://i.ibb.co/K5zfb6S/111-Table-03-10.png}
\caption{Example}
\end{figure}

Means and standard deviations

\includegraphics{https://i.ibb.co/b1wXBXN/111-Table-03-11.png}
\includegraphics{https://i.ibb.co/pbwyZsH/111-Figure-03-05.png}

\begin{figure}
\centering
\includegraphics{https://i.ibb.co/N9qjcPx/111-Figure-03-06.png}
\caption{Example. Data Set II.}
\end{figure}

\hypertarget{chebyshevs-rule}{%
\subsection{Chebyshev's Rule}\label{chebyshevs-rule}}

For any quantitative data set and any real number \(k\) greater than or equal to 1, at least
\(1 - 1/k^2\) of the observations lie within \(k\) standard deviations to
either side of the mean, that is, between \(\bar{x} -ks\) and \(\bar{x} + ks\).

\begin{itemize}
\tightlist
\item
  \textbf{k=2}: At least 75\% observations lie within 2 standard deviations.
\item
  \textbf{k=3}: At least 89\% observations lie within 3 standard deviations.
\end{itemize}

\begin{figure}
\centering
\includegraphics{https://i.ibb.co/Zx5P2rM/117-Figure-03-07.png}
\caption{Figure 8. Chebyshev's Rule with \(k=2, 3\)}
\end{figure}

\hypertarget{empirical-rule}{%
\subsection{Empirical Rule}\label{empirical-rule}}

For any quantitative data set with roughly a bell-shaped distribution, the following properties hold.

\begin{itemize}
\tightlist
\item
  \textbf{Property 1}: Approximately 68\% of the observations lie within one standard
  deviation to either side of the mean.
\item
  \textbf{Property 2}: Approximately 95\% of the observations lie within two standard
  deviations to either side of the mean.
\item
  \textbf{Property 3}: Approximately 99.7\% of the observations lie within three standard
  deviations to either side of the mean.
\end{itemize}

\begin{figure}
\centering
\includegraphics{https://i.ibb.co/hMWVFjg/119-Figure-03-09.png}
\caption{Figure 9. Empirical Rule with \(k=2, 3\)}
\end{figure}

\hypertarget{example-patients-heart-rate}{%
\subsubsection{Example: Patients' Heart Rate}\label{example-patients-heart-rate}}

We are studying the heart rates of 10 patients. We want to understand what a ``typical'' heart rate is.

\begin{Shaded}
\begin{Highlighting}[]
\NormalTok{heart\_rate }\OtherTok{\textless{}{-}} \FunctionTok{c}\NormalTok{(}\DecValTok{72}\NormalTok{, }\DecValTok{80}\NormalTok{, }\DecValTok{75}\NormalTok{, }\DecValTok{70}\NormalTok{, }\DecValTok{68}\NormalTok{, }\DecValTok{90}\NormalTok{, }\DecValTok{88}\NormalTok{, }\DecValTok{76}\NormalTok{, }\DecValTok{74}\NormalTok{, }\DecValTok{70}\NormalTok{)}
\FunctionTok{mean}\NormalTok{(heart\_rate)   }\CommentTok{\# average heart rate}
\end{Highlighting}
\end{Shaded}

\begin{verbatim}
## [1] 76.3
\end{verbatim}

\begin{Shaded}
\begin{Highlighting}[]
\FunctionTok{median}\NormalTok{(heart\_rate) }\CommentTok{\# median heart rate}
\end{Highlighting}
\end{Shaded}

\begin{verbatim}
## [1] 74.5
\end{verbatim}

\begin{Shaded}
\begin{Highlighting}[]
\NormalTok{IntroStats}\SpecialCharTok{::}\FunctionTok{getmode}\NormalTok{(heart\_rate)}
\end{Highlighting}
\end{Shaded}

\begin{verbatim}
## [1] 70
\end{verbatim}

\begin{quote}
Interpretation: The mean and median are close, which suggests a potential symmetric distribution. If there were outliers (e.g., one patient had a heart rate of 200), the median would be more reliable.
\end{quote}

\hypertarget{example-length-of-hospital-stay-days}{%
\subsubsection{Example: Length of Hospital Stay (days)}\label{example-length-of-hospital-stay-days}}

\begin{Shaded}
\begin{Highlighting}[]
\NormalTok{stay\_days }\OtherTok{\textless{}{-}} \FunctionTok{c}\NormalTok{(}\DecValTok{3}\NormalTok{, }\DecValTok{5}\NormalTok{, }\DecValTok{4}\NormalTok{, }\DecValTok{6}\NormalTok{, }\DecValTok{3}\NormalTok{, }\DecValTok{2}\NormalTok{, }\DecValTok{8}\NormalTok{, }\DecValTok{4}\NormalTok{, }\DecValTok{4}\NormalTok{, }\DecValTok{3}\NormalTok{)}
\FunctionTok{mean}\NormalTok{(stay\_days)}
\end{Highlighting}
\end{Shaded}

\begin{verbatim}
## [1] 4.2
\end{verbatim}

\begin{Shaded}
\begin{Highlighting}[]
\FunctionTok{median}\NormalTok{(stay\_days)}
\end{Highlighting}
\end{Shaded}

\begin{verbatim}
## [1] 4
\end{verbatim}

\begin{Shaded}
\begin{Highlighting}[]
\NormalTok{IntroStats}\SpecialCharTok{::}\FunctionTok{getmode}\NormalTok{(stay\_days)}
\end{Highlighting}
\end{Shaded}

\begin{verbatim}
## [1] 3 4
\end{verbatim}

\begin{quote}
These measures help us understand typical patient recovery times.
\end{quote}

\hypertarget{measures-of-spread}{%
\section{Measures of Spread}\label{measures-of-spread}}

While central tendency tells us where the data center is, \textbf{measures of spread} describe how dispersed or variable the data are. These help researchers assess how consistent or variable their data are.

Key measures include:
- \textbf{Range}: The difference between the maximum and minimum values.
- \textbf{Variance}: The average of the squared deviations from the mean. It is in squared units.
- \textbf{Standard Deviation (SD)}: The square root of the variance. It is in the same units as the data and is more interpretable.

\hypertarget{range}{%
\subsection{Range}\label{range}}

Range is the distance between the largest and smallest value in a sample. \(R = x_{max} - x_{min}\). Range is very easy to calculate but it is a poor measure of dispersion. It fails to offer any insights into the data points situated between the minimum and maximum values.

\begin{Shaded}
\begin{Highlighting}[]
      \DocumentationTok{\#\#\#\#\#\#\#\#\#\#\#\#\#\#\#\#\#\#\#\#\#\#\#\#}
        \CommentTok{\#Range}
        \DocumentationTok{\#\#\#\#\#\#\#\#\#\#\#\#\#\#\#\#\#\#\#\#\#\#\#\#}
\NormalTok{        y }\OtherTok{=} \FunctionTok{range}\NormalTok{(x) }
\NormalTok{      y}
\end{Highlighting}
\end{Shaded}

\begin{verbatim}
## [1]  8.249865 14.181595
\end{verbatim}

\begin{Shaded}
\begin{Highlighting}[]
\NormalTok{      y[}\DecValTok{2}\NormalTok{] }\SpecialCharTok{{-}}\NormalTok{ y[}\DecValTok{1}\NormalTok{]}
\end{Highlighting}
\end{Shaded}

\begin{verbatim}
## [1] 5.93173
\end{verbatim}

\hypertarget{variance-and-standard-deviation}{%
\subsection{Variance and Standard Deviation}\label{variance-and-standard-deviation}}

Sample variance (\(s^2\)) is a measure of average dispersion from the sample mean. The notation has \(^2\) because it is sample standard deviation (\(s\)) squared. Sample standard deviation (\(s\)) has the same unit as the original data.
\[s^2 = \frac{1}{n-1}\sum_{i=1}^n (x_i-\bar{x})^2\]

\[
s = \sqrt{\frac{1}{n-1} \sum (x_i - \bar{x})^2}.
\]

For a known population distribution, the variance is denoted as \(\sigma^2\), calculated as
\[
    \sigma^2 = E[(X-\mu)^2] = \int_{-\infty}^{\infty} (x-\mu)^2 f(x) dx
    \]
where \(f(x)\) is the density function. You are not required to perform the calculation in this course.

\begin{Shaded}
\begin{Highlighting}[]
    \DocumentationTok{\#\#\#\#\#\#\#\#\#\#\#\#\#\#\#\#\#\#\#\#\#\#\#\#}
        \CommentTok{\#Sample variance and sd}
        \DocumentationTok{\#\#\#\#\#\#\#\#\#\#\#\#\#\#\#\#\#\#\#\#\#\#\#\#}
    \FunctionTok{var}\NormalTok{(x) }\CommentTok{\#variance}
\end{Highlighting}
\end{Shaded}

\begin{verbatim}
## [1] 3.173902
\end{verbatim}

\begin{Shaded}
\begin{Highlighting}[]
    \FunctionTok{sd}\NormalTok{(x)  }\CommentTok{\#sd}
\end{Highlighting}
\end{Shaded}

\begin{verbatim}
## [1] 1.781545
\end{verbatim}

\begin{Shaded}
\begin{Highlighting}[]
    \FunctionTok{sd}\NormalTok{(x)}\SpecialCharTok{\^{}}\DecValTok{2} \CommentTok{\#sd\^{}2 = var}
\end{Highlighting}
\end{Shaded}

\begin{verbatim}
## [1] 3.173902
\end{verbatim}

\hypertarget{example-systolic-blood-pressure}{%
\subsection{Example: Systolic Blood Pressure}\label{example-systolic-blood-pressure}}

\begin{Shaded}
\begin{Highlighting}[]
\NormalTok{sbp }\OtherTok{\textless{}{-}} \FunctionTok{c}\NormalTok{(}\DecValTok{120}\NormalTok{, }\DecValTok{124}\NormalTok{, }\DecValTok{118}\NormalTok{, }\DecValTok{130}\NormalTok{, }\DecValTok{110}\NormalTok{, }\DecValTok{128}\NormalTok{, }\DecValTok{126}\NormalTok{, }\DecValTok{115}\NormalTok{, }\DecValTok{132}\NormalTok{, }\DecValTok{125}\NormalTok{)}
\NormalTok{range\_sbp }\OtherTok{\textless{}{-}} \FunctionTok{max}\NormalTok{(sbp) }\SpecialCharTok{{-}} \FunctionTok{min}\NormalTok{(sbp)}
\NormalTok{var\_sbp }\OtherTok{\textless{}{-}} \FunctionTok{var}\NormalTok{(sbp)}
\NormalTok{sd\_sbp }\OtherTok{\textless{}{-}} \FunctionTok{sd}\NormalTok{(sbp)}
\NormalTok{range\_sbp}
\end{Highlighting}
\end{Shaded}

\begin{verbatim}
## [1] 22
\end{verbatim}

\begin{Shaded}
\begin{Highlighting}[]
\NormalTok{var\_sbp}
\end{Highlighting}
\end{Shaded}

\begin{verbatim}
## [1] 48.4
\end{verbatim}

\begin{Shaded}
\begin{Highlighting}[]
\NormalTok{sd\_sbp}
\end{Highlighting}
\end{Shaded}

\begin{verbatim}
## [1] 6.957011
\end{verbatim}

\begin{quote}
Interpretation: The sd tells us how much the blood pressures vary from the mean. The interval of mean \(\pm 2*sd\) includes about 95\% data.
\end{quote}

\hypertarget{example-serum-glucose-levels-mgdl}{%
\subsection{Example: Serum Glucose Levels (mg/dL)}\label{example-serum-glucose-levels-mgdl}}

\begin{Shaded}
\begin{Highlighting}[]
\NormalTok{glucose }\OtherTok{\textless{}{-}} \FunctionTok{c}\NormalTok{(}\DecValTok{90}\NormalTok{, }\DecValTok{85}\NormalTok{, }\DecValTok{88}\NormalTok{, }\DecValTok{110}\NormalTok{, }\DecValTok{95}\NormalTok{, }\DecValTok{102}\NormalTok{, }\DecValTok{87}\NormalTok{, }\DecValTok{89}\NormalTok{, }\DecValTok{100}\NormalTok{, }\DecValTok{93}\NormalTok{)}
\FunctionTok{sd}\NormalTok{(glucose)}
\end{Highlighting}
\end{Shaded}

\begin{verbatim}
## [1] 7.922542
\end{verbatim}

\begin{quote}
This shows the variation in blood glucose among patients.
\end{quote}

\hypertarget{population-standard-deviation-for-a-finite-population}{%
\subsection{Population standard deviation for a finite population}\label{population-standard-deviation-for-a-finite-population}}

Recall, the sample standard deviation \(s\) is
\[
s = \sqrt{\frac{\sum (x_i-\bar{x})^2}{n-1}}
\]
For a variable \(x\), the standard deviation of all possible observations for the entire population is called the \textbf{population standard deviation} or standard deviation of the variable \(x\). It is denoted as \(\sigma_x\), or when no confusion will arise, simply \(\sigma\).
For a finite population, the defining formula is
\[
\sigma = \sqrt{\frac{\sum (x_i-\mu)^2}{N}}
\]
where \(N\) is the population size. The \textbf{population variance} is
\[
\sigma^2 = \frac{\sum (x_i-\mu)^2}{N}
\]

\hypertarget{example.-1}{%
\subsubsection{Example.}\label{example.-1}}

For the golf problem, calculate the population variance \(\sigma^2\) and population standard deviation \(\sigma\) for the golf example.

\begin{Shaded}
\begin{Highlighting}[]
\NormalTok{IntroStats}\SpecialCharTok{::}\FunctionTok{pop.var}\NormalTok{(golf)}
\end{Highlighting}
\end{Shaded}

\begin{verbatim}
## [1] 17417.05
\end{verbatim}

\begin{Shaded}
\begin{Highlighting}[]
\NormalTok{IntroStats}\SpecialCharTok{::}\FunctionTok{pop.sd}\NormalTok{(golf)}
\end{Highlighting}
\end{Shaded}

\begin{verbatim}
## [1] 131.9737
\end{verbatim}

Compare to sample variance and sample sd.

\begin{Shaded}
\begin{Highlighting}[]
\FunctionTok{var}\NormalTok{(golf)}
\end{Highlighting}
\end{Shaded}

\begin{verbatim}
## [1] 18441.58
\end{verbatim}

\begin{Shaded}
\begin{Highlighting}[]
\FunctionTok{sd}\NormalTok{(golf)}
\end{Highlighting}
\end{Shaded}

\begin{verbatim}
## [1] 135.7998
\end{verbatim}

\hypertarget{percentiles-and-quartiles}{%
\section{Percentiles and Quartiles}\label{percentiles-and-quartiles}}

\hypertarget{sample-percentiles}{%
\subsection{Sample Percentiles}\label{sample-percentiles}}

Given a sample \(x_1, \cdots, x_n\), the \(p\)th percentile, also called quantile, \(P\) is the value such that \(p\%\) or less of the observations are less than \(P\) and \((100-p)\%\) of less of the observations are greater than \(P\). Three special percentiles are quartiles \(Q_1, Q_2, Q_3\) and calculated as \(Q_1 = \frac{n+1}{4}\)th ordered observation, \(Q_2 = \frac{n+1}{2}\)th ordered observation, \(Q_3 = \frac{3(n+1)}{4}\)th ordered observation.

-\textbf{Step 1}: Arrange the data in increasing order.
-\textbf{Step 2}: Find the median of the entire data set. This value is the second quartile \(Q_2\).
-\textbf{Step 3}: Divide the ordered data set into two halves, a bottom half and a top half; if the number of observations is odd, include the median in both halves.
-\textbf{Step 4}: Find the median of the bottom half of the data set. This value is first quartile \(Q_1\).
-\textbf{Step 5}: Find the median of the top half of the data set. This value is the 3rd quartile \(Q_3\).
-\textbf{Step 6}: Summarize the results.

\hypertarget{example-14}{%
\subsubsection{Example}\label{example-14}}

A sample of 20 people reported their TV viewing hours in a week: 5, 15, 16, 20, 21, 25, 26, 27, 30, 30, 31, 32, 32, 34, 35, 38, 38, 41, 43, 66. Determine and interpret the quartile for these data.

\begin{Shaded}
\begin{Highlighting}[]
\NormalTok{x }\OtherTok{=} \FunctionTok{c}\NormalTok{(}\DecValTok{5}\NormalTok{, }\DecValTok{15}\NormalTok{, }\DecValTok{16}\NormalTok{, }\DecValTok{20}\NormalTok{, }\DecValTok{21}\NormalTok{, }\DecValTok{25}\NormalTok{, }\DecValTok{26}\NormalTok{, }\DecValTok{27}\NormalTok{, }\DecValTok{30}\NormalTok{, }\DecValTok{30}\NormalTok{, }\DecValTok{31}\NormalTok{, }\DecValTok{32}\NormalTok{, }\DecValTok{32}\NormalTok{, }\DecValTok{34}\NormalTok{, }\DecValTok{35}\NormalTok{, }\DecValTok{38}\NormalTok{, }\DecValTok{38}\NormalTok{, }\DecValTok{41}\NormalTok{, }\DecValTok{43}\NormalTok{, }\DecValTok{66}\NormalTok{)}
\end{Highlighting}
\end{Shaded}

Step 1. Order data

\begin{Shaded}
\begin{Highlighting}[]
\NormalTok{sorted.x }\OtherTok{\textless{}{-}} \FunctionTok{sort}\NormalTok{(x)}
\NormalTok{sorted.x}
\end{Highlighting}
\end{Shaded}

\begin{verbatim}
##  [1]  5 15 16 20 21 25 26 27 30 30 31 32 32 34 35 38 38 41 43 66
\end{verbatim}

Step 2. Find the median, i.e., Q2

Since there are 20 (an even number) observations in the data, the median is the average of 2 middle observations.
\[
Q_2 = \frac{30+31}{2}=30.5
\]

\begin{Shaded}
\begin{Highlighting}[]
\NormalTok{sorted.x[}\DecValTok{10}\NormalTok{]}
\end{Highlighting}
\end{Shaded}

\begin{verbatim}
## [1] 30
\end{verbatim}

\begin{Shaded}
\begin{Highlighting}[]
\NormalTok{sorted.x[}\DecValTok{11}\NormalTok{]}
\end{Highlighting}
\end{Shaded}

\begin{verbatim}
## [1] 31
\end{verbatim}

\begin{Shaded}
\begin{Highlighting}[]
\NormalTok{(sorted.x[}\DecValTok{10}\NormalTok{]}\SpecialCharTok{+}\NormalTok{sorted.x[}\DecValTok{11}\NormalTok{])}\SpecialCharTok{/}\DecValTok{2} 
\end{Highlighting}
\end{Shaded}

\begin{verbatim}
## [1] 30.5
\end{verbatim}

Step 3. Divide the ordered data into two halves. If the sample size is odd, include the median in both halves.

Bottom half is: 5, 15, 16, 20, 21, 25, 26, 27, 30, 30; top half is: 31, 32, 32, 34, 35, 38, 38, 41, 43, 66.

\begin{Shaded}
\begin{Highlighting}[]
\CommentTok{\#Bottom half}
\NormalTok{Bottom.half }\OtherTok{\textless{}{-}}\NormalTok{ sorted.x[}\DecValTok{1}\SpecialCharTok{:}\DecValTok{10}\NormalTok{]}
\NormalTok{Bottom.half}
\end{Highlighting}
\end{Shaded}

\begin{verbatim}
##  [1]  5 15 16 20 21 25 26 27 30 30
\end{verbatim}

\begin{Shaded}
\begin{Highlighting}[]
\CommentTok{\#Top half}
\NormalTok{Top.half }\OtherTok{\textless{}{-}}\NormalTok{ sorted.x[}\DecValTok{11}\SpecialCharTok{:}\DecValTok{20}\NormalTok{]}
\NormalTok{Top.half}
\end{Highlighting}
\end{Shaded}

\begin{verbatim}
##  [1] 31 32 32 34 35 38 38 41 43 66
\end{verbatim}

Step 4. Find the median of bottom half. Since there are 10 numbers, its median is the average of 5th and 6th numbers.

\[
Q_1 = \frac{21+25}{2}=23
\]

\begin{Shaded}
\begin{Highlighting}[]
\CommentTok{\#the median of the bottom half}
\NormalTok{Q1 }\OtherTok{\textless{}{-}}\NormalTok{ (Bottom.half[}\DecValTok{5}\NormalTok{]}\SpecialCharTok{+}\NormalTok{Bottom.half[}\DecValTok{6}\NormalTok{])}\SpecialCharTok{/}\DecValTok{2}
\NormalTok{Q1}
\end{Highlighting}
\end{Shaded}

\begin{verbatim}
## [1] 23
\end{verbatim}

Step 5. Find the median of top half, which is \(Q_3\). Since there are 10 numbers, its median is the average of 5th and 6th numbers.

\[
Q_3 = \frac{35+38}{2}=36.5
\]

\begin{Shaded}
\begin{Highlighting}[]
\CommentTok{\#the median of the bottom half}
\NormalTok{Q3 }\OtherTok{\textless{}{-}}\NormalTok{ (Top.half[}\DecValTok{5}\NormalTok{]}\SpecialCharTok{+}\NormalTok{Top.half[}\DecValTok{6}\NormalTok{])}\SpecialCharTok{/}\DecValTok{2}
\NormalTok{Q3}
\end{Highlighting}
\end{Shaded}

\begin{verbatim}
## [1] 36.5
\end{verbatim}

Step 6. Summarize the results.

In summary, the three quartiles for the TV-viewing times are \(Q_1 = 23\) hours, \(Q_2 = 30.5\) hours, \(Q_3=36.5\) hours. This means that 25\% TV-viewing times are less than 23 hours, 25\% are between 23 and 30.5 hours, 25\% are between 20.5 hours and 36.5 hours,and 25 \% are greater than 36.5 hours.

\begin{Shaded}
\begin{Highlighting}[]
\CommentTok{\#Use R functions}
\FunctionTok{quantile}\NormalTok{(x, }\AttributeTok{probs=}\FunctionTok{c}\NormalTok{(}\FloatTok{0.25}\NormalTok{, }\FloatTok{0.5}\NormalTok{, }\FloatTok{0.75}\NormalTok{), }\AttributeTok{type=}\DecValTok{5}\NormalTok{)  }
\end{Highlighting}
\end{Shaded}

\begin{verbatim}
##  25%  50%  75% 
## 23.0 30.5 36.5
\end{verbatim}

\hypertarget{population-percentiles}{%
\subsection{Population Percentiles}\label{population-percentiles}}

For a known population distribution, the \(p\)th percentile \(x_p\) is calculated as
\[
    \int_{-\infty}^{x_p} f(x) dx = p 
    \]
Equivalent, \(P(X < x_p) = p\) and \(x_p = F^{-1}(p)\), where \(F^{-1}(x)\) is the inverse cumulative distribution function. In this course, you are not required to calculate population percentiles.

In R, use the existing function below.

\begin{Shaded}
\begin{Highlighting}[]
        \DocumentationTok{\#\#\#\#\#\#\#\#\#\#\#\#\#\#\#\#\#\#\#\#\#\#\#\#}
        \CommentTok{\#Sample Percentile and quartiles}
        \DocumentationTok{\#\#\#\#\#\#\#\#\#\#\#\#\#\#\#\#\#\#\#\#\#\#\#\#}
        \CommentTok{\#Percentile and quartiles}
        \FunctionTok{set.seed}\NormalTok{(}\DecValTok{2023}\NormalTok{)}

\NormalTok{    x }\OtherTok{\textless{}{-}} \FunctionTok{rnorm}\NormalTok{(}\DecValTok{10}\NormalTok{)}
\NormalTok{    x}
\end{Highlighting}
\end{Shaded}

\begin{verbatim}
##  [1] -0.08378436 -0.98294375 -1.87506732 -0.18614466 -0.63348570  1.09079746
##  [7] -0.91372727  1.00163971 -0.39926660 -0.46812305
\end{verbatim}

\begin{Shaded}
\begin{Highlighting}[]
        \FunctionTok{quantile}\NormalTok{(x)}
\end{Highlighting}
\end{Shaded}

\begin{verbatim}
##         0%        25%        50%        75%       100% 
## -1.8750673 -0.8436669 -0.4336948 -0.1093744  1.0907975
\end{verbatim}

\begin{Shaded}
\begin{Highlighting}[]
        \CommentTok{\#10\% and 20\% quantiles of a sample drawn from a standard normal distribution}
        \FunctionTok{quantile}\NormalTok{(x, }\AttributeTok{probs =} \FunctionTok{c}\NormalTok{(}\FloatTok{0.1}\NormalTok{, }\FloatTok{0.2}\NormalTok{)) }
\end{Highlighting}
\end{Shaded}

\begin{verbatim}
##        10%        20% 
## -1.0721561 -0.9275706
\end{verbatim}

\hypertarget{interquartile-range-iqr}{%
\section{Interquartile Range (IQR)}\label{interquartile-range-iqr}}

IQR is defined as the difference between \(Q_3\) and \(Q_1\). Since quartiles indicate distribution of data, the IQR can reply the dispersion of the data distribution. A large IQR indicates a large variability in the middle 50\% data.

\begin{Shaded}
\begin{Highlighting}[]
        \DocumentationTok{\#\#\#\#\#\#\#\#\#\#\#\#\#\#\#\#\#\#\#\#\#\#\#\#}
        \CommentTok{\#Sample IQR}
        \DocumentationTok{\#\#\#\#\#\#\#\#\#\#\#\#\#\#\#\#\#\#\#\#\#\#\#\#}
        
        \FunctionTok{IQR}\NormalTok{(x)}
\end{Highlighting}
\end{Shaded}

\begin{verbatim}
## [1] 0.7342924
\end{verbatim}

The IQR is especially useful when dealing with skewed distributions or outliers.

\begin{Shaded}
\begin{Highlighting}[]
\NormalTok{iqr\_sbp }\OtherTok{\textless{}{-}} \FunctionTok{IQR}\NormalTok{(sbp)}
\NormalTok{iqr\_sbp}
\end{Highlighting}
\end{Shaded}

\begin{verbatim}
## [1] 9
\end{verbatim}

\begin{quote}
Interpretation: The difference between Q1 and Q3 is 9.
\end{quote}

\hypertarget{bar-plot}{%
\section{Bar plot}\label{bar-plot}}

Bar plot is commonly used to visualize values of a categorical variable in science journals. It is also used to provide comparison between two or more groups by each value of the categorical variable. It is not suitable to display values of a continuous variable, eg, CD4 blood counts. In this section, we incorporate programming codes that generate various standard bar plots.

The survival status of each class in Titanic is shown below.

\begin{Shaded}
\begin{Highlighting}[]
    \DocumentationTok{\#\#\#\#\#\#\#\#\#\#\#\#\#\#\#\#\#\#\#\#\#\#\#\#}
        \CommentTok{\#11. Bar Plot}
        \DocumentationTok{\#\#\#\#\#\#\#\#\#\#\#\#\#\#\#\#\#\#\#\#\#\#\#\#}
        \CommentTok{\#install.packages("knitr") \#install if not yet}
\NormalTok{        class1 }\OtherTok{\textless{}{-}} \FunctionTok{c}\NormalTok{(}\DecValTok{122}\NormalTok{, }\DecValTok{203}\NormalTok{) }\CommentTok{\#deaths, survivals}
\NormalTok{        class2 }\OtherTok{\textless{}{-}} \FunctionTok{c}\NormalTok{(}\DecValTok{167}\NormalTok{, }\DecValTok{118}\NormalTok{) }\CommentTok{\#deaths, survivals}
\NormalTok{        class3 }\OtherTok{\textless{}{-}} \FunctionTok{c}\NormalTok{(}\DecValTok{528}\NormalTok{, }\DecValTok{178}\NormalTok{) }\CommentTok{\#deaths, survivals}
\NormalTok{        crew }\OtherTok{\textless{}{-}} \FunctionTok{c}\NormalTok{(}\DecValTok{673}\NormalTok{, }\DecValTok{212}\NormalTok{) }\CommentTok{\#deaths, survivals}
        
\NormalTok{        classSurv }\OtherTok{\textless{}{-}} \FunctionTok{cbind}\NormalTok{(class1, class2, class3, crew)}
\NormalTok{        classSurv}
\end{Highlighting}
\end{Shaded}

\begin{verbatim}
##      class1 class2 class3 crew
## [1,]    122    167    528  673
## [2,]    203    118    178  212
\end{verbatim}

\begin{Shaded}
\begin{Highlighting}[]
        \FunctionTok{par}\NormalTok{(}\AttributeTok{mfrow=}\FunctionTok{c}\NormalTok{(}\DecValTok{1}\NormalTok{,}\DecValTok{1}\NormalTok{))  }
\NormalTok{        people }\OtherTok{\textless{}{-}} \FunctionTok{cbind}\NormalTok{(}\FunctionTok{sum}\NormalTok{(class1), }\FunctionTok{sum}\NormalTok{(class2), }\FunctionTok{sum}\NormalTok{(class3), }\FunctionTok{sum}\NormalTok{(crew))  }
        \FunctionTok{barplot}\NormalTok{(people, }\AttributeTok{main =} \StringTok{"People by Class"}\NormalTok{,}
        \AttributeTok{col=} \DecValTok{3}\NormalTok{,}
        \AttributeTok{names.arg=}\FunctionTok{c}\NormalTok{(}\StringTok{"Class 1"}\NormalTok{,}\StringTok{"Class 2"}\NormalTok{,}\StringTok{"Class 3"}\NormalTok{,}\StringTok{"Crew"}\NormalTok{))}
\end{Highlighting}
\end{Shaded}

\includegraphics{StatsTB_files/figure-latex/unnamed-chunk-78-1.pdf}

\begin{Shaded}
\begin{Highlighting}[]
        \FunctionTok{barplot}\NormalTok{(classSurv, }\AttributeTok{main =} \StringTok{"Survival by Class"}\NormalTok{,}
        \AttributeTok{col =} \FunctionTok{c}\NormalTok{(}\StringTok{"red"}\NormalTok{,}\StringTok{"green"}\NormalTok{),}
        \AttributeTok{names.arg=}\FunctionTok{c}\NormalTok{(}\StringTok{"Class 1"}\NormalTok{,}\StringTok{"Class 2"}\NormalTok{,}\StringTok{"Class 3"}\NormalTok{,}\StringTok{"Crew"}\NormalTok{))}
        \FunctionTok{legend}\NormalTok{(}\StringTok{"topleft"}\NormalTok{, }\FunctionTok{c}\NormalTok{(}\StringTok{"Not survived"}\NormalTok{,}\StringTok{"Survived"}\NormalTok{),}
        \AttributeTok{fill =} \FunctionTok{c}\NormalTok{(}\StringTok{"red"}\NormalTok{,}\StringTok{"green"}\NormalTok{), }\AttributeTok{bty=}\StringTok{"n"}\NormalTok{, }\AttributeTok{cex=}\FloatTok{0.8}\NormalTok{)}
\end{Highlighting}
\end{Shaded}

\includegraphics{StatsTB_files/figure-latex/unnamed-chunk-78-2.pdf}

\begin{Shaded}
\begin{Highlighting}[]
        \FunctionTok{barplot}\NormalTok{(people, }\AttributeTok{main =} \StringTok{"People by Class"}\NormalTok{,}
        \AttributeTok{col=} \DecValTok{3}\NormalTok{,}\AttributeTok{horiz =} \ConstantTok{TRUE}\NormalTok{,}
        \AttributeTok{names.arg=}\FunctionTok{c}\NormalTok{(}\StringTok{"Class 1"}\NormalTok{,}\StringTok{"Class 2"}\NormalTok{,}\StringTok{"Class 3"}\NormalTok{,}\StringTok{"Crew"}\NormalTok{))}
\end{Highlighting}
\end{Shaded}

\includegraphics{StatsTB_files/figure-latex/unnamed-chunk-78-3.pdf}

\begin{Shaded}
\begin{Highlighting}[]
        \FunctionTok{barplot}\NormalTok{(classSurv, }\AttributeTok{main =} \StringTok{"Survival by Class"}\NormalTok{,}
        \AttributeTok{col =} \FunctionTok{c}\NormalTok{(}\StringTok{"red"}\NormalTok{,}\StringTok{"green"}\NormalTok{),}\AttributeTok{horiz =} \ConstantTok{TRUE}\NormalTok{,}
        \AttributeTok{names.arg=}\FunctionTok{c}\NormalTok{(}\StringTok{"Class 1"}\NormalTok{,}\StringTok{"Class 2"}\NormalTok{,}\StringTok{"Class 3"}\NormalTok{,}\StringTok{"Crew"}\NormalTok{))}
        \FunctionTok{legend}\NormalTok{(}\StringTok{"bottomright"}\NormalTok{, }\FunctionTok{c}\NormalTok{(}\StringTok{"Not survived"}\NormalTok{,}\StringTok{"Survived"}\NormalTok{),}
        \AttributeTok{fill =} \FunctionTok{c}\NormalTok{(}\StringTok{"red"}\NormalTok{,}\StringTok{"green"}\NormalTok{), }\AttributeTok{bty=}\StringTok{"n"}\NormalTok{, }\AttributeTok{cex=}\FloatTok{0.8}\NormalTok{)}
\end{Highlighting}
\end{Shaded}

\includegraphics{StatsTB_files/figure-latex/unnamed-chunk-78-4.pdf}

\begin{Shaded}
\begin{Highlighting}[]
        \FunctionTok{barplot}\NormalTok{(classSurv, }\AttributeTok{main =} \StringTok{"Survival by Class"}\NormalTok{,}
        \AttributeTok{col =} \FunctionTok{c}\NormalTok{(}\StringTok{"red"}\NormalTok{,}\StringTok{"green"}\NormalTok{), }\AttributeTok{beside=}\ConstantTok{TRUE}\NormalTok{)}
        \FunctionTok{legend}\NormalTok{(}\StringTok{"topleft"}\NormalTok{, }\FunctionTok{c}\NormalTok{(}\StringTok{"Not survived"}\NormalTok{,}\StringTok{"Survived"}\NormalTok{),}
        \AttributeTok{fill =} \FunctionTok{c}\NormalTok{(}\StringTok{"red"}\NormalTok{,}\StringTok{"green"}\NormalTok{), }\AttributeTok{bty=}\StringTok{"n"}\NormalTok{)}
\end{Highlighting}
\end{Shaded}

\includegraphics{StatsTB_files/figure-latex/unnamed-chunk-78-5.pdf}

\begin{Shaded}
\begin{Highlighting}[]
        \FunctionTok{barplot}\NormalTok{(classSurv, }\AttributeTok{main =} \StringTok{"Survival by Class"}\NormalTok{,}
        \AttributeTok{col =} \FunctionTok{c}\NormalTok{(}\StringTok{"red"}\NormalTok{,}\StringTok{"green"}\NormalTok{), }\AttributeTok{beside=}\ConstantTok{TRUE}\NormalTok{, }\AttributeTok{horiz=}\ConstantTok{TRUE}\NormalTok{)}
        \FunctionTok{legend}\NormalTok{(}\StringTok{"bottomright"}\NormalTok{, }\FunctionTok{c}\NormalTok{(}\StringTok{"Not survived"}\NormalTok{,}\StringTok{"Survived"}\NormalTok{),}
        \AttributeTok{fill =} \FunctionTok{c}\NormalTok{(}\StringTok{"red"}\NormalTok{,}\StringTok{"green"}\NormalTok{), }\AttributeTok{bty=}\StringTok{"n"}\NormalTok{)}
\end{Highlighting}
\end{Shaded}

\includegraphics{StatsTB_files/figure-latex/unnamed-chunk-78-6.pdf}

\hypertarget{boxplot}{%
\section{Boxplot}\label{boxplot}}

Boxplots are graphical summaries that show:
- The median (middle line)
- The IQR (box) with the lower and upper box edges indicate the Q1 and Q3 respectively. 50\% data are included in the box.
- Possible outliers (points outside the whiskers).

*R's boxplot uses Tukey's 1.5 × IQR rule\textbf{ for whiskers and outliers, with quartiles defined by }Hyndman-Fan type 7 quantile definition**.

\textbf{Whiskers (Fences)}

By default, R defines the whiskers as the \textbf{most extreme values within 1.5 × IQR} of the quartiles:

\begin{itemize}
\item
  \textbf{Lower Whisker:}

  \[
  LW = \min\{x \in X : x \geq Q_1 - 1.5 \times IQR\}
  \]
\item
  \textbf{Upper Whisker:}

  \[
  UW = \max\{x \in X : x \leq Q_3 + 1.5 \times IQR\}
  \]
  \textbf{Outliers}
\end{itemize}

Any observations \textbf{outside the whiskers} are plotted as \textbf{outliers}:

\begin{itemize}
\item
  \textbf{Lower Outliers:}

  \[
  x < Q_1 - 1.5 \times IQR
  \]
\item
  \textbf{Upper Outliers:}

  \[
  x > Q_3 + 1.5 \times IQR
  \]
\end{itemize}

\textbf{Box Boundaries}

\begin{itemize}
\tightlist
\item
  \textbf{Lower Bound of the Box:} \(Q_1\)
\item
  \textbf{Upper Bound of the Box:} \(Q_3\)
\item
  \textbf{Middle Line (inside the box):} \(Q_2\) (Median)
\end{itemize}

\begin{figure}
\centering
\includegraphics{https://i.ibb.co/7dpdYrDg/boxplot.png}
\caption{Boxplot interpretation}
\end{figure}

\hypertarget{example-in-r}{%
\subsection{Example in R}\label{example-in-r}}

\begin{Shaded}
\begin{Highlighting}[]
\FunctionTok{boxplot.stats}\NormalTok{(sbp)}
\end{Highlighting}
\end{Shaded}

\begin{verbatim}
## $stats
## [1] 110.0 118.0 124.5 128.0 132.0
## 
## $n
## [1] 10
## 
## $conf
## [1] 119.5036 129.4964
## 
## $out
## numeric(0)
\end{verbatim}

This will return:

\begin{itemize}
\tightlist
\item
  \texttt{\$stats}: Q1, Median, Q3, and whisker bounds
\item
  \texttt{\$n}: sample size
\item
  \texttt{\$conf}: confidence interval around the median (for notch plots)
\item
  \texttt{\$out}: the outliers
\end{itemize}

\begin{Shaded}
\begin{Highlighting}[]
\FunctionTok{boxplot}\NormalTok{(sbp, }\AttributeTok{main =} \StringTok{"Boxplot of Systolic BP"}\NormalTok{,}
        \AttributeTok{ylab =} \StringTok{"Systolic BP (mmHg)"}\NormalTok{, }\AttributeTok{col =} \StringTok{"lightgreen"}\NormalTok{)}
\end{Highlighting}
\end{Shaded}

\includegraphics{StatsTB_files/figure-latex/unnamed-chunk-80-1.pdf}

\hypertarget{example-cholesterol-levels}{%
\subsection{Example: Cholesterol Levels}\label{example-cholesterol-levels}}

\begin{Shaded}
\begin{Highlighting}[]
\NormalTok{cholesterol }\OtherTok{\textless{}{-}} \FunctionTok{c}\NormalTok{(}\DecValTok{190}\NormalTok{, }\DecValTok{185}\NormalTok{, }\DecValTok{210}\NormalTok{, }\DecValTok{220}\NormalTok{, }\DecValTok{175}\NormalTok{, }\DecValTok{205}\NormalTok{, }\DecValTok{198}\NormalTok{, }\DecValTok{230}\NormalTok{, }\DecValTok{215}\NormalTok{, }\DecValTok{200}\NormalTok{)}
\FunctionTok{boxplot}\NormalTok{(cholesterol, }\AttributeTok{main =} \StringTok{"Boxplot of Cholesterol"}\NormalTok{,}
        \AttributeTok{ylab =} \StringTok{"mg/dL"}\NormalTok{, }\AttributeTok{col =} \StringTok{"lightblue"}\NormalTok{)}
\end{Highlighting}
\end{Shaded}

\includegraphics{StatsTB_files/figure-latex/unnamed-chunk-81-1.pdf}

\hypertarget{definition-five-number-summary}{%
\subsection{Definition: Five-number summary}\label{definition-five-number-summary}}

The five-number summary of a data set is Min, Q1, Q2, Q3, Max.

\begin{Shaded}
\begin{Highlighting}[]
\CommentTok{\#Use R function for 5{-}number summary directly}
\FunctionTok{summary}\NormalTok{(x, }\AttributeTok{quantile.type=}\DecValTok{5}\NormalTok{)}
\end{Highlighting}
\end{Shaded}

\begin{verbatim}
##     Min.  1st Qu.   Median     Mean  3rd Qu.     Max. 
## -1.87507 -0.91373 -0.43369 -0.34501 -0.08378  1.09080
\end{verbatim}

The \texttt{summary()} function gives a convenient overview of descriptive statistics in R. This is helpful for quickly getting a snapshot of your data.

\begin{Shaded}
\begin{Highlighting}[]
\FunctionTok{summary}\NormalTok{(sbp)}
\end{Highlighting}
\end{Shaded}

\begin{verbatim}
##    Min. 1st Qu.  Median    Mean 3rd Qu.    Max. 
##   110.0   118.5   124.5   122.8   127.5   132.0
\end{verbatim}

It includes:
- Minimum and Maximum
- 1st Quartile (Q1)
- Median (Q2)
- 3rd Quartile (Q3)
- Mean

\hypertarget{example-serum-cholesterol}{%
\subsection{Example: Serum Cholesterol}\label{example-serum-cholesterol}}

Suppose we collect serum cholesterol levels (mg/dL) from a sample of patients in a cardiovascular study:

\begin{Shaded}
\begin{Highlighting}[]
\NormalTok{cholesterol }\OtherTok{\textless{}{-}} \FunctionTok{c}\NormalTok{(}\DecValTok{190}\NormalTok{, }\DecValTok{185}\NormalTok{, }\DecValTok{210}\NormalTok{, }\DecValTok{220}\NormalTok{, }\DecValTok{175}\NormalTok{, }\DecValTok{205}\NormalTok{, }\DecValTok{198}\NormalTok{, }\DecValTok{230}\NormalTok{, }\DecValTok{215}\NormalTok{, }\DecValTok{200}\NormalTok{)}
\FunctionTok{summary}\NormalTok{(cholesterol)}
\end{Highlighting}
\end{Shaded}

\begin{verbatim}
##    Min. 1st Qu.  Median    Mean 3rd Qu.    Max. 
##   175.0   192.0   202.5   202.8   213.8   230.0
\end{verbatim}

\begin{Shaded}
\begin{Highlighting}[]
\FunctionTok{sd}\NormalTok{(cholesterol)}
\end{Highlighting}
\end{Shaded}

\begin{verbatim}
## [1] 16.75178
\end{verbatim}

\begin{Shaded}
\begin{Highlighting}[]
\FunctionTok{boxplot}\NormalTok{(cholesterol, }\AttributeTok{main =} \StringTok{"Serum Cholesterol Levels"}\NormalTok{,}
        \AttributeTok{ylab =} \StringTok{"Cholesterol (mg/dL)"}\NormalTok{, }\AttributeTok{col =} \StringTok{"lightblue"}\NormalTok{)}
\end{Highlighting}
\end{Shaded}

\includegraphics{StatsTB_files/figure-latex/unnamed-chunk-84-1.pdf}

\begin{quote}
Interpretation: The mean gives the average cholesterol level, while the SD and boxplot tell us how variable it is. This helps identify whether patients are within a healthy range or at cardiovascular risk.
\end{quote}

\hypertarget{example-body-temperature-by-sex}{%
\subsection{Example: Body Temperature by Sex}\label{example-body-temperature-by-sex}}

\begin{Shaded}
\begin{Highlighting}[]
\NormalTok{temp }\OtherTok{\textless{}{-}} \FunctionTok{c}\NormalTok{(}\FloatTok{97.6}\NormalTok{, }\FloatTok{98.7}\NormalTok{, }\FloatTok{97.2}\NormalTok{, }\FloatTok{98.0}\NormalTok{, }\FloatTok{97.9}\NormalTok{, }\FloatTok{99.4}\NormalTok{, }\FloatTok{98.8}\NormalTok{, }\FloatTok{99.1}\NormalTok{, }\FloatTok{99.5}\NormalTok{, }\FloatTok{99.3}\NormalTok{)}
\NormalTok{sex }\OtherTok{\textless{}{-}} \FunctionTok{c}\NormalTok{(}\FunctionTok{rep}\NormalTok{(}\StringTok{"Male"}\NormalTok{, }\DecValTok{5}\NormalTok{), }\FunctionTok{rep}\NormalTok{(}\StringTok{"Female"}\NormalTok{, }\DecValTok{5}\NormalTok{))}

\NormalTok{data.temp }\OtherTok{\textless{}{-}} \FunctionTok{data.frame}\NormalTok{(temp, sex)}
\NormalTok{data.temp}
\end{Highlighting}
\end{Shaded}

\begin{verbatim}
##    temp    sex
## 1  97.6   Male
## 2  98.7   Male
## 3  97.2   Male
## 4  98.0   Male
## 5  97.9   Male
## 6  99.4 Female
## 7  98.8 Female
## 8  99.1 Female
## 9  99.5 Female
## 10 99.3 Female
\end{verbatim}

\begin{Shaded}
\begin{Highlighting}[]
\CommentTok{\#summary for all}
\FunctionTok{summary}\NormalTok{(temp)}
\end{Highlighting}
\end{Shaded}

\begin{verbatim}
##    Min. 1st Qu.  Median    Mean 3rd Qu.    Max. 
##   97.20   97.92   98.75   98.55   99.25   99.50
\end{verbatim}

\begin{Shaded}
\begin{Highlighting}[]
\CommentTok{\#summary for male only}
\FunctionTok{summary}\NormalTok{(temp[sex}\SpecialCharTok{==}\StringTok{"Male"}\NormalTok{])}
\end{Highlighting}
\end{Shaded}

\begin{verbatim}
##    Min. 1st Qu.  Median    Mean 3rd Qu.    Max. 
##   97.20   97.60   97.90   97.88   98.00   98.70
\end{verbatim}

\begin{Shaded}
\begin{Highlighting}[]
\CommentTok{\#summary for female only}
\FunctionTok{summary}\NormalTok{(temp[sex}\SpecialCharTok{==}\StringTok{"Female"}\NormalTok{])}
\end{Highlighting}
\end{Shaded}

\begin{verbatim}
##    Min. 1st Qu.  Median    Mean 3rd Qu.    Max. 
##   98.80   99.10   99.30   99.22   99.40   99.50
\end{verbatim}

\begin{Shaded}
\begin{Highlighting}[]
\CommentTok{\#boxplot for all together}
\FunctionTok{boxplot}\NormalTok{(temp, }\AttributeTok{main =} \StringTok{"Body Temperature"}\NormalTok{,}
        \AttributeTok{ylab =} \StringTok{"Fahrenheit"}\NormalTok{, }\AttributeTok{col =} \StringTok{"pink"}\NormalTok{)}
\end{Highlighting}
\end{Shaded}

\includegraphics{StatsTB_files/figure-latex/unnamed-chunk-85-1.pdf}

\begin{Shaded}
\begin{Highlighting}[]
\CommentTok{\#boxplot by sex separately}
\FunctionTok{boxplot}\NormalTok{(temp }\SpecialCharTok{\textasciitilde{}}\NormalTok{ sex, }\AttributeTok{main =} \StringTok{"Body Temperature"}\NormalTok{,}
        \AttributeTok{ylab =} \StringTok{"Fahrenheit"}\NormalTok{, }\AttributeTok{col =} \FunctionTok{c}\NormalTok{(}\StringTok{"pink"}\NormalTok{, }\StringTok{"turquoise"}\NormalTok{))}
\end{Highlighting}
\end{Shaded}

\includegraphics{StatsTB_files/figure-latex/unnamed-chunk-85-2.pdf}

\begin{Shaded}
\begin{Highlighting}[]
    \DocumentationTok{\#\#\#\#\#\#\#\#\#\#\#\#\#\#\#\#\#\#\#\#\#\#\#\#}
        \CommentTok{\#Box Plot}
        \DocumentationTok{\#\#\#\#\#\#\#\#\#\#\#\#\#\#\#\#\#\#\#\#\#\#\#\#}
        \FunctionTok{set.seed}\NormalTok{(}\DecValTok{2023}\NormalTok{)}
\NormalTok{        height\_f }\OtherTok{\textless{}{-}} \FunctionTok{rnorm}\NormalTok{(}\DecValTok{100}\NormalTok{, }\AttributeTok{mean=}\FloatTok{5.2}\NormalTok{, }\AttributeTok{sd=}\FloatTok{0.3}\NormalTok{)}
\NormalTok{        height\_m }\OtherTok{\textless{}{-}} \FunctionTok{rnorm}\NormalTok{(}\DecValTok{100}\NormalTok{, }\AttributeTok{mean=}\FloatTok{5.8}\NormalTok{, }\AttributeTok{sd=}\FloatTok{0.5}\NormalTok{)}
\NormalTok{        height }\OtherTok{\textless{}{-}} \FunctionTok{c}\NormalTok{(height\_f, height\_m)}
\NormalTok{        group }\OtherTok{\textless{}{-}} \FunctionTok{c}\NormalTok{(}\FunctionTok{rep}\NormalTok{(}\DecValTok{1}\NormalTok{, }\DecValTok{100}\NormalTok{),}\FunctionTok{rep}\NormalTok{(}\DecValTok{2}\NormalTok{, }\DecValTok{100}\NormalTok{))}
        
\NormalTok{        data.ht }\OtherTok{\textless{}{-}} \FunctionTok{as.data.frame}\NormalTok{(}\FunctionTok{cbind}\NormalTok{(height, group))}
\NormalTok{        data.ht[}\DecValTok{1}\SpecialCharTok{:}\DecValTok{5}\NormalTok{,]}
\end{Highlighting}
\end{Shaded}

\begin{verbatim}
##     height group
## 1 5.174865     1
## 2 4.905117     1
## 3 4.637480     1
## 4 5.144157     1
## 5 5.009954     1
\end{verbatim}

\begin{Shaded}
\begin{Highlighting}[]
        \FunctionTok{par}\NormalTok{(}\AttributeTok{mfrow=}\FunctionTok{c}\NormalTok{(}\DecValTok{1}\NormalTok{,}\DecValTok{2}\NormalTok{))}
        
        \CommentTok{\#boxplot for all}
        \FunctionTok{boxplot}\NormalTok{(data.ht}\SpecialCharTok{$}\NormalTok{height, }\AttributeTok{col=}\StringTok{"green"}\NormalTok{, }\AttributeTok{ylab=}\StringTok{"Height"}\NormalTok{, }\AttributeTok{names =} \StringTok{"All"}\NormalTok{)}
        
        \CommentTok{\#boxplot for all {-} horizontal}
        \FunctionTok{boxplot}\NormalTok{(data.ht}\SpecialCharTok{$}\NormalTok{height, }\AttributeTok{col=}\StringTok{"green"}\NormalTok{, }\AttributeTok{ylab=}\StringTok{"Height"}\NormalTok{, }\AttributeTok{names =} \StringTok{"All"}\NormalTok{, }
        \AttributeTok{horizontal =} \ConstantTok{TRUE}\NormalTok{)}
\end{Highlighting}
\end{Shaded}

\includegraphics{StatsTB_files/figure-latex/unnamed-chunk-86-1.pdf}

\begin{Shaded}
\begin{Highlighting}[]
        \CommentTok{\#boxplot by group}
        \FunctionTok{boxplot}\NormalTok{(data.ht}\SpecialCharTok{$}\NormalTok{height[group}\SpecialCharTok{==}\DecValTok{1}\NormalTok{], data.ht}\SpecialCharTok{$}\NormalTok{height[group}\SpecialCharTok{==}\DecValTok{2}\NormalTok{],  }
        \AttributeTok{ylab=}\StringTok{"Height"}\NormalTok{, }\AttributeTok{names =} \FunctionTok{c}\NormalTok{(}\StringTok{"Female"}\NormalTok{, }\StringTok{"Male"}\NormalTok{), }\AttributeTok{col=}\FunctionTok{c}\NormalTok{(}\StringTok{"purple"}\NormalTok{, }\StringTok{"skyblue"}\NormalTok{))}
        
        \FunctionTok{boxplot}\NormalTok{(data.ht}\SpecialCharTok{$}\NormalTok{height }\SpecialCharTok{\textasciitilde{}}\NormalTok{ data.ht}\SpecialCharTok{$}\NormalTok{group,  }
        \AttributeTok{ylab=}\StringTok{"Height"}\NormalTok{, }\AttributeTok{names =} \FunctionTok{c}\NormalTok{(}\StringTok{"Female"}\NormalTok{, }\StringTok{"Male"}\NormalTok{), }\AttributeTok{col=}\FunctionTok{c}\NormalTok{(}\StringTok{"purple"}\NormalTok{, }\StringTok{"skyblue"}\NormalTok{))}
\end{Highlighting}
\end{Shaded}

\includegraphics{StatsTB_files/figure-latex/unnamed-chunk-86-2.pdf}

\begin{Shaded}
\begin{Highlighting}[]
        \CommentTok{\#boxplot by group {-} horizontal}
        \FunctionTok{boxplot}\NormalTok{(data.ht}\SpecialCharTok{$}\NormalTok{height[group}\SpecialCharTok{==}\DecValTok{1}\NormalTok{], data.ht}\SpecialCharTok{$}\NormalTok{height[group}\SpecialCharTok{==}\DecValTok{2}\NormalTok{], }
        \AttributeTok{col=}\FunctionTok{c}\NormalTok{(}\StringTok{"purple"}\NormalTok{, }\StringTok{"skyblue"}\NormalTok{),}
        \AttributeTok{ylab=}\StringTok{"Height"}\NormalTok{, }\AttributeTok{names =} \FunctionTok{c}\NormalTok{(}\StringTok{"Female"}\NormalTok{, }\StringTok{"Male"}\NormalTok{), }\AttributeTok{horizontal =} \ConstantTok{TRUE}\NormalTok{)}
\end{Highlighting}
\end{Shaded}

\includegraphics{StatsTB_files/figure-latex/unnamed-chunk-86-3.pdf}

\hypertarget{skewness-1}{%
\subsection{Skewness}\label{skewness-1}}

\begin{figure}
\centering
\includegraphics{https://i.ibb.co/F6zDD5b/133-Figure-03-16.png}
\caption{Boxplot and skewness}
\end{figure}

\hypertarget{definition-parameter-and-statistic}{%
\subsection{Definition: Parameter and Statistic}\label{definition-parameter-and-statistic}}

\begin{itemize}
\tightlist
\item
  \textbf{Parameter}: A descriptive measure for a population.
\item
  \textbf{Statistic}: A descriptive measure for a sample.
\end{itemize}

\hypertarget{standardized-variable}{%
\section{Standardized variable}\label{standardized-variable}}

For a variable \(x\), the variable
\[
z = \frac{x-\mu}{\sigma}
\]
is called the standardized version of \(x\) or the standardized variable corresponding to the variable \(x\).

\hypertarget{z-score}{%
\subsection{z-Score}\label{z-score}}

For an observed value of a variable \(x\), the corresponding value of the standardized variable z is called the z-score of the observation. The term standard score is often used instead of z-score.

\hypertarget{properties-of-a-standardized-variable}{%
\subsection{Properties of a standardized variable}\label{properties-of-a-standardized-variable}}

A standardized variable always has mean 0 and standard deviation 1.

A negative value indicates below the mean and positive \(z\) indicates greater than the mean.

\hypertarget{example-15}{%
\subsection{Example}\label{example-15}}

Consider a variable \(x\) has the observations in a finite population:
\[-1, 3, 3, 3, 5, 5\]
(1) Determine the standardized version of \(x\). (2) Find the observed value of \(z\) corresponding to an observed value of \(x\) of 5.
(3) Calculate all possible values of \(z\)
(4) Find the mean and standard deviation of \(z\)
(5) Show dotplots of both \(x\) and \(z\). Interpret the results.

\textbf{Solution.}

(a). We calculate the population mean and standard deviation as
\(\mu = 3\) and \(\sigma=2\). The standardized version of \(x\) is
\[
z = \frac{x-3}{2}
\]

\begin{enumerate}
\def\labelenumi{(\alph{enumi})}
\setcounter{enumi}{1}
\tightlist
\item
  When \(x=5\), the standardized value of \(z\) is
  \[
  z = \frac{x-3}{2} = \frac{5-3}{2} = 1
  \]
\end{enumerate}

\begin{Shaded}
\begin{Highlighting}[]
\NormalTok{x }\OtherTok{\textless{}{-}} \FunctionTok{c}\NormalTok{(}\SpecialCharTok{{-}}\DecValTok{1}\NormalTok{, }\DecValTok{3}\NormalTok{, }\DecValTok{3}\NormalTok{, }\DecValTok{3}\NormalTok{, }\DecValTok{5}\NormalTok{, }\DecValTok{5}\NormalTok{)}
\CommentTok{\#Use R function to calculate z score}
\NormalTok{IntroStats}\SpecialCharTok{::}\FunctionTok{z.score}\NormalTok{(x)}
\end{Highlighting}
\end{Shaded}

\begin{verbatim}
## [1] -2  0  0  0  1  1
\end{verbatim}

\begin{figure}
\centering
\includegraphics{https://i.ibb.co/3ygn3wx/145-Figure-03-19.png}
\caption{Population and sample}
\end{figure}

After standardization, the distribution has a shift, so the new sd is 0.

\begin{Shaded}
\begin{Highlighting}[]
\CommentTok{\#If we now the population mean and standard deviation}
\NormalTok{IntroStats}\SpecialCharTok{::}\FunctionTok{z.score}\NormalTok{(x, }\AttributeTok{mu=}\FloatTok{3.2}\NormalTok{, }\AttributeTok{sigma=}\DecValTok{2}\NormalTok{)}
\end{Highlighting}
\end{Shaded}

\begin{verbatim}
## [1] -2.1 -0.1 -0.1 -0.1  0.9  0.9
\end{verbatim}

\hypertarget{histogram-1}{%
\section{Histogram}\label{histogram-1}}

Histogram is to display the distribution of a sample with relative chance of occurrence. A peak point indicates a value the occurs the most frequently. It is often superimposed with a probability density curve, which indicates the same meaning.

\begin{Shaded}
\begin{Highlighting}[]
      \DocumentationTok{\#\#\#\#\#\#\#\#\#\#\#\#\#\#\#\#\#\#\#\#\#\#\#\#}
        \CommentTok{\#Histgram}
        \DocumentationTok{\#\#\#\#\#\#\#\#\#\#\#\#\#\#\#\#\#\#\#\#\#\#\#\#}
\NormalTok{        x }\OtherTok{=} \FunctionTok{rnorm}\NormalTok{(}\DecValTok{100}\NormalTok{)}
        \FunctionTok{hist}\NormalTok{(x) }
\end{Highlighting}
\end{Shaded}

\includegraphics{StatsTB_files/figure-latex/unnamed-chunk-89-1.pdf}

\begin{Shaded}
\begin{Highlighting}[]
    \FunctionTok{hist}\NormalTok{(x, }\AttributeTok{col =} \DecValTok{5}\NormalTok{, }\AttributeTok{main=}\StringTok{"Default Histogram"}\NormalTok{)}
\end{Highlighting}
\end{Shaded}

\includegraphics{StatsTB_files/figure-latex/unnamed-chunk-89-2.pdf}

\begin{Shaded}
\begin{Highlighting}[]
        \FunctionTok{hist}\NormalTok{(x, }\AttributeTok{col =} \DecValTok{5}\NormalTok{, }\AttributeTok{density =} \DecValTok{20}\NormalTok{, }\AttributeTok{main =} \StringTok{"Density = 20"}\NormalTok{)}
\end{Highlighting}
\end{Shaded}

\includegraphics{StatsTB_files/figure-latex/unnamed-chunk-89-3.pdf}

\begin{Shaded}
\begin{Highlighting}[]
        \FunctionTok{hist}\NormalTok{(x, }\AttributeTok{freq=}\ConstantTok{FALSE}\NormalTok{, }\AttributeTok{col =} \DecValTok{6}\NormalTok{, }\AttributeTok{main=}\StringTok{"Relative Frequency"}\NormalTok{)}
\end{Highlighting}
\end{Shaded}

\includegraphics{StatsTB_files/figure-latex/unnamed-chunk-89-4.pdf}

\begin{Shaded}
\begin{Highlighting}[]
        \FunctionTok{hist}\NormalTok{(x, }\AttributeTok{freq=}\ConstantTok{FALSE}\NormalTok{, }\AttributeTok{col =} \StringTok{"pink"}\NormalTok{, }\AttributeTok{main =} \StringTok{"Curve Overlay"}\NormalTok{)}
        \FunctionTok{lines}\NormalTok{(}\FunctionTok{density}\NormalTok{(x), }\AttributeTok{col =} \StringTok{"blue"}\NormalTok{, }\AttributeTok{lwd =} \DecValTok{3}\NormalTok{)}
\end{Highlighting}
\end{Shaded}

\includegraphics{StatsTB_files/figure-latex/unnamed-chunk-89-5.pdf}

\hypertarget{summary-3}{%
\section{Summary}\label{summary-3}}

In this chapter, we covered key descriptive statistics including

\begin{itemize}
\tightlist
\item
  \textbf{Measures of central tendency}: mean, median, mode
\item
  \textbf{Measures of spread}: range, variance, standard deviation, IQR
\item
  \textbf{Boxplot}: Visualizing the data and detecting outliers
\end{itemize}

\hypertarget{probability-concepts}{%
\chapter{Probability Concepts}\label{probability-concepts}}

\begin{Shaded}
\begin{Highlighting}[]
\FunctionTok{library}\NormalTok{(IntroStats)}
\end{Highlighting}
\end{Shaded}

\hypertarget{probability-basics}{%
\section{Probability Basics}\label{probability-basics}}

\hypertarget{definition-3}{%
\subsection{Definition}\label{definition-3}}

Probability for Equally Likely Outcomes (\textbf{f/N Rule}). Suppose an experiment has N possible outcomes, all equally likely. An event that can occur in \(f\) ways has probability \(f/N\).

\begin{figure}
\centering
\includegraphics{https://i.ibb.co/rMTrqrq/probability-f-N.png}
\caption{Figure. Probability f/N Rule}
\end{figure}

\hypertarget{example-16}{%
\subsection{Example}\label{example-16}}

When 2 balanced dice are rolled, 36 equally likely outcomes are possible, and illustrated below.

\begin{figure}
\centering
\includegraphics{https://i.ibb.co/cY5TBnj/159-Figure-04-01.png}
\caption{Figure. 36 Possible dice outcomes}
\end{figure}

Question (a). Find the probability of the sum of the dice is 11.

\textbf{Solution.}

There are exactly 2 ways to have the sum of 11. So the probability is 2/36 = 0.056.

\begin{enumerate}
\def\labelenumi{(\alph{enumi})}
\setcounter{enumi}{1}
\tightlist
\item
  Find the probability of doubles are rolled, that is, both dice come up the same number.
\end{enumerate}

\textbf{Solution.}

There are exactly 6 ways to have the doubles. So the probability is 6/36 = 0.167.

\hypertarget{basic-properties-of-probabilities}{%
\subsection{Basic Properties of probabilities}\label{basic-properties-of-probabilities}}

\emph{Property 1:} The probability of an event is always between 0 and 1, inclusive.

\emph{Property 2: }The probability of an event that cannot occur is 0. (An event that cannot occur is called an impossible event.)

\emph{Property 3:} The probability of an event that must occur is 1. (An event that must occur is called a certain event.)

Return to the dice problem. What is the probability that the sum is 1?

What is the probability that the sum of the dice is 12 or less?

\hypertarget{events}{%
\section{Events}\label{events}}

\hypertarget{definition-sample-space-and-event}{%
\subsection{Definition Sample space and event}\label{definition-sample-space-and-event}}

\textbf{Sample space:} The collection of all possible outcomes
for an experiment.

\textbf{Event:} A collection of outcomes for the experiment, that
is, any subset of the sample space. An event occurs if
and only if the outcome of the experiment is a member
of the event.

\hypertarget{example-17}{%
\subsection{Example}\label{example-17}}

\begin{figure}
\centering
\includegraphics{https://i.ibb.co/Jp1RX9p/165-Figure-04-03.png}
\caption{Figure Playing cards}
\end{figure}

For the experiment of randomly selecting one card from a deck of 52. Let

A = event the card selected is the king of hearts,

B = event the card selected is a king

C = event the card selected is a heart

D = event the card selected is a face

List the outcomes constituting each of these events.

\emph{solution.}

The outcome of A is the king of hearts card.

\begin{figure}
\centering
\includegraphics{https://i.ibb.co/1qGz7kp/165-Figure-04-04.png}
\caption{Figure Playing cards}
\end{figure}

The outcome of B:

\begin{figure}
\centering
\includegraphics{https://i.ibb.co/2W4Lvzt/165-Figure-04-05.png}
\caption{Figure Playing cards}
\end{figure}

The outcome of C:

\begin{figure}
\centering
\includegraphics{https://i.ibb.co/bHYdk2n/165-Figure-04-06.png}
\caption{Figure Playing cards}
\end{figure}

The outcome of D:

\begin{figure}
\centering
\includegraphics{https://i.ibb.co/vYQFzLm/165-Figure-04-07.png}
\caption{Figure Playing cards}
\end{figure}

\hypertarget{venn-diagrams}{%
\subsection{Venn Diagrams}\label{venn-diagrams}}

Venn Diagrams named after English Logician John Venn (1834-1923) are the best ways to show events and relationships.

Venn diagrams for (a) event (not E), (b) event (A \& B), and (c) event (A or B)

Relationships Among Events

-(not E): The event ``E does not occur''

-(A \& B): The event ``both A and B occur''

-(A or B): The event ``either A or B or both occur''

\begin{figure}
\centering
\includegraphics{https://i.ibb.co/jbNkxxn/166-Figure-04-09.png}
\caption{Figure}
\end{figure}

\hypertarget{example-18}{%
\subsection{Example}\label{example-18}}

The age and frequency for a class of students are listed below:

Age (Frequency): 17(1), 18(1), 19(9), 20(7), 21(7)

One student is randomly selected from the class. Define

A = event that the selected student is under age 21

B = event that the selected student is over 18

Determine the the following events:

\begin{enumerate}
\def\labelenumi{(\alph{enumi})}
\tightlist
\item
  (not B); (b) (A \& B); (c) (A or B)
\end{enumerate}

\textbf{Solution.}

(a): 2 students (age: 18 and 17)

(b): 16 students (Age: 19 and 20)

\hypertarget{definition.-mutually-exclusive-events}{%
\subsection{Definition. Mutually exclusive events}\label{definition.-mutually-exclusive-events}}

\textbf{Definition.} Two or more events are mutually exclusive events if no two of them have outcomes in common.

\begin{figure}
\centering
\includegraphics{https://i.ibb.co/883WR4B/169-Figure-04-14.png}
\caption{Figure}
\end{figure}

\begin{enumerate}
\def\labelenumi{(\alph{enumi})}
\tightlist
\item
  Two mutually exclusive events; (b) two non-mutually exclusive events
\end{enumerate}

\begin{figure}
\centering
\includegraphics{https://i.ibb.co/DCLjpqF/169-Figure-04-15.png}
\caption{Figure}
\end{figure}

Which ones are showing mutually exclusive events? (a) Three mutually exclusive events;
(b) three non-mutually exclusive events;(c) three non-mutually exclusive events

\hypertarget{rules-of-probability}{%
\section{Rules of probability}\label{rules-of-probability}}

\hypertarget{definition.}{%
\subsection{Definition.}\label{definition.}}

Probability Notation: If E is an event, then P(E) represents the probability that event E occurs. It is read ``the probability of E.''

\hypertarget{addition-rules}{%
\subsection{Addition rules}\label{addition-rules}}

If event A and event B are mutually exclusive, then
\[
P(A \text{ or } B) = P(A) + P(B)
\]
More generally, if events \(A_1\), \(A_2, \cdots, A_k\) are mutually exclusive, then
\[
P(A_1 \text{ or } A_2 \text{ or } \cdots, A_k) = \sum P(A_i)
\]

\hypertarget{complementation-rule}{%
\subsection{Complementation rule}\label{complementation-rule}}

For any event \(E\),
\[
P(\text{not } E) = 1- P(E)
\]

\hypertarget{example-19}{%
\subsection{Example}\label{example-19}}

Consider the relative frequencies in US arms.

\begin{figure}
\centering
\includegraphics{https://i.ibb.co/PZTQYDx/174-Table-04-05.png}
\caption{Figure}
\end{figure}

Find the probability that a randomly selected farm has (a) less than 2000 acres; (b) 50 acres or more.

\textbf{Solution.}

Define event J = the selected form has less than 2000 acres.

Then \[P(J) = 1- P(G) = 1-0.036 = 0.964\]

Define event K = the selected form has 50 acres or more.

\[
P(K) = 1-P(\text{not }K) = 1-P(A)-P(B)=0.387
\]

\hypertarget{general-addition-rule}{%
\subsection{General addition rule}\label{general-addition-rule}}

If A and B are any two events, then
\[
P(A \text{ or } B) = P(A) + P(B) - P(A \text{ & } B)
\]

\hypertarget{example-20}{%
\subsection{Example}\label{example-20}}

Characteristics of People Arrested Data on people who have been arrested are published by the Federal Bureau of Investigation in Uniform Crime Reports. Records for one year show that 73.9\% of the people arrested were male, 12.0\% were under 18 years of age, and 8.5\% were males under 18 years of age. If a person arrested that year is selected at random, what is the probability that that person is either male or under 18?

\textbf{Solution.}

Let M = event the person obtained is male and E = event the person is under 18.

then

\[
P(M \text{ or } E) = P(M) + P(E) - P(M \text{ & }E) = 0.739+0.120-0.085 = 0.774
\]

\hypertarget{addition-in-3-events}{%
\subsection{Addition in 3 events}\label{addition-in-3-events}}

If A and B are any two events, then
\[
P(A \text{ or } B \text{ or } C) = P(A) + P(B) + P(C) - P(A \text{ & } B)- P(B \text{ & } C)- P(A \text{ & } C) + P(A \text{ & } B \text{ & } C)
\]

\hypertarget{contingency-tables}{%
\section{Contingency tables}\label{contingency-tables}}

\hypertarget{understanding-contingency-tables}{%
\subsection{Understanding Contingency tables}\label{understanding-contingency-tables}}

\begin{figure}
\centering
\includegraphics{https://i.ibb.co/7Kbx5Jm/180-Table-04-06.png}
\caption{Figure}
\end{figure}

What are the two variables? How do you understand the Total row and the Total column?

Suppose a faculty member is randomly selected, determine the following probabilities.

\begin{enumerate}
\def\labelenumi{(\alph{enumi})}
\tightlist
\item
  \(P(A_1)\); (b) \(P(R_2)\); (c) \(P(A_1 \text{ & } R_2)\)
\end{enumerate}

The probabilities \(P(A_1)\) and \(P(R_2)\) are called \textbf{marginal} probabilities because they correspond to the events in the margin of the contingency table.

The probability of \(P(A_1 \text{ & } R_2)\) is called \textbf{joint} probability.

\textbf{Solution.}

\begin{enumerate}
\def\labelenumi{(\alph{enumi})}
\item
  \(P(A_1) = 68/1164=0.058\)
\item
  \(P(R_2)=381/1164=0.327\)
\item
  \(P(A_1 \text{ & } R_2) = 3/1164 = 0.003\)
\end{enumerate}

We sometimes replace the contingency table by joint probabilities and marginal probabilities below.

\begin{figure}
\centering
\includegraphics{https://i.ibb.co/yXb40sY/182-Table-04-07.png}
\caption{Figure}
\end{figure}

\hypertarget{conditional-probability}{%
\section{Conditional Probability}\label{conditional-probability}}

\hypertarget{definition-4}{%
\subsection{Definition}\label{definition-4}}

The probability that event B occurs given that event A occurs is called a \textbf{conditional probability}. It is denoted as \(P(B|A)\) which is read ``the probability of B given A.'' We
call A the given event.

\hypertarget{example-21}{%
\subsection{Example}\label{example-21}}

When a balanced die is rolled once, 6 equally likely outcomes are possible:

\includegraphics{https://i.ibb.co/sg8vjdp/185-Figure-04-22.png}
Let F = event a 5 is rolled, and O = event the die comes up odd.

Determine the probabilities:

\begin{enumerate}
\def\labelenumi{(\alph{enumi})}
\tightlist
\item
  \(P(F)\); (b) \(P(F|O)\); (c) \(P(O | (\text{not } F))\)
\end{enumerate}

\textbf{Solution.}

(a).

\[P(F) = 1/6=0.167\]

(b). Given the condition \(O\) occurs, there are only 3 possible outcomes: 1, 3, 5.

\begin{figure}
\centering
\includegraphics{https://i.ibb.co/W0LYp1S/186-Figure-04-23.png}
\caption{Figure}
\end{figure}

\[P(F|O) = 1/3 = 0.333\]

(c). Given that 5 is not rolled (not \(F\)), the possible outcomes are 5 numbers 1, 2, 3, 4, 6.

\begin{figure}
\centering
\includegraphics{https://i.ibb.co/19fwKkW/186-Figure-04-24.png}
\caption{Figure}
\end{figure}

\[
P(O|(\text{not }F)) = 2/5 = 0.4
\]

\hypertarget{conditional-probability-rule}{%
\subsection{Conditional probability rule}\label{conditional-probability-rule}}

If A and B are any two events with \(P(A) >0\), then
\[
P(B|A) = \frac{P(A\text{ & } B)}{P(A)}
\]

\hypertarget{example-22}{%
\subsection{Example}\label{example-22}}

U.S. Census Bureau publishes martial status and gender below. Single = never married.

\begin{figure}
\centering
\includegraphics{https://i.ibb.co/2jJshDS/188-Table-04-09.png}
\caption{Figure}
\end{figure}

Suppose a U.S. adult is selected at random. Determine

\begin{enumerate}
\def\labelenumi{(\alph{enumi})}
\item
  the probability that the adult is divorced, given that the adult selected is a male.
\item
  the probability that the adult is a male, given that the adault is divoiced.
\end{enumerate}

\textbf{Solution.}

(a).

\[
P(M_4 | S_1) = \frac{P(M_4\text{ & } S_1)}{P(S_1)} = \frac{0.044}{0.485} = 0.091
\]

(b).

\[
P(S_1 | M_4) = \frac{P(M_4\text{ & } S_1)}{P(M_4)} = \frac{0.044}{0.104} = 0.423
\]

\hypertarget{multiplication-rule-and-independence}{%
\section{Multiplication rule and Independence}\label{multiplication-rule-and-independence}}

\hypertarget{general-multiplication-rule}{%
\subsection{General multiplication rule}\label{general-multiplication-rule}}

If A and B are any two events, then

\[
P(A \text{ & } B) = P(A)\cdot P(B|A)
\]
This is obtained immediately from the conditional probability rule
\[
P(B|A) = \frac{P(A\text{ & } B)}{P(A)}
\]

\hypertarget{example-23}{%
\subsection{Example}\label{example-23}}

The U.S. Congress, Joint Committee on Printing, provides information on the composition of the Congress in the Congressional Directory. For the 113th Congress, 18.7\% of the members are senators and 53\% of the senators are Democrats. What is the probability that a randomly selected member of the 113th Congress is a Democratic senator?

\textbf{Solution} Let

D = event the member selected is a Democrat, and

S = event the member selected is a senator.

The event that the member selected is a Democratic senator can be expressed as
\textbf{(S \& D)}. We want to determine the probability of that event.

Because 18.7\% of members are senators,

\[P(S) = 0.187\]

and because 53\% of

senators are Democrats,
\[P(D | S) = 0.530\]

Applying the general multiplication rule, we get

\[P(S \text{ & } D) = P(S) \cdot P(D|S) = 0.187 \cdot  0.530 = 0.099.\]

The probability that a randomly selected member of the 113th Congress is a Democratic senator is 0.099.

\emph{Interpretation} 9.9\% of members of the 113th Congress are Democratic senators.

\hypertarget{example-24}{%
\subsection{Example}\label{example-24}}

In Professor Weiss's introductory statistics class, the number of males and females are as shown in the frequency distribution presented below. Two students are selected at random from the class. The first student selected is not returned to the class for possible reselection; that is, the sampling is without replacement. Find the probability that the first student selected is female and the second is male.

\begin{figure}
\centering
\includegraphics{https://i.ibb.co/gtJHWx5/194-Table-04-10.png}
\caption{Figure}
\end{figure}

\textbf{Solution}

Let

F1 = event the first student obtained is female, and

M2 = event the second student obtained is male.

We want to determine P(F1 \& M2). Using the general multiplication rule, we write
\[
P(F1 \text{ & } M2) = P(F1) \cdot P(M2 | F1).
\]
\[
P(F1) = \frac{23}{40}
\]

Next, we find \(P(M2| F1)\)-the conditional probability that the second student selected is male, given that the first one selected is female.

Given that the first student selected is female, of the 39 students remaining in the class 17 are male, so

\[
P(M2 | F1) = \frac{17}{39}
\]

Applying the general multiplication rule, we conclude that

\[
P(F1 \text{ & } M2) = P(F1) \cdot P(M2| F1) = \frac{23}{40}\cdot\frac{17}{39} = 0.251.
\]

\emph{Interpretation} When two students are randomly selected from the class, the probability is 0.251 that the first student selected is female and the second student selected is male.

\hypertarget{definition.-independence}{%
\subsection{Definition. Independence}\label{definition.-independence}}

Event B is said to be independent of event A if
\[
P(B|A) = P(B)
\]
One event is independent of another event if knowning whether the latter event occurs does not affect the probability of the former event.

\hypertarget{example-25}{%
\subsection{Example}\label{example-25}}

Consider the playing cards example again. Let

F = event a face card is selected

K = event a king is selected

H = event a heart is selected.

Question: (a) Determine whether K is independent of F; (b) whether K is independent of H.

\begin{figure}
\centering
\includegraphics{https://i.ibb.co/Jp1RX9p/165-Figure-04-03.png}
\caption{Figure Playing cards}
\end{figure}

\textbf{Solution}

\begin{enumerate}
\def\labelenumi{(\alph{enumi})}
\tightlist
\item
  Note:
\end{enumerate}

\[
P(K) = 4/52=1/13
\]
To determine whether K is independent of F, we compute \(P(K|F)\) and compare to \(P(K)\).

\[
P(K|F) = 4/12 = 1/3
\]

which does not equal P(K), so K is not independent of F.

(b). When H occurs (a heart is selected), there are 13 possible outcomes. So

\[
P(K|H) = 1/13
\]

which equals P(K). So K is independent of H.

\hypertarget{special-multiplication-rule}{%
\subsection{Special multiplication rule}\label{special-multiplication-rule}}

If A and B are independent, then
\[
P(A \text{ & } B) = P(A) \cdot P(B)
\]
\#\#\# Example.

An American roulette wheel contains 38 numbers, of which 18 are red, 18 are black, and 2 are green. When the roulette wheel is spun, the ball is equally likely to land on any of the 38 numbers. In three plays at a roulette wheel, what is the probability that the ball will land on green the first time and on black the second and third times?

\begin{figure}
\centering
\includegraphics{https://i.ibb.co/1QYKxyk/roulette.jpg}
\caption{Figure American roulette wheel}
\end{figure}

\textbf{Solution}

First, we can reasonably assume that outcomes on successive plays at the wheel are independent. Now, we let

G1 = event the ball lands on green the first time,

B2 = event the ball lands on black the second time, and

B3 = event the ball lands on black the third time.

We want to determine P(G1 \& B2 \& B3).

Because outcomes on successive plays at the wheel are independent, we know that event G1, event B2, and event B3 are independent. Applying the special multiplication rule, we conclude that

\[
P(G1 \text{ & } B2 \text{ & } B3) = P(G1) \cdot P(B2) \cdot P(B3) =\frac{2}{38} \cdot \frac{18}{38} \cdot \frac{18}{38}=0.012
\]

\emph{Interpretation} In three plays at a roulette wheel, there is a 1.2\% chance that the ball will land on green the first time and on black the second and third times.

\hypertarget{mutually-exclusive-versus-independent-events}{%
\subsection{Mutually Exclusive Versus Independent Events}\label{mutually-exclusive-versus-independent-events}}

The terms mutually exclusive and independent refer to different concepts. Mutually exclusive events are those that cannot occur simultaneously; independent events are those for which the occurrence of some does not affect the probabilities of the others occurring.

In fact, if two or more events are mutually exclusive, the occurrence of one precludes the occurrence of the others. Two events cannot be both mutually exclusive and independent.

When two events are mutually exclusive, there is addition rule:
\[P(A \text{ or } B) = P(A) + P(B)\]
Note: A and B cannot both occur together.

When two events are independent, there is multiplication rule.

\[P(A \text{ & } B) = P(A) \cdot P(B)\]

\hypertarget{bayess-rule}{%
\section{Bayes's Rule}\label{bayess-rule}}

\hypertarget{rule-of-total-probability}{%
\subsection{Rule of total probability}\label{rule-of-total-probability}}

Suppose that events \(A_1, \cdots, A_k\) are mutually exclusive and exhaustive; that is, exactly one of the events must occur \(\sum P(A_i) = 1\). Then for any event B,

\[
P(B) = \sum P(A_i \text{ & } B) = \sum P(A_i) \cdot P(B|A_i)
\]

\hypertarget{example-26}{%
\subsection{Example}\label{example-26}}

The U.S. Census Bureau presents data on age of residents and region of residence in Demographic Profiles. The first two columns of table below give a percentage distribution for region of residence; the third column shows the percentage of seniors (age 65 years or over) in each region. For instance, 17.9\% of U.S. residents live in the Northeast, and 14.1\% of those who live in the Northeast are seniors. Determine the percentage of U.S. residents that are seniors.

\begin{figure}
\centering
\includegraphics{https://i.ibb.co/XJPzwcq/202-Table-04-11.png}
\caption{Figure}
\end{figure}

\textbf{Solution}:

Probability derived from the table:

\begin{figure}
\centering
\includegraphics{https://i.ibb.co/xSR2Jj0/203-Table-04-12.png}
\caption{Figure}
\end{figure}

Let

\begin{itemize}
\tightlist
\item
  S = event the resident selected is a senior,
\end{itemize}

and

\begin{itemize}
\item
  R1 = event the resident selected lives in the Northeast,
\item
  R2 = event the resident selected lives in the Midwest,
\item
  R3 = event the resident selected lives in the South, and
\item
  R4 = event the resident selected lives in the West.
\end{itemize}

The problem is to determine the percentage of U.S. residents that are seniors, or, in terms of probability, P(S). Because a U.S. resident must reside in exactly one of the four regions, events R1, R2, R3, and R4 are mutually exclusive and ex-haustive. Therefore, by the rule of total probability we have

\[
P(S) = \sum P(R_j)\cdot P(S| R_j)
= 0.179 • 0.141 + 0.217 • 0.135 + 0.371 • 0.130 + 0.233 • 0.119
= 0.130.
\]

\hypertarget{bayes-rule}{%
\subsection{Bayes' Rule}\label{bayes-rule}}

Suppose \(A_1, \cdots, A_k\) are mutually exclusive and exhaustive (\(\sum P(A_i)=1\)), then apply the total probability rule to event B

\[
P(B) = \sum P(A_i \text{ & } B) = \sum P(A_i) \cdot P(B|A_i)
\]
Furthermore, by conditional probability rule,

\[
P(A_i|B) = \frac{P(A_i\text{ & } B)}{P(B)} = \frac{P(A_i\text{ & } B)}{\sum P(A_i) \cdot P(B|A_i)}
\]

This is \textbf{Bayes' rule}.

\hypertarget{example-27}{%
\subsection{Example}\label{example-27}}

Following the seniors problem example, we know 14.1\% of Northeast residents are seniors. What percentage of seniors are Northeast residents?

Apply Bayes' rule,

\[P(S)=\sum P(R_j)P(S|R_j) = 0.179\cdot 0.141 + 0.217\cdot 0.135 + 0.371\cdot 0.130+ 0.233\cdot 0.119 = 0.1305\]

\begin{Shaded}
\begin{Highlighting}[]
\FloatTok{0.179}\SpecialCharTok{*} \FloatTok{0.141} \SpecialCharTok{+} \FloatTok{0.217}\SpecialCharTok{*} \FloatTok{0.135} \SpecialCharTok{+} \FloatTok{0.371}\SpecialCharTok{*} \FloatTok{0.130}\SpecialCharTok{+} \FloatTok{0.233}\SpecialCharTok{*} \FloatTok{0.119}
\end{Highlighting}
\end{Shaded}

\begin{verbatim}
## [1] 0.130491
\end{verbatim}

\[
P(R_1|S) = \frac{P(R_1)P(S|R_1)}{\sum P(R_j)P(S|R_j)} = \frac{0.179\cdot 0.141}{0.1305}=0.193
\]

19.5\% seniors are Northeast residents.

\hypertarget{example-28}{%
\subsection{Example}\label{example-28}}

According to the Arizona Chapter of the American Lung Association, 7.0\% of the population has lung disease. Of those having lung disease, 90.0\% are smokers; of those not having lung disease, 25.3\% are smokers.

Determine the probability that a randomly selected smoker has lung disease.

\textbf{Solution}

\textbf{Step 1: Define events of interest}

Let

\begin{itemize}
\tightlist
\item
  S = event the person selected is a smoker,
\end{itemize}

and

\begin{itemize}
\item
  L1 = event the person selected has no lung disease, and
\item
  L2 = event the person selected has lung disease.
\end{itemize}

Note that events L1 and L2 are complementary, which implies that they are mutually exclusive and exhaustive:
\[P(L1) + P(L2) = 1; P(L1 \text{ and } L2) = 0\]

The problem is to determine the probability that a randomly selected smoker has lung disease. This is a conditional probability and the condition is \textbf{selected person is a smoker}, so need to find out

\[P(L2 | S) = ?\]

\textbf{Step 2: Find given information related to the events}

\begin{itemize}
\item
  7.0\% of the population has lung disease indicates \(P(L2) = 0.070\). This is one marginal probability.
\item
  Of those having lung disease, 90.0\% are smokers. This indicates a conditional probability and \textbf{the condition is having lung disease}. So
  \[P(S | L2) = 0.900\]
\item
  Of those not having lung disease, 25.3\% are smokers. This also indicates a conditional probability and \textbf{the condition is NOT having lung disease}. So
  \[P(S | L1) = 0.253\].
\item
  Lastly, we calculate another marginal probability
\end{itemize}

\[P(L1) = P(\text{not } L2) = 1 - P(L2) = 1 - 0.070 = 0.930\].

\begin{itemize}
\tightlist
\item
  We summarize this information together:
\end{itemize}

\begin{figure}
\centering
\includegraphics{https://i.ibb.co/gjprBV8/205-Table-04-13.png}
\caption{Figure}
\end{figure}

\textbf{Step 3: Apply Bayes' rule to calcuate the conditional probability}

Applying Bayes's rule, we obtain

\[
P(L2 | S)  = \frac{P(L_2)P(S|L_2)}{P(L_1)P(S|L_1)+P(L_2)P(S|L_2)} = \frac{0.070 • 0.900}{0.930 • 0.253 + 0.070 • 0.900}=0.211
\]

The probability is 0.211 that a randomly selected smoker has lung disease.

\textbf{Step 4. Interpretation}

Among smokers, 21.1\% have lung disease.

This shows that the rate of lung disease among smokers (21.1\%) is more than three times the rate among the general population (7.0\%).

Similarly, we can show that the probability that a randomly selected nonsmoker has lung disease is

\[P(L_2|\text{not S})=0.010\]

in other words, 1.0\% of nonsmokers have lung disease.

Please calculate this conditional probability yourself using Bayes' rule. \emph{Hint}: Find out \((\text{not } S|L1)\) and \(P(\text{not } S|L2)\), then calculate the total probability as the denominator in Bayes' rule: \[P(\text{not } S) = P(\text{not } S|L1)P(L1) + P(\text{not } S|L2)P(L2)\]

Hence the rate of lung disease among smokers (21.1\%) is more than 20 times that among nonsmokers (1.0\%). Because this study is observational, however, we can conclude that smoking causes lung disease; we can only infer that a strong positive association exists between smoking and lung disease.

\textbf{Important Note:} We are able to use Bayes's rule in this probablem because \(L1\) and \(L2\) are mutually exclusive and exthaustive.

\hypertarget{discrete-random-variables}{%
\chapter{Discrete Random Variables}\label{discrete-random-variables}}

\begin{Shaded}
\begin{Highlighting}[]
\FunctionTok{library}\NormalTok{(IntroStats)}
\end{Highlighting}
\end{Shaded}

\hypertarget{definitions}{%
\section{Definitions}\label{definitions}}

\hypertarget{random-variable-and-discrete-random-variable}{%
\subsection{Random Variable and Discrete Random Variable}\label{random-variable-and-discrete-random-variable}}

A \textbf{random variable} is a quantitative variable whose value depends on chance.

A \textbf{discrete random variable} is a random variable whose possible values can be listed. In particular, a random variable with only a finite number of possible values is a discrete random variable.

\textbf{Notations:}

\begin{itemize}
\tightlist
\item
  Lower case letters such as \(x, y, z\) to denote variables.
\item
  To represent random variables, we usually use upper case letters such as \(X\), \(Y\), and \(Z\).
\item
  \(\bigl\{X = 2\bigr\}\) denotes the event that the random variable \(X\) equals \(x\).
\item
  \(P(X = x)\) denotes the probability of the random variable \(X\) equals \(x\).
\end{itemize}

\hypertarget{example-number-of-siblings}{%
\subsubsection{Example: Number of Siblings}\label{example-number-of-siblings}}

\begin{table}

\caption{\label{tab:unnamed-chunk-93}Distribution of siblings with frequencies and relative frequencies}
\centering
\begin{tabular}[t]{l|r|r}
\hline
Siblings\_x & Frequency\_f & Relative\_Frequency\\
\hline
0 & 8 & 0.200\\
\hline
1 & 17 & 0.425\\
\hline
2 & 11 & 0.275\\
\hline
3 & 3 & 0.075\\
\hline
4 & 1 & 0.025\\
\hline
Total & 40 & 1.000\\
\hline
\end{tabular}
\end{table}

The table presents each student's number of siblings in a class. For
example, 11 of the 40 students have 2 siblings, which is about 27.5\% of
the total students.

\begin{Shaded}
\begin{Highlighting}[]
\NormalTok{x }\OtherTok{\textless{}{-}} \DecValTok{0}\SpecialCharTok{:}\DecValTok{4}\NormalTok{; f }\OtherTok{\textless{}{-}} \FunctionTok{c}\NormalTok{(}\DecValTok{8}\NormalTok{, }\DecValTok{17}\NormalTok{, }\DecValTok{11}\NormalTok{, }\DecValTok{3}\NormalTok{, }\DecValTok{1}\NormalTok{)}

\NormalTok{rel.freq }\OtherTok{\textless{}{-}}\NormalTok{ f }\SpecialCharTok{/} \FunctionTok{sum}\NormalTok{(f)}

\CommentTok{\#y \textless{}{-} c(rep(x[1],f[1]), rep(x[2],f[2]), rep(x[3],f[3]), rep(x[4],f[4]), rep(x[5],f[5]))}

\NormalTok{bp }\OtherTok{\textless{}{-}} \FunctionTok{barplot}\NormalTok{(rel.freq, }\AttributeTok{col=}\StringTok{"turquoise"}\NormalTok{, }\AttributeTok{xlab=}\StringTok{"Number of siblings"}\NormalTok{, }\AttributeTok{ylab=}\StringTok{"Relative Frequency"}\NormalTok{)}
\FunctionTok{mtext}\NormalTok{(}\AttributeTok{side =} \DecValTok{1}\NormalTok{, }\AttributeTok{at =}\NormalTok{ bp, }\AttributeTok{line =} \DecValTok{0}\NormalTok{, }\AttributeTok{text =}\NormalTok{ x)}
\end{Highlighting}
\end{Shaded}

\includegraphics{StatsTB_files/figure-latex/unnamed-chunk-94-1.pdf}

Number of siblings is a variable because it varies from one student to
another. Suppose now a student is randomly selected, then the number of
siblings of the student is called a random variable because its value
depends on chance.

\textbf{Describe an example to tell the difference between variable and random
variable.}

\hypertarget{probability-distributions-and-probability-histogram}{%
\subsection{Probability Distributions and Probability Histogram}\label{probability-distributions-and-probability-histogram}}

\textbf{Probability distribution}: A listing of the possible values and corresponding probabilities of a discrete random variable, or a formula for the probabilities.

\textbf{Probability histogram:} A graph of the probability distribution that displays the possible values of a discrete random variable on the horizontal axis and the probabilities of those values on the vertical axis. The probability of each value is represented by a vertical bar whose height equals the probability.

\hypertarget{example-29}{%
\subsubsection{Example}\label{example-29}}

\begin{figure}
\centering
\includegraphics{https://i.ibb.co/KVLVjmn/224-Table-05-01.png}
\caption{Figure Number of Siblings}
\end{figure}

Let \(X\) denote the number of siblings of a randomly selected student.

\begin{itemize}
\item
  \begin{enumerate}
  \def\labelenumi{(\alph{enumi})}
  \tightlist
  \item
    Determine the probability distribution of \(X\)
  \end{enumerate}
\item
  \begin{enumerate}
  \def\labelenumi{(\alph{enumi})}
  \setcounter{enumi}{1}
  \tightlist
  \item
    Construct a probability histogram for \(X\)
  \end{enumerate}
\end{itemize}

\textbf{Solution}

We need to determine the probability of each of the possible values of the random variable \(X\). Apply the \(f/N\) rule, we find

\[
P(X = 2) = \frac{f}{N} = 11/40 = 0.275
\]

Similarly, we can determine the other probabilities below.

\begin{figure}
\centering
\includegraphics{https://i.ibb.co/KsWjcqG/225-Table-05-02.png}
\caption{Probability Distribution for Number of
Siblings}
\end{figure}

\begin{Shaded}
\begin{Highlighting}[]
\NormalTok{prob }\OtherTok{\textless{}{-}}\NormalTok{ f}\SpecialCharTok{/}\DecValTok{40}
\NormalTok{bp }\OtherTok{\textless{}{-}} \FunctionTok{barplot}\NormalTok{(prob, }\AttributeTok{col=}\StringTok{"turquoise"}\NormalTok{, }\AttributeTok{xlab=}\StringTok{"Number of siblings"}\NormalTok{, }\AttributeTok{ylab=}\StringTok{"Probability"}\NormalTok{)}
\FunctionTok{mtext}\NormalTok{(}\AttributeTok{side =} \DecValTok{1}\NormalTok{, }\AttributeTok{at =}\NormalTok{ bp, }\AttributeTok{line =} \DecValTok{0}\NormalTok{, }\AttributeTok{text =}\NormalTok{ x)}
\end{Highlighting}
\end{Shaded}

\includegraphics{StatsTB_files/figure-latex/unnamed-chunk-95-1.pdf}

For any discrete random variable \(X\), suppose \(X\) has possible values of \(x_1, \cdots, x_k\).

\[
\sum P(X = x_i) = 1
\]

\emph{Think about why?}

\hypertarget{example-30}{%
\subsubsection{Example}\label{example-30}}

Table below shows the frequency distribution for enrollment by grade in public elementary schools in U.S. Frequencies are in thousands of students.

Let \(Y\) denote the grade level of a randomly selected elementary school student.

\begin{figure}
\centering
\includegraphics{https://i.ibb.co/7yxcbTn/226-Table-05-03.png}
\caption{Figure Elementary students by grades}
\end{figure}

\begin{itemize}
\item
  \begin{enumerate}
  \def\labelenumi{(\alph{enumi})}
  \tightlist
  \item
    Represent the event that the selected student is in 5th grade.
  \end{enumerate}
\item
  \begin{enumerate}
  \def\labelenumi{(\alph{enumi})}
  \setcounter{enumi}{1}
  \tightlist
  \item
    Determine \[P(Y = 5)\] and express in percentages.
  \end{enumerate}
\item
  \begin{enumerate}
  \def\labelenumi{(\alph{enumi})}
  \setcounter{enumi}{2}
  \tightlist
  \item
    Determine the probability distribution of \(Y\).
  \end{enumerate}
\end{itemize}

\textbf{Solution:}

(a). Use \(\bigl\{Y = 5\bigr\}\) to represent the event.

(b).
\[
P(Y=5) = \frac{3629}{34169} = 0.106
\]

A randomly selected elementary school student in the U.S. has a probability of 10.6\% are in the 5th grade.

(c). Similarly, we calculate the probability for each of grades to obtain the probability distribution of \(Y\).

\begin{figure}
\centering
\includegraphics{https://i.ibb.co/k4RV2qs/227-Table-05-04.png}
\caption{Figure Elementary students by grades}
\end{figure}

\hypertarget{example-31}{%
\subsubsection{Example}\label{example-31}}

When a balanced dime is tossed three times. Eight equally likely outcomes are possible.

\begin{figure}
\centering
\includegraphics{https://i.ibb.co/Pmrfdk3/227-Table-05-05.png}
\caption{Figure A coin tossed 3 times}
\end{figure}

Let \(X\) denote the number of heads tossed, then all possible values of \(X\) are 0, 1, 2, 3.

\begin{itemize}
\tightlist
\item
  (a). Determine \(P(X=2)\)
\item
  (b). Find the distribution of \(X\)
\item
  (c). Use notation to represent the event that at most two heads are tossed.
\item
  (d). Find \(P(X \le 2)\)
\end{itemize}

\textbf{Solution.}

(a). \[P(X=2) = \frac{3}{8}=0.375\]

(b). Similarly, we can find the distribution of \(X\)

\begin{figure}
\centering
\includegraphics{https://i.ibb.co/sPQymr2/228-Table-05-06.png}
\caption{Figure . A coin tossed 3 times}
\end{figure}

(c). At most two heads are tossed: \(\bigl\{X \le 2\bigr\}\).

(d).
\[
P(X \le 2) = P(X=0)+P(X=1)+P(X=2) = 0.125+0.375+0.375=0.875
\]

There is an 87.5\% chance of obtaining two or fewer heads in 3 tosses.

\hypertarget{simulation-tossing-a-coin}{%
\subsection{Simulation: tossing a coin}\label{simulation-tossing-a-coin}}

We perform a simulation study below, step by step, to enhance our understanding of the probability distribution concept.

\textbf{Step 1.}

\emph{Perform the experiment once: Toss a coin 3 times} Because the coin is balanced, there is 50\% chance to toss a head. The R function \texttt{toss.coin()} is to simulate the process. Try this. Each run represents one experiment of tossing 3 the coin 3 times.

\begin{Shaded}
\begin{Highlighting}[]
\FunctionTok{library}\NormalTok{(IntroStats)}
\FunctionTok{toss.coin}\NormalTok{(}\AttributeTok{n=}\DecValTok{3}\NormalTok{)}
\end{Highlighting}
\end{Shaded}

\begin{verbatim}
## $heads
## [1] 2
## 
## $outcome
## [1] "H" "H" "T"
\end{verbatim}

\textbf{Step 2.}

\emph{Perform the experiment 2 times}

\begin{Shaded}
\begin{Highlighting}[]
\NormalTok{exp1 }\OtherTok{\textless{}{-}} \FunctionTok{toss.coin}\NormalTok{(}\AttributeTok{n=}\DecValTok{3}\NormalTok{)}
\NormalTok{exp1}
\end{Highlighting}
\end{Shaded}

\begin{verbatim}
## $heads
## [1] 1
## 
## $outcome
## [1] "T" "T" "H"
\end{verbatim}

\begin{Shaded}
\begin{Highlighting}[]
\NormalTok{exp2 }\OtherTok{\textless{}{-}} \FunctionTok{toss.coin}\NormalTok{(}\AttributeTok{n=}\DecValTok{3}\NormalTok{)}
\NormalTok{exp2}
\end{Highlighting}
\end{Shaded}

\begin{verbatim}
## $heads
## [1] 2
## 
## $outcome
## [1] "H" "H" "T"
\end{verbatim}

\textbf{Step 3.}

\emph{Perform the experiment 100 times}: draw and bar plot for observed number of heads in 100 experiments.

\begin{Shaded}
\begin{Highlighting}[]
\NormalTok{n.experiments }\OtherTok{\textless{}{-}} \DecValTok{100}
\NormalTok{n.heads }\OtherTok{=} \FunctionTok{rep}\NormalTok{(}\ConstantTok{NA}\NormalTok{, n.experiments)}

\CommentTok{\#conduct 100 experiments}
\ControlFlowTok{for}\NormalTok{ (i }\ControlFlowTok{in} \DecValTok{1}\SpecialCharTok{:}\NormalTok{n.experiments)\{}
\NormalTok{    n.heads[i] }\OtherTok{\textless{}{-}} \FunctionTok{toss.coin}\NormalTok{(}\AttributeTok{n=}\DecValTok{3}\NormalTok{)}\SpecialCharTok{$}\NormalTok{heads}
\NormalTok{\}}

\CommentTok{\#summary by frequency}
\NormalTok{s }\OtherTok{\textless{}{-}} \FunctionTok{table}\NormalTok{(n.heads)}
\NormalTok{s}
\end{Highlighting}
\end{Shaded}

\begin{verbatim}
## n.heads
##  0  1  2  3 
## 17 43 29 11
\end{verbatim}

\begin{Shaded}
\begin{Highlighting}[]
\CommentTok{\#relative frequency}
\NormalTok{relative.freq }\OtherTok{=}\NormalTok{ s}\SpecialCharTok{/}\DecValTok{100}
\NormalTok{relative.freq}
\end{Highlighting}
\end{Shaded}

\begin{verbatim}
## n.heads
##    0    1    2    3 
## 0.17 0.43 0.29 0.11
\end{verbatim}

\begin{Shaded}
\begin{Highlighting}[]
\NormalTok{bp }\OtherTok{\textless{}{-}} \FunctionTok{barplot}\NormalTok{(s, }\AttributeTok{col=}\StringTok{"turquoise"}\NormalTok{, }\AttributeTok{xlab=}\StringTok{"Number of Heads in 3 tosses for 100 experiments"}\NormalTok{, }\AttributeTok{ylab=}\StringTok{"Frequency"}\NormalTok{)}
\FunctionTok{mtext}\NormalTok{(}\AttributeTok{side =} \DecValTok{1}\NormalTok{, }\AttributeTok{at =}\NormalTok{ bp, }\AttributeTok{line =} \DecValTok{0}\NormalTok{, }\AttributeTok{text =}\NormalTok{ s)}
\end{Highlighting}
\end{Shaded}

\includegraphics{StatsTB_files/figure-latex/unnamed-chunk-98-1.pdf}

\begin{Shaded}
\begin{Highlighting}[]
\FunctionTok{barplot}\NormalTok{(relative.freq, }\AttributeTok{col=}\StringTok{"turquoise"}\NormalTok{, }\AttributeTok{xlab=}\StringTok{"Number of Heads in 3 tosses for 100 experiments"}\NormalTok{, }\AttributeTok{ylab=}\StringTok{"Probability"}\NormalTok{)}
\FunctionTok{mtext}\NormalTok{(}\AttributeTok{side =} \DecValTok{1}\NormalTok{, }\AttributeTok{at =}\NormalTok{ bp, }\AttributeTok{line =} \DecValTok{0}\NormalTok{, }\AttributeTok{text =}\NormalTok{ s)}
\end{Highlighting}
\end{Shaded}

\includegraphics{StatsTB_files/figure-latex/unnamed-chunk-98-2.pdf}

\textbf{Step 4.}

Compare to the theoretical calculation:

\begin{Shaded}
\begin{Highlighting}[]
\NormalTok{p }\OtherTok{\textless{}{-}} \FunctionTok{c}\NormalTok{(}\FloatTok{0.125}\NormalTok{, }\FloatTok{0.375}\NormalTok{, }\FloatTok{0.375}\NormalTok{, }\FloatTok{0.125}\NormalTok{)}
\NormalTok{p}
\end{Highlighting}
\end{Shaded}

\begin{verbatim}
## [1] 0.125 0.375 0.375 0.125
\end{verbatim}

\begin{Shaded}
\begin{Highlighting}[]
\NormalTok{bp }\OtherTok{\textless{}{-}} \FunctionTok{barplot}\NormalTok{(p, }\AttributeTok{col=}\StringTok{"green"}\NormalTok{, }\AttributeTok{xlab=}\StringTok{"Probability of Number of Heads in Theory"}\NormalTok{, }\AttributeTok{ylab=}\StringTok{"Probability"}\NormalTok{)}
\FunctionTok{mtext}\NormalTok{(}\AttributeTok{side =} \DecValTok{1}\NormalTok{, }\AttributeTok{at =}\NormalTok{ bp, }\AttributeTok{line =} \DecValTok{0}\NormalTok{, }\AttributeTok{text =} \DecValTok{0}\SpecialCharTok{:}\DecValTok{3}\NormalTok{)}
\end{Highlighting}
\end{Shaded}

\includegraphics{StatsTB_files/figure-latex/unnamed-chunk-99-1.pdf}

\textbf{Step 5.}

\emph{Perform the experiment 1000 times}: draw and bar plot for observed number of heads in 1000 experiments.

\begin{Shaded}
\begin{Highlighting}[]
\NormalTok{n.experiments }\OtherTok{\textless{}{-}} \DecValTok{1000}
\NormalTok{n.heads }\OtherTok{=} \FunctionTok{rep}\NormalTok{(}\ConstantTok{NA}\NormalTok{, n.experiments)}

\CommentTok{\#conduct 100 experiments}
\ControlFlowTok{for}\NormalTok{ (i }\ControlFlowTok{in} \DecValTok{1}\SpecialCharTok{:}\NormalTok{n.experiments)\{}
\NormalTok{    n.heads[i] }\OtherTok{\textless{}{-}} \FunctionTok{toss.coin}\NormalTok{(}\AttributeTok{n=}\DecValTok{3}\NormalTok{)}\SpecialCharTok{$}\NormalTok{heads}
\NormalTok{\}}

\CommentTok{\#summary by frequency}
\NormalTok{s }\OtherTok{\textless{}{-}} \FunctionTok{table}\NormalTok{(n.heads)}
\NormalTok{s}
\end{Highlighting}
\end{Shaded}

\begin{verbatim}
## n.heads
##   0   1   2   3 
## 132 392 355 121
\end{verbatim}

\begin{Shaded}
\begin{Highlighting}[]
\CommentTok{\#relative frequency}
\NormalTok{relative.freq }\OtherTok{=}\NormalTok{ s}\SpecialCharTok{/}\DecValTok{1000}
\NormalTok{relative.freq}
\end{Highlighting}
\end{Shaded}

\begin{verbatim}
## n.heads
##     0     1     2     3 
## 0.132 0.392 0.355 0.121
\end{verbatim}

\begin{Shaded}
\begin{Highlighting}[]
\NormalTok{bp }\OtherTok{\textless{}{-}} \FunctionTok{barplot}\NormalTok{(s, }\AttributeTok{col=}\StringTok{"turquoise"}\NormalTok{, }\AttributeTok{xlab=}\StringTok{"Number of Heads in 3 tosses for 1000 experiments"}\NormalTok{, }\AttributeTok{ylab=}\StringTok{"Frequency"}\NormalTok{)}
\FunctionTok{mtext}\NormalTok{(}\AttributeTok{side =} \DecValTok{1}\NormalTok{, }\AttributeTok{at =}\NormalTok{ bp, }\AttributeTok{line =} \DecValTok{0}\NormalTok{, }\AttributeTok{text =}\NormalTok{ s)}
\end{Highlighting}
\end{Shaded}

\includegraphics{StatsTB_files/figure-latex/unnamed-chunk-100-1.pdf}

\begin{Shaded}
\begin{Highlighting}[]
\FunctionTok{barplot}\NormalTok{(relative.freq, }\AttributeTok{col=}\StringTok{"turquoise"}\NormalTok{, }\AttributeTok{xlab=}\StringTok{"Number of Heads in 3 tosses for 1000 experiments"}\NormalTok{, }\AttributeTok{ylab=}\StringTok{"Probability"}\NormalTok{)}
\FunctionTok{mtext}\NormalTok{(}\AttributeTok{side =} \DecValTok{1}\NormalTok{, }\AttributeTok{at =}\NormalTok{ bp, }\AttributeTok{line =} \DecValTok{0}\NormalTok{, }\AttributeTok{text =}\NormalTok{ s)}
\end{Highlighting}
\end{Shaded}

\includegraphics{StatsTB_files/figure-latex/unnamed-chunk-100-2.pdf}

\textbf{Step 6.}

\emph{Perform the experiment 10000 times}: draw and bar plot for observed
number of heads in 1000 experiments.

\begin{Shaded}
\begin{Highlighting}[]
\NormalTok{n.experiments }\OtherTok{\textless{}{-}} \DecValTok{10000}
\NormalTok{n.heads }\OtherTok{=} \FunctionTok{rep}\NormalTok{(}\ConstantTok{NA}\NormalTok{, n.experiments)}

\CommentTok{\#conduct 100 experiments}
\ControlFlowTok{for}\NormalTok{ (i }\ControlFlowTok{in} \DecValTok{1}\SpecialCharTok{:}\NormalTok{n.experiments)\{}
\NormalTok{    n.heads[i] }\OtherTok{\textless{}{-}} \FunctionTok{toss.coin}\NormalTok{(}\AttributeTok{n=}\DecValTok{3}\NormalTok{)}\SpecialCharTok{$}\NormalTok{heads}
\NormalTok{\}}

\CommentTok{\#summary by frequency}
\NormalTok{s }\OtherTok{\textless{}{-}} \FunctionTok{table}\NormalTok{(n.heads)}
\NormalTok{s}
\end{Highlighting}
\end{Shaded}

\begin{verbatim}
## n.heads
##    0    1    2    3 
## 1238 3768 3749 1245
\end{verbatim}

\begin{Shaded}
\begin{Highlighting}[]
\CommentTok{\#relative frequency}
\NormalTok{relative.freq }\OtherTok{=}\NormalTok{ s}\SpecialCharTok{/}\DecValTok{10000}
\NormalTok{relative.freq}
\end{Highlighting}
\end{Shaded}

\begin{verbatim}
## n.heads
##      0      1      2      3 
## 0.1238 0.3768 0.3749 0.1245
\end{verbatim}

\begin{Shaded}
\begin{Highlighting}[]
\NormalTok{bp }\OtherTok{\textless{}{-}} \FunctionTok{barplot}\NormalTok{(s, }\AttributeTok{col=}\StringTok{"turquoise"}\NormalTok{, }\AttributeTok{xlab=}\StringTok{"Number of Heads in 3 tosses for 10000 experiments"}\NormalTok{, }\AttributeTok{ylab=}\StringTok{"Frequency"}\NormalTok{)}
\FunctionTok{mtext}\NormalTok{(}\AttributeTok{side =} \DecValTok{1}\NormalTok{, }\AttributeTok{at =}\NormalTok{ bp, }\AttributeTok{line =} \DecValTok{0}\NormalTok{, }\AttributeTok{text =}\NormalTok{ s)}
\end{Highlighting}
\end{Shaded}

\includegraphics{StatsTB_files/figure-latex/unnamed-chunk-101-1.pdf}

\begin{Shaded}
\begin{Highlighting}[]
\FunctionTok{barplot}\NormalTok{(relative.freq, }\AttributeTok{col=}\StringTok{"turquoise"}\NormalTok{, }\AttributeTok{xlab=}\StringTok{"Number of Heads in 3 tosses for 10000 experiments"}\NormalTok{, }\AttributeTok{ylab=}\StringTok{"Probability"}\NormalTok{)}
\FunctionTok{mtext}\NormalTok{(}\AttributeTok{side =} \DecValTok{1}\NormalTok{, }\AttributeTok{at =}\NormalTok{ bp, }\AttributeTok{line =} \DecValTok{0}\NormalTok{, }\AttributeTok{text =}\NormalTok{ s)}
\end{Highlighting}
\end{Shaded}

\includegraphics{StatsTB_files/figure-latex/unnamed-chunk-101-2.pdf}

\textbf{What are your conclusions from this simulation study?}

\hypertarget{interpretation-of-probability-distribution}{%
\subsection{Interpretation of Probability Distribution}\label{interpretation-of-probability-distribution}}

In a large number of independent observations of a random variable \(X\), the proportion of times each possible value occurs will approximate the probability distribution of \(X\); or, equivalently, the proportion histogram will approximate the probability histogram for \(X\).

\hypertarget{the-mean-and-sd-of-a-discrete-random-variable}{%
\section{The Mean and SD of a Discrete Random Variable}\label{the-mean-and-sd-of-a-discrete-random-variable}}

\hypertarget{definition-mean-of-a-discrete-random-variable}{%
\subsection{Definition: Mean of a Discrete Random Variable}\label{definition-mean-of-a-discrete-random-variable}}

For a variable \(x\), the mean of all possible observations for the entire
population is called the \textbf{population mean}, and calculated as

\[
\mu=\frac{\sum x_i}{N}
\]

\hypertarget{example-32}{%
\subsection{Example}\label{example-32}}

Consider a population of 8 students whose ages are 19, 20, 20, 19, 21,
27, 20, 21. Let \(X\) denote the age of a randomly selected student. What
is the mean of \(X\)? What is the probability distribution of \(X\)?

\begin{Shaded}
\begin{Highlighting}[]
\NormalTok{x }\OtherTok{\textless{}{-}}  \FunctionTok{c}\NormalTok{(}\DecValTok{19}\NormalTok{, }\DecValTok{20}\NormalTok{, }\DecValTok{20}\NormalTok{, }\DecValTok{19}\NormalTok{, }\DecValTok{21}\NormalTok{, }\DecValTok{27}\NormalTok{, }\DecValTok{20}\NormalTok{, }\DecValTok{21}\NormalTok{) }\CommentTok{\#Ages of the entire population}

\CommentTok{\#Population mean}
\NormalTok{mu }\OtherTok{\textless{}{-}} \FunctionTok{mean}\NormalTok{(x)}
\NormalTok{mu}
\end{Highlighting}
\end{Shaded}

\begin{verbatim}
## [1] 20.875
\end{verbatim}

\begin{Shaded}
\begin{Highlighting}[]
\CommentTok{\#frequency summary}
\NormalTok{s }\OtherTok{\textless{}{-}} \FunctionTok{table}\NormalTok{(x)}
\NormalTok{s}
\end{Highlighting}
\end{Shaded}

\begin{verbatim}
## x
## 19 20 21 27 
##  2  3  2  1
\end{verbatim}

\begin{Shaded}
\begin{Highlighting}[]
\NormalTok{N }\OtherTok{\textless{}{-}} \FunctionTok{length}\NormalTok{(x)}
\CommentTok{\#Probability distribution}
\NormalTok{p }\OtherTok{\textless{}{-}}\NormalTok{ s }\SpecialCharTok{/}\NormalTok{ N}
\NormalTok{p}
\end{Highlighting}
\end{Shaded}

\begin{verbatim}
## x
##    19    20    21    27 
## 0.250 0.375 0.250 0.125
\end{verbatim}

Let's reconsider the calculation of population mean by frequency.

\begin{itemize}
\item
  2 students: age 19
\item
  3 students: age 20
\item
  2 students: age 21
\item
  1 student: age 27
\item
  Population mean = sum of all students ages / total number of
  students
\end{itemize}

\[
\mu = \frac{2\times 19 + 3\times 20 + 2\times 21 + 1\times 27}{8} = 19\cdot \frac{2}{8}+20\cdot \frac{3}{8}+21\cdot \frac{2}{8}+27\cdot \frac{1}{8}
\]

\[
\mu = 19P(X=19)+20P(X=20)+21P(X=21)+27P(X=27) = \sum x_iP(X=x_i)
\]

\hypertarget{definition-mean-of-a-discrete-random-variable-1}{%
\subsection{Definition: Mean of a discrete random variable}\label{definition-mean-of-a-discrete-random-variable-1}}

The \textbf{mean of a discrete random variable} \(X\) is denoted \(\mu_X\), or
when no confusion will arise, simply \(\mu\). It is defined by \[
\mu = \sum xP(X=x)
\] The terms \textbf{expected value} and \textbf{expectation} are commonly used in
place of the term \textbf{mean}.

\hypertarget{example-33}{%
\subsection{Example}\label{example-33}}

Consider \(X\) as random variable that denotes the number of tellers busy
with customers at 1:00pm in a bank branch. Suppose the probability
distribution of \(X\), i.e., \(P(X = x)\), is

\begin{table}

\caption{\label{tab:unnamed-chunk-103}Probability distribution with expected value}
\centering
\begin{tabular}[t]{l|l|r}
\hline
x & P\_X\_eq\_x & xP\_X\_eq\_x\\
\hline
0 & 0.029 & 0.000\\
\hline
1 & 0.049 & 0.049\\
\hline
2 & 0.078 & 0.156\\
\hline
3 & 0.155 & 0.465\\
\hline
4 & 0.212 & 0.848\\
\hline
5 & 0.262 & 1.310\\
\hline
6 & 0.215 & 1.290\\
\hline
 &  & 4.118\\
\hline
\end{tabular}
\end{table}

Find the expected number of tellers servicing customers at 1:00pm, i.e.,
the mean of \(X\).

\textbf{Solution:}

\[
\mu = \sum xP(X=x) = 0\times 0.029+1\times 0.049+\cdots+6\times 0.215 = 4.118
\]

The mean number of tellers servicing customers at 1:00pm is 4.118.

\hypertarget{interpretation-of-the-mean-of-a-random-variable}{%
\subsection{Interpretation of the Mean of a Random Variable}\label{interpretation-of-the-mean-of-a-random-variable}}

In a large number of independent observations of a random variable \(X\),
the average value of those observations will approximately equal the
mean, \(\mu\) of \(X\). The larger the number of observations, the closer
the average tends to be to \(\mu\).

Graphs showing the average number of busy tellers versus the number of
observations for two simulations of 100 observations each

\begin{figure}
\centering
\includegraphics{https://i.ibb.co/4WZHBQL/234-Figure-05-03.png}
\caption{Figure . Tellers}
\end{figure}

\hypertarget{standard-deviation-of-a-discrete-random-variable}{%
\subsection{Standard deviation of a discrete random variable}\label{standard-deviation-of-a-discrete-random-variable}}

The standard deviation of a discrete random variable \(X\) is denoted
\(\sigma_X\) or when no confusion will arise, simply \(\sigma\). It is
defined as
\[
\sigma = \sqrt{\sum (x-\mu)^2P(X=x)}
\]

The standard deviation of a discrete random variable can also be obtained from the computing formula

\[
\sigma = \sqrt{\sum x^2P(X=x)-\mu^2}
\]
\#\#\#\# Example

For the teller example, following the formula,

\begin{table}

\caption{\label{tab:unnamed-chunk-104}Probability distribution with squared terms and variance calculation}
\centering
\begin{tabular}[t]{l|l|l|r}
\hline
x & P\_X\_eq\_x & x\_squared & x2P\_X\_eq\_x\\
\hline
0 & 0.029 & 0 & 0.000\\
\hline
1 & 0.049 & 1 & 0.049\\
\hline
2 & 0.078 & 4 & 0.312\\
\hline
3 & 0.155 & 9 & 1.395\\
\hline
4 & 0.212 & 16 & 3.392\\
\hline
5 & 0.262 & 25 & 6.550\\
\hline
6 & 0.215 & 36 & 7.740\\
\hline
 &  &  & 19.438\\
\hline
\end{tabular}
\end{table}

\[
\sigma = \sqrt{\sum x^2P(X=x)-\mu^2}=\sqrt{19.438-4.118^2} = 1.6
\]

\hypertarget{example-expected-side-effects-in-a-cancer-drug-trial}{%
\subsubsection{Example: Expected Side Effects in a Cancer Drug Trial}\label{example-expected-side-effects-in-a-cancer-drug-trial}}

Consider \(X\) as a random variable that denotes the \textbf{number of moderate side effects} experienced by a patient during the course of a new cancer drug trial. Based on data from previous trials, the probability distribution of \(X\), i.e., \(P(X = x)\), is given by the table below:

\begin{longtable}[]{@{}ll@{}}
\toprule\noalign{}
\(x\) (Number of Side Effects) & \(P(X = x)\) \\
\midrule\noalign{}
\endhead
\bottomrule\noalign{}
\endlastfoot
0 & 0.10 \\
1 & 0.20 \\
2 & 0.35 \\
3 & 0.25 \\
4 & 0.10 \\
\end{longtable}

We will compute the \textbf{expected value} (mean) and the \textbf{standard deviation} of \(X\) to understand the average burden of side effects and its variability across patients.

\begin{Shaded}
\begin{Highlighting}[]
\CommentTok{\# Define x values and corresponding probabilities}
\NormalTok{x }\OtherTok{\textless{}{-}} \DecValTok{0}\SpecialCharTok{:}\DecValTok{4}
\NormalTok{p }\OtherTok{\textless{}{-}} \FunctionTok{c}\NormalTok{(}\FloatTok{0.10}\NormalTok{, }\FloatTok{0.20}\NormalTok{, }\FloatTok{0.35}\NormalTok{, }\FloatTok{0.25}\NormalTok{, }\FloatTok{0.10}\NormalTok{)}

\CommentTok{\# Calculate expected value (mean)}
\NormalTok{mu }\OtherTok{\textless{}{-}} \FunctionTok{sum}\NormalTok{(x }\SpecialCharTok{*}\NormalTok{ p)}
\CommentTok{\# Calculate variance using computing formula}
\NormalTok{variance }\OtherTok{\textless{}{-}} \FunctionTok{sum}\NormalTok{(x}\SpecialCharTok{\^{}}\DecValTok{2} \SpecialCharTok{*}\NormalTok{ p) }\SpecialCharTok{{-}}\NormalTok{ mu}\SpecialCharTok{\^{}}\DecValTok{2}
\CommentTok{\# Standard deviation}
\NormalTok{sigma }\OtherTok{\textless{}{-}} \FunctionTok{sqrt}\NormalTok{(variance)}

\CommentTok{\# Output results}
\NormalTok{mu}
\end{Highlighting}
\end{Shaded}

\begin{verbatim}
## [1] 2.05
\end{verbatim}

\begin{Shaded}
\begin{Highlighting}[]
\NormalTok{sigma}
\end{Highlighting}
\end{Shaded}

\begin{verbatim}
## [1] 1.116915
\end{verbatim}

\textbf{Interpretation}

The expected number of moderate side effects per patient is
\(\mu = 2.05\), which means that, on average, a patient is likely to experience approximately 2 moderate side effects during the trial.

The standard deviation is \(\sigma = 1.12\), indicating a moderate level of variability around the mean. This suggests that while many patients may experience around two side effects, others may experience significantly fewer or more---highlighting differences in individual tolerance to the drug.

\hypertarget{binomial-distribution}{%
\section{Binomial Distribution}\label{binomial-distribution}}

The total population of New Jersey is estimated at 8,938,175 people with 4,361,952 male and 4,576,223 female. The New Jersey Gender Ratio is 95 men to 100 women (95:100) or 0.95. New Jersey's gender ratio is lower than the national average of 97 men to 100 women (97:100) or 0.97. Source: States101.com.

When randomly pick a person in New Jersey, what's the probability that the person is female? It is \(p = 4576223 / 8938175 \approx 0.5120\), which is the population probability of female. Let \(X\) as the random variable of gender being female for a person in New Jersey (0 = male, 1 = female), then \(X\sim Bernoulli(p)\) Bernoulli distribution with parameter \(p\).

The mean of \(X\) is \(\mu_X = E[X] = 0(1-p) + 1(p) = p\) and variance is \(\sigma_X^2 = E[(X-\mu)^2] = (0-p)^2(1-p) + (1-p)^2p = p(1-p)\).

It's intuitively understandable that ``on-average'' a random person as female is \(p\). However, the variance is not that intuitive. \(\sigma_X^2 = 0.2499\). For Bernoulli(\(p\)) distribution, among all possible \(0\le p\le 1\), the largest variance occurs when \(p=0.5\). This observation is helpful when estimating sample size for experiments with binary outcome.

Now consider the scenario that we randomly pick 20 people from New Jersey, what is the probability of having exactly 2 females?

Denote \(X_1\) as the outcome of the first person being female (\(X_1=0\) or 1), and similarly define \(X_2, \cdots X_{20}\). Then
\[
X^* = X_1 + \cdots, X_{20}
\]
is the random variable as the number of females among the 20 random people. The problem is to find out \(P(X^* = 2)\).

The possible values of \(X^*\) are \((0, 1, \cdots, 20)\). The distribution of \(X^*\) follows a \textbf{binomial distribution}, which is the sum of multiple independent Bernoulli random variables.

How do we calculate \(P(X^* = x)\)?

\hypertarget{definition-factorials}{%
\subsection{Definition: factorials}\label{definition-factorials}}

The product of the first \(K\) positive integers (counting numbers) is
called \(k\) factorial and is denoted \(k!\). In symbols,
\[
k! = k(k-1)\cdots 2\cdot 1
\]
We also define \(0!=1\).

\textbf{Example}

Find 3!, 4! and 5!.

\[
3! = 3\times 2\times 1 = 6
\]
\[
4! = 4\times 3\times 2\times 1 = 24
\]

\[
5! = 5\times 4\times 3\times 2\times 1 = 120
\]

\hypertarget{definition-binomial-coefficient}{%
\subsection{Definition: Binomial coefficient}\label{definition-binomial-coefficient}}

If n is a positive integer and \(x\) is a non-negative integer less than or
equal to n, then the binomial coefficient \(\binom{n}{x}\) is defined as
\[
\binom{n}{x} = \frac{n!}{x!(n-x)!}
\]

This is the number of ways to pick \(x\) items without considering order
from a total of \(n\) items.

\hypertarget{example-34}{%
\subsubsection{Example}\label{example-34}}

Calculate

\[
\binom{6}{1} = \frac{6!}{1!5!}=6
\]
\[
\binom{5}{3} = \frac{5!}{3!2!}=10
\]

\[
\binom{4}{4} = \frac{4!}{4!0!}=1
\]

\hypertarget{definition-bernoulli-trials}{%
\subsection{Definition: Bernoulli Trials}\label{definition-bernoulli-trials}}

Repeated trials of an experiment are called Bernoulli trials if the following three conditions are satisfied:

\begin{itemize}
\item
  The experiment (each trial) has two possible outcomes, denoted generically \(s\), for success, and \(f\), for failure.
\item
  The trials are independent, meaning that the outcome on one trial in no way affects the outcome on other trials.
\item
  The probability of a success, called the success probability and denoted \(p\), remains the same from trial to trial.
\end{itemize}

\hypertarget{definition-binomial-distribution}{%
\subsection{Definition: Binomial distribution}\label{definition-binomial-distribution}}

Binomial distribution is the probability distribution for the \textbf{number of successes} in a sequence of Bernoulli trials.

\hypertarget{example-35}{%
\subsubsection{Example}\label{example-35}}

Reconsider the example of tossing 3 coins in a sequence. How do we calculate the probability of observing 2 heads, i.e., \(P(X=2)=?\)

There are 3 outcomes to observe 2 heads: HHT, THH, and HTH. This is equivalent to pick 2 tosses and assign to H. The total number of ways to assign 2 heads are exactly the binomial coefficient:
\[
\binom{3}{2}=3
\]

\hypertarget{number-of-outcomes-containing-a-specified-number-of-successes}{%
\subsection{Number of Outcomes Containing a Specified Number of Successes}\label{number-of-outcomes-containing-a-specified-number-of-successes}}

In \(n\) Bernoulli trials, the number of outcomes that contain exactly \(x\) successes equals the binomial coefficient
\[
\binom{n}{x}
\]

Each toss has 0.5 probability to observe a head, and 0.5 probability to observe a tail. So for each outcome of 2 heads, the probability is \(0.5^2(1-0.5)^1\). There are
\[
\binom{3}{2}=3
\]
such outcomes, so the probability of observing 2 heads is exactly
\[
P(X=2) = \binom{3}{2}0.5^2(1-0.5)^1 = 0.375
\]

\hypertarget{binomial-probability-formula}{%
\subsection{Binomial Probability Formula}\label{binomial-probability-formula}}

Let \(X\) denote the total number of successes in \(n\) Bernoulli trials with success probability \(p\). Then the probability distribution of the random variable \(X\) is given by
\[
P(X=x) = \binom{n}{x}p^x(1-p)^{n-x}
\]
for \(x=0, 1, \cdots, n\).

The random variable \(X\) is called a binomial random variable and is said to have the binomial distribution with parameters \(n\) and \(p\).

\hypertarget{binomial-probability-formula-1}{%
\subsection{Binomial Probability Formula}\label{binomial-probability-formula-1}}

\begin{itemize}
\item
  \(n\) trials are to be performed.
\item
  Two outcomes, success or failure, are possible for each trial.
\item
  The trials are independent.
\item
  The success probability, \(p\), remains the same from trial to trial.
\item
  \textbf{Step 1} Identify a success.
\item
  \textbf{Step 2} Determine \(p\), the success probability.
\item
  \textbf{Step 3} Determine \(n\), the number of trials.
\item
  \textbf{Step 4} The binomial probability formula for the number of successes, \(X\), is
  \[
  P(X=x) = \binom{n}{x}p^x(1-p)^{n-x}
  \]
\end{itemize}

\hypertarget{example-binomial-distribution-of-survival-to-age-65}{%
\subsubsection{Example: Binomial Distribution of Survival to Age 65}\label{example-binomial-distribution-of-survival-to-age-65}}

There is roughly an 80\% chance that a person of age 20 years will be alive at age 65 years. Suppose that 3 people of age 20 are randomly selected. Find the probability that the number alive at age 65 years of age:

\begin{enumerate}
\def\labelenumi{(\alph{enumi})}
\tightlist
\item
  exactly 2; (b) at most one; (c) at least 1. (d) determine the probability distribution of the number alive at age 65.
\end{enumerate}

\textbf{Solution:}

Let \(X\) denote the number of people of the three who are alive age 65.

\textbf{Step 1} Identify a success. A success is that a person currently at age 20 will be alive at age 65.

\textbf{Step 2} Determine \(p\), the success probability. The success probability is \(p=0.8\).

\textbf{Step 3} Determine \(n\), the number of trials. \(n=3\) in this example because we randomly select 3 people.

\textbf{Step 4} The binomial probability formula for the number of successes, \(X\)

\[
P(X=x) = \binom{3}{x}0.8^x(0.2)^{3-x}
\]

(a).

\[
P(X=2) = \binom{3}{2}0.8^2(0.2)^{3-2}=0.384
\]

\begin{Shaded}
\begin{Highlighting}[]
\FunctionTok{dbinom}\NormalTok{(}\AttributeTok{x=}\DecValTok{2}\NormalTok{, }\AttributeTok{size=}\DecValTok{3}\NormalTok{, }\AttributeTok{p=}\FloatTok{0.8}\NormalTok{)}
\end{Highlighting}
\end{Shaded}

\begin{verbatim}
## [1] 0.384
\end{verbatim}

(b).

\[
P(X\le 1) = P(X=0) + P(X=1) = \binom{3}{0} 0.8^0(0.2)^{3-0} + \binom{3}{1}0.8^1(0.2)^{3-1}=0.104
\]

\begin{Shaded}
\begin{Highlighting}[]
\FunctionTok{dbinom}\NormalTok{(}\AttributeTok{x=}\DecValTok{0}\NormalTok{, }\AttributeTok{size=}\DecValTok{3}\NormalTok{, }\AttributeTok{p=}\FloatTok{0.8}\NormalTok{)}\SpecialCharTok{+}\FunctionTok{dbinom}\NormalTok{(}\AttributeTok{x=}\DecValTok{1}\NormalTok{, }\AttributeTok{size=}\DecValTok{3}\NormalTok{, }\AttributeTok{p=}\FloatTok{0.8}\NormalTok{)}
\end{Highlighting}
\end{Shaded}

\begin{verbatim}
## [1] 0.104
\end{verbatim}

(c).

\[
P(X\ge 1) = 1-P(X=0)=1-\binom{3}{0}0.8^0(0.2)^{3-0}=0.992
\]

\begin{Shaded}
\begin{Highlighting}[]
\DecValTok{1}\SpecialCharTok{{-}}\FunctionTok{dbinom}\NormalTok{(}\AttributeTok{x=}\DecValTok{0}\NormalTok{, }\AttributeTok{size=}\DecValTok{3}\NormalTok{, }\AttributeTok{p=}\FloatTok{0.8}\NormalTok{)}
\end{Highlighting}
\end{Shaded}

\begin{verbatim}
## [1] 0.992
\end{verbatim}

(d). There are 4 possible outcomes for the number of successes: 0-3. The probability distribution is

\begin{Shaded}
\begin{Highlighting}[]
\FunctionTok{dbinom}\NormalTok{(}\AttributeTok{x=}\DecValTok{0}\SpecialCharTok{:}\DecValTok{3}\NormalTok{, }\AttributeTok{size=}\DecValTok{3}\NormalTok{, }\AttributeTok{p=}\FloatTok{0.8}\NormalTok{)}
\end{Highlighting}
\end{Shaded}

\begin{verbatim}
## [1] 0.008 0.096 0.384 0.512
\end{verbatim}

\hypertarget{mean-and-sd-of-a-binomial-random-variable}{%
\subsection{Mean and SD of a Binomial Random Variable}\label{mean-and-sd-of-a-binomial-random-variable}}

The mean and standard deviation of a binomial random variable with parameters n and p are

\[
\mu = np\\ \sigma = \sqrt{np(1-p)}
\]

\hypertarget{example-36}{%
\subsubsection{Example}\label{example-36}}

Find the mean and sd of \(X\).

\textbf{Solution:}

\[
\mu = np=3\cdot 0.8 = 2.4\\
\sigma = \sqrt{np(1-p)} = \sqrt{3\cdot 0.8 \cdot 0.2} = 0.7
\]

\hypertarget{example-hpv-vaccine-immune-response}{%
\subsubsection{Example: HPV Vaccine Immune Response}\label{example-hpv-vaccine-immune-response}}

In a clinical study, each participant receives a series of \textbf{3 doses} of the HPV vaccine. According to previous research, each dose independently has an \textbf{85\% chance} of generating a measurable immune response.

Let the number of successful immune responses \(X\) follow a \textbf{binomial distribution}:

\[
X \sim (n = 3, p = 0.85)
\]

We apply the formulas to compute:

\begin{itemize}
\tightlist
\item
  the \textbf{mean}: \(\mu = np\)
\item
  the \textbf{standard deviation}: \(\sigma = \sqrt{np(1 - p)}\)
\end{itemize}

\begin{Shaded}
\begin{Highlighting}[]
\CommentTok{\# Parameters}
\NormalTok{n }\OtherTok{\textless{}{-}} \DecValTok{3}        \CommentTok{\# number of trials (vaccine doses)}
\NormalTok{p }\OtherTok{\textless{}{-}} \FloatTok{0.85}     \CommentTok{\# probability of success per dose}

\CommentTok{\# Mean and standard deviation}
\NormalTok{mu }\OtherTok{\textless{}{-}}\NormalTok{ n }\SpecialCharTok{*}\NormalTok{ p}
\NormalTok{sigma }\OtherTok{\textless{}{-}} \FunctionTok{sqrt}\NormalTok{(n }\SpecialCharTok{*}\NormalTok{ p }\SpecialCharTok{*}\NormalTok{ (}\DecValTok{1} \SpecialCharTok{{-}}\NormalTok{ p))}

\CommentTok{\# Output results}
\NormalTok{mu}
\end{Highlighting}
\end{Shaded}

\begin{verbatim}
## [1] 2.55
\end{verbatim}

\begin{Shaded}
\begin{Highlighting}[]
\NormalTok{sigma}
\end{Highlighting}
\end{Shaded}

\begin{verbatim}
## [1] 0.6184658
\end{verbatim}

\textbf{Interpretation}

On average, a participant is expected to exhibit immune response to approximately 2.55 doses, meaning most individuals will respond positively to 2 or all 3 doses of the HPV vaccine.

The standard deviation \(\sigma = 0.63\) reflects moderate variability, suggesting that while the response rate is high, a small number of participants may have only 1 or even 0 successful responses. Understanding this distribution helps evaluate the consistency and effectiveness of the vaccine in a population.

\hypertarget{the-poisson-distribution}{%
\section{The Poisson Distribution}\label{the-poisson-distribution}}

\hypertarget{poisson-probability-formula}{%
\subsection{Poisson probability formula}\label{poisson-probability-formula}}

Probabilities for a random variable \(X\) that has a Poisson distribution are given by the formula
\[
P(X= x) = e^{-\lambda}\cdot\frac{\lambda^x}{x!}
\]
where \(\lambda\) is a positive real number and The random variable \(X\) is called a \textbf{Poisson random variable} and is said to have the Poisson distribution with parameter \(\lambda\).

\hypertarget{example-er-patient-arrival}{%
\subsubsection{Example: ER Patient Arrival}\label{example-er-patient-arrival}}

The number of patients arriving one ER department at 6:00pm-7:00pm has a Poisson distribution with parameter \(\lambda = 6.9\). Determine the probability for the number of patients arriving at the ER at 6:00-7:00pm for

\begin{enumerate}
\def\labelenumi{(\alph{enumi})}
\tightlist
\item
  exactly 4; (b) at most 2; (c) between 4 and 10 inclusive; (d) Probability distribution for \(X\).
\end{enumerate}

\textbf{Solutions:}

(a).

\[
P(X=4) = e^{-6.9}\frac{6.9^4}{4!} = 0.095
\]

\begin{Shaded}
\begin{Highlighting}[]
\FunctionTok{dpois}\NormalTok{(}\AttributeTok{x=}\DecValTok{4}\NormalTok{, }\AttributeTok{lambda=}\FloatTok{6.9}\NormalTok{)}
\end{Highlighting}
\end{Shaded}

\begin{verbatim}
## [1] 0.09518164
\end{verbatim}

(b).

\[
P(X \le 2) = P(X=0)+P(X=1)+P(X=2) = e^{-6.9}\left(\frac{6.9^0}{0!}+\frac{6.9^1}{1!}+\frac{6.9^2}{2!}\right)
\]

(c).

\[
P(4\le X \le 10) = P(X=4)+P(X=5)+\cdots+P(X=10) \\ =e^{-6.9}\left(\frac{6.9^4}{4!}+\frac{6.9^5}{5!}+\cdots+\frac{6.9^10}{10!}\right)\\
=0.821
\]

\begin{Shaded}
\begin{Highlighting}[]
\FunctionTok{dpois}\NormalTok{(}\AttributeTok{x=}\DecValTok{4}\NormalTok{, }\AttributeTok{lambda=}\FloatTok{6.9}\NormalTok{)}\SpecialCharTok{+}\FunctionTok{dpois}\NormalTok{(}\AttributeTok{x=}\DecValTok{5}\NormalTok{, }\AttributeTok{lambda=}\FloatTok{6.9}\NormalTok{)}\SpecialCharTok{+}\FunctionTok{dpois}\NormalTok{(}\AttributeTok{x=}\DecValTok{6}\NormalTok{, }\AttributeTok{lambda=}\FloatTok{6.9}\NormalTok{)}\SpecialCharTok{+}\FunctionTok{dpois}\NormalTok{(}\AttributeTok{x=}\DecValTok{7}\NormalTok{, }\AttributeTok{lambda=}\FloatTok{6.9}\NormalTok{)}\SpecialCharTok{+}\FunctionTok{dpois}\NormalTok{(}\AttributeTok{x=}\DecValTok{8}\NormalTok{, }\AttributeTok{lambda=}\FloatTok{6.9}\NormalTok{)}\SpecialCharTok{+}\FunctionTok{dpois}\NormalTok{(}\AttributeTok{x=}\DecValTok{9}\NormalTok{, }\AttributeTok{lambda=}\FloatTok{6.9}\NormalTok{)}\SpecialCharTok{+}\FunctionTok{dpois}\NormalTok{(}\AttributeTok{x=}\DecValTok{10}\NormalTok{, }\AttributeTok{lambda=}\FloatTok{6.9}\NormalTok{)}
\end{Highlighting}
\end{Shaded}

\begin{verbatim}
## [1] 0.8212956
\end{verbatim}

(d). Probability distribution

\begin{Shaded}
\begin{Highlighting}[]
\NormalTok{prob }\OtherTok{\textless{}{-}} \FunctionTok{dpois}\NormalTok{(}\AttributeTok{x=}\DecValTok{0}\SpecialCharTok{:}\DecValTok{20}\NormalTok{, }\AttributeTok{lambda=}\FloatTok{6.9}\NormalTok{)}
\FunctionTok{barplot}\NormalTok{(prob, }\AttributeTok{col=}\StringTok{"turquoise"}\NormalTok{)}
\end{Highlighting}
\end{Shaded}

\includegraphics{StatsTB_files/figure-latex/unnamed-chunk-113-1.pdf}

\hypertarget{shape-of-a-poisson-distribute}{%
\subsection{Shape of a Poisson distribute}\label{shape-of-a-poisson-distribute}}

All Poisson distributions are right skewed.

\hypertarget{mean-and-sd-of-a-poisson-random-variable}{%
\subsection{Mean and SD of a Poisson random variable}\label{mean-and-sd-of-a-poisson-random-variable}}

The mean and standard deviation of a Poisson random variable with parameter \(\lambda\) are

\[
\mu=\lambda; \sigma=\sqrt{\lambda}
\]

One most important property of a Poisson distribution is the mean and variance are the same and equal the parameter \(\lambda\). This is also convenient to apply.

\hypertarget{example-37}{%
\subsubsection{Example}\label{example-37}}

Let \(X\) denote the number of patients arriving at the emergency room between 6 and 7 pm. Assuming \(X\) follows a Poisson distribution with \(\lambda=6.9\),

\begin{itemize}
\tightlist
\item
  (a). Determine and interpret the mean of the random
  variable \(X\).
\item
  (b). Determine the standard deviation of \(X\)
\end{itemize}

\textbf{Solution:}

(a).

\[
\mu = 6.9
\]

On average, 6.9 patients arrive at the ER between 6 and 7pm.

(b).

\[
\sigma^2=\lambda=6.9; \sigma = \sqrt{\lambda} = 2.6
\]

\hypertarget{poisson-distribution-features}{%
\subsection{Poisson distribution features}\label{poisson-distribution-features}}

To further understand Poisson distribution, try out the technology below. Display the Poisson probability distribution with parameter \(\lambda\)

\begin{Shaded}
\begin{Highlighting}[]
\CommentTok{\#Poisson (lambda = 10)}
\NormalTok{x }\OtherTok{\textless{}{-}} \DecValTok{0}\SpecialCharTok{:}\DecValTok{30}
\NormalTok{p10 }\OtherTok{\textless{}{-}} \FunctionTok{dpois}\NormalTok{(x, }\AttributeTok{lambda=}\DecValTok{10}\NormalTok{)}
\NormalTok{p15 }\OtherTok{\textless{}{-}} \FunctionTok{dpois}\NormalTok{(x, }\AttributeTok{lambda=}\DecValTok{15}\NormalTok{)}

\NormalTok{bp10 }\OtherTok{\textless{}{-}} \FunctionTok{barplot}\NormalTok{(p10, }\AttributeTok{col=}\StringTok{"turquoise"}\NormalTok{, }\AttributeTok{main=}\StringTok{"Poisson (lambda = 10)"}\NormalTok{)}
\FunctionTok{mtext}\NormalTok{(}\AttributeTok{side =} \DecValTok{1}\NormalTok{, }\AttributeTok{at =}\NormalTok{ bp10, }\AttributeTok{line =} \DecValTok{0}\NormalTok{, }\AttributeTok{text =}\NormalTok{x, }\AttributeTok{cex=}\FloatTok{0.8}\NormalTok{)}
\end{Highlighting}
\end{Shaded}

\includegraphics{StatsTB_files/figure-latex/unnamed-chunk-114-1.pdf}

\begin{Shaded}
\begin{Highlighting}[]
\NormalTok{bp15 }\OtherTok{\textless{}{-}} \FunctionTok{barplot}\NormalTok{(p15, }\AttributeTok{col=}\StringTok{"turquoise"}\NormalTok{, }\AttributeTok{main=}\StringTok{"Poisson (lambda = 15)"}\NormalTok{)}
\FunctionTok{mtext}\NormalTok{(}\AttributeTok{side =} \DecValTok{1}\NormalTok{, }\AttributeTok{at =}\NormalTok{ bp15, }\AttributeTok{line =} \DecValTok{0}\NormalTok{, }\AttributeTok{text =}\NormalTok{x, }\AttributeTok{cex=}\FloatTok{0.8}\NormalTok{)}
\end{Highlighting}
\end{Shaded}

\includegraphics{StatsTB_files/figure-latex/unnamed-chunk-114-2.pdf}

Poisson distribution is similar to Binomial distribution. Can we use Poisson to approximate binomial?

\hypertarget{poisson-approximation-to-binomial-distribution}{%
\subsection{Poisson Approximation to Binomial Distribution}\label{poisson-approximation-to-binomial-distribution}}

Recall, a binomial probability distribution is
\[
P(X=x) = \binom{n}{x}p^x(1-p)^{n-x}
\]

Mean of Binomial random variable \(X\) is \(np\). Mean of Poisson random variable is \(\lambda\). How do both distributions look like when \(\lambda = np\)?

\textbf{Try below:}

Plot the Binomial distribution \(Binom(n=50, p=0.3)\) and
\(Poisson (\lambda = np=15)\).

\textbf{Scenario 1: n = 50; p = 0.3}

\begin{Shaded}
\begin{Highlighting}[]
\FunctionTok{pois.binom}\NormalTok{(}\AttributeTok{n=}\DecValTok{50}\NormalTok{, }\AttributeTok{p=}\FloatTok{0.3}\NormalTok{)}
\end{Highlighting}
\end{Shaded}

\includegraphics{StatsTB_files/figure-latex/unnamed-chunk-115-1.pdf}

\textbf{Scenario 2: n = 100; p = 0.3}

\begin{Shaded}
\begin{Highlighting}[]
\FunctionTok{pois.binom}\NormalTok{(}\AttributeTok{n=}\DecValTok{100}\NormalTok{, }\AttributeTok{p=}\FloatTok{0.3}\NormalTok{)}
\end{Highlighting}
\end{Shaded}

\includegraphics{StatsTB_files/figure-latex/unnamed-chunk-116-1.pdf}

\textbf{Scenario 3: n = 500; p = 0.3}

\begin{Shaded}
\begin{Highlighting}[]
\FunctionTok{pois.binom}\NormalTok{(}\AttributeTok{n=}\DecValTok{500}\NormalTok{, }\AttributeTok{p=}\FloatTok{0.3}\NormalTok{)}
\end{Highlighting}
\end{Shaded}

\includegraphics{StatsTB_files/figure-latex/unnamed-chunk-117-1.pdf}

\textbf{Observation:}

Event when \(n\) is very large, the approximation still has some issues.

What about large \(n\) and small \(p\)? Try below:

\textbf{Scenario 4: n = 1000; p = 0.005}

\begin{Shaded}
\begin{Highlighting}[]
\FunctionTok{pois.binom}\NormalTok{(}\AttributeTok{n=}\DecValTok{1000}\NormalTok{, }\AttributeTok{p=}\FloatTok{0.005}\NormalTok{)}
\end{Highlighting}
\end{Shaded}

\includegraphics{StatsTB_files/figure-latex/unnamed-chunk-118-1.pdf}

\hypertarget{procedure-for-approximation}{%
\subsection{Procedure for Approximation}\label{procedure-for-approximation}}

To Approximate Binomial Probabilities by Using a Poisson Probability Formula

\begin{itemize}
\item
  \textbf{Step 1} Find \(n\), the number of trials, and \(p\), the
  success probability.
\item
  \textbf{Step 2} Continue only if \(n \ge 100\) and
  \(np \le 10\) and
\item
  \textbf{Step 3} Approximate the binomial probabilities by
  using the Poisson probability formula
\end{itemize}

\[
P(X=x) = e^{-np}\frac{(np)^x}{x!}
\]

\hypertarget{example-estimate-infant-deaths-in-finland}{%
\subsubsection{Example: Estimate Infant Deaths in Finland}\label{example-estimate-infant-deaths-in-finland}}

The infant mortality rate is the number of deaths of children under 1 year old per 1000 live births in a year. It is reported in Finland the mortality rate is 3.4. Use Poisson distribution to approximate the probability that of 500 randomly selected live births in Finland, there are

\begin{enumerate}
\def\labelenumi{(\alph{enumi})}
\tightlist
\item
  no infant deaths; (b). at most 3 infant births
\end{enumerate}

\textbf{Solution:}

Let \(X\) denote the number of infant deaths out of 500 live births in Finland.

Follow the procedure, \(n=500\), \(p=3.4/1000 = 0.0034\) which is the probability of one infant death in 1 year.

Check whether it is acceptable for Poisson approximation:

\[
np = 1.7; n > 100
\]

So it is OK to proceed the approximation.

\textbf{(a).}

\[
P(X = 0) = e^{-1.7}\frac{1.7^0}{0!} = 0.183
\]

\textbf{(b).}

\[
P(X \le 3) = P(X=0)+P(X=1)+P(X=2)+P(X=3)\\
=e^{-1.7}\left(\frac{1.7^0}{0!}+\frac{1.7^1}{1!}+\frac{1.7^2}{2!}+\frac{1.7^3}{3!}\right) = 0.907
\]

\begin{Shaded}
\begin{Highlighting}[]
\CommentTok{\#(a)}
\FunctionTok{dpois}\NormalTok{(}\AttributeTok{x=}\DecValTok{0}\NormalTok{, }\AttributeTok{lambda=}\FloatTok{1.7}\NormalTok{)}
\end{Highlighting}
\end{Shaded}

\begin{verbatim}
## [1] 0.1826835
\end{verbatim}

\begin{Shaded}
\begin{Highlighting}[]
\CommentTok{\#(b). ppois(q) = P(X \textless{}= q) is the cumulative distribution function}
\FunctionTok{ppois}\NormalTok{(}\AttributeTok{q=}\DecValTok{3}\NormalTok{, }\AttributeTok{lambda=}\FloatTok{1.7}\NormalTok{)}
\end{Highlighting}
\end{Shaded}

\begin{verbatim}
## [1] 0.9068106
\end{verbatim}

\textbf{Interpretation}

Using the Poisson approximation:

\begin{itemize}
\item
  There is approximately an 18.3\% chance that no infant deaths will occur among 500 live births in Finland.
\item
  There is approximately a 90.7\% chance that 3 or fewer infant deaths will occur in such a group.
\item
  This low mortality rate reflects Finland's high standard of prenatal and infant care. The Poisson model is appropriate here due to the small probability of a single event and large number of trials.
\end{itemize}

\hypertarget{sensitivity-and-specificity}{%
\chapter{Sensitivity and specificity}\label{sensitivity-and-specificity}}

\begin{Shaded}
\begin{Highlighting}[]
\FunctionTok{library}\NormalTok{(IntroStats)}
\end{Highlighting}
\end{Shaded}

\hypertarget{marginal-probability}{%
\section{Marginal probability}\label{marginal-probability}}

Figure below illustrates the marginal probability and events. \[P(A) = \sum_j^n P(A\cap B_j)\], where \(B_1, \cdots, B_n\) are \(n\) disjoint events and \(\sum_j P(B_j) = 1\).

\begin{figure}
\centering
\includegraphics{https://i.ibb.co/8PVnsjw/Marginal.png}
\caption{Marginal Probability}
\end{figure}

\hypertarget{definitions-sensitivity-and-specificity}{%
\section{Definitions: Sensitivity and specificity}\label{definitions-sensitivity-and-specificity}}

Figure below illustrates the definitions of sensitivity, specificity, positive predictive value, and negative predictive value when characterizing testing results. Both sensitivity and specificity can be computed based on the current sample data, assuming that this data accurately reflects the technical proficiency of testing within the general population. Predictive value positive denotes the probability that a patient actually has the disease when a patient from the general population tests positive. Calculating this correctly necessitates knowledge of the disease's prevalence within the general population, which should ideally be estimated from a large sample size rather than solely relying on the current sample data. In cases where an existing estimate of disease prevalence isn't available, one can use the current sample to make an estimate.

\begin{figure}
\centering
\includegraphics{https://i.ibb.co/xsFjkXZ/sensitivity-and-specificity.png}
\caption{Sensitivity and Specificity}
\end{figure}

\hypertarget{confidence-intervals}{%
\section{Confidence Intervals}\label{confidence-intervals}}

The confidence interval (CI) is usually reported for sensitivity and specificity estimates. Both sensitivity and specificity are estimated based on binary samples. So the CI can be derived from the underlying Bernoulli distribution. Many methods can be used for computing confidence intervals for a single proportion. Five methods are described below: (1) Exact (Clopper-Pearson); (2) Score (Wilson); (3) Wilson Score with continuity correction; (4) Normal approximation; (5)Normal approximation with continuity correction. For a comparison of methods, see Newcombe (1998). For each of the following methods, let \(p\) be the population sensitivity, and let \(r\) represent the number of true positives with \(n\) total positives. Let \(\hat{p} = r / n\). The CI for specificity can be similarly obtained.

Exact (Clopper-Pearson): Using a mathematical relationship (see Fleiss et al (2003), p.~25) between the F distribution and the cumulative binomial distribution, the lower and upper confidence limits of a \(100(1-\alpha)\)\% exact confidence interval for the true proportion \(p\) are given by
\[
      \left[\frac{r}{r+(n-r+1)F_{1-\alpha/2, 2(n-r+1), 2r}}, \frac{(r+1)F_{1-\alpha/2;2(r+1),2(n-r)}}{(n-r) + (r + 1) F_{1-\alpha/2; 2(r+1), 2(n-r)}}\right]
\]

Score (Wilson): The Wilson Score confidence interval, which is based on inverting the z-test for a single proportion.

\[
      \frac{(2n\hat{p}+z_{1-\alpha/2}^2)\pm z_{1-\alpha/2}\sqrt{z_{1-\alpha/2}^2+4n\hat{p}(1-\hat{p})}}{2(n+z_{1-\alpha/2}^2)^2}
\]

Score with continuity correction: The Score confidence interval with continuity correction is based on inverting the z-test for a single proportion with continuity correction.
\[
Lower Limit = \frac{(2n\hat{p}+z_{1-\alpha/2}^2-1)- z_{1-\alpha/2}\sqrt{z_{1-\alpha/2}^2 - (2 + (1/n)) + 4\hat{p}(n(1-\hat{p})+1)}}{2(n+z_{1-\alpha/2}^2)^2}
\]

\[
Upper Limit = \frac{(2n\hat{p}+z_{1-\alpha/2}^2+1)+ z_{1-\alpha/2}\sqrt{z_{1-\alpha/2}^2 + (2 - (1/n)) + 4\hat{p}(n(1-\hat{p})-1)}}{2(n+z_{1-\alpha/2}^2)^2}
\]

Normal approximation: The simple asymptotic formula is based on the normal approximation to the binomial distribution. The approximation is close only for very large sample sizes.
\[
      \hat{p} \pm z_{1-\alpha/2}\sqrt{\frac{\hat{p}(1-\hat{p})}{n}}
\]

Normal approximation with continuity correction: Incorporating continuity correction,
\[
      \left(\hat{p} - z_{1-\alpha/2}\sqrt{\frac{\hat{p}(1-\hat{p})}{n}} - \frac{1}{2n}, \hat{p} + z_{1-\alpha/2}\sqrt{\frac{\hat{p}(1-\hat{p})}{n}} + \frac{1}{2n} \right)
\]

\hypertarget{example-38}{%
\section{Example}\label{example-38}}

A medical research team proposed a screening test for Alzheimer's disease. A random sample 450 patients and another independent random sample of 500 patients are tested and the results are listed in Table \ref{tab:sensitivity}. Both samples were drawn from US population of 65 years old or above.

\begin{center}
        \begin{table}[]
        \centering
        \caption{\label{tab:sensitivity} Alzheimer's Diagnosis}
        \begin{tabular}{@{}cccc@{}}
            \toprule
            Test & Disease Present($D$) & Disease Absent ($\bar{D}$) & Total \\ 
            \midrule
            Positive ($T$)      & 436 & 5 & 441   \\
            Negative ($\bar{T}$)& 14 & 495 & 509  \\
            Total             & 450 & 500 & 950   \\ 
            \bottomrule
        \end{tabular}
        
    \end{table}
    \end{center}

The sensitivity is \(P(T|D) = 436/450 = 0.97\), and the specificity is \(P(\bar{T}|\bar{D}) = 495/500=0.99\). What is the predictive value positive for a person of age 65 or older form US population? The predictive value positive is \(P(D|T)\) and calculated using Bayes' theorem

\[
P(D|T) = \frac{P(T|D)P(D)}{P(T|D)P(D)+P(T|\bar{D})P(\bar{D})} = \frac{0.97P(D)}{0.97P(D)+0.01(1-P(D))}
\]

According to Evans et al, the Alzeimer's prevalence in US population aged 65 and above is 11.3 percent, so \(P(D) = 0.113\). So \(P(D|T)\) is calculated as 0.93. Similarly, one can calculate the predictive value negative.

The 95\%CI for sensitivity using 5 different methods can be calculated below.

\begin{Shaded}
\begin{Highlighting}[]
\DocumentationTok{\#\#\#\#\#\#\#\#\#\#\#\#\#\#\#\#\#\#\#\#\#\#}
\CommentTok{\#14. Calculation of 5 CIs}
\DocumentationTok{\#\#\#\#\#\#\#\#\#\#\#\#\#\#\#\#\#\#\#\#\#\#}
\FunctionTok{ci.p}\NormalTok{(}\AttributeTok{r=}\DecValTok{436}\NormalTok{, }\AttributeTok{n=}\DecValTok{450}\NormalTok{, }\AttributeTok{conflev=}\FloatTok{0.95}\NormalTok{, }\AttributeTok{method=}\StringTok{"Clopper{-}Pearson Exact"}\NormalTok{)}
\end{Highlighting}
\end{Shaded}

\begin{verbatim}
## [1] 0.9483514 0.9828890
\end{verbatim}

\begin{Shaded}
\begin{Highlighting}[]
\FunctionTok{ci.p}\NormalTok{(}\AttributeTok{r=}\DecValTok{436}\NormalTok{, }\AttributeTok{n=}\DecValTok{450}\NormalTok{, }\AttributeTok{conflev=}\FloatTok{0.95}\NormalTok{, }\AttributeTok{method=}\StringTok{"Wilson Score"}\NormalTok{)}
\end{Highlighting}
\end{Shaded}

\begin{verbatim}
## [1] 0.9484612 0.9813789
\end{verbatim}

\begin{Shaded}
\begin{Highlighting}[]
\FunctionTok{ci.p}\NormalTok{(}\AttributeTok{r=}\DecValTok{436}\NormalTok{, }\AttributeTok{n=}\DecValTok{450}\NormalTok{, }\AttributeTok{conflev=}\FloatTok{0.95}\NormalTok{, }\AttributeTok{method=}\StringTok{"Wilson Score with Continuity"}\NormalTok{)}
\end{Highlighting}
\end{Shaded}

\begin{verbatim}
## [1] 0.9470963 0.9822124
\end{verbatim}

\begin{Shaded}
\begin{Highlighting}[]
\FunctionTok{ci.p}\NormalTok{(}\AttributeTok{r=}\DecValTok{436}\NormalTok{, }\AttributeTok{n=}\DecValTok{450}\NormalTok{, }\AttributeTok{conflev=}\FloatTok{0.95}\NormalTok{, }\AttributeTok{method=}\StringTok{"Asymptotic Normal"}\NormalTok{)}
\end{Highlighting}
\end{Shaded}

\begin{verbatim}
## [1] 0.9528477 0.9849301
\end{verbatim}

\begin{Shaded}
\begin{Highlighting}[]
\FunctionTok{ci.p}\NormalTok{(}\AttributeTok{r=}\DecValTok{436}\NormalTok{, }\AttributeTok{n=}\DecValTok{450}\NormalTok{, }\AttributeTok{conflev=}\FloatTok{0.95}\NormalTok{, }\AttributeTok{method=}\StringTok{"Asymptotic Normal with Continuity"}\NormalTok{)}
\end{Highlighting}
\end{Shaded}

\begin{verbatim}
## [1] 0.9517366 0.9860412
\end{verbatim}

Calculate the specificity and its 95\%CI.

\begin{Shaded}
\begin{Highlighting}[]
\FunctionTok{ci.p}\NormalTok{(}\AttributeTok{r=}\DecValTok{495}\NormalTok{, }\AttributeTok{n=}\DecValTok{500}\NormalTok{, }\AttributeTok{conflev=}\FloatTok{0.95}\NormalTok{, }\AttributeTok{method=}\StringTok{"Clopper{-}Pearson Exact"}\NormalTok{)}
\end{Highlighting}
\end{Shaded}

\begin{verbatim}
## [1] 0.9768186 0.9967453
\end{verbatim}

\begin{Shaded}
\begin{Highlighting}[]
\FunctionTok{ci.p}\NormalTok{(}\AttributeTok{r=}\DecValTok{495}\NormalTok{, }\AttributeTok{n=}\DecValTok{500}\NormalTok{, }\AttributeTok{conflev=}\FloatTok{0.95}\NormalTok{, }\AttributeTok{method=}\StringTok{"Wilson Score"}\NormalTok{)}
\end{Highlighting}
\end{Shaded}

\begin{verbatim}
## [1] 0.9768069 0.9957212
\end{verbatim}

\begin{Shaded}
\begin{Highlighting}[]
\FunctionTok{ci.p}\NormalTok{(}\AttributeTok{r=}\DecValTok{495}\NormalTok{, }\AttributeTok{n=}\DecValTok{500}\NormalTok{, }\AttributeTok{conflev=}\FloatTok{0.95}\NormalTok{, }\AttributeTok{method=}\StringTok{"Wilson Score with Continuity"}\NormalTok{)}
\end{Highlighting}
\end{Shaded}

\begin{verbatim}
## [1] 0.9754307 0.9963127
\end{verbatim}

\begin{Shaded}
\begin{Highlighting}[]
\FunctionTok{ci.p}\NormalTok{(}\AttributeTok{r=}\DecValTok{495}\NormalTok{, }\AttributeTok{n=}\DecValTok{500}\NormalTok{, }\AttributeTok{conflev=}\FloatTok{0.95}\NormalTok{, }\AttributeTok{method=}\StringTok{"Asymptotic Normal"}\NormalTok{)}
\end{Highlighting}
\end{Shaded}

\begin{verbatim}
## [1] 0.9812787 0.9987213
\end{verbatim}

\begin{Shaded}
\begin{Highlighting}[]
\FunctionTok{ci.p}\NormalTok{(}\AttributeTok{r=}\DecValTok{495}\NormalTok{, }\AttributeTok{n=}\DecValTok{500}\NormalTok{, }\AttributeTok{conflev=}\FloatTok{0.95}\NormalTok{, }\AttributeTok{method=}\StringTok{"Asymptotic Normal with Continuity"}\NormalTok{)}
\end{Highlighting}
\end{Shaded}

\begin{verbatim}
## [1] 0.9802787 0.9997213
\end{verbatim}

\hypertarget{the-normal-distribution}{%
\chapter{The Normal Distribution}\label{the-normal-distribution}}

\begin{Shaded}
\begin{Highlighting}[]
\FunctionTok{library}\NormalTok{(IntroStats)}
\end{Highlighting}
\end{Shaded}

The most important distribution in statistics is the normal distribution. The formula was first published by Abraham De Moivre (1667 - 1754) on November 12, 1733. It is also called Gaussian distribution to honor Carl Friedrich Gauss (1777-1855). The density function is given by

\[
    f(x) = \frac{1}{\sqrt{2\pi}\sigma} e^{-\frac{(x-\mu)^2}{2\sigma^2}}, \hspace{1in} -\infty < x < \infty.
\]

The probability of interest \(P(a<x<b)\) can be calculated using the density function as
\[
\int_a^b f(x)dx.
\]

Denote the cumulative distribution function \[F(x) = P(X \le x) = \int_{-\infty}^x f(u)du\], i.e., the area under the curve from \(-\infty\) to \(x\). As a result, \[P(a<x<b) = F(b) - F(a)\], which is the area under the curve from \(a\) to \(b\).

\begin{figure}
\centering
\includegraphics{https://i.ibb.co/kyHHYFb/normal.png}
\caption{Normal Density Curves}
\end{figure}

\begin{Shaded}
\begin{Highlighting}[]
    \DocumentationTok{\#\#\#\#\#\#\#\#\#\#\#\#\#\#\#\#\#\#\#\#\#\#\#\#\#\#\#\#\#\#\#\#}
        \CommentTok{\#18. Normal distribution}
        \DocumentationTok{\#\#\#\#\#\#\#\#\#\#\#\#\#\#\#\#\#\#\#\#\#\#\#\#\#\#\#\#\#\#\#\#}
\NormalTok{        x }\OtherTok{=} \FunctionTok{seq}\NormalTok{(}\SpecialCharTok{{-}}\DecValTok{3}\NormalTok{, }\DecValTok{5}\NormalTok{, }\AttributeTok{by=}\FloatTok{0.01}\NormalTok{)}
        \FunctionTok{plot}\NormalTok{(x, }\FunctionTok{dnorm}\NormalTok{(x), }\AttributeTok{type=}\StringTok{"l"}\NormalTok{, }\AttributeTok{lwd=}\DecValTok{2}\NormalTok{, }\AttributeTok{ylab=}\StringTok{""}\NormalTok{, }\AttributeTok{main=}\StringTok{"Normal Density Curves"}\NormalTok{)}
        \FunctionTok{lines}\NormalTok{(x, }\FunctionTok{dnorm}\NormalTok{(x, }\AttributeTok{mean=}\DecValTok{1}\NormalTok{, }\AttributeTok{sd=}\DecValTok{2}\NormalTok{), }\AttributeTok{col=}\StringTok{"red"}\NormalTok{, }\AttributeTok{lwd=}\DecValTok{2}\NormalTok{)}
        \FunctionTok{lines}\NormalTok{(}\FunctionTok{c}\NormalTok{(}\DecValTok{0}\NormalTok{,}\DecValTok{0}\NormalTok{),}\FunctionTok{c}\NormalTok{(}\DecValTok{0}\NormalTok{,}\FunctionTok{dnorm}\NormalTok{(}\DecValTok{0}\NormalTok{)), }\AttributeTok{lty=}\DecValTok{2}\NormalTok{)}
        \FunctionTok{lines}\NormalTok{(}\FunctionTok{c}\NormalTok{(}\DecValTok{1}\NormalTok{,}\DecValTok{1}\NormalTok{),}\FunctionTok{c}\NormalTok{(}\DecValTok{0}\NormalTok{,}\FunctionTok{dnorm}\NormalTok{(}\DecValTok{1}\NormalTok{, }\AttributeTok{mean=}\DecValTok{1}\NormalTok{, }\AttributeTok{sd=}\DecValTok{2}\NormalTok{)), }\AttributeTok{col=}\StringTok{"red"}\NormalTok{, }\AttributeTok{lty=}\DecValTok{2}\NormalTok{)}
        \FunctionTok{legend}\NormalTok{(}\StringTok{"topright"}\NormalTok{, }\AttributeTok{legend=}\FunctionTok{c}\NormalTok{(}\StringTok{"N(0, 1)"}\NormalTok{, }\StringTok{"N(1, sd=2)"}\NormalTok{), }
        \AttributeTok{lwd=}\FunctionTok{c}\NormalTok{(}\DecValTok{2}\NormalTok{,}\DecValTok{2}\NormalTok{), }\AttributeTok{col=}\FunctionTok{c}\NormalTok{(}\StringTok{"black"}\NormalTok{,}\StringTok{"red"}\NormalTok{), }\AttributeTok{bty=}\StringTok{"n"}\NormalTok{)}
\end{Highlighting}
\end{Shaded}

\includegraphics{StatsTB_files/figure-latex/unnamed-chunk-124-1.pdf}

\begin{Shaded}
\begin{Highlighting}[]
        \CommentTok{\#Probability within 1 sd }
        \CommentTok{\#(1)P({-}1 \textless{} X \textless{} 1) from the standard normal distribution}
        \FunctionTok{pnorm}\NormalTok{(}\DecValTok{1}\NormalTok{) }\SpecialCharTok{{-}} \FunctionTok{pnorm}\NormalTok{(}\SpecialCharTok{{-}}\DecValTok{1}\NormalTok{) }
\end{Highlighting}
\end{Shaded}

\begin{verbatim}
## [1] 0.6826895
\end{verbatim}

\begin{Shaded}
\begin{Highlighting}[]
        \CommentTok{\#(2)P({-}2 \textless{} X \textless{} 2) from the N(0, sd=2) distribution}
        \FunctionTok{pnorm}\NormalTok{(}\DecValTok{2}\NormalTok{, }\AttributeTok{mean=}\DecValTok{0}\NormalTok{, }\AttributeTok{sd=}\DecValTok{2}\NormalTok{) }\SpecialCharTok{{-}} \FunctionTok{pnorm}\NormalTok{(}\SpecialCharTok{{-}}\DecValTok{2}\NormalTok{, }\AttributeTok{mean=}\DecValTok{0}\NormalTok{, }\AttributeTok{sd=}\DecValTok{2}\NormalTok{)}
\end{Highlighting}
\end{Shaded}

\begin{verbatim}
## [1] 0.6826895
\end{verbatim}

\begin{Shaded}
\begin{Highlighting}[]
        \CommentTok{\#Quantile function}
        \CommentTok{\#Find the x value such that the area under the curve for X \textgreater{} x is 0.025.}
        \FunctionTok{qnorm}\NormalTok{(}\DecValTok{1}\FloatTok{{-}0.025}\NormalTok{)}
\end{Highlighting}
\end{Shaded}

\begin{verbatim}
## [1] 1.959964
\end{verbatim}

\begin{Shaded}
\begin{Highlighting}[]
        \CommentTok{\#Find the x value such that the area under the curve for X \textgreater{} x is 0.05.}
        \FunctionTok{qnorm}\NormalTok{(}\DecValTok{1}\FloatTok{{-}0.05}\NormalTok{)}
\end{Highlighting}
\end{Shaded}

\begin{verbatim}
## [1] 1.644854
\end{verbatim}

\hypertarget{normally-distributed-variables}{%
\section{Normally distributed variables}\label{normally-distributed-variables}}

\hypertarget{definition-density-curve}{%
\subsection{Definition: Density curve}\label{definition-density-curve}}

A density curve represents the distribution of a continuous random variable. It can also approximate the distribution of a discrete random variable.

\begin{itemize}
\tightlist
\item
  A density curve is always on or above the horizontal axis. \textbf{Why?}
\item
  The total area under a density curve equals 1. \textbf{Why?}
\end{itemize}

\begin{figure}
\centering
\includegraphics{https://i.ibb.co/xMTS5Cs/263-Figure-06-01.png}
\caption{Density curve: Total area = 1}
\end{figure}

\hypertarget{area-under-the-curve-within-a-range}{%
\subsection{Area under the curve within a range}\label{area-under-the-curve-within-a-range}}

The percentage of of ALL possible observations of a random variable that lie within a range equals the corresponding area under the curve in that range, and expressed as a percentage.

\includegraphics{https://i.ibb.co/28sc1Mq/264-Figure-06-02a.png}
\includegraphics{https://i.ibb.co/6YdByfy/264-Figure-06-02b.png}

\begin{figure}
\centering
\includegraphics{https://i.ibb.co/vmmCMF0/264-Figure-06-02c.png}
\caption{Area under the curve within a range}
\end{figure}

\hypertarget{example-39}{%
\subsubsection{Example}\label{example-39}}

Consider a density curve of a continuous random variable. Suppose the area under the density curve between 3 and 4 is 0.387, and the area under the curve to the right of 2 is 0.762.

\textbf{Question:} What percentage of all possible observations of the variable are at most 2? What percentage of all possible observations are between 3 and 4?

\textbf{Solution:} 0.238 and 0.387.

\hypertarget{example-40}{%
\subsubsection{Example}\label{example-40}}

The crystals of a certain mineral are cubes. The length of the sides is a variable that has the density curve shown below. The equation of this density curve is
\[
y = \frac{x}{2}
\]

for \(0<x<2\) mm.

\begin{figure}
\centering
\includegraphics{https://i.ibb.co/fH1L8TP/265-Figure-06-03a.png}
\caption{Crystals}
\end{figure}

\begin{figure}
\centering
\includegraphics{https://i.ibb.co/q05HW39/265-Figure-06-03b.png}
\caption{Crystals}
\end{figure}

Determine the percentage of these crystals that have side lengths: (a) \(<0.5mm\); (b) between 1 and 1.5 mm. (c) at least \(0.25mm\).

\textbf{Solution:}

The triangular area under the density curve from 0 to \(x\) is \(\frac{1}{2}(x)\frac{x}{2} = \frac{x^2}{4}\).

\begin{enumerate}
\def\labelenumi{(\alph{enumi})}
\tightlist
\item
  The percentage of \(<0.5mm\) is \(\frac{0.5^2}{4}=0.0625\).
\end{enumerate}

\begin{Shaded}
\begin{Highlighting}[]
\FunctionTok{area.under.density.curve}\NormalTok{(}\AttributeTok{cdf =} \ControlFlowTok{function}\NormalTok{(x)\{x}\SpecialCharTok{\^{}}\DecValTok{2}\SpecialCharTok{/}\DecValTok{4}\NormalTok{\}, }\AttributeTok{interval =} \FunctionTok{c}\NormalTok{(}\DecValTok{0}\NormalTok{, }\FloatTok{0.5}\NormalTok{))}
\end{Highlighting}
\end{Shaded}

\begin{verbatim}
## [1] 0.0625
\end{verbatim}

\includegraphics{https://i.ibb.co/56xBNf6/265-Figure-06-04.png}
(b) between 1 and 1.5 mm.

The percentage is \(\frac{1.5^2}{4}-\frac{1^2}{4}=0.3125\)

\begin{Shaded}
\begin{Highlighting}[]
\FunctionTok{area.under.density.curve}\NormalTok{(}\AttributeTok{cdf =} \ControlFlowTok{function}\NormalTok{(x)\{x}\SpecialCharTok{\^{}}\DecValTok{2}\SpecialCharTok{/}\DecValTok{4}\NormalTok{\}, }\AttributeTok{interval =} \FunctionTok{c}\NormalTok{(}\DecValTok{1}\NormalTok{, }\FloatTok{1.5}\NormalTok{))}
\end{Highlighting}
\end{Shaded}

\begin{verbatim}
## [1] 0.3125
\end{verbatim}

\begin{figure}
\centering
\includegraphics{https://i.ibb.co/Fnyk6Xv/265-Figure-06-05.png}
\caption{Crystals}
\end{figure}

\begin{enumerate}
\def\labelenumi{(\alph{enumi})}
\setcounter{enumi}{2}
\tightlist
\item
  The percentage is \(\frac{2^2}{4}-\frac{0.25^2}{4}=0.984375\)
\end{enumerate}

\begin{Shaded}
\begin{Highlighting}[]
\FunctionTok{area.under.density.curve}\NormalTok{(}\AttributeTok{cdf =} \ControlFlowTok{function}\NormalTok{(x)\{x}\SpecialCharTok{\^{}}\DecValTok{2}\SpecialCharTok{/}\DecValTok{4}\NormalTok{\}, }\AttributeTok{interval =} \FunctionTok{c}\NormalTok{(}\FloatTok{0.25}\NormalTok{, }\DecValTok{2}\NormalTok{))}
\end{Highlighting}
\end{Shaded}

\begin{verbatim}
## [1] 0.984375
\end{verbatim}

\begin{Shaded}
\begin{Highlighting}[]
\CommentTok{\#or calculate from the left}
\DecValTok{1} \SpecialCharTok{{-}} \FunctionTok{area.under.density.curve}\NormalTok{(}\AttributeTok{cdf =} \ControlFlowTok{function}\NormalTok{(x)\{x}\SpecialCharTok{\^{}}\DecValTok{2}\SpecialCharTok{/}\DecValTok{4}\NormalTok{\}, }\AttributeTok{interval =} \FunctionTok{c}\NormalTok{(}\DecValTok{0}\NormalTok{,}\FloatTok{0.25}\NormalTok{))}
\end{Highlighting}
\end{Shaded}

\begin{verbatim}
## [1] 0.984375
\end{verbatim}

\begin{figure}
\centering
\includegraphics{https://i.ibb.co/hgLxGPc/265-Figure-06-06.png}
\caption{Crystals}
\end{figure}

\hypertarget{normally-distributed-variable}{%
\subsection{Normally Distributed Variable}\label{normally-distributed-variable}}

A variable is said to be a normally distributed variable or to have a normal distribution if its distribution has the shape of a normal curve.

\begin{figure}
\centering
\includegraphics{https://i.ibb.co/rcpZ9mn/266-Figure-06-07.png}
\caption{Normal}
\end{figure}

\begin{figure}
\centering
\includegraphics{https://i.ibb.co/t4xsfTm/266-Figure-06-08.png}
\caption{Normal}
\end{figure}

The normal density curve has mean parameter \(\mu\) and standard deviation parameter \(\sigma\) with the following form:

\[
f(x) = \frac{1}{\sqrt{2\pi\sigma}}e^{-\frac{1}{2}\cdot\frac{(x-\mu)^2}{\sigma^2}}
\]

The total area under the normal density curve is 1, i.e.,
\[
\int_{-\infty}^{\infty} f(x)dx = 1
\]

Please try out different \(N(\mu, \sigma)\) distributions below.

\begin{Shaded}
\begin{Highlighting}[]
\FunctionTok{draw.normal.dist}\NormalTok{(}\AttributeTok{mu=}\DecValTok{0}\NormalTok{, }\AttributeTok{sigma=}\DecValTok{3}\NormalTok{, }\AttributeTok{xlim=}\FunctionTok{c}\NormalTok{(}\SpecialCharTok{{-}}\DecValTok{30}\NormalTok{, }\DecValTok{40}\NormalTok{), }\AttributeTok{ylim=}\FunctionTok{c}\NormalTok{(}\DecValTok{0}\NormalTok{,}\FloatTok{0.13}\NormalTok{))}
\end{Highlighting}
\end{Shaded}

\includegraphics{StatsTB_files/figure-latex/unnamed-chunk-128-1.pdf}

\begin{Shaded}
\begin{Highlighting}[]
\FunctionTok{draw.normal.dist}\NormalTok{(}\AttributeTok{mu=}\DecValTok{0}\NormalTok{, }\AttributeTok{sigma=}\DecValTok{5}\NormalTok{, }\AttributeTok{xlim=}\FunctionTok{c}\NormalTok{(}\SpecialCharTok{{-}}\DecValTok{30}\NormalTok{, }\DecValTok{40}\NormalTok{), }\AttributeTok{ylim=}\FunctionTok{c}\NormalTok{(}\DecValTok{0}\NormalTok{,}\FloatTok{0.13}\NormalTok{))}
\end{Highlighting}
\end{Shaded}

\includegraphics{StatsTB_files/figure-latex/unnamed-chunk-128-2.pdf}

\begin{Shaded}
\begin{Highlighting}[]
\FunctionTok{draw.normal.dist}\NormalTok{(}\AttributeTok{mu=}\DecValTok{0}\NormalTok{, }\AttributeTok{sigma=}\DecValTok{10}\NormalTok{, }\AttributeTok{xlim=}\FunctionTok{c}\NormalTok{(}\SpecialCharTok{{-}}\DecValTok{30}\NormalTok{, }\DecValTok{40}\NormalTok{), }\AttributeTok{ylim=}\FunctionTok{c}\NormalTok{(}\DecValTok{0}\NormalTok{,}\FloatTok{0.13}\NormalTok{))}
\end{Highlighting}
\end{Shaded}

\includegraphics{StatsTB_files/figure-latex/unnamed-chunk-128-3.pdf}

\begin{Shaded}
\begin{Highlighting}[]
\FunctionTok{draw.normal.dist}\NormalTok{(}\AttributeTok{mu=}\DecValTok{10}\NormalTok{, }\AttributeTok{sigma=}\DecValTok{3}\NormalTok{, }\AttributeTok{xlim=}\FunctionTok{c}\NormalTok{(}\SpecialCharTok{{-}}\DecValTok{30}\NormalTok{, }\DecValTok{40}\NormalTok{), }\AttributeTok{ylim=}\FunctionTok{c}\NormalTok{(}\DecValTok{0}\NormalTok{,}\FloatTok{0.13}\NormalTok{))}
\end{Highlighting}
\end{Shaded}

\includegraphics{StatsTB_files/figure-latex/unnamed-chunk-128-4.pdf}

\begin{Shaded}
\begin{Highlighting}[]
\FunctionTok{draw.normal.dist}\NormalTok{(}\AttributeTok{mu=}\DecValTok{10}\NormalTok{, }\AttributeTok{sigma=}\DecValTok{5}\NormalTok{, }\AttributeTok{xlim=}\FunctionTok{c}\NormalTok{(}\SpecialCharTok{{-}}\DecValTok{30}\NormalTok{, }\DecValTok{40}\NormalTok{), }\AttributeTok{ylim=}\FunctionTok{c}\NormalTok{(}\DecValTok{0}\NormalTok{,}\FloatTok{0.13}\NormalTok{))}
\end{Highlighting}
\end{Shaded}

\includegraphics{StatsTB_files/figure-latex/unnamed-chunk-128-5.pdf}

\begin{Shaded}
\begin{Highlighting}[]
\FunctionTok{draw.normal.dist}\NormalTok{(}\AttributeTok{mu=}\DecValTok{10}\NormalTok{, }\AttributeTok{sigma=}\DecValTok{10}\NormalTok{, }\AttributeTok{xlim=}\FunctionTok{c}\NormalTok{(}\SpecialCharTok{{-}}\DecValTok{30}\NormalTok{, }\DecValTok{40}\NormalTok{), }\AttributeTok{ylim=}\FunctionTok{c}\NormalTok{(}\DecValTok{0}\NormalTok{,}\FloatTok{0.13}\NormalTok{))}
\end{Highlighting}
\end{Shaded}

\includegraphics{StatsTB_files/figure-latex/unnamed-chunk-128-6.pdf}

\hypertarget{example-heights}{%
\subsubsection{Example: Heights}\label{example-heights}}

Table below shows the relative frequencies of a college's 3264 female freshman students' heights. The mean is 64.4 inches and sd is 2.4 inches.

\begin{table}

\caption{\label{tab:unnamed-chunk-129}Height distribution with frequencies and relative frequencies}
\centering
\begin{tabular}[t]{l|r|r}
\hline
Height\_in & Frequency\_f & Relative\_freq\\
\hline
56–under 57 & 3 & 0.0009\\
\hline
57–under 58 & 6 & 0.0018\\
\hline
58–under 59 & 26 & 0.0080\\
\hline
59–under 60 & 74 & 0.0227\\
\hline
60–under 61 & 147 & 0.0450\\
\hline
61–under 62 & 247 & 0.0757\\
\hline
62–under 63 & 382 & 0.1170\\
\hline
63–under 64 & 483 & 0.1480\\
\hline
64–under 65 & 559 & 0.1713\\
\hline
65–under 66 & 514 & 0.1575\\
\hline
66–under 67 & 359 & 0.1100\\
\hline
67–under 68 & 240 & 0.0735\\
\hline
68–under 69 & 122 & 0.0374\\
\hline
69–under 70 & 65 & 0.0199\\
\hline
70–under 71 & 24 & 0.0074\\
\hline
71–under 72 & 7 & 0.0021\\
\hline
72–under 73 & 5 & 0.0015\\
\hline
73–under 74 & 1 & 0.0003\\
\hline
Total & 3264 & 1.0000\\
\hline
\end{tabular}
\end{table}

\begin{enumerate}
\def\labelenumi{(\alph{enumi})}
\tightlist
\item
  Show that the variable height is approximately normally distributed.
\end{enumerate}

\textbf{Solution.}

Superimpose the normal density curve with mean 64.4 and sd 2.4 on top of the relative frequency. If they are consistent, then it indicates the variable height is approximately normal.

\begin{Shaded}
\begin{Highlighting}[]
\CommentTok{\#Data entry}
\NormalTok{height }\OtherTok{\textless{}{-}} \DecValTok{56}\SpecialCharTok{:}\DecValTok{73}
\NormalTok{freq }\OtherTok{\textless{}{-}} \FunctionTok{c}\NormalTok{(}\DecValTok{3}\NormalTok{, }\DecValTok{6}\NormalTok{, }\DecValTok{26}\NormalTok{, }\DecValTok{74}\NormalTok{, }\DecValTok{147}\NormalTok{, }\DecValTok{247}\NormalTok{, }\DecValTok{382}\NormalTok{, }\DecValTok{483}\NormalTok{, }\DecValTok{559}\NormalTok{, }\DecValTok{514}\NormalTok{, }\DecValTok{359}\NormalTok{, }\DecValTok{240}\NormalTok{, }\DecValTok{122}\NormalTok{, }\DecValTok{65}\NormalTok{, }\DecValTok{24}\NormalTok{, }\DecValTok{7}\NormalTok{, }\DecValTok{5}\NormalTok{, }\DecValTok{1}\NormalTok{)}

\NormalTok{n }\OtherTok{\textless{}{-}} \FunctionTok{sum}\NormalTok{(freq)}
\NormalTok{n}
\end{Highlighting}
\end{Shaded}

\begin{verbatim}
## [1] 3264
\end{verbatim}

\begin{Shaded}
\begin{Highlighting}[]
\NormalTok{rel.freq }\OtherTok{\textless{}{-}}\NormalTok{ freq}\SpecialCharTok{/}\NormalTok{n}

\NormalTok{rel.freq}
\end{Highlighting}
\end{Shaded}

\begin{verbatim}
##  [1] 0.0009191176 0.0018382353 0.0079656863 0.0226715686 0.0450367647
##  [6] 0.0756740196 0.1170343137 0.1479779412 0.1712622549 0.1574754902
## [11] 0.1099877451 0.0735294118 0.0373774510 0.0199142157 0.0073529412
## [16] 0.0021446078 0.0015318627 0.0003063725
\end{verbatim}

\begin{Shaded}
\begin{Highlighting}[]
\NormalTok{data }\OtherTok{\textless{}{-}} \FunctionTok{data.frame}\NormalTok{(}\FunctionTok{cbind}\NormalTok{(height, freq, rel.freq))}
\NormalTok{data}
\end{Highlighting}
\end{Shaded}

\begin{verbatim}
##    height freq     rel.freq
## 1      56    3 0.0009191176
## 2      57    6 0.0018382353
## 3      58   26 0.0079656863
## 4      59   74 0.0226715686
## 5      60  147 0.0450367647
## 6      61  247 0.0756740196
## 7      62  382 0.1170343137
## 8      63  483 0.1479779412
## 9      64  559 0.1712622549
## 10     65  514 0.1574754902
## 11     66  359 0.1099877451
## 12     67  240 0.0735294118
## 13     68  122 0.0373774510
## 14     69   65 0.0199142157
## 15     70   24 0.0073529412
## 16     71    7 0.0021446078
## 17     72    5 0.0015318627
## 18     73    1 0.0003063725
\end{verbatim}

Draw relative frequency plot and superimpose the normal curve. They are quite consistent.

\begin{Shaded}
\begin{Highlighting}[]
\FunctionTok{draw.normal.dist}\NormalTok{(}\AttributeTok{mu=}\FloatTok{64.4}\NormalTok{, }\AttributeTok{sigma=}\FloatTok{2.4}\NormalTok{)}
\FunctionTok{lines}\NormalTok{(height, rel.freq, }\AttributeTok{type=}\StringTok{"h"}\NormalTok{, }\AttributeTok{lwd=}\DecValTok{5}\NormalTok{, }\AttributeTok{col=}\StringTok{"gray"}\NormalTok{)}
\end{Highlighting}
\end{Shaded}

\includegraphics{StatsTB_files/figure-latex/unnamed-chunk-131-1.pdf}

\hypertarget{standard-normal-distribution}{%
\subsection{Standard Normal Distribution}\label{standard-normal-distribution}}

A normal distribution with mean 0 and standard deviation 1 is called standard normal distribution.

For \(x\sim N(\mu, \sigma)\), we can standardize x by transformation
\[
z = \frac{x-\mu}{\sigma}
\]
so \(z\) follows a standard normal distribution.

\begin{figure}
\centering
\includegraphics{https://i.ibb.co/cNGR7H6/269-Figure-06-12.png}
\caption{Standardization}
\end{figure}

We can use the standardized normal distribution to calculate the area under the curve because of the relationship below.

\begin{figure}
\centering
\includegraphics{https://i.ibb.co/nLbzRM0/270-Figure-06-13.png}
\caption{Standardization}
\end{figure}

\hypertarget{simulating-a-normally-distributed-variable}{%
\subsection{Simulating a normally distributed variable}\label{simulating-a-normally-distributed-variable}}

Consider gestation periods of humans are normally distributed with a mean of 266 days and sd of 16 days. Simulate 10000 human gestation periods, obtain a histogram of the simulated data.

\begin{Shaded}
\begin{Highlighting}[]
\NormalTok{x }\OtherTok{\textless{}{-}} \FunctionTok{rnorm}\NormalTok{(}\DecValTok{10000}\NormalTok{, }\AttributeTok{mean=}\DecValTok{266}\NormalTok{, }\AttributeTok{sd=}\DecValTok{16}\NormalTok{) }\CommentTok{\#simulating 10000 normal data}
\FunctionTok{hist}\NormalTok{(x, }\AttributeTok{col=}\StringTok{"turquoise"}\NormalTok{, }\AttributeTok{main=}\StringTok{"Histogram of 10000 simulated gestation periods of humans"}\NormalTok{, }\AttributeTok{freq=}\ConstantTok{FALSE}\NormalTok{, }\AttributeTok{xlab=}\StringTok{"Gestation Periods (days)"}\NormalTok{)}

\FunctionTok{lines}\NormalTok{(}\FunctionTok{density}\NormalTok{(x),}\AttributeTok{col=}\StringTok{"red"}\NormalTok{, }\AttributeTok{lwd=}\DecValTok{3}\NormalTok{)}
\end{Highlighting}
\end{Shaded}

\includegraphics{StatsTB_files/figure-latex/unnamed-chunk-132-1.pdf}

\hypertarget{areas-under-the-n0-1-density-curve}{%
\section{Areas under the N(0, 1) density curve}\label{areas-under-the-n0-1-density-curve}}

\hypertarget{basic-properties-of-n0-1}{%
\subsection{Basic properties of N(0, 1)}\label{basic-properties-of-n0-1}}

Let's have a closer look at the N(0, 1) curve.

\begin{Shaded}
\begin{Highlighting}[]
\FunctionTok{draw.normal.dist}\NormalTok{(}\AttributeTok{mu=}\DecValTok{0}\NormalTok{, }\AttributeTok{sigma=}\DecValTok{1}\NormalTok{)}
\end{Highlighting}
\end{Shaded}

\includegraphics{StatsTB_files/figure-latex/unnamed-chunk-133-1.pdf}

\begin{itemize}
\tightlist
\item
  \textbf{Property 1:} The total area under the standard normal curve is 1.
\item
  \textbf{Property 2:} The standard normal curve extends indefinitely in both directions, approaching, but never touching, the horizontal axis as it does so.
\item
  \textbf{Property 3:} The standard normal curve is symmetric about 0; that is, the part of the curve to the left of the dashed line in Fig. 8.14 is the mirror image of the part of the curve to the right of it.
\item
  \textbf{Property 4:} Almost all the area under the standard normal curve lies between −3 and 3.
\end{itemize}

\hypertarget{example-41}{%
\subsection{Example}\label{example-41}}

Find the area under the standard normal density curve to the left of \(z=1.23\).

\begin{Shaded}
\begin{Highlighting}[]
\FunctionTok{draw.shade.under.density}\NormalTok{(}\AttributeTok{dfun =}\NormalTok{ dnorm, }\AttributeTok{range=}\FunctionTok{c}\NormalTok{(}\SpecialCharTok{{-}}\FloatTok{3.5}\NormalTok{, }\FloatTok{3.5}\NormalTok{), }
 \AttributeTok{shade.interval=}\FunctionTok{c}\NormalTok{(}\SpecialCharTok{{-}}\FloatTok{3.5}\NormalTok{, }\FloatTok{1.23}\NormalTok{), }
 \AttributeTok{curve.col =} \StringTok{"aquamarine4"}\NormalTok{, }\AttributeTok{shade.col =} \StringTok{"turquoise"}\NormalTok{,}
 \AttributeTok{xlab=}\StringTok{"z"}\NormalTok{, }\AttributeTok{ylab=}\StringTok{"Density Curve of N(0, 1)"}\NormalTok{, }\AttributeTok{mean=}\DecValTok{0}\NormalTok{, }\AttributeTok{sd=}\DecValTok{1}\NormalTok{)}
\end{Highlighting}
\end{Shaded}

\includegraphics{StatsTB_files/figure-latex/unnamed-chunk-134-1.pdf}
\textbf{Solution}

The area under the curve to the left of 1.23 is calculated as
\[
\int_{-\infty}^{1.23} f(x)dx = \Phi(1.23)
\]

where \(\Phi()\) is the cumulative distribution function (CDF) of the standard normal. It can be calculated as \[\Phi(1.23)\]

\begin{Shaded}
\begin{Highlighting}[]
\FunctionTok{pnorm}\NormalTok{(}\FloatTok{1.23}\NormalTok{)}
\end{Highlighting}
\end{Shaded}

\begin{verbatim}
## [1] 0.8906514
\end{verbatim}

You can also calculate using the general function below.

\begin{Shaded}
\begin{Highlighting}[]
\FunctionTok{area.under.density.curve}\NormalTok{(}\AttributeTok{cdf=}\NormalTok{pnorm, }\AttributeTok{interval=}\FunctionTok{c}\NormalTok{(}\SpecialCharTok{{-}}\ConstantTok{Inf}\NormalTok{, }\FloatTok{1.23}\NormalTok{))}
\end{Highlighting}
\end{Shaded}

\begin{verbatim}
## [1] 0.8906514
\end{verbatim}

\hypertarget{example-42}{%
\subsection{Example}\label{example-42}}

Find the area under the standard normal curve to the right of 0.76.

\begin{Shaded}
\begin{Highlighting}[]
\FunctionTok{draw.shade.under.density}\NormalTok{(}\AttributeTok{dfun =}\NormalTok{ dnorm, }\AttributeTok{range=}\FunctionTok{c}\NormalTok{(}\SpecialCharTok{{-}}\FloatTok{3.5}\NormalTok{, }\FloatTok{3.5}\NormalTok{), }
 \AttributeTok{shade.interval=}\FunctionTok{c}\NormalTok{(}\FloatTok{0.76}\NormalTok{, }\FloatTok{3.5}\NormalTok{), }
 \AttributeTok{curve.col =} \StringTok{"aquamarine4"}\NormalTok{, }\AttributeTok{shade.col =} \StringTok{"turquoise"}\NormalTok{,}
 \AttributeTok{xlab=}\StringTok{"z"}\NormalTok{, }\AttributeTok{ylab=}\StringTok{"Density Curve of N(0, 1)"}\NormalTok{, }\AttributeTok{mean=}\DecValTok{0}\NormalTok{, }\AttributeTok{sd=}\DecValTok{1}\NormalTok{)}
\end{Highlighting}
\end{Shaded}

\includegraphics{StatsTB_files/figure-latex/unnamed-chunk-137-1.pdf}

This area can be calculated as the 1 minus the area to the left \[1-\Phi(0.76)\]

\begin{Shaded}
\begin{Highlighting}[]
\DecValTok{1}\SpecialCharTok{{-}}\FunctionTok{pnorm}\NormalTok{(}\FloatTok{0.76}\NormalTok{)}
\end{Highlighting}
\end{Shaded}

\begin{verbatim}
## [1] 0.2236273
\end{verbatim}

You can also calculate using the general function below.

\begin{Shaded}
\begin{Highlighting}[]
\FunctionTok{area.under.density.curve}\NormalTok{(}\AttributeTok{cdf=}\NormalTok{pnorm, }\AttributeTok{interval=}\FunctionTok{c}\NormalTok{(}\FloatTok{0.76}\NormalTok{, }\ConstantTok{Inf}\NormalTok{))}
\end{Highlighting}
\end{Shaded}

\begin{verbatim}
## [1] 0.2236273
\end{verbatim}

\hypertarget{example-43}{%
\subsection{Example}\label{example-43}}

Find the area under the density curve of N(0, 1) between -0.68 and 1.82.

\begin{Shaded}
\begin{Highlighting}[]
\FunctionTok{draw.shade.under.density}\NormalTok{(}\AttributeTok{dfun =}\NormalTok{ dnorm, }\AttributeTok{range=}\FunctionTok{c}\NormalTok{(}\SpecialCharTok{{-}}\FloatTok{3.5}\NormalTok{, }\FloatTok{3.5}\NormalTok{), }
 \AttributeTok{shade.interval=}\FunctionTok{c}\NormalTok{(}\SpecialCharTok{{-}}\FloatTok{0.68}\NormalTok{, }\FloatTok{1.82}\NormalTok{), }
 \AttributeTok{curve.col =} \StringTok{"aquamarine4"}\NormalTok{, }\AttributeTok{shade.col =} \StringTok{"turquoise"}\NormalTok{,}
 \AttributeTok{xlab=}\StringTok{"z"}\NormalTok{, }\AttributeTok{ylab=}\StringTok{"Density Curve of N(0, 1)"}\NormalTok{, }\AttributeTok{mean=}\DecValTok{0}\NormalTok{, }\AttributeTok{sd=}\DecValTok{1}\NormalTok{)}
\end{Highlighting}
\end{Shaded}

\includegraphics{StatsTB_files/figure-latex/unnamed-chunk-140-1.pdf}

This area can be calculated as \[\Phi(1.82)-\Phi(-0.68)\]

\begin{Shaded}
\begin{Highlighting}[]
\FunctionTok{pnorm}\NormalTok{(}\FloatTok{1.82}\NormalTok{)}\SpecialCharTok{{-}}\FunctionTok{pnorm}\NormalTok{(}\SpecialCharTok{{-}}\FloatTok{0.68}\NormalTok{)}
\end{Highlighting}
\end{Shaded}

\begin{verbatim}
## [1] 0.7173683
\end{verbatim}

You can also calculate using the general function below.

\begin{Shaded}
\begin{Highlighting}[]
\FunctionTok{area.under.density.curve}\NormalTok{(}\AttributeTok{cdf=}\NormalTok{pnorm, }\AttributeTok{interval=}\FunctionTok{c}\NormalTok{(}\SpecialCharTok{{-}}\FloatTok{0.68}\NormalTok{, }\FloatTok{1.82}\NormalTok{))}
\end{Highlighting}
\end{Shaded}

\begin{verbatim}
## [1] 0.7173683
\end{verbatim}

In summary.

\begin{figure}
\centering
\includegraphics{https://i.ibb.co/BtjhNxz/276-Figure-06-18.png}
\caption{Area under the Curve}
\end{figure}

\hypertarget{examplefinding-the-z-score-having-a-specified-area-to-its-left}{%
\subsection{\texorpdfstring{\url{Example:Finding} the z-score having a specified area to its left}{Example:Finding the z-score having a specified area to its left}}\label{examplefinding-the-z-score-having-a-specified-area-to-its-left}}

What is the z-score having an area of 0.04 to its left under the standard normal distribution?

\begin{figure}
\centering
\includegraphics{https://i.ibb.co/n7pdm7G/277-Figure-06-19.png}
\caption{Quantile}
\end{figure}

The CDF (\(\Phi(z)\)) is to find out the area under the curve. In other words, \[\Phi(z)\] means the area under the curve of N(0, 1) to the left of \(z\).

\[
\Phi(z) = P(Z < z)
\]

Given the area under the curve to the left of \(z\), the z-score can be determined by inverse calculation.

\[
p = \Phi(z) \\
z = \Phi^{-1}(p)
\]

The second function is called quantile function. It is calculated in R as \texttt{qnorm}. Given \(p\), we can find \(z\), and the \(z\) is called the \(p\)th quantile.

\begin{Shaded}
\begin{Highlighting}[]
\FunctionTok{qnorm}\NormalTok{(}\FloatTok{0.04}\NormalTok{)}
\end{Highlighting}
\end{Shaded}

\begin{verbatim}
## [1] -1.750686
\end{verbatim}

So the 4\% quantile is -1.75.

\hypertarget{find-the-quantile}{%
\subsection{Find the Quantile}\label{find-the-quantile}}

Find the quantile such that the area to the right of the quantile is 0.025.

\begin{Shaded}
\begin{Highlighting}[]
\FunctionTok{qnorm}\NormalTok{(}\DecValTok{1}\FloatTok{{-}0.025}\NormalTok{)}
\end{Highlighting}
\end{Shaded}

\begin{verbatim}
## [1] 1.959964
\end{verbatim}

Since the area to the right of the quantile is 0.025, then the area to its left is 1-0.025. So the quantile can be calculated as qnorm(0.975).

\hypertarget{notations}{%
\subsection{Notations}\label{notations}}

Sometimes people use notation \(z_{\alpha}\) to indicate the quantile of the area to its right is \(\alpha\). However, this notation is not uniformly acceptable and can be confusing. Another notation is \(z_{1-\alpha}\), which means the same quantile but more explicit and consistent with the quantile function \(qnorm()\). To avoid confusions, I recommend using the notation of \(z_{1-\alpha}\), which explicitly means the \((1-\alpha)\times 100\%\) percentile.

\hypertarget{normally-distributed-variable-1}{%
\section{Normally distributed variable}\label{normally-distributed-variable-1}}

\hypertarget{procedure-to-determine-a-probability-of-a-nmu-sigma-variable}{%
\subsection{\texorpdfstring{Procedure to determine a probability of a \(N(\mu, \sigma)\) variable}{Procedure to determine a probability of a N(\textbackslash mu, \textbackslash sigma) variable}}\label{procedure-to-determine-a-probability-of-a-nmu-sigma-variable}}

To Determine a Percentage or Probability for a Normally Distributed Variable
- \emph{Step 1} Sketch the normal curve associated with the variable.
- \emph{Step 2} Shade the region of interest and mark its delimiting x-value(s).
- \emph{Step 3} Find the z-score(s) for the delimiting x-value(s) found in Step 2.
- \emph{Step 4} Use Table II to find the area under the standard normal curve delimited by the z-score(s) found in Step 3.

\begin{figure}
\centering
\includegraphics{https://i.ibb.co/tLXFJMR/normal-prob.png}
\caption{Procedure to determine a probability of a \(N(\mu, \sigma)\) variable}
\end{figure}

For \(X \sim N(\mu, \sigma)\), the probability
\[
P(a < X < b) = P \left(\frac{a-\mu}{\sigma} < Z < \frac{b-\mu}{\sigma}\right) = \Phi\left( \frac{b-\mu}{\sigma}\right) - \Phi\left(\frac{a-\mu}{\sigma}\right)
\]

\hypertarget{example-iq}{%
\subsection{Example: IQ}\label{example-iq}}

IQ is normally distributed with a mean of 100 and standard deviation of 16. Determine the percentage of patients whose IQ between 115 and 140.

\textbf{solution.}

Let \(X\) denote the IQ score. Then mean of \(X\) is 100 and standard deviation is 16.

\[
P(115 < X < 140) = \Phi\left( \frac{140-100}{16}\right) - \Phi\left(\frac{115-100}{16}\right)=\Phi(2.5)-\Phi(0.94) = 0.168
\]

Once convert to \(Phi\) function, we can calculate as

\begin{Shaded}
\begin{Highlighting}[]
\FunctionTok{pnorm}\NormalTok{((}\DecValTok{140{-}100}\NormalTok{)}\SpecialCharTok{/}\DecValTok{16}\NormalTok{) }\SpecialCharTok{{-}} \FunctionTok{pnorm}\NormalTok{((}\DecValTok{115{-}100}\NormalTok{)}\SpecialCharTok{/}\DecValTok{16}\NormalTok{)}
\end{Highlighting}
\end{Shaded}

\begin{verbatim}
## [1] 0.168041
\end{verbatim}

You can also use the function

\begin{Shaded}
\begin{Highlighting}[]
\FunctionTok{area.under.density.curve}\NormalTok{(}\AttributeTok{cdf=}\NormalTok{pnorm, }\AttributeTok{interval =} \FunctionTok{c}\NormalTok{(}\DecValTok{115}\NormalTok{, }\DecValTok{140}\NormalTok{), }\AttributeTok{mean=}\DecValTok{100}\NormalTok{, }\AttributeTok{sd=}\DecValTok{16}\NormalTok{)}
\end{Highlighting}
\end{Shaded}

\begin{verbatim}
## [1] 0.168041
\end{verbatim}

\hypertarget{empirical-rule-1}{%
\subsection{Empirical Rule}\label{empirical-rule-1}}

\begin{figure}
\centering
\includegraphics{https://i.ibb.co/hMWVFjg/119-Figure-03-09.png}
\caption{Empirical rule}
\end{figure}

\hypertarget{normal-probability-plot}{%
\section{Normal probability plot}\label{normal-probability-plot}}

\hypertarget{how-do-we-check-normality}{%
\subsection{How do we check normality?}\label{how-do-we-check-normality}}

Given a sample, how do we know the sample is likely from normal distribution?

\begin{itemize}
\tightlist
\item
  If the sample size is large, we can draw histogram, then superimpose the normal density curve with the sample mean and sample standard deviation. The normality is assessed by superimposing the density curve and check consistency.
\item
  If the sample size is small, we need more sensitive technique to check normality.
\end{itemize}

\hypertarget{definition-5}{%
\subsection{Definition}\label{definition-5}}

Normal probability plot is a plot of the observed values of the variable versus the normal scores - the observations expected for a variable having standard normal distribution.

\hypertarget{assessing-normality}{%
\subsection{Assessing normality}\label{assessing-normality}}

Guidelines for Assessing Normality Using a Normal Probability Plot

\begin{itemize}
\tightlist
\item
  To assess the normality of a variable using sample data, construct a normal probability plot.
\item
  If the plot is roughly linear, you can assume that the variable is approximately normally distributed.
\item
  If the plot is not roughly linear, you can assume that the variable is not approximately normally distributed.
\item
  These guidelines should be interpreted loosely for small samples but usually interpreted strictly for large samples.
\end{itemize}

\hypertarget{example-gross-income}{%
\subsection{Example: Gross Income}\label{example-gross-income}}

IRS publishes data for adjusted gross income (\$1000s).

\begin{Shaded}
\begin{Highlighting}[]
\NormalTok{AGI }\OtherTok{\textless{}{-}} \FunctionTok{c}\NormalTok{(}\FloatTok{9.7}\NormalTok{, }\FloatTok{93.1}\NormalTok{, }\DecValTok{33}\NormalTok{, }\FloatTok{21.2}\NormalTok{, }
         \FloatTok{81.4}\NormalTok{, }\FloatTok{51.1}\NormalTok{, }\FloatTok{43.5}\NormalTok{, }\FloatTok{10.6}\NormalTok{, }
         \FloatTok{12.8}\NormalTok{, }\FloatTok{7.8}\NormalTok{, }\FloatTok{18.1}\NormalTok{, }\FloatTok{12.7}\NormalTok{)}
\end{Highlighting}
\end{Shaded}

Plot the normal probability plot. In R, the expected normal score is expected quantile if the sample is normally distributed. The normal score used in \texttt{qqnorm()} function is calculated by:
- first ordering the sample in increasing order,
- then the ith observation is corresponding to a normal score, calculated by

\[
z_i =\Phi^{-1}\left( \frac{i-a}{n+1-2a}\right)
\]

where \(n\) is the sample size, and \(a = 3/8\) for \(n \le 10\), and \(a=0.5\) for \(n>10\). Note: This is slightly different from the textbook Table III.

\begin{Shaded}
\begin{Highlighting}[]
\FunctionTok{normal.prob.plot}\NormalTok{(AGI, }\AttributeTok{col=}\StringTok{"turquoise"}\NormalTok{, }\AttributeTok{xlab=}\StringTok{"AGI"}\NormalTok{, }\AttributeTok{ylab=}\StringTok{"Expected Normal Score"}\NormalTok{)}
\end{Highlighting}
\end{Shaded}

\includegraphics{StatsTB_files/figure-latex/unnamed-chunk-148-1.pdf} \includegraphics{StatsTB_files/figure-latex/unnamed-chunk-148-2.pdf}

\begin{verbatim}
##       y normal.score
## 1   7.8   -1.7316644
## 2   9.7   -1.1503494
## 3  10.6   -0.8122178
## 4  12.7   -0.5485223
## 5  12.8   -0.3186394
## 6  18.1   -0.1046335
## 7  21.2    0.1046335
## 8  33.0    0.3186394
## 9  43.5    0.5485223
## 10 51.1    0.8122178
## 11 81.4    1.1503494
## 12 93.1    1.7316644
\end{verbatim}

The normal probability plot is apparently not linear, so the AGI is not normally distributed.

You can also check normality by boxplot in a rough fashion.

\begin{Shaded}
\begin{Highlighting}[]
\FunctionTok{boxplot}\NormalTok{(AGI, }\AttributeTok{col=}\StringTok{"turquoise"}\NormalTok{)}
\end{Highlighting}
\end{Shaded}

\includegraphics{StatsTB_files/figure-latex/unnamed-chunk-149-1.pdf}

\hypertarget{use-normal-probability-plot-to-detect-outliers}{%
\subsection{Use normal probability plot to detect outliers}\label{use-normal-probability-plot-to-detect-outliers}}

Chicken consumption in pounds is shown below.

\begin{Shaded}
\begin{Highlighting}[]
\NormalTok{y }\OtherTok{\textless{}{-}} \FunctionTok{c}\NormalTok{(}\DecValTok{57}\NormalTok{, }\DecValTok{69}\NormalTok{, }\DecValTok{63}\NormalTok{, }\DecValTok{49}\NormalTok{, }\DecValTok{63}\NormalTok{, }\DecValTok{61}\NormalTok{, }\DecValTok{72}\NormalTok{, }\DecValTok{65}\NormalTok{, }\DecValTok{91}\NormalTok{, }\DecValTok{59}\NormalTok{, }\DecValTok{0}\NormalTok{, }\DecValTok{82}\NormalTok{, }\DecValTok{60}\NormalTok{, }\DecValTok{75}\NormalTok{, }\DecValTok{55}\NormalTok{, }\DecValTok{80}\NormalTok{, }\DecValTok{73}\NormalTok{)}

\FunctionTok{normal.prob.plot}\NormalTok{(y, }\AttributeTok{col=}\StringTok{"turquoise"}\NormalTok{, }\AttributeTok{xlab=}\StringTok{"Chicken Consumption (LB)"}\NormalTok{, }\AttributeTok{ylab=}\StringTok{"Expected Normal Score"}\NormalTok{)}
\end{Highlighting}
\end{Shaded}

\includegraphics{StatsTB_files/figure-latex/unnamed-chunk-150-1.pdf} \includegraphics{StatsTB_files/figure-latex/unnamed-chunk-150-2.pdf}

\begin{verbatim}
##     y normal.score
## 1   0   -1.8895100
## 2  49   -1.3517022
## 3  55   -1.0491314
## 4  57   -0.8207921
## 5  59   -0.6289042
## 6  60   -0.4578519
## 7  61   -0.2993069
## 8  63   -0.1479871
## 9  63    0.0000000
## 10 65    0.1479871
## 11 69    0.2993069
## 12 72    0.4578519
## 13 73    0.6289042
## 14 75    0.8207921
## 15 80    1.0491314
## 16 82    1.3517022
## 17 91    1.8895100
\end{verbatim}

The observation 0 is detected as one outlier.

\hypertarget{the-sampling-distribution-of-the-sample-mean-barx}{%
\chapter{\texorpdfstring{The Sampling Distribution of the Sample Mean (\(\bar{x}\))}{The Sampling Distribution of the Sample Mean (\textbackslash bar\{x\})}}\label{the-sampling-distribution-of-the-sample-mean-barx}}

\begin{Shaded}
\begin{Highlighting}[]
\FunctionTok{library}\NormalTok{(IntroStats)}
\end{Highlighting}
\end{Shaded}

\hypertarget{sampling-error}{%
\section{Sampling Error}\label{sampling-error}}

\hypertarget{definition-sampling-error}{%
\subsection{Definition: Sampling Error}\label{definition-sampling-error}}

\textbf{Sampling error} is the error resulting from using a sample to
estimate a population characteristic.

\hypertarget{example-sampling-error-in-irs-tax-return-estimates}{%
\subsection{Example: Sampling Error in IRS Tax Return Estimates}\label{example-sampling-error-in-irs-tax-return-estimates}}

IRS publishes annual figures on individual income tax returns. For the
year 2010, the IRS reported that the mean tax of individual income tax
returns was \$11266. Actually, the IRS reported the mean tax of a sample
of 308,946 individual tax returns from a total of more than 130 million
such returns.

\textbf{Question:}

\begin{itemize}
\item
  Is the reported mean tax by IRS a sample mean or population mean?
\item
  Should we expect the mean of all 130 million tax returns is exactly
  the same as \$11266?
\item
  Why would IRS do this way?
\item
  How can we answer the question: is the sample mean tax \(\bar{x}\)
  likely to be within \$100 of the population mean?
\end{itemize}

\hypertarget{definition-sampling-distribution-of-the-sample-mean}{%
\subsection{Definition: Sampling Distribution of the Sample Mean}\label{definition-sampling-distribution-of-the-sample-mean}}

For a variable \(x\) and a given sample size, the distribution of the
variable \(\bar{x}\) is called the sampling distribution of the sampling
mean.

\hypertarget{example-heights-of-starring-players}{%
\subsection{Example: Heights of Starring Players}\label{example-heights-of-starring-players}}

Consider 5 players' heights:

\begin{Shaded}
\begin{Highlighting}[]
\NormalTok{Player }\OtherTok{\textless{}{-}} \FunctionTok{c}\NormalTok{(}\StringTok{"A"}\NormalTok{, }\StringTok{"B"}\NormalTok{, }\StringTok{"C"}\NormalTok{, }\StringTok{"D"}\NormalTok{, }\StringTok{"E"}\NormalTok{)}
\NormalTok{Height }\OtherTok{\textless{}{-}} \FunctionTok{c}\NormalTok{(}\DecValTok{76}\NormalTok{, }\DecValTok{78}\NormalTok{, }\DecValTok{79}\NormalTok{, }\DecValTok{81}\NormalTok{, }\DecValTok{86}\NormalTok{)}

\NormalTok{dat }\OtherTok{\textless{}{-}} \FunctionTok{data.frame}\NormalTok{(}\FunctionTok{cbind}\NormalTok{(Player, Height))}
\NormalTok{dat}
\end{Highlighting}
\end{Shaded}

\begin{verbatim}
##   Player Height
## 1      A     76
## 2      B     78
## 3      C     79
## 4      D     81
## 5      E     86
\end{verbatim}

Consider we sample 2 from this population of 5 players.

\begin{enumerate}
\def\labelenumi{\alph{enumi}.}
\tightlist
\item
  Obtain the sampling distribution of the sample mean of sample size

  \begin{enumerate}
  \def\labelenumii{\arabic{enumii}.}
  \setcounter{enumii}{1}
  \tightlist
  \item
  \end{enumerate}
\item
  Find the probability that the sampling error made in estimating the
  population mean by the sample mean will be 1 inch or less?
\end{enumerate}

Try sampling below

\begin{Shaded}
\begin{Highlighting}[]
\FunctionTok{sample}\NormalTok{(Height, }\AttributeTok{size =} \DecValTok{2}\NormalTok{)}
\end{Highlighting}
\end{Shaded}

\begin{verbatim}
## [1] 79 78
\end{verbatim}

\textbf{Solution:}

\begin{enumerate}
\def\labelenumi{\alph{enumi}.}
\tightlist
\item
  Because there are only 5 players in this population, all possible
  samples are listed below. For each sample, we can calculate the
  sample mean. There a total of \(\binom{5}{2}=10\) possible samples.
\end{enumerate}

\begin{table}

\caption{\label{tab:unnamed-chunk-154}Samples with corresponding heights and their means}
\centering
\begin{tabular}[t]{l|l|r}
\hline
Sample & Heights & Mean\_xbar\\
\hline
A, B & 76, 78 & 77.0\\
\hline
A, C & 76, 79 & 77.5\\
\hline
A, D & 76, 81 & 78.5\\
\hline
A, E & 76, 86 & 81.0\\
\hline
B, C & 78, 79 & 78.5\\
\hline
B, D & 78, 81 & 79.5\\
\hline
B, E & 78, 86 & 82.0\\
\hline
C, D & 79, 81 & 80.0\\
\hline
C, E & 79, 86 & 82.5\\
\hline
D, E & 81, 86 & 83.5\\
\hline
\end{tabular}
\end{table}

The population mean is 80.

\begin{Shaded}
\begin{Highlighting}[]
\FunctionTok{mean}\NormalTok{(Height)}
\end{Highlighting}
\end{Shaded}

\begin{verbatim}
## [1] 80
\end{verbatim}

However, only 1 sample has the sample mean 80!

To obtain the sample mean (\(\bar{x}\))'s distribution, draw all possible
sample means.

\begin{Shaded}
\begin{Highlighting}[]
\NormalTok{xbar }\OtherTok{\textless{}{-}} \FunctionTok{c}\NormalTok{(}\DecValTok{77}\NormalTok{, }\FloatTok{77.5}\NormalTok{, }\FloatTok{78.5}\NormalTok{, }\DecValTok{81}\NormalTok{, }\FloatTok{78.5}\NormalTok{, }\FloatTok{79.5}\NormalTok{, }\DecValTok{82}\NormalTok{, }\DecValTok{80}\NormalTok{, }\FloatTok{82.5}\NormalTok{, }\FloatTok{83.5}\NormalTok{)}
\FunctionTok{stripchart}\NormalTok{(xbar, }\AttributeTok{method=}\StringTok{"stack"}\NormalTok{, }\AttributeTok{pch=}\DecValTok{1}\NormalTok{,}\AttributeTok{cex=}\DecValTok{2}\NormalTok{)}
\FunctionTok{abline}\NormalTok{(}\AttributeTok{v=}\DecValTok{80}\NormalTok{, }\AttributeTok{col=}\StringTok{"red"}\NormalTok{)}
\end{Highlighting}
\end{Shaded}

\includegraphics{StatsTB_files/figure-latex/unnamed-chunk-156-1.pdf}

\begin{figure}
\centering
\includegraphics{https://i.ibb.co/PTF3DTF/310-Figure-07-01.png}
\caption{Players}
\end{figure}

The mean of sample means is also 80.

\begin{Shaded}
\begin{Highlighting}[]
\FunctionTok{mean}\NormalTok{(xbar)}
\end{Highlighting}
\end{Shaded}

\begin{verbatim}
## [1] 80
\end{verbatim}

\textbf{Is this coincidence?}

\begin{enumerate}
\def\labelenumi{\alph{enumi}.}
\setcounter{enumi}{1}
\tightlist
\item
  Because this the dot plot shows all samples of size 2. There are 3
  samples having means within 1 inch of the population mean 80. The
  probability is 3/10 = 0.3.
\end{enumerate}

This means a random sample of size 2 will have 0,3 probability having
its sample mean \(\bar{x}\) falls within 1 inch of the population mean.

\hypertarget{example-heights-1}{%
\subsection{Example: Heights}\label{example-heights-1}}

Consider we sample 4 plays. There are a total of \(\binom{5}{4}=5\)
samples.

\%\includegraphics{https://i.ibb.co/bLTCYm6/310-Table-07-03.png}

\begin{table}

\caption{\label{tab:unnamed-chunk-158}Four-element samples with corresponding heights and means}
\centering
\begin{tabular}[t]{l|l|r}
\hline
Sample & Heights & Mean\_xbar\\
\hline
A, B, C, D & 76, 78, 79, 81 & 78.50\\
\hline
A, B, C, E & 76, 78, 79, 86 & 79.75\\
\hline
A, B, D, E & 76, 78, 81, 86 & 80.25\\
\hline
A, C, D, E & 76, 79, 81, 86 & 80.50\\
\hline
B, C, D, E & 78, 79, 81, 86 & 81.00\\
\hline
\end{tabular}
\end{table}

The dot plot of the sample means is below.

\begin{figure}
\centering
\includegraphics{https://i.ibb.co/fMJpHQb/310-Figure-07-02.png}
\caption{Players}
\end{figure}

None of the 5 samples has a sample mean of 80! However, the mean of
sample means is 80!

\begin{Shaded}
\begin{Highlighting}[]
\FunctionTok{mean}\NormalTok{(}\FunctionTok{c}\NormalTok{(}\FloatTok{78.5}\NormalTok{, }\FloatTok{79.75}\NormalTok{, }\FloatTok{80.25}\NormalTok{, }\FloatTok{80.5}\NormalTok{, }\DecValTok{81}\NormalTok{))}
\end{Highlighting}
\end{Shaded}

\begin{verbatim}
## [1] 80
\end{verbatim}

Compared to the previous sample size of 2, when the sample size is 5,
the dots are getting closer to each other. The probability of sample
mean within 1 inch of population mean 80 is 4/5 = 0.8. It is much larger
than 0.3. This indicates the sampling error is smaller!

\hypertarget{sampling-error-vs-sample-size}{%
\subsection{Sampling error vs sample size}\label{sampling-error-vs-sample-size}}

\begin{figure}
\centering
\includegraphics{https://i.ibb.co/89cfSZ0/311-Figure-07-03.png}
\caption{Players}
\end{figure}

We calculate the probability of within 1 inch and within 0.5 inch of
population mean \(\mu\).

\begin{figure}
\centering
\includegraphics{https://i.ibb.co/6W8CNCn/311-Table-07-04.png}
\caption{Players}
\end{figure}

The larger the sample size, the smaller the sampling error tends to be
in estimating a population mean \(\mu\) by a sample mean \(\bar{x}\).

\hypertarget{example-systolic-blood-pressure-1}{%
\subsection{Example: Systolic Blood Pressure}\label{example-systolic-blood-pressure-1}}

\textbf{Introduction}

In this simulation, we use the publicly available \textbf{NHANES} dataset to
investigate how \textbf{sampling error} varies with \textbf{sample size} when
estimating a population mean --- in this case, the \textbf{average systolic
blood pressure (SBP)} of adults.

We demonstrate that as sample size increases, the sampling error (i.e.,
standard deviation of sample mean \(\bar{x}\)) decreases.

\textbf{Load Data and Prepare Sample}

\begin{Shaded}
\begin{Highlighting}[]
\FunctionTok{library}\NormalTok{(NHANES)}
\FunctionTok{library}\NormalTok{(dplyr)}
\FunctionTok{library}\NormalTok{(ggplot2)}

\CommentTok{\# Filter data to include only adults with non{-}missing SBP}
\NormalTok{data }\OtherTok{\textless{}{-}}\NormalTok{ NHANES }\SpecialCharTok{\%\textgreater{}\%}
  \FunctionTok{filter}\NormalTok{(}\SpecialCharTok{!}\FunctionTok{is.na}\NormalTok{(BPSysAve), Age }\SpecialCharTok{\textgreater{}=} \DecValTok{18}\NormalTok{) }\SpecialCharTok{\%\textgreater{}\%}
\NormalTok{  dplyr}\SpecialCharTok{::}\FunctionTok{select}\NormalTok{(BPSysAve)}

\CommentTok{\# View the population mean SBP (used as "true" mean)}
\NormalTok{true\_mu }\OtherTok{\textless{}{-}} \FunctionTok{mean}\NormalTok{(data}\SpecialCharTok{$}\NormalTok{BPSysAve)}
\NormalTok{true\_mu}
\end{Highlighting}
\end{Shaded}

\begin{verbatim}
## [1] 120.7274
\end{verbatim}

\textbf{Simulate Sampling with Different Sample Sizes}

We take repeated random samples (with replacement) of varying sizes and compute the mean of each sample.

\begin{Shaded}
\begin{Highlighting}[]
\FunctionTok{set.seed}\NormalTok{(}\DecValTok{123}\NormalTok{)}

\NormalTok{sample\_sizes }\OtherTok{\textless{}{-}} \FunctionTok{c}\NormalTok{(}\DecValTok{10}\NormalTok{, }\DecValTok{30}\NormalTok{, }\DecValTok{100}\NormalTok{, }\DecValTok{300}\NormalTok{)}
\NormalTok{n\_sim }\OtherTok{\textless{}{-}} \DecValTok{1000}

\CommentTok{\# Simulate sampling process}
\NormalTok{sampling\_results }\OtherTok{\textless{}{-}} \FunctionTok{lapply}\NormalTok{(sample\_sizes, }\ControlFlowTok{function}\NormalTok{(n) \{}
  \FunctionTok{replicate}\NormalTok{(n\_sim, \{}
\NormalTok{    sample\_mean }\OtherTok{\textless{}{-}} \FunctionTok{mean}\NormalTok{(}\FunctionTok{sample}\NormalTok{(data}\SpecialCharTok{$}\NormalTok{BPSysAve, n, }\AttributeTok{replace =} \ConstantTok{TRUE}\NormalTok{))}
\NormalTok{    sample\_mean}
\NormalTok{  \})}
\NormalTok{\})}

\FunctionTok{names}\NormalTok{(sampling\_results) }\OtherTok{\textless{}{-}} \FunctionTok{paste0}\NormalTok{(}\StringTok{"n = "}\NormalTok{, sample\_sizes)}

\CommentTok{\# Combine results into a data frame for plotting}
\NormalTok{df\_plot }\OtherTok{\textless{}{-}} \FunctionTok{do.call}\NormalTok{(rbind, }\FunctionTok{lapply}\NormalTok{(}\FunctionTok{names}\NormalTok{(sampling\_results), }\ControlFlowTok{function}\NormalTok{(name) \{}
  \FunctionTok{data.frame}\NormalTok{(}\AttributeTok{SampleMean =}\NormalTok{ sampling\_results[[name]],}
             \AttributeTok{SampleSize =}\NormalTok{ name)}
\NormalTok{\}))}

\FunctionTok{ggplot}\NormalTok{(df\_plot, }\FunctionTok{aes}\NormalTok{(}\AttributeTok{x =}\NormalTok{ SampleMean, }\AttributeTok{fill =}\NormalTok{ SampleSize)) }\SpecialCharTok{+}
  \FunctionTok{geom\_density}\NormalTok{(}\AttributeTok{alpha =} \FloatTok{0.5}\NormalTok{) }\SpecialCharTok{+}
  \FunctionTok{geom\_vline}\NormalTok{(}\AttributeTok{xintercept =}\NormalTok{ true\_mu, }\AttributeTok{linetype =} \StringTok{"dashed"}\NormalTok{, }\AttributeTok{color =} \StringTok{"red"}\NormalTok{) }\SpecialCharTok{+}
  \FunctionTok{labs}\NormalTok{(}\AttributeTok{title =} \StringTok{"Distribution of Sample Means for Different Sample Sizes"}\NormalTok{,}
       \AttributeTok{x =} \StringTok{"Sample Mean of SBP"}\NormalTok{,}
       \AttributeTok{y =} \StringTok{"Density"}\NormalTok{,}
       \AttributeTok{fill =} \StringTok{"Sample Size"}\NormalTok{) }\SpecialCharTok{+}
  \FunctionTok{theme\_minimal}\NormalTok{()}
\end{Highlighting}
\end{Shaded}

\includegraphics{StatsTB_files/figure-latex/unnamed-chunk-161-1.pdf}

\textbf{Calculate Sampling Error}

\begin{Shaded}
\begin{Highlighting}[]
\NormalTok{sampling\_errors }\OtherTok{\textless{}{-}} \FunctionTok{sapply}\NormalTok{(sampling\_results, sd)}
\NormalTok{sampling\_errors\_df }\OtherTok{\textless{}{-}} \FunctionTok{data.frame}\NormalTok{(}
  \AttributeTok{SampleSize =}\NormalTok{ sample\_sizes,}
  \AttributeTok{SamplingError =}\NormalTok{ sampling\_errors}
\NormalTok{)}
\NormalTok{sampling\_errors\_df}
\end{Highlighting}
\end{Shaded}

\begin{verbatim}
##         SampleSize SamplingError
## n = 10          10     5.2898176
## n = 30          30     3.0337289
## n = 100        100     1.6559829
## n = 300        300     0.9993809
\end{verbatim}

\textbf{Interpretation}

As shown in both the plot and the summary table, sampling error
decreases as sample size increases. When \(n\) is small (e.g., 10), the
sample means vary widely. When \(n\) is large (e.g., 300), the sample
means cluster tightly around the population mean \(\mu\).

\hypertarget{the-mean-and-standard-deviation-of-the-sample-mean}{%
\section{The mean and standard deviation of the sample mean}\label{the-mean-and-standard-deviation-of-the-sample-mean}}

\hypertarget{mean-of-the-sample-mean}{%
\subsection{Mean of the Sample Mean}\label{mean-of-the-sample-mean}}

The mean of the sample mean equals the population mean for any sample
size \(n\).

\[
\mu_{\bar{x}} = \mu
\]

\hypertarget{example-heights-of-players}{%
\subsection{Example: Heights of players}\label{example-heights-of-players}}

Consider a sample of size 2 from the 5 players.

\begin{enumerate}
\def\labelenumi{\alph{enumi}.}
\tightlist
\item
  Determine the population mean \(\mu\)
\item
  Obtain the mean of the sample mean \(\mu_{\bar{x}}\)
\end{enumerate}

\textbf{solution:}

The population mean is the average of the entire population: \(\mu = 80\).

The mean of a sample mean of size 2 equals the population mean \(80\).

\hypertarget{standard-deviation-of-the-sample-mean}{%
\subsection{Standard deviation of the sample mean}\label{standard-deviation-of-the-sample-mean}}

For samples of size n, the standard deviation of \(\bar{x}\) equals the
population standard deviation divided by the square root of the sample
size.

\[
\sigma_{\bar{x}} = \frac{\sigma}{\sqrt{n}}
\]

\hypertarget{example-living-space}{%
\subsection{Example: Living space}\label{example-living-space}}

The mean living space for single-family detached homes is 1742 sq ft.
Assume a standard deviation of 568 sq. ft.

\begin{enumerate}
\def\labelenumi{\alph{enumi}.}
\tightlist
\item
  For a sample of 25 homes, determine the mean and standard deviation
  of \(\bar{x}\).
\item
  Repeat for a sample size of 500.
\end{enumerate}

\textbf{solution.}

\begin{enumerate}
\def\labelenumi{\alph{enumi}.}
\item
  \(\mu_{\bar{x}} = \mu=1742\) sq. ft. and
  \(\sigma_{\bar{x}} = 568/\sqrt{25} = 113.6\) sq. ft
\item
  \(\mu_{\bar{x}} = \mu=1742\) sq. ft. and
  \(\sigma_{\bar{x}} = 568/\sqrt{500} = 25.4\) sq. ft
\end{enumerate}

\hypertarget{definition-standard-error}{%
\subsection{Definition: Standard Error}\label{definition-standard-error}}

Because the standard deviation of \(\bar{x}\) determines the amount of
sampling error to be expected when a population mean is estimated by a
sample mean, it is often referred as the \textbf{standard error of the sample
mean}.

In general, the standard deviation of a statistic used to estimate a
parameter is called the \textbf{standard error (SE)} of the statistic.

\hypertarget{example-estimating-mean-blood-pressure-and-standard-error}{%
\subsection{Example: Estimating Mean Blood Pressure and Standard Error}\label{example-estimating-mean-blood-pressure-and-standard-error}}

Suppose we are interested in estimating the average systolic blood
pressure (SBP) of adult females in a population. The true population
mean is unknown, but from previous studies, the population standard
deviation is assumed to be \(\sigma = 12\) mmHg.

We take a random sample of \(n = 36\) adult females and find that their
sample mean is \(\bar{x} = 122\) mmHg.

\begin{Shaded}
\begin{Highlighting}[]
\CommentTok{\# Given values}
\NormalTok{sigma }\OtherTok{\textless{}{-}} \DecValTok{12}       \CommentTok{\# population standard deviation}
\NormalTok{n }\OtherTok{\textless{}{-}} \DecValTok{36}           \CommentTok{\# sample size}
\NormalTok{x\_bar }\OtherTok{\textless{}{-}} \DecValTok{122}      \CommentTok{\# sample mean}

\CommentTok{\# Standard Error of the Mean}
\NormalTok{SE }\OtherTok{\textless{}{-}}\NormalTok{ sigma }\SpecialCharTok{/} \FunctionTok{sqrt}\NormalTok{(n)}
\NormalTok{SE}
\end{Highlighting}
\end{Shaded}

\begin{verbatim}
## [1] 2
\end{verbatim}

\textbf{Interpretation} The population standard deviation is \(\sigma = 12\)
mmHg. The sample size is \(n = 36\), so the standard error of the sample
mean is: \[
SE = \frac{\sigma}{\sqrt{n}} = \frac{12}{\sqrt{36}} = 2 mmHg
\] This means that \(\bar{x} = 122\) mmHg is expected to vary by
approximately \(\pm 2\) mmHg from sample to sample due to sampling
variability.

\hypertarget{the-sampling-distribution-of-the-sample-mean-barx-1}{%
\section{\texorpdfstring{The sampling distribution of the sample mean \(\bar{x}\)}{The sampling distribution of the sample mean \textbackslash bar\{x\}}}\label{the-sampling-distribution-of-the-sample-mean-barx-1}}

\hypertarget{distribution-of-sample-mean-barx}{%
\subsection{\texorpdfstring{Distribution of sample mean \(\bar{x}\)}{Distribution of sample mean \textbackslash bar\{x\}}}\label{distribution-of-sample-mean-barx}}

Recall, sample mean \(\bar{x}\) with a sample size \(n\) is \[
\bar{x} = \frac{1}{n}(x_1+\cdots+x_n)
\] where each \(x_i\) is a random variable, drawn from a normal
distribution. Then \(\bar{x}\) is also a random variable following a
normal distribution, with mean \(\mu\) same as the mean of \(x_i\); and the
variance of \(\bar{x}\) is variance of \(x_i\) divided by \(n\). In other
words, let

\[
x_i\sim N(\mu, \sigma)
\] then \[
\bar{x} \sim N(\mu, \sigma^2/n)
\] \(\bar{x}\) and \(x_i\) have the same mean, but the variance of \(\bar{x}\)
is a fraction \(\frac{1}{n}\) of variance of \(x\).

\hypertarget{illustration-of-sampling-distribution-of-x_bar}{%
\subsection{Illustration of sampling distribution of x\_bar}\label{illustration-of-sampling-distribution-of-x_bar}}

\begin{Shaded}
\begin{Highlighting}[]
\CommentTok{\#Illustration of sampling distribution of x\_bar}
\FunctionTok{draw.dist.xbar.normal}\NormalTok{(}\AttributeTok{n=}\DecValTok{10}\NormalTok{, }\AttributeTok{sigma=}\DecValTok{2}\NormalTok{, }\AttributeTok{mu=}\DecValTok{10}\NormalTok{, }
                      \AttributeTok{main=}\StringTok{"Distribution of x\_bar: A sample of 10 from N(2, 10)"}\NormalTok{)       }
\end{Highlighting}
\end{Shaded}

\includegraphics{StatsTB_files/figure-latex/unnamed-chunk-164-1.pdf}

\begin{Shaded}
\begin{Highlighting}[]
\FunctionTok{draw.dist.xbar.normal}\NormalTok{(}\AttributeTok{n=}\DecValTok{50}\NormalTok{, }\AttributeTok{sigma=}\DecValTok{2}\NormalTok{, }\AttributeTok{mu=}\DecValTok{10}\NormalTok{, }
                      \AttributeTok{main=}\StringTok{"Distribution of x\_bar: A sample of 50 from N(2, 10)"}\NormalTok{)       }
\end{Highlighting}
\end{Shaded}

\includegraphics{StatsTB_files/figure-latex/unnamed-chunk-164-2.pdf}

\hypertarget{example-cholesterol-level}{%
\subsection{Example: Cholesterol Level}\label{example-cholesterol-level}}

According to the national health and nutrition examination survey of
1988-1994, the estimated mean serum cholesterol level \((\mu)\) for US
females aged 20-74 years to be 204 mg/dl and the standard deviation
\((\sigma)\) is 44. Consider these as the US population mean and standard
deviation.

\textbf{Question}. For a random sample of 50 women in this age group, what is
the probability that the sample mean (\(\bar{x}\)) is greater than 220
mg/dl?

\textbf{Solution:}

The sample mean \(\bar{x}\) approximately follows a normal distribution
with mean \(\mu\) and variance \(\sigma^2/n\). In this problem, \(n=50\),
\(\sigma=44\), so the standard error is \(44/\sqrt{50} = 6.22\). Then \[
            P(\bar{x} > 220) = 1- P(\bar{x} \le 220)            
\]\\
which is calculated as \texttt{1-pnorm(220,\ mean=204,\ sd=44/sqrt(50))=0.005}.

\begin{Shaded}
\begin{Highlighting}[]
\DecValTok{1}\SpecialCharTok{{-}}\FunctionTok{pnorm}\NormalTok{(}\DecValTok{220}\NormalTok{, }\AttributeTok{mean=}\DecValTok{204}\NormalTok{, }\AttributeTok{sd=}\DecValTok{44}\SpecialCharTok{/}\FunctionTok{sqrt}\NormalTok{(}\DecValTok{50}\NormalTok{))}
\end{Highlighting}
\end{Shaded}

\begin{verbatim}
## [1] 0.005065914
\end{verbatim}

\hypertarget{sample-mean-implications-in-scientific-research}{%
\subsection{Sample Mean: Implications in Scientific Research}\label{sample-mean-implications-in-scientific-research}}

In scientific research, the population mean \(\mu\) is unknown and the
objective of the research is to estimate \(\mu\). The distribution of the
sample mean implies:

\begin{itemize}
\tightlist
\item
  It provides one way to estimate \(\mu\).
\item
  As \(n\) becomes large, the standard error becomes small.
\item
  Furthermore, the mean of \(\bar{x}\) is equal to \(\mu\).
\item
  As a result, if we draw a large random sample, the average of the
  observed sample data can be a good estimate of \(\mu\).
\item
  In statistics, we use \(\hat{\mu}\) to denote the estimated mean based
  on a sample, herein \(\hat{\mu} = \bar{x}\), where \(\bar{x}\) is the
  average of the observed values \((x_1, \cdots, x_n)\).
\item
  The sample mean's variance is inversely proportional to \(n\).
\end{itemize}

\hypertarget{central-limit-theorem}{%
\subsection{Central limit theorem}\label{central-limit-theorem}}

The sample mean \(\bar{x}\) approximately follows a normal distribution
with mean \(\mu\) and variance \(\sigma^2/n\) when the sample size \(n\) is
large, where \(\mu\) is the mean of \(x\) and \(\sigma^2\) is the variance of
\(X\), for any distribution of \(x\). As a result,
\[z=\frac{\bar{x}-\mu}{\sigma/\sqrt{n}}\] approximately follows the
standard normal distribution \(N(0, 1)\).

\hypertarget{sampling-distribution}{%
\subsection{Sampling Distribution}\label{sampling-distribution}}

\begin{enumerate}
\def\labelenumi{\alph{enumi}.}
\tightlist
\item
  Sample mean \(\bar{x}\) has a mean equals population mean
  \(\mu_{\bar{x}} = \mu\).
\item
  The standard deviation of \(\bar{x}\):
  \(\sigma_{\bar{x}} = \sigma/\sqrt{n}\).
\item
  If \(x\) is normally distributed, so is \(\bar{x}\), regardless of
  sample size.
\item
  if \(n\) is large, \(\bar{x}\) is approximately normally distributed,
  regardless of the distribution of \(x\).
\end{enumerate}

\begin{figure}
\centering
\includegraphics{https://i.ibb.co/TB36LnD/323-Figure-07-06.png}
\caption{Central Limit Theorem}
\end{figure}

\hypertarget{example-birth-weight}{%
\subsection{Example: Birth weight}\label{example-birth-weight}}

Birth weighs of male babies have a standard deviation of 1.33 lb.
Determine the percentage of all samples of 400 male babies that have
mean birth weights within 0.125 lb of the population mean birth weight
of all male babies.

\textbf{Solution}

Let \(\mu\) denote the population mean birth weight of all male babies.
For a sample of size 400, the sample mean birth weight \(\bar{x}\) is
approximately normally distributed with

\[
\mu_{\bar{x}} = \mu
\]
and

\[
\sigma_{\bar{x}} = 1.33/\sqrt{400} = 0.0665
\]
The probability of \(\bar{x}\) within 0.125 lb of \(\mu\) can be calculated
as

\[
P(-0.125 < \bar{x} - \mu < 0.125) = P\left(\frac{-0.125}{0.0665} < \frac{\bar{x} - \mu}{0.0665} < \frac{0.125}{0.0665}\right) = \Phi\left(\frac{0.125}{0.0665}\right) - \Phi\left(\frac{-0.125}{0.0665}\right) = 0.9398
\]

\begin{Shaded}
\begin{Highlighting}[]
\FunctionTok{pnorm}\NormalTok{(}\FloatTok{0.125}\SpecialCharTok{/}\FloatTok{0.0665}\NormalTok{) }\SpecialCharTok{{-}} \FunctionTok{pnorm}\NormalTok{(}\SpecialCharTok{{-}}\FloatTok{0.125}\SpecialCharTok{/}\FloatTok{0.0665}\NormalTok{)}
\end{Highlighting}
\end{Shaded}

\begin{verbatim}
## [1] 0.9398509
\end{verbatim}

\hypertarget{confidence-interval-of-the-population-mean}{%
\chapter{Confidence Interval of the Population Mean}\label{confidence-interval-of-the-population-mean}}

\begin{Shaded}
\begin{Highlighting}[]
\FunctionTok{library}\NormalTok{(IntroStats)}
\end{Highlighting}
\end{Shaded}

\hypertarget{estimating-a-population-mean}{%
\section{Estimating a Population Mean}\label{estimating-a-population-mean}}

\hypertarget{sampling-distribution-of-barx}{%
\subsection{\texorpdfstring{Sampling distribution of \(\bar{x}\)}{Sampling distribution of \textbackslash bar\{x\}}}\label{sampling-distribution-of-barx}}

Recall, sample mean \(\bar{x}\) with a sample size \(n\) is \[
\bar{x} = \frac{1}{n}(x_1+\cdots+x_n)
\] where each \(x_i\) is a random variable, drawn from a normal
distribution. Then \(\bar{x}\) is also a random variable following a
normal distribution, with mean \(\mu\) same as the mean of \(x_i\); and the
variance of \(\bar{x}\) is variance of \(x_i\) divided by \(n\). In other
words, let

\[
x_i\sim N(\mu, \sigma)
\] then \[
\bar{x} \sim N(\mu, \sigma^2/n)
\] \(\bar{x}\) and \(x_i\) have the same mean, but the variance of \(\bar{x}\)
is a fraction \(\frac{1}{n}\) of variance of \(x\).

\hypertarget{definition-point-estimate-and-ci-estimate}{%
\subsection{Definition: Point estimate and CI estimate}\label{definition-point-estimate-and-ci-estimate}}

A common problem in statistics is to estimate the population mean. Why
do we need to estimate a population mean? For example, * The mean cost
of a wedding in New Jersey, * The mean starting salary of a college
graduate with a statistics major, * The mean prolongation of survival
time for patients diagnosed with stage IV lung cancer when treated with
one new drug compared to the old therapy.

A \textbf{point estimate} of a parameter is the value of a statistic used to
estimate the parameter.

\textbf{Unbiased estimator} of a parameter means the mean of all its possible
values equals the parameter. The sample mean \(\bar{x}\) is an unbiased
estimator of the population mean. Why?

A sample mean is usually not equal to the population mean. Therefore we
should accompany any point estimate of \(\mu\) with information indicating
the accuracy. This information is called a \textbf{confidence interval
estimate} for \(\mu\).

\hypertarget{definition-confidence-interval-estimate}{%
\subsection{Definition: confidence interval estimate}\label{definition-confidence-interval-estimate}}

\begin{itemize}
\tightlist
\item
  \textbf{Confidence interval (CI)}: An interval of numbers obtained from a
  point estimate of a parameter.
\item
  \textbf{Confidence level}: The confidence we have that the parameter lies
  in the confidence interval (i.e., that the confidence interval
  contains the parameter).
\item
  \textbf{Confidence-interval estimate}: The confidence level and
  confidence interval.
\end{itemize}

\includegraphics{https://i.ibb.co/SNXWSGx/336-Figure-08-02.png} 25 samples were
obtained and each sample has its own sample mean as an estimate of
population mean \(\mu\), and each sample results in a 95\% confidence
interval. One confidence interval here does not include the population
mean \(\mu\).

This is exactly the reason why we say we are 95\% confident that the CI
covers the population mean. This means that if we generate a large
number of samples (i.e.~if a sample represents one scientific
experiment, we repeat a large number of times), we expect 95\% of the
experiments will generate the confidence intervals covering the true
population mean \(\mu\). It is very important to have the correct
interpretation of confidence interval.

\hypertarget{example-estimating-the-average-commute-time}{%
\subsection{Example: Estimating the Average Commute Time}\label{example-estimating-the-average-commute-time}}

Suppose a transportation researcher wants to estimate the \textbf{average
commute time} \(\mu\) for all full-time workers in a large city. Since it
is impractical to survey every worker, she takes a \emph{random sample} of
\(n = 64\) workers and records their commute times. The sample has:

\begin{itemize}
\tightlist
\item
  Sample mean: \(\bar{x} = 35\) minutes\\
\item
  Sample standard deviation: \(s = 8\) minutes
\end{itemize}

She wants to construct a \textbf{95\% confidence interval} for the population
mean \(\mu\).

\textbf{Step 1: Use the confidence interval formula}

\[
\bar{x} \pm z^* \cdot \frac{s}{\sqrt{n}}
\]

Since \(n = 64\) is large, we can use the normal distribution. The
critical value for 95\% confidence is \(z^* = 1.96\).

\textbf{Step 2: Calculate standard error (SE)}

\[
\text{SE} = \frac{s}{\sqrt{n}} = \frac{8}{\sqrt{64}} = \frac{8}{8} = 1
\]

\textbf{Step 3: Calculate margin of error}

\[
\text{Margin of error} = z^* \cdot \text{SE} = 1.96 \cdot 1 = 1.96
\]

\textbf{Step 4: Construct confidence interval}

\[
\bar{x} \pm 1.96 = 35 \pm 1.96 = (33.04,\ 36.96)
\]

\textbf{Interpretation}

We are 95\% confident that the true average commute time \(\mu\) for all
full-time workers in this city lies between 33.04 minutes and 36.96
minutes.

\begin{quote}
\textbf{Note:} This does \emph{not} mean that 95\% of the population has commute
times in this interval.\\
Instead, it means that if we took many random samples of 64 workers,
approximately 95\% of the confidence intervals we compute would contain
the true population mean \(\mu\).
\end{quote}

\textbf{Simulating Confidence Interval Coverage}

To illustrate how confidence intervals behave across repeated sampling,
we simulate 25 random samples from a population with known mean
\(\mu = 35\) and standard deviation \(\sigma = 8\). Each sample consists of
\(n = 64\) observations. We construct a 95\% confidence interval for each
sample and check whether it contains the true population mean. The goal
is to visualize the fact that approximately 95\% of confidence intervals
generated from repeated sampling will contain the true mean.

\begin{Shaded}
\begin{Highlighting}[]
\FunctionTok{set.seed}\NormalTok{(}\DecValTok{2025}\NormalTok{)}

\NormalTok{mu }\OtherTok{\textless{}{-}} \DecValTok{35}       \CommentTok{\# true population mean}
\NormalTok{sigma }\OtherTok{\textless{}{-}} \DecValTok{8}     \CommentTok{\# population standard deviation}
\NormalTok{n }\OtherTok{\textless{}{-}} \DecValTok{64}        \CommentTok{\# sample size}
\NormalTok{z }\OtherTok{\textless{}{-}} \FloatTok{1.96}      \CommentTok{\# z* for 95\% confidence}

\CommentTok{\# Generate 25 confidence intervals from repeated samples}
\NormalTok{samples }\OtherTok{\textless{}{-}} \FunctionTok{replicate}\NormalTok{(}\DecValTok{25}\NormalTok{, \{}
\NormalTok{  xbar }\OtherTok{\textless{}{-}} \FunctionTok{mean}\NormalTok{(}\FunctionTok{rnorm}\NormalTok{(n, mu, sigma))}
\NormalTok{  se }\OtherTok{\textless{}{-}}\NormalTok{ sigma }\SpecialCharTok{/} \FunctionTok{sqrt}\NormalTok{(n)}
  \FunctionTok{c}\NormalTok{(}\AttributeTok{lower =}\NormalTok{ xbar }\SpecialCharTok{{-}}\NormalTok{ z }\SpecialCharTok{*}\NormalTok{ se, }\AttributeTok{upper =}\NormalTok{ xbar }\SpecialCharTok{+}\NormalTok{ z }\SpecialCharTok{*}\NormalTok{ se)}
\NormalTok{\})}

\CommentTok{\# Convert to data frame for plotting}
\NormalTok{cis }\OtherTok{\textless{}{-}} \FunctionTok{as.data.frame}\NormalTok{(}\FunctionTok{t}\NormalTok{(samples))}
\NormalTok{cis}\SpecialCharTok{$}\NormalTok{sample }\OtherTok{\textless{}{-}} \DecValTok{1}\SpecialCharTok{:}\FunctionTok{nrow}\NormalTok{(cis)}
\NormalTok{cis}\SpecialCharTok{$}\NormalTok{contains\_mu }\OtherTok{\textless{}{-}} \FunctionTok{with}\NormalTok{(cis, mu }\SpecialCharTok{\textgreater{}=}\NormalTok{ lower }\SpecialCharTok{\&}\NormalTok{ mu }\SpecialCharTok{\textless{}=}\NormalTok{ upper)}

\FunctionTok{library}\NormalTok{(ggplot2)}

\FunctionTok{ggplot}\NormalTok{(cis, }\FunctionTok{aes}\NormalTok{(}\AttributeTok{y =} \FunctionTok{reorder}\NormalTok{(}\FunctionTok{factor}\NormalTok{(sample), }\SpecialCharTok{{-}}\NormalTok{sample))) }\SpecialCharTok{+}
  \FunctionTok{geom\_errorbarh}\NormalTok{(}\FunctionTok{aes}\NormalTok{(}\AttributeTok{xmin =}\NormalTok{ lower, }\AttributeTok{xmax =}\NormalTok{ upper, }\AttributeTok{color =}\NormalTok{ contains\_mu), }\AttributeTok{height =} \FloatTok{0.4}\NormalTok{) }\SpecialCharTok{+}
  \FunctionTok{geom\_vline}\NormalTok{(}\AttributeTok{xintercept =}\NormalTok{ mu, }\AttributeTok{linetype =} \StringTok{"dashed"}\NormalTok{, }\AttributeTok{color =} \StringTok{"black"}\NormalTok{) }\SpecialCharTok{+}
  \FunctionTok{scale\_color\_manual}\NormalTok{(}\AttributeTok{values =} \FunctionTok{c}\NormalTok{(}\StringTok{"TRUE"} \OtherTok{=} \StringTok{"steelblue"}\NormalTok{, }\StringTok{"FALSE"} \OtherTok{=} \StringTok{"red"}\NormalTok{)) }\SpecialCharTok{+}
  \FunctionTok{labs}\NormalTok{(}
    \AttributeTok{x =} \StringTok{"Confidence Interval"}\NormalTok{,}
    \AttributeTok{y =} \StringTok{"Sample ID"}\NormalTok{,}
    \AttributeTok{color =} \StringTok{"Contains μ?"}\NormalTok{,}
    \AttributeTok{title =} \StringTok{"95\% Confidence Intervals from 25 Samples"}
\NormalTok{  ) }\SpecialCharTok{+}
  \FunctionTok{theme\_minimal}\NormalTok{()}
\end{Highlighting}
\end{Shaded}

\includegraphics{StatsTB_files/figure-latex/unnamed-chunk-168-1.pdf}

\hypertarget{ci-for-population-mean-when-sigma-is-known}{%
\section{\texorpdfstring{CI for population mean when \(\sigma\) is known}{CI for population mean when \textbackslash sigma is known}}\label{ci-for-population-mean-when-sigma-is-known}}

\hypertarget{the-general-formulation-of-confidence-interval}{%
\subsection{The general formulation of confidence interval}\label{the-general-formulation-of-confidence-interval}}

The general formulation of confidence interval is expressed as \[
        estimator \pm (\text{specified reliability coefficient})\times (\text{standard error})
        \]

For the estimation of \(\mu\), the \((1-\alpha)\times 100\%\) CI is \[
    (\bar{x}-z_{1-\alpha/2}\times se, \bar{x}+z_{1-\alpha/2}\times se) = 
    (\bar{x}-z_{1-\alpha/2}\times \sigma/\sqrt{n}, \bar{x}+z_{1-\alpha/2}\times \sigma/\sqrt{n})
\] where \(z_{1-\alpha/2}\) is the \((1-\alpha/2)\) quantile of the standard
normal distribution \texttt{qnorm(1-alpha/2)}.

\hypertarget{one-mean-z-interval-procedure}{%
\subsection{One-mean z-interval procedure}\label{one-mean-z-interval-procedure}}

\begin{figure}
\centering
\includegraphics{https://i.ibb.co/NW4RpR0/procedure8-1.png}
\caption{One-mean z-interval
procedure}
\end{figure}

We have this procedure because of the observation below.

\begin{figure}
\centering
\includegraphics{https://i.ibb.co/C1W5dQL/339-Figure-08-03.png}
\caption{Approximately 95\% of all samples have means within 2 standard
deviations of \(\mu\)}
\end{figure}

\hypertarget{example-99-ci-for-mean-muscle-strength-with-known-variance-sigma2-144}{%
\subsection{\texorpdfstring{Example: 99\% CI for Mean Muscle Strength with Known Variance (\(\sigma^2 = 144\))}{Example: 99\% CI for Mean Muscle Strength with Known Variance (\textbackslash sigma\^{}2 = 144)}}\label{example-99-ci-for-mean-muscle-strength-with-known-variance-sigma2-144}}

Consider a physical therapist would like to estimate the mean maximal
strength of a particular muscle in a certain population who was
diagnosed with muscular dystrophy. It's assumed the maximal muscle
strength is approximately normally distributed with a variance of 144. A
random sample of 15 subjects participated in the research project and
resulted in a sample mean of \(\bar{x} = 84.3\).

\textbf{Question.} Find a 99\% confidence interval for the mean maximal
strength.

The standard error is \(\sigma/\sqrt{n}=12/\sqrt{15}=3.0984\). For 99\%CI,
\(\alpha = 1-0.99 = 0.01\).

The reliability coefficient, sometimes also called critical value,
\(z_{1-\alpha/2} = qnorm(1-0.01/2) = qnorm(0.995) = 2.58\).

As a result, the 99\% CI of \(\mu\) is
\(84.3\pm 2.58 \sqrt{\frac{144}{15}}=84.3\pm 8.0 = (76.3, 92.3)\).

\begin{Shaded}
\begin{Highlighting}[]
\CommentTok{\#Use R function}
\FunctionTok{ci.mu}\NormalTok{(}\AttributeTok{xbar=}\FloatTok{84.3}\NormalTok{, }\AttributeTok{n=}\DecValTok{15}\NormalTok{, }\AttributeTok{sd =} \FunctionTok{sqrt}\NormalTok{(}\DecValTok{144}\NormalTok{), }\AttributeTok{conflev=}\FloatTok{0.99}\NormalTok{, }\AttributeTok{method =} \StringTok{"Known Variance"}\NormalTok{)}
\end{Highlighting}
\end{Shaded}

\begin{verbatim}
## [1] 76.31908 92.28092
\end{verbatim}

\begin{Shaded}
\begin{Highlighting}[]
        \CommentTok{\#[1] 76.31908 92.28092}
\end{Highlighting}
\end{Shaded}

\textbf{Interpretation:} We can be 95\% confident that the mean maximal
strength is somewhere between 76.32 and 92.28.

\hypertarget{definition-margin-of-error}{%
\subsection{Definition: Margin of Error}\label{definition-margin-of-error}}

\textbf{Margin of Error} for the estimate of \(\mu\) is denoted as \(E\) \[
E = z_{1-\alpha/2}\cdot \sigma /\sqrt{n}
\]

\begin{figure}
\centering
\includegraphics{https://i.ibb.co/BqGrCbJ/342-Figure-08-05.png}
\caption{Margin of Error}
\end{figure}

The margin of error is half the CI length.

\textbf{Example}

The margin of error of 99\%CI in the above example is \[
E = z_{1-\alpha/2}\cdot \sigma /\sqrt{n} = 2.58 \sqrt{\frac{144}{15}} = 7.99
\]

The margin of error of 95\%CI in the above example is \[
E = z_{1-\alpha/2}\cdot \sigma /\sqrt{n} = 1.96 \sqrt{\frac{144}{15}} = 6.07
\]

\begin{Shaded}
\begin{Highlighting}[]
\NormalTok{x }\OtherTok{\textless{}{-}} \FunctionTok{rnorm}\NormalTok{(}\DecValTok{30}\NormalTok{)}
\FunctionTok{margin.error.mu}\NormalTok{(}\AttributeTok{x=}\NormalTok{x, }\AttributeTok{sigma=}\DecValTok{1}\NormalTok{, }\AttributeTok{conflev=}\FloatTok{0.95}\NormalTok{, }\AttributeTok{method=}\StringTok{"Known Variance"}\NormalTok{)}
\end{Highlighting}
\end{Shaded}

\begin{verbatim}
## [1] 0.4092223
\end{verbatim}

\hypertarget{margin-of-error-vs-sample-size-and-confidence-level}{%
\subsection{Margin of Error vs Sample Size and Confidence Level}\label{margin-of-error-vs-sample-size-and-confidence-level}}

The margin of error decreases when the confidence level decreases.

\begin{Shaded}
\begin{Highlighting}[]
\NormalTok{conflev }\OtherTok{\textless{}{-}} \FunctionTok{seq}\NormalTok{(}\FloatTok{0.9}\NormalTok{, }\FloatTok{0.99}\NormalTok{, }\FloatTok{0.01}\NormalTok{)}
\NormalTok{E }\OtherTok{\textless{}{-}} \FunctionTok{margin.error.mu}\NormalTok{(}\AttributeTok{x=}\NormalTok{x, }\AttributeTok{sigma=}\DecValTok{1}\NormalTok{, }\AttributeTok{conflev=}\NormalTok{conflev, }\AttributeTok{method=}\StringTok{"Known Variance"}\NormalTok{)}
\FunctionTok{plot}\NormalTok{(conflev, E, }\AttributeTok{type=}\StringTok{"l"}\NormalTok{, }\AttributeTok{ylab=}\StringTok{"Margin of Error (E)"}\NormalTok{, }\AttributeTok{xlab=}\StringTok{"Confidence Level"}\NormalTok{,}
     \AttributeTok{col=}\NormalTok{mycol, }\AttributeTok{lwd=}\DecValTok{3}\NormalTok{)}
\end{Highlighting}
\end{Shaded}

\includegraphics{StatsTB_files/figure-latex/unnamed-chunk-171-1.pdf}

In addition, the margin of error decreases when \(n\) increases.

\begin{Shaded}
\begin{Highlighting}[]
\NormalTok{n }\OtherTok{\textless{}{-}} \FunctionTok{seq}\NormalTok{(}\DecValTok{20}\NormalTok{, }\DecValTok{50}\NormalTok{, }\DecValTok{1}\NormalTok{)}
\NormalTok{E }\OtherTok{\textless{}{-}} \FunctionTok{margin.error.mu}\NormalTok{(}\AttributeTok{sigma=}\DecValTok{1}\NormalTok{, }\AttributeTok{conflev=}\FloatTok{0.95}\NormalTok{, }\AttributeTok{n=}\NormalTok{n, }\AttributeTok{method=}\StringTok{"Known Variance"}\NormalTok{)}
\FunctionTok{plot}\NormalTok{(n, E, }\AttributeTok{type=}\StringTok{"l"}\NormalTok{, }\AttributeTok{ylab=}\StringTok{"Margin of Error (E)"}\NormalTok{, }\AttributeTok{xlab=}\StringTok{"Sample Size"}\NormalTok{,}
     \AttributeTok{col=}\NormalTok{mycol, }\AttributeTok{lwd=}\DecValTok{3}\NormalTok{, }\AttributeTok{main=}\StringTok{"95\% CI of mu"}\NormalTok{)}
\end{Highlighting}
\end{Shaded}

\includegraphics{StatsTB_files/figure-latex/unnamed-chunk-172-1.pdf}

\hypertarget{sample-size-required-to-estimate-mu}{%
\subsection{\texorpdfstring{Sample Size required to estimate \(\mu\)}{Sample Size required to estimate \textbackslash mu}}\label{sample-size-required-to-estimate-mu}}

The sample size required for a \((1-\alpha)\) level confidence interval
for \(\mu\) with a specified margin of error \(E\) is given by \[
n = \left(\frac{z_{1-\alpha/2}\cdot \sigma}{E}\right)^2
\]

round up to a whole number.

\textbf{Example}

Determine the sample size needed in order to be 95\% confident that \(\mu\)
is within 5 of the point estimate \(\bar{x}\). Then find the 95\%CI with
the calculated sample size.

\textbf{Solution.}

\[
n = \left(\frac{z_{1-\alpha/2}\cdot \sigma}{E}\right)^2 = \left(\frac{z_{0.975}\cdot 12}{5}\right)^2 = 22.13 \approx 23
\]

round up to 23. Why?

The minimum sample size of 23 is required in order to achieve the margin
of error of 5.

\begin{Shaded}
\begin{Highlighting}[]
\FunctionTok{n.mu.CI}\NormalTok{(}\AttributeTok{E=}\DecValTok{5}\NormalTok{, }\AttributeTok{sigma=}\DecValTok{12}\NormalTok{, }\AttributeTok{conflev =} \FloatTok{0.95}\NormalTok{, }\AttributeTok{method =} \StringTok{"Z"}\NormalTok{)}
\end{Highlighting}
\end{Shaded}

\begin{verbatim}
## [1] 23
\end{verbatim}

The 95\% confidence interval is \[
\bar{x} \pm z_{0.975}\cdot \frac{12}{\sqrt{23}} = 84.3 \pm 4.90 = (79.4, 89.2)
\]

\begin{Shaded}
\begin{Highlighting}[]
\FunctionTok{ci.mu}\NormalTok{(}\AttributeTok{xbar=}\FloatTok{84.3}\NormalTok{, }\AttributeTok{n=}\DecValTok{23}\NormalTok{, }\AttributeTok{sd=}\DecValTok{12}\NormalTok{, }\AttributeTok{conflev =} \FloatTok{0.95}\NormalTok{, }\AttributeTok{method=}\StringTok{"Known Variance"}\NormalTok{)}
\end{Highlighting}
\end{Shaded}

\begin{verbatim}
## [1] 79.39583 89.20417
\end{verbatim}

\hypertarget{confidence-interval-of-population-mean-mu-when-sigma-unknown}{%
\section{\texorpdfstring{Confidence interval of population mean \(\mu\) when \(\sigma\) unknown}{Confidence interval of population mean \textbackslash mu when \textbackslash sigma unknown}}\label{confidence-interval-of-population-mean-mu-when-sigma-unknown}}

In the previous section, the confidence interval for \(\mu\) is derived
based on assumption of \(\sigma^2\) known. However, in scientific
research, the population variance is not known.

What is the solution?

\hypertarget{distribution}{%
\subsection{Distribution}\label{distribution}}

When the population standard deviation \(\sigma\) is replaced by sample
standard deviation \(s\), the distribution below \[
        t = \frac{\bar{x}-\mu}{s/\sqrt{n}}
        \] follows a \(t\) distribution with \(n-1\) degree of freedom,
where \(s=\sqrt{\sum (x_i-\bar{x})^2/(n-1)}\). As expected, the \(t\)
distribution is very similar to the normal distribution, but
incorporating more variability in \(s\).

\hypertarget{compare-the-t-distribution-with-n0-1}{%
\subsection{Compare the t-distribution with N(0, 1)}\label{compare-the-t-distribution-with-n0-1}}

\begin{Shaded}
\begin{Highlighting}[]
    \DocumentationTok{\#\#\#\#\#\#\#\#\#\#\#\#\#\#\#\#\#\#\#\#\#\#\#\#\#\#\#\#\#\#\#\#\#\#\#\#\#\#}
        \CommentTok{\# t{-}distribution}
        \DocumentationTok{\#\#\#\#\#\#\#\#\#\#\#\#\#\#\#\#\#\#\#\#\#\#\#\#\#\#\#\#\#\#\#\#\#\#\#\#\#\#}
\NormalTok{        x }\OtherTok{\textless{}{-}} \FunctionTok{seq}\NormalTok{(}\SpecialCharTok{{-}}\DecValTok{4}\NormalTok{, }\DecValTok{4}\NormalTok{, }\AttributeTok{by=}\FloatTok{0.01}\NormalTok{)}
\NormalTok{        z }\OtherTok{\textless{}{-}} \FunctionTok{dnorm}\NormalTok{(x) }\CommentTok{\#standard normal}
\NormalTok{        t2 }\OtherTok{\textless{}{-}} \FunctionTok{dt}\NormalTok{(x, }\AttributeTok{df=}\DecValTok{2}\NormalTok{)}
\NormalTok{        t5 }\OtherTok{\textless{}{-}} \FunctionTok{dt}\NormalTok{(x, }\AttributeTok{df=}\DecValTok{5}\NormalTok{)}
\NormalTok{        t20 }\OtherTok{\textless{}{-}} \FunctionTok{dt}\NormalTok{(x, }\AttributeTok{df=}\DecValTok{20}\NormalTok{)}
\NormalTok{        t30 }\OtherTok{\textless{}{-}} \FunctionTok{dt}\NormalTok{(x, }\AttributeTok{df=}\DecValTok{30}\NormalTok{)}
        \FunctionTok{plot}\NormalTok{(x, z, }\AttributeTok{type=}\StringTok{"l"}\NormalTok{, }\AttributeTok{lwd=}\DecValTok{2}\NormalTok{, }
             \AttributeTok{main=}\StringTok{"t distribution approaches to Normal as df increases"}\NormalTok{)}
        \FunctionTok{lines}\NormalTok{(x, t2,  }\AttributeTok{col=}\DecValTok{2}\NormalTok{, }\AttributeTok{lwd=}\DecValTok{2}\NormalTok{, }\AttributeTok{lty=}\DecValTok{2}\NormalTok{)}
        \FunctionTok{lines}\NormalTok{(x, t5,  }\AttributeTok{col=}\DecValTok{3}\NormalTok{, }\AttributeTok{lwd=}\DecValTok{2}\NormalTok{, }\AttributeTok{lty=}\DecValTok{3}\NormalTok{)}
        \FunctionTok{lines}\NormalTok{(x, t20, }\AttributeTok{col=}\DecValTok{4}\NormalTok{, }\AttributeTok{lwd=}\DecValTok{2}\NormalTok{, }\AttributeTok{lty=}\DecValTok{4}\NormalTok{)}
        \FunctionTok{lines}\NormalTok{(x, t30, }\AttributeTok{col=}\DecValTok{5}\NormalTok{, }\AttributeTok{lwd=}\DecValTok{2}\NormalTok{, }\AttributeTok{lty=}\DecValTok{5}\NormalTok{)}
        \FunctionTok{legend}\NormalTok{(}\StringTok{"topleft"}\NormalTok{, }\AttributeTok{legend=}\FunctionTok{c}\NormalTok{(}\StringTok{"N(0, 1)"}\NormalTok{, }\StringTok{"t(df=2)"}\NormalTok{, }\StringTok{"t(df=5)"}\NormalTok{, }\StringTok{"t(df=20)"}\NormalTok{, }\StringTok{"t(df=30)"}\NormalTok{),}
        \AttributeTok{bty=}\StringTok{"n"}\NormalTok{, }\AttributeTok{lwd=}\FunctionTok{rep}\NormalTok{(}\DecValTok{2}\NormalTok{,}\DecValTok{5}\NormalTok{), }\AttributeTok{col=}\DecValTok{1}\SpecialCharTok{:}\DecValTok{5}\NormalTok{, }\AttributeTok{lty=}\DecValTok{1}\SpecialCharTok{:}\DecValTok{5}\NormalTok{)}
\end{Highlighting}
\end{Shaded}

\includegraphics{StatsTB_files/figure-latex/unnamed-chunk-175-1.pdf}

\hypertarget{basic-properties-of-t-curve}{%
\subsection{\texorpdfstring{Basic properties of \(t\)-curve}{Basic properties of t-curve}}\label{basic-properties-of-t-curve}}

Basic Properties of t-Curves * Property 1: The total area under a
t-curve equals 1. * Property 2: A t-curve extends indefinitely in both
directions, approaching, but never touching, the horizontal axis as it
does so. * Property 3: A t-curve is symmetric about 0. * Property 4:
As the number of degrees of freedom becomes larger, t-curves look
increasingly like the standard normal curve.

\hypertarget{confidence-interval-using-t}{%
\subsection{\texorpdfstring{Confidence interval using \(t\)}{Confidence interval using t}}\label{confidence-interval-using-t}}

When sampling is from The \(100(1-\alpha)\%\) CI of \(\mu\) is given by \[
    \bar{x} \pm t_{1-\alpha/2}\frac{s}{\sqrt{n}}
    \] where \(t_{1-\alpha/2}\) is the \((1-\alpha/2)\)th quantile of the
\(t\) distribution with degree of freedom \(n-1\), in R, use
\verb+qt(1-alpha/2, df=n-1)+. This CI can be similarly derived.

\begin{figure}
\centering
\includegraphics{https://i.ibb.co/PG3Sbpf/margin-of-error-t.png}
\caption{Margin of Error and
t-interval}
\end{figure}

\hypertarget{example-blood-pressure}{%
\subsection{Example: Blood Pressure}\label{example-blood-pressure}}

Consider a random sample of blood pressure is (90, 88, 92, 93, 91, 88,
92, 93, 91, 84, 93, 91, 84, 93, 91, 84, 70, 86, 75, 100, 120, 110). (a)
If the population variance is assumed as 120, what's the 95\%CI of the
population mean \(\mu\)? (b) If the population variance is unknown, what's
the 95\%CI of the population mean \(\mu\)?

\textbf{Solution}.

\begin{enumerate}
\def\labelenumi{(\alph{enumi})}
\tightlist
\item
  With the population variance known, use the normal distribution to
  calculate the CI as (86.28615, 95.44113). The 95\%CI of the
  population mean \(\mu\) is
\end{enumerate}

\[
(\bar{x}-z_{0.975}\times \sigma/\sqrt{n}, \bar{x}+z_{0.975}\times \sigma/\sqrt{n})
= (90.86-1.96\sqrt{120/22}, 90.86+1.96\sqrt{120/22})=(86.29, 95.44).
\] The sample mean \(\bar{x} = 90.86\) and sample size \(n=22\).

\begin{Shaded}
\begin{Highlighting}[]
\NormalTok{x }\OtherTok{\textless{}{-}} \FunctionTok{c}\NormalTok{(}\DecValTok{90}\NormalTok{, }\DecValTok{88}\NormalTok{, }\DecValTok{92}\NormalTok{, }\DecValTok{93}\NormalTok{, }\DecValTok{91}\NormalTok{, }\DecValTok{88}\NormalTok{, }\DecValTok{92}\NormalTok{, }\DecValTok{93}\NormalTok{, }\DecValTok{91}\NormalTok{, }\DecValTok{84}\NormalTok{, }\DecValTok{93}\NormalTok{, }\DecValTok{91}\NormalTok{, }\DecValTok{84}\NormalTok{, }\DecValTok{93}\NormalTok{, }\DecValTok{91}\NormalTok{, }\DecValTok{84}\NormalTok{, }\DecValTok{70}\NormalTok{, }\DecValTok{86}\NormalTok{, }\DecValTok{75}\NormalTok{, }\DecValTok{100}\NormalTok{, }\DecValTok{120}\NormalTok{, }\DecValTok{110}\NormalTok{)}
\NormalTok{n }\OtherTok{\textless{}{-}} \FunctionTok{length}\NormalTok{(x)}
\NormalTok{n}
\end{Highlighting}
\end{Shaded}

\begin{verbatim}
## [1] 22
\end{verbatim}

\begin{Shaded}
\begin{Highlighting}[]
\NormalTok{xbar }\OtherTok{\textless{}{-}} \FunctionTok{mean}\NormalTok{(x)}
\NormalTok{xbar}
\end{Highlighting}
\end{Shaded}

\begin{verbatim}
## [1] 90.86364
\end{verbatim}

\begin{Shaded}
\begin{Highlighting}[]
\CommentTok{\#use R function}
\FunctionTok{ci.mu}\NormalTok{(}\AttributeTok{x=}\NormalTok{x, }\AttributeTok{sd=}\FunctionTok{sqrt}\NormalTok{(}\DecValTok{120}\NormalTok{), }\AttributeTok{n=}\FunctionTok{length}\NormalTok{(x), }\AttributeTok{conflev=}\FloatTok{0.95}\NormalTok{, }\AttributeTok{method =} \StringTok{"Known Variance"}\NormalTok{)}
\end{Highlighting}
\end{Shaded}

\begin{verbatim}
## [1] 86.28615 95.44113
\end{verbatim}

\begin{enumerate}
\def\labelenumi{(\arabic{enumi})}
\setcounter{enumi}{1}
\tightlist
\item
  Without known population variance, use the t-distribution to
  calculate the CI as (86.33733, 95.38994). See below the R codes for
  implementation of the formula.
\end{enumerate}

The \(95\%\) CI of \(\mu\) is given by \[
    \bar{x} \pm t_{0.975}\frac{s}{\sqrt{n}} 
    \]

The 0.975 quantile of t-distribution with \(n-1 = 21\) degrees of freedom
is

\begin{Shaded}
\begin{Highlighting}[]
\FunctionTok{qt}\NormalTok{(}\FloatTok{0.975}\NormalTok{, }\AttributeTok{df=}\DecValTok{21}\NormalTok{)}
\end{Highlighting}
\end{Shaded}

\begin{verbatim}
## [1] 2.079614
\end{verbatim}

So the \(95\%\) CI of \(\mu\) is \[
    \bar{x} \pm t_{0.975}\frac{s}{\sqrt{n}} = 90.86 \pm 2.080\frac{\sqrt{120}}{\sqrt{22}}=(86.34, 95.39)
\]

\begin{Shaded}
\begin{Highlighting}[]
\CommentTok{\#use R function}
\FunctionTok{ci.mu}\NormalTok{(}\AttributeTok{x=}\NormalTok{x, }\AttributeTok{sd=}\FunctionTok{sqrt}\NormalTok{(}\DecValTok{120}\NormalTok{), }\AttributeTok{n=}\FunctionTok{length}\NormalTok{(x), }\AttributeTok{conflev=}\FloatTok{0.95}\NormalTok{, }\AttributeTok{method =} \StringTok{"Unknown Variance"}\NormalTok{)}
\end{Highlighting}
\end{Shaded}

\begin{verbatim}
## [1] 86.33733 95.38994
\end{verbatim}

\hypertarget{example-estimate-mean-cholesterol-level-in-diabetic-patients}{%
\subsection{Example: Estimate Mean Cholesterol Level in Diabetic Patients}\label{example-estimate-mean-cholesterol-level-in-diabetic-patients}}

A medical researcher collects fasting blood cholesterol levels from a
sample of 25 patients diagnosed with type 2 diabetes. The population
variance is unknown. The researcher wants to estimate the mean
cholesterol level using a 95\% confidence interval.

\begin{Shaded}
\begin{Highlighting}[]
\CommentTok{\# Simulated cholesterol data (in mg/dL)}
\NormalTok{chol }\OtherTok{\textless{}{-}} \FunctionTok{c}\NormalTok{(}\DecValTok{205}\NormalTok{, }\DecValTok{189}\NormalTok{, }\DecValTok{212}\NormalTok{, }\DecValTok{198}\NormalTok{, }\DecValTok{220}\NormalTok{, }\DecValTok{208}\NormalTok{, }\DecValTok{210}\NormalTok{, }\DecValTok{190}\NormalTok{, }\DecValTok{195}\NormalTok{, }\DecValTok{202}\NormalTok{, }
          \DecValTok{215}\NormalTok{, }\DecValTok{225}\NormalTok{, }\DecValTok{218}\NormalTok{, }\DecValTok{194}\NormalTok{, }\DecValTok{197}\NormalTok{, }\DecValTok{210}\NormalTok{, }\DecValTok{211}\NormalTok{, }\DecValTok{222}\NormalTok{, }\DecValTok{207}\NormalTok{, }\DecValTok{219}\NormalTok{, }
          \DecValTok{200}\NormalTok{, }\DecValTok{196}\NormalTok{, }\DecValTok{213}\NormalTok{, }\DecValTok{204}\NormalTok{, }\DecValTok{209}\NormalTok{)}

\NormalTok{n }\OtherTok{\textless{}{-}} \FunctionTok{length}\NormalTok{(chol)}
\NormalTok{xbar }\OtherTok{\textless{}{-}} \FunctionTok{mean}\NormalTok{(chol)}
\NormalTok{s }\OtherTok{\textless{}{-}} \FunctionTok{sd}\NormalTok{(chol)}
\NormalTok{alpha }\OtherTok{\textless{}{-}} \FloatTok{0.05}
\NormalTok{tcrit }\OtherTok{\textless{}{-}} \FunctionTok{qt}\NormalTok{(}\DecValTok{1} \SpecialCharTok{{-}}\NormalTok{ alpha}\SpecialCharTok{/}\DecValTok{2}\NormalTok{, }\AttributeTok{df =}\NormalTok{ n }\SpecialCharTok{{-}} \DecValTok{1}\NormalTok{)}

\CommentTok{\# Calculate confidence interval}
\NormalTok{se }\OtherTok{\textless{}{-}}\NormalTok{ s }\SpecialCharTok{/} \FunctionTok{sqrt}\NormalTok{(n)}
\NormalTok{lower }\OtherTok{\textless{}{-}}\NormalTok{ xbar }\SpecialCharTok{{-}}\NormalTok{ tcrit }\SpecialCharTok{*}\NormalTok{ se}
\NormalTok{upper }\OtherTok{\textless{}{-}}\NormalTok{ xbar }\SpecialCharTok{+}\NormalTok{ tcrit }\SpecialCharTok{*}\NormalTok{ se}
\FunctionTok{c}\NormalTok{(lower, upper)}
\end{Highlighting}
\end{Shaded}

\begin{verbatim}
## [1] 202.5804 210.9396
\end{verbatim}

\textbf{Interpretation}

Based on the sample of 38 diabetic patients, the \textbf{sample mean} serum
cholesterol level is:

\[
\bar{x} = 214.03 \text{ mg/dl}
\]

with a \textbf{standard deviation} of:

\[
s = 24.89 \text{ mg/dl}
\]

To estimate the population mean \(\mu\), we construct a 95\% confidence interval.Using the formula for the confidence interval of the mean with unknown
population standard deviation (i.e., using \(t\) distribution):

\[
\bar{x} \pm t \cdot \frac{s}{\sqrt{n}}
\]
Plugging in the values:

\begin{itemize}
\tightlist
\item
  \(\bar{x} = 214.03\)
\item
  \(s = 24.89\)
\item
  \(n = 38\)
\item
  degrees of freedom = \(n - 1 = 37\)
\item
  \(t^* \approx 2.026\) (from \(t\) table for 95\% CI and df = 37)
\end{itemize}

The margin of error is:

\[
ME = 2.026 \cdot \frac{24.89}{\sqrt{38}} \approx 8.17
\]

So the 95\% confidence interval is:

\[
[214.03 - 8.17,\ 214.03 + 8.17] = [205.86,\ 222.20]
\]
We are 95\% confident that the \textbf{true mean serum cholesterol level} among all diabetic patients is \textbf{between 205.86 and 222.20 mg/dl}. This interval reflects the uncertainty due to sampling variability. If we were to take many such samples of 38 diabetic patients, about 95\% of the resulting intervals would contain the true mean \(\mu\).

\hypertarget{comparison-of-t-and-z-confidence-interval}{%
\subsection{\texorpdfstring{Comparison of \(t\) and \(z\) Confidence Interval}{Comparison of t and z Confidence Interval}}\label{comparison-of-t-and-z-confidence-interval}}

The only difference between \(z\) confidence interval and \(t\) confidence
interval is the quantile \(z_{1-\alpha/2}\) or \(t_{1-\alpha/2}\), which are
applied to known or unknown variance. In practice, most problems have
the situation of unknown variance. However, when the sample size is
large, their difference is ignorable.

Let's try out some cases to understand better.

\begin{Shaded}
\begin{Highlighting}[]
\CommentTok{\#Scenario 1. moderate sample size n = 30, normally distributed}
\NormalTok{y }\OtherTok{\textless{}{-}} \FunctionTok{rnorm}\NormalTok{(}\DecValTok{30}\NormalTok{, }\AttributeTok{mean=}\DecValTok{10}\NormalTok{, }\AttributeTok{sd=}\DecValTok{2}\NormalTok{) }\CommentTok{\#generate random data}

\CommentTok{\#z CI, assuming population sd is the same as the sample sd}
\FunctionTok{ci.mu}\NormalTok{(}\AttributeTok{x=}\NormalTok{y, }\AttributeTok{sd=}\FunctionTok{sd}\NormalTok{(y), }\AttributeTok{n=}\FunctionTok{length}\NormalTok{(y), }\AttributeTok{method=}\StringTok{"Known Variance"}\NormalTok{)}
\end{Highlighting}
\end{Shaded}

\begin{verbatim}
## [1] 8.783864 9.970804
\end{verbatim}

\begin{Shaded}
\begin{Highlighting}[]
\CommentTok{\#t CI}
\FunctionTok{ci.mu}\NormalTok{(}\AttributeTok{x=}\NormalTok{y, }\AttributeTok{n=}\FunctionTok{length}\NormalTok{(y), }\AttributeTok{method=}\StringTok{"Unknown Variance"}\NormalTok{)}
\end{Highlighting}
\end{Shaded}

\begin{verbatim}
## [1] 8.758046 9.996623
\end{verbatim}

\begin{Shaded}
\begin{Highlighting}[]
\CommentTok{\#Scenario 2. increase sample size to n = 100, normally distributed}
\NormalTok{y }\OtherTok{\textless{}{-}} \FunctionTok{runif}\NormalTok{(}\DecValTok{100}\NormalTok{, }\AttributeTok{min=}\DecValTok{5}\NormalTok{, }\AttributeTok{max=}\DecValTok{15}\NormalTok{) }\CommentTok{\#generate random data}

\CommentTok{\#z CI, assuming population sd is the same as the sample sd}
\FunctionTok{ci.mu}\NormalTok{(}\AttributeTok{x=}\NormalTok{y, }\AttributeTok{sd=}\FunctionTok{sd}\NormalTok{(y), }\AttributeTok{n=}\FunctionTok{length}\NormalTok{(y), }\AttributeTok{method=}\StringTok{"Known Variance"}\NormalTok{)}
\end{Highlighting}
\end{Shaded}

\begin{verbatim}
## [1]  9.346256 10.515601
\end{verbatim}

\begin{Shaded}
\begin{Highlighting}[]
\CommentTok{\#t CI}
\FunctionTok{ci.mu}\NormalTok{(}\AttributeTok{x=}\NormalTok{y, }\AttributeTok{n=}\FunctionTok{length}\NormalTok{(y), }\AttributeTok{method=}\StringTok{"Unknown Variance"}\NormalTok{)}
\end{Highlighting}
\end{Shaded}

\begin{verbatim}
## [1]  9.339021 10.522836
\end{verbatim}

Check normality of sample

\begin{Shaded}
\begin{Highlighting}[]
\FunctionTok{normal.prob.plot}\NormalTok{(y)}
\end{Highlighting}
\end{Shaded}

\includegraphics{StatsTB_files/figure-latex/unnamed-chunk-182-1.pdf} \includegraphics{StatsTB_files/figure-latex/unnamed-chunk-182-2.pdf}

\begin{verbatim}
##             y normal.score
## 1    5.031208  -2.57582930
## 2    5.148692  -2.17009038
## 3    5.219274  -1.95996398
## 4    5.263227  -1.81191067
## 5    5.339283  -1.69539771
## 6    5.415322  -1.59819314
## 7    5.441777  -1.51410189
## 8    5.467191  -1.43953147
## 9    5.483946  -1.37220381
## 10   5.497340  -1.31057911
## 11   5.648867  -1.25356544
## 12   5.773727  -1.20035886
## 13   5.892934  -1.15034938
## 14   6.126185  -1.10306256
## 15   6.460414  -1.05812162
## 16   6.495523  -1.01522203
## 17   6.499562  -0.97411388
## 18   6.590502  -0.93458929
## 19   6.906946  -0.89647336
## 20   7.139368  -0.85961736
## 21   7.190694  -0.82389363
## 22   7.361542  -0.78919165
## 23   7.387913  -0.75541503
## 24   7.422961  -0.72247905
## 25   7.556390  -0.69030882
## 26   7.562756  -0.65883769
## 27   7.695367  -0.62800601
## 28   7.728853  -0.59776013
## 29   7.779660  -0.56805150
## 30   7.952629  -0.53883603
## 31   7.958800  -0.51007346
## 32   8.197201  -0.48172685
## 33   8.260574  -0.45376219
## 34   8.325473  -0.42614801
## 35   8.383069  -0.39885507
## 36   8.406904  -0.37185609
## 37   8.468342  -0.34512553
## 38   8.550455  -0.31863936
## 39   8.605275  -0.29237490
## 40   8.613822  -0.26631061
## 41   8.634403  -0.24042603
## 42   8.715113  -0.21470157
## 43   8.898201  -0.18911843
## 44   9.156098  -0.16365849
## 45   9.190212  -0.13830421
## 46   9.222787  -0.11303854
## 47   9.388055  -0.08784484
## 48   9.389894  -0.06270678
## 49   9.703686  -0.03760829
## 50   9.891614  -0.01253347
## 51   9.946975   0.01253347
## 52  10.150339   0.03760829
## 53  10.246924   0.06270678
## 54  10.253530   0.08784484
## 55  10.443937   0.11303854
## 56  10.446160   0.13830421
## 57  10.508668   0.16365849
## 58  10.787170   0.18911843
## 59  10.845014   0.21470157
## 60  10.972494   0.24042603
## 61  11.031316   0.26631061
## 62  11.066312   0.29237490
## 63  11.098482   0.31863936
## 64  11.143459   0.34512553
## 65  11.180918   0.37185609
## 66  11.197997   0.39885507
## 67  11.310726   0.42614801
## 68  11.395879   0.45376219
## 69  11.462636   0.48172685
## 70  11.488255   0.51007346
## 71  11.637244   0.53883603
## 72  11.764927   0.56805150
## 73  11.814204   0.59776013
## 74  12.074182   0.62800601
## 75  12.520860   0.65883769
## 76  12.644420   0.69030882
## 77  12.761729   0.72247905
## 78  12.893231   0.75541503
## 79  13.191015   0.78919165
## 80  13.344458   0.82389363
## 81  13.408174   0.85961736
## 82  13.414182   0.89647336
## 83  13.457970   0.93458929
## 84  13.458308   0.97411388
## 85  13.470941   1.01522203
## 86  13.483005   1.05812162
## 87  13.578108   1.10306256
## 88  13.734525   1.15034938
## 89  14.088297   1.20035886
## 90  14.126563   1.25356544
## 91  14.466866   1.31057911
## 92  14.473793   1.37220381
## 93  14.536933   1.43953147
## 94  14.594970   1.51410189
## 95  14.635440   1.59819314
## 96  14.639696   1.69539771
## 97  14.716288   1.81191067
## 98  14.873767   1.95996398
## 99  14.900604   2.17009038
## 100 14.970968   2.57582930
\end{verbatim}

\hypertarget{extension-what-if-it-is-not-normal}{%
\subsection{Extension: What if it is not normal?}\label{extension-what-if-it-is-not-normal}}

Use the nonparametric version Wilcoxon confidence interval of the
population median.

\begin{Shaded}
\begin{Highlighting}[]
\FunctionTok{wilcox.test}\NormalTok{(}\AttributeTok{x=}\NormalTok{y, }\AttributeTok{mu=}\DecValTok{10}\NormalTok{, }\AttributeTok{conf.level =} \FloatTok{0.95}\NormalTok{, }\AttributeTok{conf.int =} \ConstantTok{TRUE}\NormalTok{)}
\end{Highlighting}
\end{Shaded}

\begin{verbatim}
## 
##  Wilcoxon signed rank test with continuity correction
## 
## data:  y
## V = 2443, p-value = 0.7793
## alternative hypothesis: true location is not equal to 10
## 95 percent confidence interval:
##   9.314496 10.565490
## sample estimates:
## (pseudo)median 
##       9.928342
\end{verbatim}

\hypertarget{random-sampling}{%
\chapter{Random Sampling}\label{random-sampling}}

\begin{Shaded}
\begin{Highlighting}[]
\FunctionTok{library}\NormalTok{(IntroStats)}
\end{Highlighting}
\end{Shaded}

\hypertarget{random-number-generators}{%
\section{Random Number Generators}\label{random-number-generators}}

The \texttt{runif()} function generates random numbers from a \textbf{uniform distribution} over the interval {[}0, 1{]} by default.

\begin{Shaded}
\begin{Highlighting}[]
\FunctionTok{set.seed}\NormalTok{(}\DecValTok{123}\NormalTok{)}
\NormalTok{uniform\_sample }\OtherTok{\textless{}{-}} \FunctionTok{runif}\NormalTok{(}\DecValTok{100}\NormalTok{)}
\FunctionTok{hist}\NormalTok{(uniform\_sample, }\AttributeTok{col =} \StringTok{"skyblue"}\NormalTok{, }\AttributeTok{main =} \StringTok{"Uniform(0,1) Distribution"}\NormalTok{, }\AttributeTok{xlab =} \StringTok{"Value"}\NormalTok{)}
\end{Highlighting}
\end{Shaded}

\includegraphics{StatsTB_files/figure-latex/unnamed-chunk-185-1.pdf}

You can also generate numbers from any interval {[}a,b{]}:

\begin{Shaded}
\begin{Highlighting}[]
\FunctionTok{runif}\NormalTok{(}\DecValTok{5}\NormalTok{, }\AttributeTok{min =} \DecValTok{10}\NormalTok{, }\AttributeTok{max =} \DecValTok{20}\NormalTok{)}
\end{Highlighting}
\end{Shaded}

\begin{verbatim}
## [1] 15.99989 13.32824 14.88613 19.54474 14.82902
\end{verbatim}

\hypertarget{random-numbers-sampling-using-a-seed}{%
\subsection{Random Numbers Sampling Using a Seed}\label{random-numbers-sampling-using-a-seed}}

To ensure reproducibility, use set.seed() before generating random numbers. This ensures the same output each time.

\begin{Shaded}
\begin{Highlighting}[]
\FunctionTok{set.seed}\NormalTok{(}\DecValTok{2025}\NormalTok{)}
\FunctionTok{runif}\NormalTok{(}\DecValTok{3}\NormalTok{)}
\end{Highlighting}
\end{Shaded}

\begin{verbatim}
## [1] 0.7326202 0.4757614 0.5142159
\end{verbatim}

\hypertarget{random-numbers-from-a-normal-distribution}{%
\subsection{Random Numbers from a Normal Distribution}\label{random-numbers-from-a-normal-distribution}}

\begin{Shaded}
\begin{Highlighting}[]
\FunctionTok{set.seed}\NormalTok{(}\DecValTok{42}\NormalTok{)}
\NormalTok{normal\_sample }\OtherTok{\textless{}{-}} \FunctionTok{rnorm}\NormalTok{(}\DecValTok{1000}\NormalTok{, }\AttributeTok{mean =} \DecValTok{0}\NormalTok{, }\AttributeTok{sd =} \DecValTok{1}\NormalTok{)}
\FunctionTok{hist}\NormalTok{(normal\_sample, }\AttributeTok{col =} \StringTok{"salmon"}\NormalTok{, }\AttributeTok{main =} \StringTok{"Standard Normal Distribution"}\NormalTok{, }\AttributeTok{xlab =} \StringTok{"Value"}\NormalTok{)}
\end{Highlighting}
\end{Shaded}

\includegraphics{StatsTB_files/figure-latex/unnamed-chunk-188-1.pdf}

\hypertarget{random-numbers-from-a-binomial-distribution}{%
\subsection{Random Numbers from a Binomial Distribution}\label{random-numbers-from-a-binomial-distribution}}

\begin{Shaded}
\begin{Highlighting}[]
\FunctionTok{set.seed}\NormalTok{(}\DecValTok{10}\NormalTok{)}
\NormalTok{binomial\_sample }\OtherTok{\textless{}{-}} \FunctionTok{rbinom}\NormalTok{(}\DecValTok{1000}\NormalTok{, }\AttributeTok{size =} \DecValTok{10}\NormalTok{, }\AttributeTok{prob =} \FloatTok{0.5}\NormalTok{)}
\FunctionTok{hist}\NormalTok{(binomial\_sample, }\AttributeTok{col =} \StringTok{"lightgreen"}\NormalTok{, }\AttributeTok{main =} \StringTok{"Binomial(n=10, p=0.5)"}\NormalTok{, }\AttributeTok{xlab =} \StringTok{"Successes"}\NormalTok{)}
\end{Highlighting}
\end{Shaded}

\includegraphics{StatsTB_files/figure-latex/unnamed-chunk-189-1.pdf}

\hypertarget{random-numbers-from-a-poisson-distribution}{%
\subsection{Random Numbers from a Poisson Distribution}\label{random-numbers-from-a-poisson-distribution}}

Use rpois(n, lambda) where lambda is the average rate of occurrence.

\begin{Shaded}
\begin{Highlighting}[]
\NormalTok{poisson\_sample }\OtherTok{\textless{}{-}} \FunctionTok{rpois}\NormalTok{(}\DecValTok{1000}\NormalTok{, }\AttributeTok{lambda =} \DecValTok{5}\NormalTok{)}
\FunctionTok{hist}\NormalTok{(poisson\_sample, }\AttributeTok{col =} \StringTok{"lightblue"}\NormalTok{, }\AttributeTok{main =} \StringTok{"Poisson(λ = 5)"}\NormalTok{, }\AttributeTok{xlab =} \StringTok{"Count"}\NormalTok{)}
\end{Highlighting}
\end{Shaded}

\includegraphics{StatsTB_files/figure-latex/unnamed-chunk-190-1.pdf}

\hypertarget{random-numbers-from-a-t-distribution}{%
\subsection{Random Numbers from a t Distribution}\label{random-numbers-from-a-t-distribution}}

Use rt(n, df) for random values from a Student's t distribution.

\begin{Shaded}
\begin{Highlighting}[]
\NormalTok{t\_sample }\OtherTok{\textless{}{-}} \FunctionTok{rt}\NormalTok{(}\DecValTok{1000}\NormalTok{, }\AttributeTok{df =} \DecValTok{10}\NormalTok{)}
\FunctionTok{hist}\NormalTok{(t\_sample, }\AttributeTok{col =} \StringTok{"orange"}\NormalTok{, }\AttributeTok{main =} \StringTok{"t{-}distribution (df=10)"}\NormalTok{, }\AttributeTok{xlab =} \StringTok{"Value"}\NormalTok{)}
\end{Highlighting}
\end{Shaded}

\includegraphics{StatsTB_files/figure-latex/unnamed-chunk-191-1.pdf}

\hypertarget{random-numbers-from-an-f-distribution}{%
\subsection{Random Numbers from an F Distribution}\label{random-numbers-from-an-f-distribution}}

Use rf(n, df1, df2) to generate values from an F-distribution.

\begin{Shaded}
\begin{Highlighting}[]
\NormalTok{f\_sample }\OtherTok{\textless{}{-}} \FunctionTok{rf}\NormalTok{(}\DecValTok{1000}\NormalTok{, }\AttributeTok{df1 =} \DecValTok{5}\NormalTok{, }\AttributeTok{df2 =} \DecValTok{10}\NormalTok{)}
\FunctionTok{hist}\NormalTok{(f\_sample, }\AttributeTok{col =} \StringTok{"plum"}\NormalTok{, }\AttributeTok{main =} \StringTok{"F(5,10) Distribution"}\NormalTok{, }\AttributeTok{xlab =} \StringTok{"Value"}\NormalTok{, }\AttributeTok{xlim =} \FunctionTok{c}\NormalTok{(}\DecValTok{0}\NormalTok{,}\DecValTok{5}\NormalTok{))}
\end{Highlighting}
\end{Shaded}

\includegraphics{StatsTB_files/figure-latex/unnamed-chunk-192-1.pdf}

\hypertarget{random-numbers-from-a-chi-square-distribution}{%
\subsection{Random Numbers from a Chi-square Distribution}\label{random-numbers-from-a-chi-square-distribution}}

Use rchisq(n, df) for Chi-square distributed values.

\begin{Shaded}
\begin{Highlighting}[]
\NormalTok{chi\_sample }\OtherTok{\textless{}{-}} \FunctionTok{rchisq}\NormalTok{(}\DecValTok{1000}\NormalTok{, }\AttributeTok{df =} \DecValTok{4}\NormalTok{)}
\FunctionTok{hist}\NormalTok{(chi\_sample, }\AttributeTok{col =} \StringTok{"khaki"}\NormalTok{, }\AttributeTok{main =} \StringTok{"Chi{-}square(df=4)"}\NormalTok{, }\AttributeTok{xlab =} \StringTok{"Value"}\NormalTok{)}
\end{Highlighting}
\end{Shaded}

\includegraphics{StatsTB_files/figure-latex/unnamed-chunk-193-1.pdf}

\hypertarget{examples}{%
\subsection{Examples}\label{examples}}

\textbf{Example 1}

\begin{Shaded}
\begin{Highlighting}[]
\CommentTok{\#Sample from population}
\NormalTok{officials }\OtherTok{=} \FunctionTok{c}\NormalTok{(}\StringTok{"G"}\NormalTok{, }\StringTok{"L"}\NormalTok{, }\StringTok{"S"}\NormalTok{, }\StringTok{"A"}\NormalTok{, }\StringTok{"T"}\NormalTok{)}

\CommentTok{\#Draw a sample of 2 officials without replacement}
\FunctionTok{sample}\NormalTok{(officials, }\AttributeTok{size =} \DecValTok{2}\NormalTok{, }\AttributeTok{replace =} \ConstantTok{FALSE}\NormalTok{)}
\end{Highlighting}
\end{Shaded}

\begin{verbatim}
## [1] "A" "S"
\end{verbatim}

\begin{Shaded}
\begin{Highlighting}[]
\CommentTok{\#Draw a sample of 4 officials without replacement}
\FunctionTok{sample}\NormalTok{(officials, }\AttributeTok{size =} \DecValTok{4}\NormalTok{, }\AttributeTok{replace =} \ConstantTok{FALSE}\NormalTok{)}
\end{Highlighting}
\end{Shaded}

\begin{verbatim}
## [1] "S" "A" "G" "T"
\end{verbatim}

\begin{Shaded}
\begin{Highlighting}[]
\CommentTok{\#Draw a sample of 4 officials with replacement}
\FunctionTok{sample}\NormalTok{(officials, }\AttributeTok{size =} \DecValTok{4}\NormalTok{, }\AttributeTok{replace =} \ConstantTok{TRUE}\NormalTok{)}
\end{Highlighting}
\end{Shaded}

\begin{verbatim}
## [1] "A" "G" "S" "S"
\end{verbatim}

\textbf{Example2}

\begin{Shaded}
\begin{Highlighting}[]
\CommentTok{\#Stratified sampling proportional allocation}

\NormalTok{stratum1 }\OtherTok{=} \DecValTok{1}\SpecialCharTok{:}\DecValTok{400} \CommentTok{\#a set of integers from 1 to 400}
\NormalTok{stratum2 }\OtherTok{=} \DecValTok{401}\SpecialCharTok{:}\DecValTok{1000} \CommentTok{\#a set of integers from 401 to 1000}
\NormalTok{stratum3 }\OtherTok{=} \DecValTok{1001}\SpecialCharTok{:}\DecValTok{1800}
\NormalTok{stratum4 }\OtherTok{=} \DecValTok{1801}\SpecialCharTok{:}\DecValTok{2000}

\CommentTok{\#Total sample size}
\NormalTok{N }\OtherTok{=} \DecValTok{10}

\CommentTok{\#size by each stratum}

\NormalTok{n1 }\OtherTok{=}\NormalTok{ N }\SpecialCharTok{*} \DecValTok{400} \SpecialCharTok{/} \DecValTok{2000}
\NormalTok{n2 }\OtherTok{=}\NormalTok{ N }\SpecialCharTok{*} \DecValTok{600} \SpecialCharTok{/} \DecValTok{2000}
\NormalTok{n3 }\OtherTok{=}\NormalTok{ N }\SpecialCharTok{*} \DecValTok{800} \SpecialCharTok{/} \DecValTok{2000}
\NormalTok{n4 }\OtherTok{=}\NormalTok{ N }\SpecialCharTok{*} \DecValTok{200} \SpecialCharTok{/} \DecValTok{2000}

\CommentTok{\#sample by each stratum}
\NormalTok{s1 }\OtherTok{=} \FunctionTok{sample}\NormalTok{(stratum1, }\AttributeTok{size=}\NormalTok{n1, }\AttributeTok{replace=}\ConstantTok{FALSE}\NormalTok{)}
\NormalTok{s2 }\OtherTok{=} \FunctionTok{sample}\NormalTok{(stratum2, }\AttributeTok{size=}\NormalTok{n2, }\AttributeTok{replace=}\ConstantTok{FALSE}\NormalTok{)}
\NormalTok{s3 }\OtherTok{=} \FunctionTok{sample}\NormalTok{(stratum3, }\AttributeTok{size=}\NormalTok{n3, }\AttributeTok{replace=}\ConstantTok{FALSE}\NormalTok{)}
\NormalTok{s4 }\OtherTok{=} \FunctionTok{sample}\NormalTok{(stratum4, }\AttributeTok{size=}\NormalTok{n4, }\AttributeTok{replace=}\ConstantTok{FALSE}\NormalTok{)}

\CommentTok{\#combine stratum samples together}
\NormalTok{s }\OtherTok{=} \FunctionTok{c}\NormalTok{(s1, s2, s3, s4)}
\NormalTok{s1}
\end{Highlighting}
\end{Shaded}

\begin{verbatim}
## [1] 125 376
\end{verbatim}

\begin{Shaded}
\begin{Highlighting}[]
\NormalTok{s2}
\end{Highlighting}
\end{Shaded}

\begin{verbatim}
## [1] 528 534 607
\end{verbatim}

\begin{Shaded}
\begin{Highlighting}[]
\NormalTok{s3}
\end{Highlighting}
\end{Shaded}

\begin{verbatim}
## [1] 1221 1008 1071 1398
\end{verbatim}

\begin{Shaded}
\begin{Highlighting}[]
\NormalTok{s4}
\end{Highlighting}
\end{Shaded}

\begin{verbatim}
## [1] 1993
\end{verbatim}

\begin{Shaded}
\begin{Highlighting}[]
\NormalTok{s}
\end{Highlighting}
\end{Shaded}

\begin{verbatim}
##  [1]  125  376  528  534  607 1221 1008 1071 1398 1993
\end{verbatim}

\textbf{Example3}

\begin{Shaded}
\begin{Highlighting}[]
\CommentTok{\#Stratified sampling proportional allocation}

\NormalTok{stratum1 }\OtherTok{=} \DecValTok{1}\SpecialCharTok{:}\DecValTok{25} \CommentTok{\#a set of integers from 1 to 25}
\NormalTok{stratum2 }\OtherTok{=} \DecValTok{26}\SpecialCharTok{:}\DecValTok{200} \CommentTok{\#a set of integers from 26 to 200}
\NormalTok{stratum3 }\OtherTok{=} \DecValTok{201}\SpecialCharTok{:}\DecValTok{250}

\CommentTok{\#Total sample size}
\NormalTok{N }\OtherTok{=} \DecValTok{20}

\CommentTok{\#size by each stratum}

\NormalTok{n1 }\OtherTok{=}\NormalTok{ N }\SpecialCharTok{*} \DecValTok{25} \SpecialCharTok{/} \DecValTok{250}
\NormalTok{n2 }\OtherTok{=}\NormalTok{ N }\SpecialCharTok{*} \DecValTok{175} \SpecialCharTok{/} \DecValTok{250}
\NormalTok{n3 }\OtherTok{=}\NormalTok{ N }\SpecialCharTok{*} \DecValTok{50} \SpecialCharTok{/} \DecValTok{250}

\CommentTok{\#sample by each stratum}
\NormalTok{s1 }\OtherTok{=} \FunctionTok{sample}\NormalTok{(stratum1, }\AttributeTok{size=}\NormalTok{n1, }\AttributeTok{replace=}\ConstantTok{FALSE}\NormalTok{)}
\NormalTok{s2 }\OtherTok{=} \FunctionTok{sample}\NormalTok{(stratum2, }\AttributeTok{size=}\NormalTok{n2, }\AttributeTok{replace=}\ConstantTok{FALSE}\NormalTok{)}
\NormalTok{s3 }\OtherTok{=} \FunctionTok{sample}\NormalTok{(stratum3, }\AttributeTok{size=}\NormalTok{n3, }\AttributeTok{replace=}\ConstantTok{FALSE}\NormalTok{)}

\CommentTok{\#combine stratum samples together}
\NormalTok{s }\OtherTok{=} \FunctionTok{c}\NormalTok{(s1, s2, s3)}
\NormalTok{s1}
\end{Highlighting}
\end{Shaded}

\begin{verbatim}
## [1] 6 5
\end{verbatim}

\begin{Shaded}
\begin{Highlighting}[]
\NormalTok{s2}
\end{Highlighting}
\end{Shaded}

\begin{verbatim}
##  [1] 129  93 156 188 157 143  57 144 172 180 109  31  34 190
\end{verbatim}

\begin{Shaded}
\begin{Highlighting}[]
\NormalTok{s3}
\end{Highlighting}
\end{Shaded}

\begin{verbatim}
## [1] 242 232 241 250
\end{verbatim}

\begin{Shaded}
\begin{Highlighting}[]
\NormalTok{s}
\end{Highlighting}
\end{Shaded}

\begin{verbatim}
##  [1]   6   5 129  93 156 188 157 143  57 144 172 180 109  31  34 190 242 232 241
## [20] 250
\end{verbatim}

\hypertarget{random-sampling-from-a-sample}{%
\section{Random Sampling from a Sample}\label{random-sampling-from-a-sample}}

Random sampling is often used to estimate population quantities or build simulations.

\hypertarget{bootstrap-sampling-with-replacement}{%
\subsection{Bootstrap Sampling with Replacement}\label{bootstrap-sampling-with-replacement}}

Bootstrap is a resampling technique that involves repeatedly drawing samples with replacement from the original dataset. This method allows us to estimate the sampling distribution of a statistic, such as the mean.

\begin{Shaded}
\begin{Highlighting}[]
\NormalTok{data }\OtherTok{\textless{}{-}} \FunctionTok{c}\NormalTok{(}\DecValTok{5}\NormalTok{, }\DecValTok{7}\NormalTok{, }\DecValTok{8}\NormalTok{, }\DecValTok{10}\NormalTok{, }\DecValTok{13}\NormalTok{)}
\FunctionTok{set.seed}\NormalTok{(}\DecValTok{1}\NormalTok{)}
\NormalTok{bootstrap\_sample }\OtherTok{\textless{}{-}} \FunctionTok{sample}\NormalTok{(data, }\AttributeTok{size =} \DecValTok{5}\NormalTok{, }\AttributeTok{replace =} \ConstantTok{TRUE}\NormalTok{)}
\NormalTok{bootstrap\_sample}
\end{Highlighting}
\end{Shaded}

\begin{verbatim}
## [1]  5 10  5  7 13
\end{verbatim}

In the code below, we generate a single bootstrap sample and then repeat this process 1,000 times to compute the mean of each sample. The histogram displays the distribution of these bootstrap means.

\begin{Shaded}
\begin{Highlighting}[]
\NormalTok{data }\OtherTok{\textless{}{-}} \FunctionTok{c}\NormalTok{(}\DecValTok{5}\NormalTok{, }\DecValTok{7}\NormalTok{, }\DecValTok{8}\NormalTok{, }\DecValTok{10}\NormalTok{, }\DecValTok{13}\NormalTok{)}
\FunctionTok{set.seed}\NormalTok{(}\DecValTok{1}\NormalTok{)}
\NormalTok{B }\OtherTok{\textless{}{-}} \DecValTok{1000}
\NormalTok{means }\OtherTok{\textless{}{-}} \FunctionTok{replicate}\NormalTok{(B, }\FunctionTok{mean}\NormalTok{(}\FunctionTok{sample}\NormalTok{(data, }\AttributeTok{size =} \DecValTok{5}\NormalTok{, }\AttributeTok{replace =} \ConstantTok{TRUE}\NormalTok{)))}
\FunctionTok{length}\NormalTok{(means) }
\end{Highlighting}
\end{Shaded}

\begin{verbatim}
## [1] 1000
\end{verbatim}

\begin{Shaded}
\begin{Highlighting}[]
\FunctionTok{head}\NormalTok{(means)   }
\end{Highlighting}
\end{Shaded}

\begin{verbatim}
## [1] 8.0 7.2 9.0 9.8 8.4 7.4
\end{verbatim}

\begin{Shaded}
\begin{Highlighting}[]
\FunctionTok{hist}\NormalTok{(means,}
     \AttributeTok{breaks =} \DecValTok{30}\NormalTok{,}
     \AttributeTok{col =} \StringTok{"gray"}\NormalTok{,}
     \AttributeTok{main =} \StringTok{"Bootstrap Sampling Distribution of Mean"}\NormalTok{,}
     \AttributeTok{xlab =} \StringTok{"Sample Mean"}\NormalTok{)}
\end{Highlighting}
\end{Shaded}

\includegraphics{StatsTB_files/figure-latex/unnamed-chunk-198-1.pdf}

\hypertarget{random-sampling-without-replacement}{%
\subsection{Random Sampling without Replacement}\label{random-sampling-without-replacement}}

Use sample() without the replace=TRUE argument to select unique items.

\begin{Shaded}
\begin{Highlighting}[]
\FunctionTok{set.seed}\NormalTok{(}\DecValTok{123}\NormalTok{)}
\FunctionTok{sample}\NormalTok{(data, }\AttributeTok{size =} \DecValTok{3}\NormalTok{, }\AttributeTok{replace =} \ConstantTok{FALSE}\NormalTok{)}
\end{Highlighting}
\end{Shaded}

\begin{verbatim}
## [1]  8  7 13
\end{verbatim}

\hypertarget{stratified-sampling-with-proportional-allocation-1}{%
\section{Stratified sampling with proportional allocation}\label{stratified-sampling-with-proportional-allocation-1}}

The following R function is to sample data with proportional allocation

\begin{Shaded}
\begin{Highlighting}[]
\NormalTok{strat.sample }\OtherTok{=} \ControlFlowTok{function}\NormalTok{(x, strata, size, }\AttributeTok{proportional =} \ConstantTok{TRUE}\NormalTok{)\{}
  \CommentTok{\#x: an array that represents the entire population}
  \CommentTok{\#strata: stratum indicator}
  \CommentTok{\#size: sample size}
  \CommentTok{\#proportional: TRUE or FALSE. If true, the sample is proportional to the strata size.}
  
\NormalTok{  n }\OtherTok{=} \FunctionTok{length}\NormalTok{(x)}
\NormalTok{  u.strata }\OtherTok{=} \FunctionTok{unique}\NormalTok{(strata)}
\NormalTok{  n.strata }\OtherTok{=} \FunctionTok{length}\NormalTok{(u.strata)}
\NormalTok{  m }\OtherTok{=} \FunctionTok{rep}\NormalTok{(}\ConstantTok{NA}\NormalTok{, n.strata) }\CommentTok{\#size of each stratum}
  
  \ControlFlowTok{for}\NormalTok{ (i }\ControlFlowTok{in} \DecValTok{1}\SpecialCharTok{:}\NormalTok{n.strata)\{m[i] }\OtherTok{=} \FunctionTok{length}\NormalTok{(strata[strata }\SpecialCharTok{==}\NormalTok{ u.strata[i]])\}}
\NormalTok{  ni }\OtherTok{=} \FunctionTok{round}\NormalTok{(size}\SpecialCharTok{*}\NormalTok{m}\SpecialCharTok{/}\NormalTok{n)}
  \ControlFlowTok{if}\NormalTok{ (}\FunctionTok{sum}\NormalTok{(ni) }\SpecialCharTok{\textgreater{}}\NormalTok{ size) \{ni[n.strata] }\OtherTok{=}\NormalTok{ ni[n.strata] }\SpecialCharTok{{-}}\NormalTok{ (}\FunctionTok{sum}\NormalTok{(ni) }\SpecialCharTok{{-}}\NormalTok{ size)\} }\ControlFlowTok{else} \ControlFlowTok{if}\NormalTok{ (}\FunctionTok{sum}\NormalTok{(ni) }\SpecialCharTok{\textless{}}\NormalTok{ size) \{ni[n.strata] }\OtherTok{=}\NormalTok{ ni[n.strata] }\SpecialCharTok{+}\NormalTok{ (size}\SpecialCharTok{{-}}\FunctionTok{sum}\NormalTok{(ni))\}}
  
  \ControlFlowTok{if}\NormalTok{(proportional)\{}
\NormalTok{      s }\OtherTok{=} \ConstantTok{NULL}
      \ControlFlowTok{for}\NormalTok{ (i }\ControlFlowTok{in} \DecValTok{1}\SpecialCharTok{:}\NormalTok{n.strata)\{}
\NormalTok{        s }\OtherTok{=} \FunctionTok{c}\NormalTok{(s, }\FunctionTok{sample}\NormalTok{(x[strata }\SpecialCharTok{==}\NormalTok{ strata[i]], }\AttributeTok{size=}\NormalTok{ni[i], }\AttributeTok{replace=}\ConstantTok{FALSE}\NormalTok{))}
\NormalTok{      \}}
\NormalTok{  \} }\ControlFlowTok{else}\NormalTok{ \{}
\NormalTok{    s}\OtherTok{=} \FunctionTok{sample}\NormalTok{(x, }\AttributeTok{size=}\NormalTok{size, }\AttributeTok{replace=}\ConstantTok{FALSE}\NormalTok{)}
\NormalTok{  \}}
  \FunctionTok{return}\NormalTok{(s)}
\NormalTok{\}}

\FunctionTok{strat.sample}\NormalTok{(}\AttributeTok{x=}\FunctionTok{rnorm}\NormalTok{(}\DecValTok{100}\NormalTok{), }\AttributeTok{strata=}\FunctionTok{c}\NormalTok{(}\FunctionTok{rep}\NormalTok{(}\DecValTok{1}\NormalTok{, }\DecValTok{20}\NormalTok{), }\FunctionTok{rep}\NormalTok{(}\DecValTok{2}\NormalTok{, }\DecValTok{40}\NormalTok{), }\FunctionTok{rep}\NormalTok{(}\DecValTok{3}\NormalTok{, }\DecValTok{40}\NormalTok{)), }\AttributeTok{size=}\DecValTok{10}\NormalTok{, }\AttributeTok{proportional =} \ConstantTok{TRUE}\NormalTok{)}
\end{Highlighting}
\end{Shaded}

\begin{verbatim}
##  [1]  0.54909674 -0.05568601 -0.33491276  1.69018435 -0.10896597 -0.11724196
##  [7] -0.33491276  0.23821292 -0.78438222  2.52833655
\end{verbatim}

\begin{Shaded}
\begin{Highlighting}[]
\FunctionTok{strat.sample}\NormalTok{(}\AttributeTok{x=}\FunctionTok{rnorm}\NormalTok{(}\DecValTok{100}\NormalTok{), }\AttributeTok{strata=}\FunctionTok{c}\NormalTok{(}\FunctionTok{rep}\NormalTok{(}\DecValTok{1}\NormalTok{, }\DecValTok{20}\NormalTok{), }\FunctionTok{rep}\NormalTok{(}\DecValTok{2}\NormalTok{, }\DecValTok{40}\NormalTok{), }\FunctionTok{rep}\NormalTok{(}\DecValTok{3}\NormalTok{, }\DecValTok{40}\NormalTok{)), }\AttributeTok{size=}\DecValTok{12}\NormalTok{, }\AttributeTok{proportional =} \ConstantTok{TRUE}\NormalTok{)}
\end{Highlighting}
\end{Shaded}

\begin{verbatim}
##  [1]  0.8538955 -0.1803943  0.3399565  0.1516114  0.8538955  1.3974267
##  [7]  0.4856014  0.4187967  1.7636530 -0.7661688 -0.2419765  0.5964251
\end{verbatim}

\hypertarget{introduction-to-hypothesis-testing}{%
\chapter{Introduction to Hypothesis Testing}\label{introduction-to-hypothesis-testing}}

\begin{Shaded}
\begin{Highlighting}[]
\FunctionTok{library}\NormalTok{(IntroStats)}
\end{Highlighting}
\end{Shaded}

\hypertarget{research-hypothesis-and-statistical-hypotheses}{%
\section{Research Hypothesis and Statistical Hypotheses}\label{research-hypothesis-and-statistical-hypotheses}}

Hypothesis testing is a fundamental concept in statistics, and it involves a set of terms and procedures to describe the statistical methodology.

\hypertarget{research-hypothesis}{%
\subsection{Research Hypothesis}\label{research-hypothesis}}

A \textbf{research hypothesis} is a conjecture or supposition that motivates the research.

\hypertarget{null-hypothesis-h0}{%
\subsection{Null Hypothesis (H0)}\label{null-hypothesis-h0}}

\textbf{Statistical hypotheses} include \textbf{null hypothesis}, denoted as \(H_0\), and \textbf{alternative hypothesis}, denoted as \(H_a\) or \(H_1\), which are framed in a particular way that can be evaluated by suitable statistical techniques.

Null Hypothesis (H0) is the initial assumption or hypothesis that there is no significant difference or effect in the population. It is often a statement of no effect or no relationship.

\hypertarget{alternative-hypothesis-ha-or-h1}{%
\subsection{Alternative Hypothesis (Ha or H1)}\label{alternative-hypothesis-ha-or-h1}}

Alternative Hypothesis, often denoted as \(H_a\) or \(H_1\), is the opposite of the null hypothesis, representing the hypothesis that there is a significant difference or effect in the population.

\hypertarget{example-44}{%
\subsection{Example}\label{example-44}}

A pre-clinical laboratory scientist has observed the potential of a novel agent to suppress tumor progression in an animal model. This finding serves as the foundation for further research into its potential application in human cancer treatment, forming the basis of a \textbf{research hypothesis}.

The \textbf{statistical hypothesis} can be formulated based on the research hypothesis and subsequently tested using data collected through an appropriately designed experiment, assuming reasonable conditions.

In this particular example, the \textbf{null hypothesis} could be structured as follows: ``The novel agent does not possess the ability to suppress tumor progression in lung cancer patients,'' implying that the tumor size over time for patients treated with the novel agent is indistinguishable from those who receive no treatment.

Conversely, the \textbf{alternative hypothesis} would be: ``The novel agent has the capacity to inhibit tumor progression in lung cancer patients,'' indicating a reduction in tumor size among those treated with the novel agent compared to those receiving no treatment.

The rejection of the null hypothesis, or equivalently, the acceptance of the alternative hypothesis (\(H_a\)), would signify that the novel agent can indeed suppress tumor progression in lung cancer patients.

\hypertarget{hypothesis-test}{%
\section{Hypothesis Test}\label{hypothesis-test}}

The set of terms related to hypothesis testing is described below.

\hypertarget{test-statistics}{%
\subsection{Test Statistics}\label{test-statistics}}

Test Statistics is a mathematical formula or statistic that summarizes the information from the sample data and is used to make a decision about whether to reject the null hypothesis.

\hypertarget{significance-level}{%
\subsection{Significance Level}\label{significance-level}}

Significance Level is the predetermined level of significance that determines the threshold for rejecting the null hypothesis. Common values include 0.05 and 0.01, indicating a 5\% and 1\% chance of making a Type I error, respectively.

\hypertarget{type-i-error}{%
\subsection{Type I Error}\label{type-i-error}}

\textbf{Type I Error} Occurs when the null hypothesis is incorrectly rejected when it is actually true. This is often denoted as (\(\alpha\)), the significance level.

\hypertarget{type-ii-error}{%
\subsection{Type II Error}\label{type-ii-error}}

\textbf{Type II Error} Occurs when the null hypothesis is incorrectly not rejected when it is actually false. It is often denoted as (\(\beta\)).

\hypertarget{one-tailed-one-sided-test}{%
\subsection{One-Tailed (One-Sided) Test}\label{one-tailed-one-sided-test}}

There are 3 types of tests: two-sided (also called two-tailed), left-sided and right-sided.

One-sided test is to test a \textbf{specific direction} (greater than or less than) according to the question of interest. For example, if we are interested in testing whether a blood pressure drug can reduce the blood pressure at least 10 mmHg after 1 month. The question of interest is reduction of blood pressure. So the alternative hypothesis is framed as \(H_a: \mu \le 10\) mmHg, where \(\mu\) denotes the mean change of blood pressure after 1 month.

\begin{itemize}
\item
  \textbf{right-sided}: \(H_a\) has key word ``greater''. The corresponding p-value and alpha are also right-sided. There is only 1 critical value, which is on right.
\item
  \textbf{left-sided}: \(H_a\) has key word ``smaller''. The corresponding p-value and alpha are also left-sided. There is only 1 critical value, which is on left
\end{itemize}

\begin{figure}
\centering
\includegraphics{https://i.ibb.co/qpXjNfw/left-sided-test.png}
\caption{Figure 4. Left-sided}
\end{figure}

\begin{figure}
\centering
\includegraphics{https://i.ibb.co/VYGVMKF/right-sided-test.png}
\caption{Figure 5. right-sided}
\end{figure}

\hypertarget{two-tailed-two-sided-test}{%
\subsection{Two-Tailed (Two-Sided) Test}\label{two-tailed-two-sided-test}}

Two-Sided test is to test \textbf{difference} without a specific direction. For example, if we are interested in testing whether a blood pressure drug A has any difference in reducing the blood pressure after 1 month compared to another drug B. The question of interest is difference of blood pressure reduction. So the alternative hypothesis is framed as \(H_a: \mu_A \ne \mu_B\), where \(\mu_A\) and \(mu_B\) denote the mean change of blood pressure after 1 month for those who received drug A and drug B, respectively.

\begin{itemize}
\tightlist
\item
  \textbf{two-sided}: \(H_a\) has key word ``differ''. The corresponding p-value and alpha are also two-sided. There are 2 critical values on both sides.
\end{itemize}

\includegraphics{https://i.ibb.co/VWvgwWN/2-sided-test.png}
The choice between one-tailed and two-tailed depends on the \textbf{research question}.

\hypertarget{p-value}{%
\subsection{P value}\label{p-value}}

P value is the probability associated with the test statistic, indicating the likelihood of obtaining such extreme results if the null hypothesis were true. A smaller p-value suggests stronger evidence against the null hypothesis. Since the calculation of p-value depends on the type of test (one-sided or two-sided), the p-value should be reported as one-sided or two-sided as well for clarity.

\hypertarget{critical-values}{%
\subsection{Critical Value(s)}\label{critical-values}}

A critical value is a point on the scale of a test statistic beyond which we reject the null hypothesis. It separates the region where the null hypothesis is rejected (the rejection region) from the region where it is not. One-sided test has one critical value, while two-sided test has two critical values.

In hypothesis testing, critical values are used to determine the thresholds for statistical significance. They vary depending on: the chosen significance level \(\alpha\), the type of test (one-tailed or two-tailed), and the sampling distribution of the test statistic (e.g., normal, t-distribution).

\hypertarget{rejection-region}{%
\subsection{Rejection Region}\label{rejection-region}}

Rejection Region is the range of values of the test statistic that, if observed, leads to the rejection of the null hypothesis. The rejection region is determined by critical value(s).

\hypertarget{hypothesis-testing-approaches}{%
\section{Hypothesis testing approaches}\label{hypothesis-testing-approaches}}

When performing hypothesis testing, there are two main methods to decide whether to reject the null hypothesis: critical value(s) approach and p-value approach.

\hypertarget{critical-values-approach}{%
\subsection{Critical value(s) approach}\label{critical-values-approach}}

Calculate a test statistic (like \(z\) or \(t\)) from the sample data. Then determine whether the test statistic falls in the rejection region. If yes, reject the null hypothesis; otherwise do not reject.

\hypertarget{p-value-approach}{%
\subsection{p-value Approach}\label{p-value-approach}}

Compute a p-value from the test statistic based on assumed sampling distribution of the test statistic under null hypothesis. Then determine whether the p-value is less than or equal to the prespecified significance level \(\alpha\). If yes, reject the null hypothesis; otherwise do not reject.

Both methods lead to the \textbf{same statistical conclusion}, but the p-value approach is more common in scientific research because it gives more flexibility and precision.

\hypertarget{example-testing-an-iron-supplement}{%
\subsection{Example: Testing an Iron Supplement}\label{example-testing-an-iron-supplement}}

Suppose a health researcher believes that a new iron supplement increases hemoglobin levels in women. The changes of hemoglobin level after receiving 1 month of the iron supplement for 15 women diagnosed with anemia are 0.3, 2.1, 1.5, 2.9, 5.2, 0, -0.5, 2.6, 2.7, 2.0, 2.1, 3.3, 1.6, 2.5, 2.2. At 5\% significance level, is this sample suggesting the iron supplement can increase hemoglobin level?

\begin{enumerate}
\def\labelenumi{\arabic{enumi}.}
\tightlist
\item
  Statistical Hypotheses:
\end{enumerate}

\[
H_0: \mu = 0 \quad \text{(no change in hemoglobin level from supplement)}
\]

\[
H_a: \mu > 0 \quad \text{(supplement increases hemoglobin)}
\]

\begin{enumerate}
\def\labelenumi{\arabic{enumi}.}
\setcounter{enumi}{1}
\item
  Significance level (one-sided): 0.05
\item
  The critical value is 1.76 based on t-test. The method to determine the critical value will be described in the next chapter.
\end{enumerate}

\begin{Shaded}
\begin{Highlighting}[]
\NormalTok{x}\OtherTok{=}\FunctionTok{c}\NormalTok{(}\FloatTok{0.3}\NormalTok{, }\FloatTok{2.1}\NormalTok{, }\FloatTok{1.5}\NormalTok{, }\FloatTok{2.9}\NormalTok{, }\FloatTok{5.2}\NormalTok{, }\DecValTok{0}\NormalTok{, }\SpecialCharTok{{-}}\FloatTok{0.5}\NormalTok{, }\FloatTok{2.6}\NormalTok{, }\FloatTok{2.7}\NormalTok{, }\FloatTok{2.0}\NormalTok{, }\FloatTok{2.1}\NormalTok{, }\FloatTok{3.3}\NormalTok{, }\FloatTok{1.6}\NormalTok{, }\FloatTok{2.5}\NormalTok{, }\FloatTok{2.2}\NormalTok{)}
\FunctionTok{qt}\NormalTok{(}\FloatTok{0.95}\NormalTok{, }\AttributeTok{df=}\FunctionTok{length}\NormalTok{(x)}\SpecialCharTok{{-}}\DecValTok{1}\NormalTok{)}
\end{Highlighting}
\end{Shaded}

\begin{verbatim}
## [1] 1.76131
\end{verbatim}

\begin{enumerate}
\def\labelenumi{\arabic{enumi}.}
\setcounter{enumi}{3}
\item
  The rejection region is \(t > 1.76\). The direction of the rejection region is always consistent with the alternative hypothesis \(H_a\).
\item
  The test statistic is \(t\) value 5.63, using t-test. The method to determine the \(t\) value will be described in the next chapter.
\item
  Using the critical value approach, the \(t\) value 5.63 falls in the rejection region, so reject the null hypothesis.
\item
  The p-value is 0.00005, which is less than the significance level \(\alpha = 0.05\), so reject the null hypothesis. Both the critical value approach and p-value approach are consistent. The method to determine the \(p\) value will be described in the next chapter, calculated as \(p = P(t>5.63)\), where \(t\) is a random variable with \(n-1=14\) degrees of freedom.
\end{enumerate}

\begin{Shaded}
\begin{Highlighting}[]
\DecValTok{1}\SpecialCharTok{{-}}\FunctionTok{pt}\NormalTok{(}\FloatTok{5.63}\NormalTok{, }\AttributeTok{df=}\FunctionTok{length}\NormalTok{(x)}\SpecialCharTok{{-}}\DecValTok{1}\NormalTok{)}
\end{Highlighting}
\end{Shaded}

\begin{verbatim}
## [1] 3.103721e-05
\end{verbatim}

\begin{enumerate}
\def\labelenumi{\arabic{enumi}.}
\setcounter{enumi}{7}
\tightlist
\item
  Conclusion. At 5\% significance level, there is sufficient evidence that the new iron supplement is able to increase hemoglobin level for anemia women.
\end{enumerate}

The hypothesis testing process is implemented in the function below:

\begin{Shaded}
\begin{Highlighting}[]
\FunctionTok{one.mean.t}\NormalTok{(}\AttributeTok{x=}\NormalTok{x, }\AttributeTok{mu0=}\DecValTok{0}\NormalTok{, }\AttributeTok{tail =} \StringTok{"right"}\NormalTok{, }\AttributeTok{alpha=}\FloatTok{0.05}\NormalTok{)}
\end{Highlighting}
\end{Shaded}

\begin{verbatim}
## $CI
## [1] 1.258133 2.808533
## 
## $xbar
## [1] 2.033333
## 
## $n
## [1] 15
## 
## $margin
## [1] 0.7752
## 
## $s
## [1] 1.39983
## 
## $p
## [1] 3.127274e-05
## 
## $t0
## [1] 5.625731
## 
## $CV
## [1] 1.76131
## 
## $df
## [1] 14
## 
## $conclusion
## [1] "H0 Rejected"
\end{verbatim}

\hypertarget{inference-for-comparing-one-population-mean-to-a-threshold}{%
\chapter{Inference for Comparing One Population Mean to a Threshold}\label{inference-for-comparing-one-population-mean-to-a-threshold}}

\begin{Shaded}
\begin{Highlighting}[]
\FunctionTok{library}\NormalTok{(IntroStats)}
\end{Highlighting}
\end{Shaded}

\hypertarget{hypothesis-test-for-one-population-mean-when-population-standard-deviation-known-one-sample-z-test-or-one-mean-z-test}{%
\section{Hypothesis Test for One Population Mean when Population Standard Deviation Known (one-sample z-test or one-mean z-test)}\label{hypothesis-test-for-one-population-mean-when-population-standard-deviation-known-one-sample-z-test-or-one-mean-z-test}}

\hypertarget{motivating-example}{%
\subsection{Motivating Example}\label{motivating-example}}

One common estimate of cheetah's top speed is 60 mph. Table below shows
the top speeds of 35 cheetahs.

\begin{figure}
\centering
\includegraphics{https://i.ibb.co/18n1tzQ/392-Table-09-11.png}
\caption{Cheetah Top Speeds}
\end{figure}

At the 5\% significance level, do the data provide sufficient evidence
that cheetahs' top speeds differ from 60 mph? Assume the population
deviation (\(\sigma\)) is 3.2 mph.

\hypertarget{test-statistic}{%
\subsection{Test Statistic}\label{test-statistic}}

The test statistic with population variance \(\sigma^2\) known is

\[z = \frac{\bar{x}-\mu}{\sigma/\sqrt{n}} \sim N(0, 1)\]

This test is called \textbf{one-sample z-test or one-mean z-test}, because the test statistic follows the standard normal distribution.

\hypertarget{the-hypothesis-testing-procedure}{%
\subsection{The hypothesis Testing Procedure}\label{the-hypothesis-testing-procedure}}

\begin{figure}
\centering
\includegraphics{https://i.ibb.co/HppVVMX/Procedure9-1a.png}
\caption{Procedure One-Mean z-Test}
\end{figure}

\begin{figure}
\centering
\includegraphics{https://i.ibb.co/GnJ0ZBH/Procedure9-1b.png}
\caption{Procedure One-Mean z-Test}
\end{figure}

\hypertarget{example-cheetah-top-speeds}{%
\subsection{Example: Cheetah Top Speeds}\label{example-cheetah-top-speeds}}

\textbf{Solution}:

We follow the steps to perform the z-test for the cheetah top speed problem.

\begin{enumerate}
\def\labelenumi{\arabic{enumi}.}
\tightlist
\item
  State the null and alternative hypotheses explicitly.
  \(H_0: \mu = 60\) and \(H_a: \mu \ne 60\).
\item
  Determine the test statistic and its distribution. The test
  statistic is \(z\) above.
\item
  Determine the significance level. \(\alpha = 0.05\).
\item
  Determine the rejection region and critical values. For
  \(\alpha=0.05\), the critical values are \(\pm z_{1-0.05/2}=\pm 1.96\).
  The rejection region is \(z \ge 1.96\) or \(z \le -1.96\).
\item
  Calculate the test statistic under \(H_0\). We usually use lower case
  \(z\) to denote the observed value of the random variable \(Z\).
  \[z_0 = \frac{\bar{x}-\mu_0}{\sigma/\sqrt{n}} = \frac{59.526-60}{3.2/\sqrt{35}}=-0.877\].
\item
  Calculate the p value. The \(P-\)value is
  \(2P(z > |-0.877|)=0.3806 > 0.05\).
\item
  Make a statistical decision based on the p value or the calculated
  test statistic. Since \(z = -0.877\) does not fall in the rejection
  region, \(H_0\) is not rejected. Equivalently, \(p > 0.05\), so \(H_0\) is
  not rejected.
\item
  Conclusion: At the 5\% significance level, the data do not provide
  sufficient evidence that cheetahs' top speeds differ from 60 mph.
\end{enumerate}

This test is implemented in R as below. The corresponding
\((1-\alpha)100\%\) CI for \(\mu\) is also produced.

\begin{Shaded}
\begin{Highlighting}[]
\NormalTok{x }\OtherTok{\textless{}{-}} \FunctionTok{c}\NormalTok{( }\FloatTok{57.3}\NormalTok{, }\FloatTok{57.5}\NormalTok{, }\FloatTok{59.0}\NormalTok{, }\FloatTok{56.5}\NormalTok{, }\FloatTok{61.3}\NormalTok{,}
        \FloatTok{57.6}\NormalTok{, }\FloatTok{59.2}\NormalTok{, }\FloatTok{65.0}\NormalTok{, }\FloatTok{60.1}\NormalTok{, }\FloatTok{59.7}\NormalTok{,}
         \FloatTok{62.6}\NormalTok{, }\FloatTok{52.6}\NormalTok{, }\FloatTok{60.7}\NormalTok{, }\FloatTok{62.3}\NormalTok{, }\FloatTok{65.2}\NormalTok{,}
        \FloatTok{54.8}\NormalTok{, }\FloatTok{55.4}\NormalTok{, }\FloatTok{55.5}\NormalTok{, }\FloatTok{57.8}\NormalTok{, }\FloatTok{58.7}\NormalTok{,}
        \FloatTok{57.8}\NormalTok{, }\FloatTok{60.9}\NormalTok{, }\FloatTok{75.3}\NormalTok{, }\FloatTok{60.6}\NormalTok{, }\FloatTok{58.1}\NormalTok{,}
       \FloatTok{55.9}\NormalTok{, }\FloatTok{61.6}\NormalTok{, }\FloatTok{59.6}\NormalTok{, }\FloatTok{59.8}\NormalTok{, }\FloatTok{63.4}\NormalTok{,}
        \FloatTok{54.7}\NormalTok{, }\FloatTok{60.2}\NormalTok{, }\FloatTok{52.4}\NormalTok{, }\FloatTok{58.3}\NormalTok{, }\FloatTok{66.0}\NormalTok{)}
\FunctionTok{one.mean.z}\NormalTok{(}\AttributeTok{x=}\NormalTok{x, }\AttributeTok{mu0=}\DecValTok{60}\NormalTok{, }\AttributeTok{sigma=}\FloatTok{3.2}\NormalTok{, }\AttributeTok{tail=}\StringTok{"two"}\NormalTok{, }\AttributeTok{alpha=}\FloatTok{0.05}\NormalTok{)}
\end{Highlighting}
\end{Shaded}

\begin{verbatim}
## $CI
## [1] 58.46557 60.58586
## 
## $xbar
## [1] 59.52571
## 
## $n
## [1] 35
## 
## $margin
## [1] 1.060142
## 
## $p
## [1] 0.3805695
## 
## $z0
## [1] -0.8768475
## 
## $CV
## [1] -1.959964  1.959964
## 
## $conclusion
## [1] "H0 Not Rejected"
\end{verbatim}

\hypertarget{testing-h_0-by-confidence-interval}{%
\subsection{\texorpdfstring{Testing \(H_0\) by confidence interval}{Testing H\_0 by confidence interval}}\label{testing-h_0-by-confidence-interval}}

The hypothesized value 60 mph is within the 95\% CI, so it is also
equivalent to conclude there is no sufficient evidence to conclude the
cheetahs' top speeds differ from 60 mph at significance level of 5\%. The
CI is (58.46557, 60.58586) which includes 60. So do not reject H0.

So CI method is equivalent to the p-value method and critical value
method.

\hypertarget{checking-for-outliers}{%
\subsection{Checking for Outliers}\label{checking-for-outliers}}

Do we miss something here? Yes, check normality and outliers! One
outlier is identified.

\includegraphics{https://i.ibb.co/ysYBbtb/393-Figure-09-16a.png}
\includegraphics{https://i.ibb.co/ncTVB8W/393-Figure-09-16b.png}
\includegraphics{https://i.ibb.co/k8d2PtH/393-Figure-09-16c.png}
\includegraphics{https://i.ibb.co/n1QxTTH/393-Figure-09-16d.png}

\hypertarget{pragmatic-approach}{%
\subsection{Pragmatic Approach}\label{pragmatic-approach}}

\begin{enumerate}
\def\labelenumi{\arabic{enumi}.}
\tightlist
\item
  Perform the z-test using \textbf{all} data;
\item
  Perform the z-test \textbf{excluding the outlier}.
\end{enumerate}

\textbf{Solution with Outlier Removal}

\begin{itemize}
\tightlist
\item
  With all data included, \(H_0\) is not rejected because
  \(P=0.3788>0.05\) and \(z = -0.88>-1.96\).
\item
  After excluding the outlier, \(z=-1.71>-1.96\) and
  \(P=2\cdot 0.0436=0.0872>0.05\). So \(H_0\) is not rejected.
\end{itemize}

\hypertarget{interpretation}{%
\subsection{Interpretation:}\label{interpretation}}

At the 5\% significance level, the data do not provide sufficient
evidence to conclude that the cheetahs' mean top speed differs from 60
mph. After removing the outlier, the conclusion still holds.

\hypertarget{statistical-significance-vs-practical-significance}{%
\subsection{Statistical Significance vs Practical Significance}\label{statistical-significance-vs-practical-significance}}

Rejection of \(H_0\) at significance level \(\alpha\) establishes
\textbf{statistical significance}, i.e., the sampled data provide sufficient
evidence to conclude the truth is different from \(H_0\). However, this
doesn't necessarily mean the difference is \textbf{practically important}.

\begin{itemize}
\tightlist
\item
  With a larger sample size, it is easier to reject \(H_0\) for the same
  hypothesis test. This is related to the \textbf{power} concept.
\item
  A statistical test could detect a \textbf{negligible difference} in
  practice when the sample size is large enough, but practically such
  a difference is not meaningful.
\item
  When interpreting data, always assess the statistical significance
  by hypothesis testing and assess whether the finding is practically
  meaningful.
\end{itemize}

\hypertarget{example-mean-blood-glucose-level}{%
\subsection{Example: Mean Blood Glucose Level}\label{example-mean-blood-glucose-level}}

A lab claims that the average fasting blood glucose level for adults is
95 mg/dL. To assess this claim, a researcher collects a random sample of
36 adults and finds a sample mean of
\[
\bar{x} = 98.4 \text{ mg/dL}.
\]
Assume the population standard deviation is

\[
\sigma = 12 \text{ mg/dL}.
\]

Test, at the 5\% significance level, whether the mean fasting blood
glucose level is significantly different from 95 mg/dL.

\textbf{Step 1: Hypotheses}

\[
H_0: \mu = 95
\]

\[
H_a: \mu \neq 95
\]

\textbf{Step 2: Test Statistic}

\[
z = \frac{\bar{x} - \mu_0}{\sigma / \sqrt{n}} = \frac{98.4 - 95}{12 / \sqrt{36}} = \frac{3.4}{2} = 1.70
\]

\textbf{Step 3: Critical Value}

For the significance level \(\alpha = 0.05\), the critical values are \(\quad z_{0.975} = \pm 1.96\).

\textbf{Step 4: Decision}

\[
z = 1.70 < 1.96 \quad \Rightarrow \quad \text{Fail to reject } H_0
\]

\textbf{Conclusion:}

At the 5\% significance level, there is insufficient evidence to conclude
that the mean fasting blood glucose level differs from 95 mg/dL.

\hypertarget{hypothesis-tests-for-one-population-mean-when-sigma-is-unknown-one-mean-t-test-or-one-sample-t-test}{%
\section{\texorpdfstring{Hypothesis Tests for One Population Mean When \(\sigma\) Is Unknown (One-Mean t-test or One-Sample t-Test)}{Hypothesis Tests for One Population Mean When \textbackslash sigma Is Unknown (One-Mean t-test or One-Sample t-Test)}}\label{hypothesis-tests-for-one-population-mean-when-sigma-is-unknown-one-mean-t-test-or-one-sample-t-test}}

\hypertarget{test-statistic-1}{%
\subsection{Test statistic}\label{test-statistic-1}}

In the previous section, it was assumed that the standard deviation
\(\sigma\) is known, then

\[
z = \frac{\bar{x}-\mu}{\sigma/\sqrt{n}}\sim N(0,1)
\]

However, it is usually unknown.

\[
t = \frac{\bar{x}-\mu}{s/\sqrt{n}}\sim t_{n-1}, 
\]

where \(s\) is the sample standard deviation.

The one-mean t-test is constructed similarly to the z-test, with the critical value and \(P\)-value determined according to the t-distribution. The testing procedure is the same.

\hypertarget{procedure-one-mean-t-test}{%
\subsection{Procedure: One-Mean t-Test}\label{procedure-one-mean-t-test}}

\begin{figure}
\centering
\includegraphics{https://i.ibb.co/nMff9gW/Procedure9-2a.png}
\caption{One-Mean t-Test}
\end{figure}

\begin{figure}
\centering
\includegraphics{https://i.ibb.co/72Fb2CQ/Procedure9-2b.png}
\caption{One-Mean t-Test}
\end{figure}

\hypertarget{example-one-mean-t-test}{%
\subsection{Example: One-Mean t-Test}\label{example-one-mean-t-test}}

Table below shows the pH levels for 15 high mountain lakes in Southern
Alps. A lake is considered non-acidic if its pH is greater than 6.

\includegraphics{https://i.ibb.co/pZ5P5WV/401-Table-09-12.png}
\includegraphics{https://i.ibb.co/BPJryCb/402-Figure-09-21.png}

\hypertarget{hypothesis-test-1}{%
\subsection{Hypothesis Test}\label{hypothesis-test-1}}

At the 5\% significance level, do the data provide sufficient evidence to
conclude that on average the high mountain lakes in Southern Alps are
non-acidic?

\textbf{Steps for One-Mean t-Test}:

\begin{enumerate}
\def\labelenumi{\arabic{enumi}.}
\tightlist
\item
  \textbf{State the Hypotheses}: \(H_0: \mu = 6\) and \(H_a: \mu > 6\)
  (non-acidic).
\item
  \textbf{Significance Level}: \(\alpha = 0.05\).
\item
  \textbf{Compute Test Statistic}:
  \(t = \frac{\bar{x}-\mu_0}{s/\sqrt{n}}=\frac{6.6-6}{3.2/\sqrt{15}}=3.46\),
  where \(\bar{x} = 6.6\) and \(s=0.672\).
\item
  \textbf{Critical Value Approach}: For \(\alpha=0.05\), the critical value
  for a right-tailed \(t\) test with df = 14 is \(t_{0.05}=1.761\). Since
  \(t=3.46\) is in the rejection region, we reject \(H_0\).
\item
  \(P\)-Value Approach: The test statistic \(t=3.46\), and the \(P\)-value
  is \(P=P(t_{14}>3.46)=0.0019<0.05\). Reject \(H_0\).
\end{enumerate}

\begin{figure}
\centering
\includegraphics{https://i.ibb.co/fxDDjW2/403-Figure-09-22a.png}
\caption{One-Mean t-Test}
\end{figure}

\begin{figure}
\centering
\includegraphics{https://i.ibb.co/n8Qw88J/403-Figure-09-22b.png}
\caption{One-Mean t-Test}
\end{figure}

\begin{Shaded}
\begin{Highlighting}[]
\NormalTok{x }\OtherTok{=} \FunctionTok{c}\NormalTok{(}\FloatTok{7.2}\NormalTok{, }\FloatTok{7.3}\NormalTok{, }\FloatTok{6.1}\NormalTok{, }\FloatTok{6.9}\NormalTok{, }\FloatTok{6.6}\NormalTok{,}
      \FloatTok{7.3}\NormalTok{, }\FloatTok{6.3}\NormalTok{, }\FloatTok{5.5}\NormalTok{, }\FloatTok{6.3}\NormalTok{, }\FloatTok{6.5}\NormalTok{,}
      \FloatTok{5.7}\NormalTok{, }\FloatTok{6.9}\NormalTok{, }\FloatTok{6.7}\NormalTok{, }\FloatTok{7.9}\NormalTok{, }\FloatTok{5.8}\NormalTok{)}

\FunctionTok{one.mean.t}\NormalTok{(}\AttributeTok{x=}\NormalTok{x, }\AttributeTok{mu0=}\DecValTok{6}\NormalTok{, }\AttributeTok{tail=}\StringTok{"right"}\NormalTok{)}
\end{Highlighting}
\end{Shaded}

\begin{verbatim}
## $CI
## [1] 6.227923 6.972077
## 
## $xbar
## [1] 6.6
## 
## $n
## [1] 15
## 
## $margin
## [1] 0.3720771
## 
## $s
## [1] 0.6718843
## 
## $p
## [1] 0.001919161
## 
## $t0
## [1] 3.458616
## 
## $CV
## [1] 1.76131
## 
## $df
## [1] 14
## 
## $conclusion
## [1] "H0 Rejected"
\end{verbatim}

\hypertarget{example-mean-hba1c-level-in-diabetic-patients}{%
\subsection{Example: Mean HbA1c Level in Diabetic Patients}\label{example-mean-hba1c-level-in-diabetic-patients}}

The following dataset contains glycated hemoglobin (HbA1c) levels (in \%)
for a sample of 12 diabetic patients. Clinical guidelines recommend that
well-controlled diabetes should have HbA1c \textless{} 7\%.

We want to test whether the mean HbA1c level in this sample exceeds 7\%,
suggesting poor glycemic control. At the 5\% significance level, do the data provide sufficient evidence to conclude that the mean HbA1c level in this diabetic population is
greater than 7\%?

\textbf{Solution:}

\begin{itemize}
\item
  State the Hypotheses: \(H_0: \mu = 7\) \(H_a: \mu > 7\)
\item
  Significance Level: \(\alpha = 0.05\)
\item
  Compute Test Statistic:
\end{itemize}

Sample mean: \(\bar{x} = 7.55\). Sample standard deviation: \(s = 0.47\). Sample size: \(n = 12\). The test statistic is

\[
t = \frac{\bar{x} - \mu_0}{s / \sqrt{n}} = \frac{7.55 - 7}{\frac{0.47}{\sqrt{12}}} = 4.04
\]
- Critical Value Approach:

Degrees of freedom: \(df = 12 - 1 = 11\). Critical value: \(t_{0.05,11} = 1.796\). Since \(t = 4.04 > 1.796\), we reject \(H_0\).

\begin{itemize}
\tightlist
\item
  P-Value Approach:
\end{itemize}

\(P = P(t_{11} > 4.04) = 0.0012 < 0.05\). Again, we reject \(H_0\).

The procedure is implemented below:

\begin{Shaded}
\begin{Highlighting}[]
\NormalTok{hba1c }\OtherTok{=} \FunctionTok{c}\NormalTok{(}\FloatTok{7.2}\NormalTok{, }\FloatTok{7.4}\NormalTok{, }\FloatTok{6.9}\NormalTok{, }\FloatTok{8.1}\NormalTok{, }\FloatTok{7.7}\NormalTok{, }\FloatTok{7.5}\NormalTok{, }
          \FloatTok{8.3}\NormalTok{, }\FloatTok{7.9}\NormalTok{, }\FloatTok{6.8}\NormalTok{, }\FloatTok{7.6}\NormalTok{, }\FloatTok{7.0}\NormalTok{, }\FloatTok{8.0}\NormalTok{)}
\FunctionTok{one.mean.t}\NormalTok{(}\AttributeTok{x=}\NormalTok{hba1c, }\AttributeTok{mu0=}\DecValTok{7}\NormalTok{, }\AttributeTok{tail=}\StringTok{"right"}\NormalTok{)}
\end{Highlighting}
\end{Shaded}

\begin{verbatim}
## $CI
## [1] 7.221674 7.844992
## 
## $xbar
## [1] 7.533333
## 
## $n
## [1] 12
## 
## $margin
## [1] 0.3116591
## 
## $s
## [1] 0.4905161
## 
## $p
## [1] 0.001559681
## 
## $t0
## [1] 3.766484
## 
## $CV
## [1] 1.795885
## 
## $df
## [1] 11
## 
## $conclusion
## [1] "H0 Rejected"
\end{verbatim}

Using R built-in function,

\begin{Shaded}
\begin{Highlighting}[]
\FunctionTok{t.test}\NormalTok{(hba1c, }\AttributeTok{mu =} \DecValTok{7}\NormalTok{, }\AttributeTok{alternative =} \StringTok{"greater"}\NormalTok{)}
\end{Highlighting}
\end{Shaded}

\begin{verbatim}
## 
##  One Sample t-test
## 
## data:  hba1c
## t = 3.7665, df = 11, p-value = 0.00156
## alternative hypothesis: true mean is greater than 7
## 95 percent confidence interval:
##  7.279036      Inf
## sample estimates:
## mean of x 
##  7.533333
\end{verbatim}

\hypertarget{inference-for-comparing-two-population-means}{%
\chapter{Inference for Comparing Two Population Means}\label{inference-for-comparing-two-population-means}}

\begin{Shaded}
\begin{Highlighting}[]
\FunctionTok{library}\NormalTok{(IntroStats)}
\end{Highlighting}
\end{Shaded}

\hypertarget{motivating-examples}{%
\section{Motivating Examples}\label{motivating-examples}}

\hypertarget{example-faculty-salaries}{%
\subsection{Example: Faculty Salaries}\label{example-faculty-salaries}}

\textbf{Faculty Salaries study:} compare salaries from private institutions and public institutions. Randomly select 35 faculty members from private institutions and 30 from public institutions, then compare.

\begin{figure}
\centering
\includegraphics{https://i.ibb.co/QQGx3YG/441-Table-10-03.png}
\caption{Faculty Salaries study}
\end{figure}

\hypertarget{example-ruxolitinib}{%
\subsection{Example: Ruxolitinib}\label{example-ruxolitinib}}

\textbf{Randomized Trial of Ruxolitinib} in Antiretroviral-Treated Adults With Human Immunodeficiency Virus (HIV). \emph{Clinical Infectious Diseases, 2022(74)1}
\url{https://doi.org/10.1093/cid/ciab212}

\begin{itemize}
\item
  \textbf{Primary Interest}: Change in plasma interleukin 6 (IL-6) by week 5.
\item
  \textbf{Participants}: 40 received ruxolitinib and 20 received control.
\item
  \textbf{Statistical Design}: 80\% \textbf{power} to detect a 0.20 log10 pg/mL difference between arms for change in IL-6 from baseline to week 4/5, using a \textbf{2-sided t-test} with \textbf{10\% type I error}.
\item
  \textbf{Study Results}: By week 5, differences in IL-6 (mean fold change, 0.93 vs 1.10; \textbf{P = 0.18}).
\item
  \textbf{Study Conclusion}: Ruxolitinib was well-tolerated. Baseline IL-6 levels were normal and \textbf{showed no significant reduction}.
\end{itemize}

\hypertarget{sampling-distribution-of-barx_1-barx_2}{%
\section{\texorpdfstring{Sampling Distribution of \(\bar{X_1}-\bar{X_2}\)}{Sampling Distribution of \textbackslash bar\{X\_1\}-\textbackslash bar\{X\_2\}}}\label{sampling-distribution-of-barx_1-barx_2}}

In scientific research, a common practice involves comparing two groups, such as an experimental group and a control group. Denote the sample mean \(\bar{x}_1\) for the first group and \(\bar{x}_2\) for the second group with the sample size \(n_1\) and \(n_2\) respectively. If \(X_i\) has mean \(\mu_i\) and variance \(\sigma_i^2\) for \(i=1, 2\), then \(\bar{x_1}-\bar{x_2}\) has a mean \(\mu_1-\mu_2\) and variance \(\sigma_1^2/n_1+\sigma_2^2/n_2\).

\[
\bar{x}_1-\bar{x}_2 \sim N(\mu_1-\mu_2, \frac{\sigma_1^2}{n_1}+\frac{\sigma_2^2}{n_2})
\]

\textbf{Assumptions}:

\begin{itemize}
\item
  Suppose \(x_i \sim N(\mu_i, \sigma_i^2)\) for \(i = 1, 2\).
\item
  Then the sample mean \(\bar{x}_i\) for sample size \(n_i\) follows a \(N(\mu_i, \frac{\sigma_i^2}{n_i})\) distribution.
\end{itemize}

\textbf{Sampling Distribution}:

\begin{itemize}
\tightlist
\item
  \(\bar{x}_1 - \bar{x}_2 \sim N(\mu_1 - \mu_2, \frac{\sigma_1^2}{n_1} + \frac{\sigma_2^2}{n_2})\)
\end{itemize}

\textbf{Standardized Test Statistic}:

\begin{itemize}
\item
  Define \(z = \frac{(\bar{x}_1 - \bar{x}_2) - (\mu_1 - \mu_2)}{\sqrt{\frac{\sigma_1^2}{n_1} + \frac{\sigma_2^2}{n_2}}}\)
\item
  Then \(z \sim N(0, 1)\).
\end{itemize}

\textbf{Interpretation of CI for the Difference Between Two Sample Means}:

\begin{itemize}
\item
  Suppose the 95\% CI for \(\mu_1 - \mu_2\) is from 15 to 20.
\item
  We can be 95\% confident that \(\mu_1 - \mu_2\) lies somewhere between 15 and 20.
\end{itemize}

\hypertarget{example-serum-cholesterol-level}{%
\subsection{Example: Serum cholesterol level}\label{example-serum-cholesterol-level}}

According to the national health and nutrition examination survey of 1988-1994, the estimated mean serum cholesterol level \((\mu)\) for US females aged 20-74 years to be 204 mg/dl and the standard deviation \((\sigma)\) is 44. If U.S. males have a mean 208 mg/dl and the standard deviation is also 44.

\textbf{Question.} For a random sample of 50 women and another independent sample of 50 men in this age group, what is the probability that the sample mean difference (\(\bar{x}_1-\bar{x}_2\)) is less than 3 mg/dl?

\textbf{Solution}

Let \(\bar{x}_1\) and \(\bar{x}_2\) denote the sample means for men and women respectively. Then \(\bar{x}_1-\bar{x}_2\) follows a normal distribution with mean 208-204 = 4 mg/gl and variance \(44/50 + 44/50 = 1.76\) mg/dl.

\[
\bar{x}_1-\bar{x}_2 \sim N(4, 1.76)
\]
So
\[
P(\bar{x}_1-\bar{x}_2 < 3) = 0.225
\]

\begin{Shaded}
\begin{Highlighting}[]
\FunctionTok{pnorm}\NormalTok{(}\DecValTok{3}\NormalTok{, }\AttributeTok{mean=}\DecValTok{4}\NormalTok{, }\AttributeTok{sd=}\FunctionTok{sqrt}\NormalTok{(}\FloatTok{1.76}\NormalTok{))}
\end{Highlighting}
\end{Shaded}

\begin{verbatim}
## [1] 0.2254912
\end{verbatim}

\hypertarget{standardized-variable-1}{%
\subsection{Standardized variable}\label{standardized-variable-1}}

We can convert \(\bar{x}_1-\bar{x}_2\) to

\[
z = \frac{(\bar{x}_1-\bar{x}_2) - (\mu_1-\mu_2)}{\sqrt{\frac{\sigma_1^2}{n_1}+\frac{\sigma_2^2}{n_2}}}
\]
and \(z \sim N(0, 1)\). Then the probability is equivalent to the calculation regarding \(z\) below.

\[
P(\bar{x}_1-\bar{x}_2 < 3) = \Phi\left(\frac{3-4}{\sqrt{44/50+44/50}}\right)
\]

\begin{Shaded}
\begin{Highlighting}[]
\FunctionTok{pnorm}\NormalTok{((}\DecValTok{3{-}4}\NormalTok{)}\SpecialCharTok{/}\FunctionTok{sqrt}\NormalTok{(}\DecValTok{44}\SpecialCharTok{/}\DecValTok{50}\SpecialCharTok{+}\DecValTok{44}\SpecialCharTok{/}\DecValTok{50}\NormalTok{))}
\end{Highlighting}
\end{Shaded}

\begin{verbatim}
## [1] 0.2254912
\end{verbatim}

Similarly,

\[
\begin{eqnarray*}
P(1<\bar{x}_1-\bar{x}_2 < 3) &=& P(\bar{x}_1-\bar{x}_2<3) - P(\bar{x}_1-\bar{x}_2\le 1) \\
&=& \Phi\left(\frac{3-4}{\sqrt{44/50+44/50}}\right) - \Phi\left(\frac{1-4}{\sqrt{44/50+44/50}}\right)\\
&=&0.214
\end{eqnarray*}
\]

\begin{Shaded}
\begin{Highlighting}[]
\FunctionTok{pnorm}\NormalTok{((}\DecValTok{3{-}4}\NormalTok{)}\SpecialCharTok{/}\FunctionTok{sqrt}\NormalTok{(}\DecValTok{44}\SpecialCharTok{/}\DecValTok{50}\SpecialCharTok{+}\DecValTok{44}\SpecialCharTok{/}\DecValTok{50}\NormalTok{)) }\SpecialCharTok{{-}} \FunctionTok{pnorm}\NormalTok{((}\DecValTok{1{-}4}\NormalTok{)}\SpecialCharTok{/}\FunctionTok{sqrt}\NormalTok{(}\DecValTok{44}\SpecialCharTok{/}\DecValTok{50}\SpecialCharTok{+}\DecValTok{44}\SpecialCharTok{/}\DecValTok{50}\NormalTok{)) }
\end{Highlighting}
\end{Shaded}

\begin{verbatim}
## [1] 0.2136219
\end{verbatim}

\hypertarget{two-sample-hypothesis-test-concept}{%
\section{Two-Sample Hypothesis Test Concept}\label{two-sample-hypothesis-test-concept}}

\textbf{1. Objective:}

Determine whether the mean salaries of college faculty in private and public institutions are different.

\begin{itemize}
\item
  Pose the problem as a hypothesis test.
\item
  How would you solve this problem intuitively?
\end{itemize}

\textbf{2. Defining Populations}

\begin{itemize}
\item
  \textbf{Population 1}: All faculty in private institutions.
\item
  \textbf{Population 2}: All faculty in public institutions.
\end{itemize}

\textbf{3. Denoting the Means of the Variable ``Salary'':}

\begin{itemize}
\item
  \textbf{\(\mu_1\)}: Mean salary of all faculty in private institutions.
\item
  \textbf{\(\mu_2\)}: Mean salary of all faculty in public institutions.
\end{itemize}

\textbf{4. Stating the Hypothesis}

\begin{itemize}
\item
  \textbf{Null Hypothesis} (\(H_0\)): \(\mu_1=\mu_2\) (Mean salaries are the same).
\item
  \textbf{Alternative Hypothesis} (\(H_a\)): \(\mu_1 \ne \mu_2\)₂ (Mean salaries are different).
\end{itemize}

\begin{figure}
\centering
\includegraphics{https://i.ibb.co/kqQWGj7/442-Figure-10-01.png}
\caption{Two population means}
\end{figure}

\begin{figure}
\centering
\includegraphics{https://i.ibb.co/QQGx3YG/441-Table-10-03.png}
\caption{Faculty Salaries study}
\end{figure}

\textbf{5. Observations}

\begin{itemize}
\item
  \(\bar{x}_1 = 98.19\), \(\bar{x}_2 = 83.18\), \(\bar{x}_1 - \bar{x}_2 = 15.01\)
\item
  \textbf{A big question}: Is the difference of 15.01 (\$15,010) reasonably attributed to \emph{sampling error}, or is the difference large enough to indicate the two \textbf{populations} have different means?
\end{itemize}

\hypertarget{pooled-t-test-assuming-population-variances-unknown-but-equal}{%
\section{Pooled t-test assuming population variances unknown but equal}\label{pooled-t-test-assuming-population-variances-unknown-but-equal}}

\hypertarget{pooled-sample-standard-deviation}{%
\subsection{Pooled Sample Standard Deviation}\label{pooled-sample-standard-deviation}}

\textbf{Assuming Equal Standard Deviations}

\begin{itemize}
\tightlist
\item
  For \(\sigma_1 = \sigma_2 = \sigma\),\\
  \[
  z = \frac{(\bar{x}_1 - \bar{x}_2) - (\mu_1 - \mu_2)}{\sigma \sqrt{\frac{1}{n_1} + \frac{1}{n_2}}} \sim N(0, 1)
  \]
\item
  Since \(\sigma\) is unknown, \textbf{\(z\) cannot be used} as a basis for hypothesis testing.
\end{itemize}

\textbf{Using the Sample Estimate \(s_p^2\) for \(\sigma^2\)}:

\begin{itemize}
\tightlist
\item
  Subscript \(_p\) indicates the estimate is from \textbf{pooling} of \(s_1^2\) and \(s_2^2\).
\item
  \[
  s_p^2 = \frac{(n_1 - 1)s_1^2 + (n_2 - 1)s_2^2}{n_1 + n_2 - 2} = w_1s_1^2 + (1 - w_1)s_2^2,
  \]
  where \(w_1 = \frac{n_1 - 1}{n_1 + n_2 - 2}\) (proportion of sample size).
\end{itemize}

\textbf{Pooled sample standard deviation}:

\[s_p = \sqrt{\frac{(n_1 - 1)s_1^2 + (n_2 - 1)s_2^2}{n_1 + n_2 - 2}}\]

\hypertarget{distribution-of-the-pooled-t-statistic}{%
\subsection{Distribution of the Pooled t-Statistic}\label{distribution-of-the-pooled-t-statistic}}

\begin{itemize}
\item
  Suppose \(x_i \sim N(\mu_i, \sigma)\) for \(i = 1, 2\) with the same standard deviation. For independent samples of sizes \(n_1\) and \(n_2\) from the two populations, the variable:\\
  \[
  t = \frac{(\bar{x}_1 - \bar{x}_2) - (\mu_1 - \mu_2)}{s_p \sqrt{\frac{1}{n_1} + \frac{1}{n_2}}}
  \]
  follows the \textbf{t-distribution} with \(df = n_1 + n_2 - 2\).
\item
  \(t\) can be used as the test statistic to obtain the critical value \(t_\alpha\) or the \(P\)-value from the t-table.
\item
  This is called the \textbf{pooled t-test}.
\end{itemize}

\hypertarget{pooled-t-test-procedure}{%
\subsection{Pooled t-test procedure}\label{pooled-t-test-procedure}}

\begin{figure}
\centering
\includegraphics{https://i.ibb.co/dtDgpS9/Procedure10-1a.png}
\caption{Pooled t-test}
\end{figure}

\begin{figure}
\centering
\includegraphics{https://i.ibb.co/W0RkkKH/Procedure10-1b.png}
\caption{Pooled t-test}
\end{figure}

\hypertarget{example-the-pooled-t-test}{%
\subsubsection{Example: The Pooled t-Test}\label{example-the-pooled-t-test}}

*Objective**:

Independent simple random samples of 35 faculty members in private institutions and 30 faculty members in public institutions data.

At the 5\% significance level, do the data provide sufficient evidence to conclude that mean salaries for faculty in private and public institutions differ?

\begin{Shaded}
\begin{Highlighting}[]
\NormalTok{    private}\OtherTok{=}\FunctionTok{c}\NormalTok{( }\FloatTok{97.3}\NormalTok{, }\FloatTok{85.9}\NormalTok{,}\FloatTok{118.8}\NormalTok{, }\FloatTok{93.9}\NormalTok{, }\FloatTok{66.6}\NormalTok{,}\FloatTok{109.2}\NormalTok{, }\FloatTok{64.9}\NormalTok{,}
                   \FloatTok{83.1}\NormalTok{,}\FloatTok{100.6}\NormalTok{, }\FloatTok{99.3}\NormalTok{, }\FloatTok{94.9}\NormalTok{, }\FloatTok{94.4}\NormalTok{,}\FloatTok{139.3}\NormalTok{,}\FloatTok{108.8}\NormalTok{,}
                  \FloatTok{158.1}\NormalTok{,}\FloatTok{142.4}\NormalTok{, }\DecValTok{85}\NormalTok{,  }\FloatTok{108.2}\NormalTok{,}\FloatTok{116.3}\NormalTok{,}\FloatTok{141.5}\NormalTok{, }\FloatTok{51.4}\NormalTok{,}
                  \FloatTok{125.6}\NormalTok{, }\FloatTok{70.6}\NormalTok{, }\FloatTok{74.6}\NormalTok{, }\FloatTok{69.9}\NormalTok{,}\FloatTok{115.4}\NormalTok{, }\FloatTok{84.6}\NormalTok{, }\DecValTok{92}\NormalTok{,}
                   \FloatTok{97.2}\NormalTok{, }\FloatTok{55.1}\NormalTok{,}\FloatTok{126.6}\NormalTok{,}\FloatTok{116.7}\NormalTok{, }\DecValTok{76}\NormalTok{,  }\FloatTok{109.6}\NormalTok{, }\DecValTok{63}\NormalTok{)}
\NormalTok{  public}\OtherTok{=}\FunctionTok{c}\NormalTok{(}\FloatTok{59.9}\NormalTok{,}\FloatTok{115.7}\NormalTok{,}\FloatTok{126.1}\NormalTok{, }\FloatTok{50.3}\NormalTok{,}\FloatTok{133.1}\NormalTok{, }\FloatTok{89.3}\NormalTok{,}
                 \FloatTok{82.5}\NormalTok{, }\FloatTok{67.1}\NormalTok{, }\FloatTok{60.7}\NormalTok{, }\FloatTok{79.9}\NormalTok{, }\FloatTok{50.1}\NormalTok{, }\FloatTok{81.7}\NormalTok{,}
                 \FloatTok{83.9}\NormalTok{,}\FloatTok{102.5}\NormalTok{,}\FloatTok{109.9}\NormalTok{,}\FloatTok{105.1}\NormalTok{, }\FloatTok{67.9}\NormalTok{,}\FloatTok{107.5}\NormalTok{,}
                 \FloatTok{54.9}\NormalTok{, }\FloatTok{41.5}\NormalTok{, }\FloatTok{59.5}\NormalTok{, }\FloatTok{65.9}\NormalTok{, }\FloatTok{76.9}\NormalTok{, }\FloatTok{66.9}\NormalTok{,}
                 \FloatTok{85.9}\NormalTok{,}\FloatTok{113.9}\NormalTok{, }\FloatTok{70.3}\NormalTok{, }\FloatTok{90.1}\NormalTok{, }\FloatTok{99.7}\NormalTok{, }\FloatTok{96.7}\NormalTok{)}
\end{Highlighting}
\end{Shaded}

\textbf{Checking Assumptions}

\textbf{Assumption 1: Random Samples}

\begin{itemize}
\tightlist
\item
  \textbf{Simple random samples} for each population: Yes.
\end{itemize}

\textbf{Assumption 2: Independence}

\begin{itemize}
\tightlist
\item
  \textbf{Independent samples} for both populations: Yes.
\end{itemize}

\textbf{Assumption 3: Equal Standard Deviations}

\begin{itemize}
\tightlist
\item
  The sample standard deviations are 26.21 and 23.95. Close enough, so the assumption is met.
\end{itemize}

\begin{figure}
\centering
\includegraphics{https://i.ibb.co/t8R9wvq/449-Table-10-06.png}
\caption{sample statistics}
\end{figure}

\begin{Shaded}
\begin{Highlighting}[]
\NormalTok{x1bar }\OtherTok{=} \FunctionTok{mean}\NormalTok{(private)}
\NormalTok{s1 }\OtherTok{=} \FunctionTok{sd}\NormalTok{(private)}
\NormalTok{n1}\OtherTok{=}\FunctionTok{length}\NormalTok{(private)}

\NormalTok{x2bar }\OtherTok{=} \FunctionTok{mean}\NormalTok{(public)}
\NormalTok{s2 }\OtherTok{=} \FunctionTok{sd}\NormalTok{(public)}
\NormalTok{n2}\OtherTok{=}\FunctionTok{length}\NormalTok{(public)}

\NormalTok{x1bar}
\end{Highlighting}
\end{Shaded}

\begin{verbatim}
## [1] 98.19429
\end{verbatim}

\begin{Shaded}
\begin{Highlighting}[]
\NormalTok{s1}
\end{Highlighting}
\end{Shaded}

\begin{verbatim}
## [1] 26.20776
\end{verbatim}

\begin{Shaded}
\begin{Highlighting}[]
\NormalTok{n1}
\end{Highlighting}
\end{Shaded}

\begin{verbatim}
## [1] 35
\end{verbatim}

\begin{Shaded}
\begin{Highlighting}[]
\NormalTok{x2bar}
\end{Highlighting}
\end{Shaded}

\begin{verbatim}
## [1] 83.18
\end{verbatim}

\begin{Shaded}
\begin{Highlighting}[]
\NormalTok{s2}
\end{Highlighting}
\end{Shaded}

\begin{verbatim}
## [1] 23.9528
\end{verbatim}

\begin{Shaded}
\begin{Highlighting}[]
\NormalTok{n2}
\end{Highlighting}
\end{Shaded}

\begin{verbatim}
## [1] 30
\end{verbatim}

\textbf{Normality Check}

\begin{itemize}
\tightlist
\item
  \textbf{Normal populations or large samples}:\\
  The sample sizes are 35 and 30, both large.\\
  Normality check passed.
\end{itemize}

\begin{figure}
\centering
\includegraphics{https://i.ibb.co/VjWwmw0/450-Figure-10-02.png}
\caption{normality}
\end{figure}

\begin{Shaded}
\begin{Highlighting}[]
\FunctionTok{normal.prob.plot}\NormalTok{(private)}
\end{Highlighting}
\end{Shaded}

\includegraphics{StatsTB_files/figure-latex/unnamed-chunk-216-1.pdf} \includegraphics{StatsTB_files/figure-latex/unnamed-chunk-216-2.pdf}

\begin{verbatim}
##        y normal.score
## 1   51.4  -2.18934976
## 2   55.1  -1.71845154
## 3   63.0  -1.46523379
## 4   64.9  -1.28155157
## 5   66.6  -1.13317003
## 6   69.9  -1.00626999
## 7   70.6  -0.89380063
## 8   74.6  -0.79163861
## 9   76.0  -0.69714143
## 10  83.1  -0.60849813
## 11  84.6  -0.52440051
## 12  85.0  -0.44386131
## 13  85.9  -0.36610636
## 14  92.0  -0.29050677
## 15  93.9  -0.21653412
## 16  94.4  -0.14372923
## 17  94.9  -0.07167928
## 18  97.2   0.00000000
## 19  97.3   0.07167928
## 20  99.3   0.14372923
## 21 100.6   0.21653412
## 22 108.2   0.29050677
## 23 108.8   0.36610636
## 24 109.2   0.44386131
## 25 109.6   0.52440051
## 26 115.4   0.60849813
## 27 116.3   0.69714143
## 28 116.7   0.79163861
## 29 118.8   0.89380063
## 30 125.6   1.00626999
## 31 126.6   1.13317003
## 32 139.3   1.28155157
## 33 141.5   1.46523379
## 34 142.4   1.71845154
## 35 158.1   2.18934976
\end{verbatim}

\begin{Shaded}
\begin{Highlighting}[]
\FunctionTok{normal.prob.plot}\NormalTok{(public)}
\end{Highlighting}
\end{Shaded}

\includegraphics{StatsTB_files/figure-latex/unnamed-chunk-216-3.pdf} \includegraphics{StatsTB_files/figure-latex/unnamed-chunk-216-4.pdf}

\begin{verbatim}
##        y normal.score
## 1   41.5   -2.1280452
## 2   50.1   -1.6448536
## 3   50.3   -1.3829941
## 4   54.9   -1.1918162
## 5   59.5   -1.0364334
## 6   59.9   -0.9027348
## 7   60.7   -0.7835004
## 8   65.9   -0.6744898
## 9   66.9   -0.5729675
## 10  67.1   -0.4770404
## 11  67.9   -0.3853205
## 12  70.3   -0.2967378
## 13  76.9   -0.2104284
## 14  79.9   -0.1256613
## 15  81.7   -0.0417893
## 16  82.5    0.0417893
## 17  83.9    0.1256613
## 18  85.9    0.2104284
## 19  89.3    0.2967378
## 20  90.1    0.3853205
## 21  96.7    0.4770404
## 22  99.7    0.5729675
## 23 102.5    0.6744898
## 24 105.1    0.7835004
## 25 107.5    0.9027348
## 26 109.9    1.0364334
## 27 113.9    1.1918162
## 28 115.7    1.3829941
## 29 126.1    1.6448536
## 30 133.1    2.1280452
\end{verbatim}

\textbf{Boxplot of Data}

\begin{itemize}
\tightlist
\item
  A descriptive boxplot shows a considerable difference between the groups.
\end{itemize}

\begin{figure}
\centering
\includegraphics{https://i.ibb.co/ygP3QXs/450-Figure-10-03.png}
\caption{Boxplots}
\end{figure}

\begin{Shaded}
\begin{Highlighting}[]
\FunctionTok{boxplot}\NormalTok{(}\FunctionTok{c}\NormalTok{(private, public) }\SpecialCharTok{\textasciitilde{}} \FunctionTok{c}\NormalTok{(}\FunctionTok{rep}\NormalTok{(}\StringTok{"private"}\NormalTok{, }\FunctionTok{length}\NormalTok{(private)), }\FunctionTok{rep}\NormalTok{(}\StringTok{"public"}\NormalTok{, }\FunctionTok{length}\NormalTok{(public))), }\AttributeTok{col=}\StringTok{"turquoise"}\NormalTok{, }\AttributeTok{horizontal =}\ConstantTok{TRUE}\NormalTok{)}
\end{Highlighting}
\end{Shaded}

\includegraphics{StatsTB_files/figure-latex/unnamed-chunk-217-1.pdf}

\textbf{Performing the Pooled t-Test}

Step 1: Hypotheses:

\begin{itemize}
\tightlist
\item
  \(H_0: \mu_1 = \mu_2\) (mean salaries are the same).
\item
  \(H_a: \mu_1 \neq \mu_2\) (mean salaries are different).
\end{itemize}

Step 2: Significance Level:

\begin{itemize}
\tightlist
\item
  \(\alpha = 0.05\).
\end{itemize}

Step 3: Compute the Test Statistic:

\begin{itemize}
\item
  Formula:\\
  \(t = \frac{\bar{x}_1 - \bar{x}_2}{s_p \sqrt{\frac{1}{n_1} + \frac{1}{n_2}}}\)\\
  where \(s_p = \sqrt{\frac{(n_1 - 1)s_1^2 + (n_2 - 1)s_2^2}{n_1 + n_2 - 2}}\).
\item
  Given \(s_1 = 26.21\), \(s_2 = 23.95\), compute:\\
  \(s_p = \sqrt{\frac{(35 - 1)26.21^2 + (30 - 1)23.95^2}{35 + 30 - 2}} = 25.19\).
\item
  Compute \(t\):\\
  \(t = \frac{\bar{x}_1 - \bar{x}_2}{25.19 \cdot \sqrt{\frac{1}{35} + \frac{1}{30}}} = 2.395\).
\end{itemize}

Step 4: Determine Critical Values:

\begin{itemize}
\tightlist
\item
  Two-tailed test critical values: \(\pm t_{\alpha/2}\) with \(df = n_1 + n_2 - 2 = 63\). Use R: \texttt{qt(1\ -\ 0.025,\ df\ =\ 63)} to obtain \(t_{0.975} = 1.998\).
\end{itemize}

Step 5: Decision:

\begin{itemize}
\tightlist
\item
  Observed \(t = 2.395\) falls in the rejection region.\\
  Reject \(H_0\).
\end{itemize}

Step 4: Compute P-Value:

\begin{itemize}
\tightlist
\item
  \(P(|t| > 2.395)\) with \(df = 63\). Use R: \texttt{2\ *\ (1\ -\ pt(q\ =\ 2.395,\ df\ =\ 63))\ =\ 0.0196}.
\end{itemize}

Step 5: Decision:

\begin{itemize}
\tightlist
\item
  Since \(P < 0.05\), reject \(H_0\).
\end{itemize}

\textbf{Use R function}

\begin{Shaded}
\begin{Highlighting}[]
\FunctionTok{two.mean.t}\NormalTok{(}\AttributeTok{x1=}\NormalTok{private, }\AttributeTok{x2=}\NormalTok{public, }\AttributeTok{equal.variance =} \ConstantTok{TRUE}\NormalTok{, }\AttributeTok{tail=}\StringTok{"two"}\NormalTok{, }\AttributeTok{alpha=}\FloatTok{0.05}\NormalTok{)}
\end{Highlighting}
\end{Shaded}

\begin{verbatim}
## $CI
## [1]  2.487398 27.541173
## 
## $xbar
## [1] 98.19429 83.18000
## 
## $mu_diff
## [1] 15.01429
## 
## $n
## [1] 35 30
## 
## $margin
## [1] 12.52689
## 
## $p
## [1] 0.0195989
## 
## $s
## [1] 26.20776 23.95280
## 
## $sp
## [1] 25.19484
## 
## $t0
## [1] 2.39514
## 
## $df
## [1] 63
## 
## $CV
## [1] -1.998341  1.998341
## 
## $conclusion
## [1] "H0 Rejected"
## 
## $equal.variance
## [1] TRUE
\end{verbatim}

\textbf{Using R built-in function}

\begin{Shaded}
\begin{Highlighting}[]
    \FunctionTok{t.test}\NormalTok{(}\AttributeTok{x=}\NormalTok{private, }\AttributeTok{y=}\NormalTok{public, }\AttributeTok{var.equal =} \ConstantTok{TRUE}\NormalTok{) }\CommentTok{\#variance equal}
\end{Highlighting}
\end{Shaded}

\begin{verbatim}
## 
##  Two Sample t-test
## 
## data:  private and public
## t = 2.3951, df = 63, p-value = 0.0196
## alternative hypothesis: true difference in means is not equal to 0
## 95 percent confidence interval:
##   2.487398 27.541173
## sample estimates:
## mean of x mean of y 
##  98.19429  83.18000
\end{verbatim}

\hypertarget{example-comparing-two-cholesterol-lowering-drugs}{%
\subsection{Example: Comparing Two Cholesterol-Lowering Drugs}\label{example-comparing-two-cholesterol-lowering-drugs}}

Researchers test whether two different drugs (Drug X and Drug Y) result in different reductions in LDL cholesterol levels.

\begin{itemize}
\item
  Group 1 (Drug X): \(n_1=30\), sample mean 102 mg/dL, sample standard deviation \(s_1=12\)
\item
  Group 2 (Drug Y): \(n_2=30\), sample mean 110 mg/dL, sample standard deviation \(s_2=14\)
\end{itemize}

We assume equal population variances and calculate the \textbf{pooled sample variance}:

\[
s_p^2 = \frac{(30 - 1)(12)^2 + (30 - 1)(14)^2}{30 + 30 - 2} = \frac{29(144) + 29(196)}{58} = \frac{4176 + 5684}{58} = \frac{9860}{58} \approx 170
\]

Now compute the test statistic:

\[
t = \frac{102 - 110}{\sqrt{170 \left( \frac{1}{30} + \frac{1}{30} \right)}} = \frac{-8}{\sqrt{170 \cdot \frac{2}{30}}} = \frac{-8}{\sqrt{11.33}} \approx \frac{-8}{3.36} \approx -2.38
\]

With \textbf{df = 58}, and a significance level of \(\alpha = 0.05\), the critical t-value for a two-tailed test is approximately \(\pm 2.001\).

Since \(-2.38 < -2.001\), we reject \(H_0\). There is a statistically significant difference in LDL reduction between the two drugs.

\textbf{Summary}

The \textbf{pooled t-test} is used when population variances are assumed to be equal, even though they are unknown. This assumption must be justified --- often through pre-analysis tests or prior studies.

\textbf{Using the R function:}

\begin{Shaded}
\begin{Highlighting}[]
\FunctionTok{two.mean.t}\NormalTok{(}\AttributeTok{x1bar=}\DecValTok{102}\NormalTok{, }\AttributeTok{n1=}\DecValTok{30}\NormalTok{, }\AttributeTok{s1=}\DecValTok{12}\NormalTok{,}
          \AttributeTok{x2bar=}\DecValTok{110}\NormalTok{, }\AttributeTok{n2=}\DecValTok{30}\NormalTok{, }\AttributeTok{s2=}\DecValTok{14}\NormalTok{,}
          \AttributeTok{tail=}\StringTok{"two"}\NormalTok{, }\AttributeTok{alpha=}\FloatTok{0.05}\NormalTok{, }\AttributeTok{equal.variance =} \ConstantTok{TRUE}\NormalTok{)}
\end{Highlighting}
\end{Shaded}

\begin{verbatim}
## $CI
## [1] -14.738785  -1.261215
## 
## $xbar
## [1] 102 110
## 
## $mu_diff
## [1] -8
## 
## $n
## [1] 30 30
## 
## $margin
## [1] 6.738785
## 
## $p
## [1] 0.02080692
## 
## $s
## [1] 12 14
## 
## $sp
## [1] 13.0384
## 
## $t0
## [1] -2.376354
## 
## $df
## [1] 58
## 
## $CV
## [1] -2.001717  2.001717
## 
## $conclusion
## [1] "H0 Rejected"
## 
## $equal.variance
## [1] TRUE
\end{verbatim}

\hypertarget{pooled-t-interval}{%
\section{Pooled t-interval}\label{pooled-t-interval}}

For a confidence level of \(1-\alpha\), the CI for \(\mu_1 - \mu_2\) is

\[
\bar{x}_1-\bar{x}_2 \pm t_{1-\alpha/2}\cdot s_p \sqrt{\frac{1}{n_1}+\frac{1}{n_2}}
\]
where \(t_{1-\alpha/2}\) is the \((1-\alpha/2)\) quantile of the t-distribution with \(df = n_1+n_2-2\). Using R code: \(t_{1-\alpha/2}\) = qt(1-alpha/2, df=n1+n2-2).

\begin{figure}
\centering
\includegraphics{https://i.ibb.co/H7v6sS4/Procedure10-2.png}
\caption{Interval}
\end{figure}

\begin{Shaded}
\begin{Highlighting}[]
\FunctionTok{ci.mu.diff}\NormalTok{(}\AttributeTok{x1=}\NormalTok{private, }\AttributeTok{x2=}\NormalTok{public, }\AttributeTok{known.variance =} \ConstantTok{FALSE}\NormalTok{, }\AttributeTok{equal.variance =} \ConstantTok{TRUE}\NormalTok{, }\AttributeTok{conflev =} \FloatTok{0.95}\NormalTok{)}
\end{Highlighting}
\end{Shaded}

\begin{verbatim}
## $CI
## [1]  2.487398 27.541173
## 
## $mean.diff
## [1] 15.01429
## 
## $se.mean.diff
## [1] 6.268645
## 
## $known.variance
## [1] FALSE
## 
## $equal.variance
## [1] TRUE
## 
## $sp
## [1] 25.19484
\end{verbatim}

\hypertarget{two-sample-z-test}{%
\section{Two-sample z-test}\label{two-sample-z-test}}

It is rare that the population variances of both independent groups are \textbf{known}. Nevertheless, in such case, we can use a \textbf{two-sample z-test} to compare their means.

\hypertarget{assumptions}{%
\subsection{Assumptions}\label{assumptions}}

\begin{itemize}
\item
  Both populations are normally distributed or the sample sizes are large.
\item
  The population variances \(\sigma_1^2\) and \(\sigma_2^2\) are known.
\item
  Samples are independent and randomly selected
\end{itemize}

\hypertarget{test-statistic-2}{%
\subsection{Test Statistic}\label{test-statistic-2}}

\[
z = \frac{\bar{x}_1 - \bar{x}_2}{\sqrt{\frac{\sigma_1^2}{n_1} + \frac{\sigma_2^2}{n_2}}}
\]

Under null hypothesis, \(z \sim N(0, 1)\). We compare the calculated z-value to the critical value from the standard normal distribution, or we compute a \textbf{p-value}.

\hypertarget{example-comparing-two-types-of-insulin}{%
\subsection{Example: Comparing Two Types of Insulin}\label{example-comparing-two-types-of-insulin}}

A researcher wants to compare the effectiveness of \textbf{Insulin A} and \textbf{Insulin B} in reducing fasting blood glucose levels in diabetic patients.

\begin{itemize}
\item
  Group 1 (Insulin A): \(n_1=50\), sample mean 110 mg/dL, population standard deviation \(\sigma_1=15\)
\item
  Group 2 (Insulin B): \(n_2=50\), sample mean 118 mg/dL, population standard deviation \(\sigma_2=12\)
\end{itemize}

Using the two-sample z-test:

\[
z = \frac{110 - 118}{\sqrt{\frac{15^2}{50} + \frac{12^2}{50}}} = \frac{-8}{\sqrt{4.5 + 2.88}} = \frac{-8}{\sqrt{7.38}} \approx -2.94
\]

At \(\alpha = 0.05\), the critical value is \(\pm 1.96\).

Since \(-2.94 < -1.96\), we \textbf{reject the null hypothesis}. There is a statistically significant difference in fasting glucose levels between the two insulin types.

\textbf{Note}

The z-test is rarely used in practice because population variances are often unknown. However, when the variances are known (e.g., from large-scale historical data), it may be useful.

\textbf{Using the R function:}

\begin{Shaded}
\begin{Highlighting}[]
\FunctionTok{two.mean.z}\NormalTok{(}\AttributeTok{x1bar=}\DecValTok{110}\NormalTok{, }\AttributeTok{n1=}\DecValTok{50}\NormalTok{, }\AttributeTok{sigma1=}\DecValTok{15}\NormalTok{,}
          \AttributeTok{x2bar=}\DecValTok{118}\NormalTok{, }\AttributeTok{n2=}\DecValTok{50}\NormalTok{, }\AttributeTok{sigma2=}\DecValTok{12}\NormalTok{,}
          \AttributeTok{tail=}\StringTok{"two"}\NormalTok{, }\AttributeTok{alpha=}\FloatTok{0.05}\NormalTok{)}
\end{Highlighting}
\end{Shaded}

\begin{verbatim}
## $CI
## [1] -13.324469  -2.675531
## 
## $mu_diff
## [1] -8
## 
## $n
## [1] 50 50
## 
## $margin
## [1] 5.324469
## 
## $p
## [1] 0.00323121
## 
## $z0
## [1] -2.944841
## 
## $CV
## [1] -1.959964  1.959964
## 
## $conclusion
## [1] "H0 Rejected"
\end{verbatim}

\hypertarget{two-sample-unpooled-t-test-assuming-population-variances-unknown-and-unequal}{%
\section{Two-sample unpooled t-test assuming population variances unknown and unequal}\label{two-sample-unpooled-t-test-assuming-population-variances-unknown-and-unequal}}

\hypertarget{welchs-t-test}{%
\subsection{Welch's t-Test}\label{welchs-t-test}}

When comparing two independent population means but \textbf{cannot assume equal variances}, we use the \textbf{unpooled t-test}, also called \textbf{Welch's t-test}.

Unlike the pooled t-test, this test does not pool the variances and adjusts the degrees of freedom accordingly.

The test statistic

\[
t = \frac{\bar{x}_1 - \bar{x}_2}{\sqrt{ \frac{s_1^2}{n_1} + \frac{s_2^2}{n_2} }}
\]
has approximately a \textbf{t-distribution}, and the degrees of freedom are calculated using the Welch--Satterthwaite equation:

\[
df = \frac{ \left( \frac{s_1^2}{n_1} + \frac{s_2^2}{n_2} \right)^2 }
{ \frac{ \left( \frac{s_1^2}{n_1} \right)^2 }{n_1 - 1} + \frac{ \left( \frac{s_2^2}{n_2} \right)^2 }{n_2 - 1} }
\]
Please note: R uses the exact df without approximation. So there might be a small numerical difference when the approximate df is used from the formula above.

\hypertarget{cochran-method}{%
\subsection{Cochran method}\label{cochran-method}}

The test statistic is
\[t^\prime = \frac{(\bar{x}_1-\bar{x}_2)-(\mu_1-\mu_2)}{\sqrt{\frac{s_1^2}{n_1}+\frac{s_2^2}{n_2}}}\]

The critical value of \(t^\prime\) for \(\alpha\) level 2-sided test is\\
\[t^\prime_{1-\alpha/2} = \frac{w_1t_1+w_2t_2}{w_1+w_2}\]

where \[w_1=s_1^2/n_1, w_2=s_2^2/n_2\],
\[t_1=t_{1-\alpha/2, df=n_1-1}\]
and
\[t_2=t_{1-\alpha/2, df=n_2-1}\].

The \(100(1-\alpha)\%\) confidence interval for \(\mu_1-\mu_2\) is given by

\[
(\bar{x}_1-\bar{x}_2) \pm t_{1-\alpha/2}^\prime\sqrt{\frac{s_1^2}{n_1}+\frac{s_2^2}{n_2}}
\]

The hypothesis testing procedure is, if the test is 2-sided \(H_0: \mu_1 = \mu_2\), then

-The significance level \(\alpha\) test has rejection region \(t > t_{1-\alpha/2}^\prime\) or \(t < -t_{1-\alpha/2}^\prime\), where \(t_{1-\alpha/2}^\prime\), \(t_{1-\alpha/2}\) is the \((1-\alpha/2)100\%\)th percentile of \(t\) distribution with \(df=n_1+n_2-2\).
-The observed value of the test statistic is calculated as \(t_0^\prime = \frac{(\bar{X}_1-\bar{X}_2)}{\sqrt{\frac{s_1^2}{n_1}+\frac{s_2^2}{n_2}}}\).
-If \(t_0^\prime\) falls into the rejection region, then reject \(H_0\).

For right-tailed test, the rejection region is \(t > t_{1-\alpha}^\prime\). For left-tailed test, the rejection region is \(t < -t_{1-\alpha}^\prime\).

\hypertarget{example-45}{%
\subsection{Example}\label{example-45}}

One study was to determine if there were differing levels of the anticardiolipin antibody IgG in subjects with and without thrombosis. The researchers observed the following data:

-Thrombosis group: mean IgG level 59.01, n = 53, sd = 44.89
-No Thrombosis group: mean IgG level 46.61, n = 54, sd = 34.85

\textbf{Question}.

Perform a hypothesis test of significance level \(\alpha=0.05\) to address the question: Are we able to conclude persons with thrombosis have, on average, higher IgG levels than those without thrombosis?

\textbf{Solution}.

\textbf{Welch Method}

\begin{Shaded}
\begin{Highlighting}[]
\FunctionTok{two.mean.t}\NormalTok{(}\AttributeTok{x1bar=}\FloatTok{59.01}\NormalTok{, }\AttributeTok{n1=}\DecValTok{53}\NormalTok{, }\AttributeTok{s1=}\FloatTok{44.89}\NormalTok{,}
                     \AttributeTok{x2bar=}\FloatTok{46.61}\NormalTok{, }\AttributeTok{n2=}\DecValTok{54}\NormalTok{, }\AttributeTok{s2=}\FloatTok{34.85}\NormalTok{,}
                     \AttributeTok{tail=}\StringTok{"right"}\NormalTok{, }\AttributeTok{alpha=}\FloatTok{0.05}\NormalTok{,}
                     \AttributeTok{equal.variance=}\ConstantTok{FALSE}\NormalTok{, }\AttributeTok{uneqvar.method =} \StringTok{"Welch"}\NormalTok{)}
\end{Highlighting}
\end{Shaded}

\begin{verbatim}
## $CI
## [1] -3.036979 27.836979
## 
## $xbar
## [1] 59.01 46.61
## 
## $mu_diff
## [1] 12.4
## 
## $n
## [1] 53 54
## 
## $margin
## [1] 15.43698
## 
## $p
## [1] 0.05707229
## 
## $s
## [1] 44.89 34.85
## 
## $se
## [1] 7.778955
## 
## $t0
## [1] 1.594044
## 
## $df
## [1] 98.05299
## 
## $CV
## [1] 1.660543
## 
## $conclusion
## [1] "H0 Not Rejected"
## 
## $equal.variance
## [1] FALSE
## 
## $uneqvar.method
## [1] "Welch"
\end{verbatim}

\textbf{Cochran Method}

\begin{Shaded}
\begin{Highlighting}[]
\FunctionTok{two.mean.t}\NormalTok{(}\AttributeTok{x1bar=}\FloatTok{59.01}\NormalTok{, }\AttributeTok{n1=}\DecValTok{53}\NormalTok{, }\AttributeTok{s1=}\FloatTok{44.89}\NormalTok{,}
                     \AttributeTok{x2bar=}\FloatTok{46.61}\NormalTok{, }\AttributeTok{n2=}\DecValTok{54}\NormalTok{, }\AttributeTok{s2=}\FloatTok{34.85}\NormalTok{,}
                     \AttributeTok{tail=}\StringTok{"right"}\NormalTok{, }\AttributeTok{alpha=}\FloatTok{0.05}\NormalTok{,}
                     \AttributeTok{equal.variance=}\ConstantTok{FALSE}\NormalTok{, }\AttributeTok{uneqvar.method =} \StringTok{"Cochran"}\NormalTok{)}
\end{Highlighting}
\end{Shaded}

\begin{verbatim}
## $CI
## [1] -0.6256751 25.4256751
## 
## $xbar
## [1] 59.01 46.61
## 
## $mu_diff
## [1] 12.4
## 
## $n
## [1] 53 54
## 
## $margin
## [1] 13.02568
## 
## $s
## [1] 44.89 34.85
## 
## $se
## [1] 7.778955
## 
## $t0
## [1] 1.594044
## 
## $df
## [1] 52 53
## 
## $CV
## [1] 1.674476
## 
## $conclusion
## [1] "H0 Not Rejected"
## 
## $equal.variance
## [1] FALSE
## 
## $uneqvar.method
## [1] "Cochran"
\end{verbatim}

\textbf{Try out the pooled t-test:}

\begin{Shaded}
\begin{Highlighting}[]
\FunctionTok{two.mean.t}\NormalTok{(}\AttributeTok{x1bar=}\FloatTok{59.01}\NormalTok{, }\AttributeTok{n1=}\DecValTok{53}\NormalTok{, }\AttributeTok{s1=}\FloatTok{44.89}\NormalTok{,}
                     \AttributeTok{x2bar=}\FloatTok{46.61}\NormalTok{, }\AttributeTok{n2=}\DecValTok{54}\NormalTok{, }\AttributeTok{s2=}\FloatTok{34.85}\NormalTok{,}
                     \AttributeTok{tail=}\StringTok{"right"}\NormalTok{, }\AttributeTok{alpha=}\FloatTok{0.05}\NormalTok{,}
                     \AttributeTok{equal.variance=}\ConstantTok{TRUE}\NormalTok{)}
\end{Highlighting}
\end{Shaded}

\begin{verbatim}
## $CI
## [1] -2.988198 27.788198
## 
## $xbar
## [1] 59.01 46.61
## 
## $mu_diff
## [1] 12.4
## 
## $n
## [1] 53 54
## 
## $margin
## [1] 15.3882
## 
## $p
## [1] 0.05654888
## 
## $s
## [1] 44.89 34.85
## 
## $sp
## [1] 40.13733
## 
## $t0
## [1] 1.597777
## 
## $df
## [1] 105
## 
## $CV
## [1] 1.659495
## 
## $conclusion
## [1] "H0 Not Rejected"
## 
## $equal.variance
## [1] TRUE
\end{verbatim}

\hypertarget{unpooled-t-test-procedure}{%
\subsection{Unpooled t-test Procedure}\label{unpooled-t-test-procedure}}

\begin{figure}
\centering
\includegraphics{https://i.ibb.co/LPvpy9j/Procedure10-3a.png}
\caption{Unpooled t-test}
\end{figure}

\includegraphics{https://i.ibb.co/8D8hkSz/Procedure10-3b.png}
\#\#\# Example.

Several neurosurgeons wanted to determine whether a dynamic system reduced the operative time relative to a static system (Table 10.7). At the 5\% significance level, do the data provide sufficient evidence to conclude that mean operative time is less with the dynamic system than with the static system?

\begin{figure}
\centering
\includegraphics{https://i.ibb.co/z4mNmXY/460-Table-10-07.png}
\caption{Example}
\end{figure}

The summary statistics are calculated below. The two sample standard deviations are \textcolor{blue}{considerably different}, so the pooled t-test is not appropriate.

\begin{figure}
\centering
\includegraphics{https://i.ibb.co/DkGrwHx/460-Table-10-08.png}
\caption{Example}
\end{figure}

\begin{Shaded}
\begin{Highlighting}[]
\NormalTok{dynamic }\OtherTok{=} \FunctionTok{c}\NormalTok{(}\DecValTok{370}\NormalTok{, }\DecValTok{360}\NormalTok{, }\DecValTok{510}\NormalTok{, }\DecValTok{445}\NormalTok{, }\DecValTok{295}\NormalTok{, }\DecValTok{315}\NormalTok{, }\DecValTok{490}\NormalTok{,}\DecValTok{345}\NormalTok{, }
    \DecValTok{450}\NormalTok{, }\DecValTok{505}\NormalTok{,}\DecValTok{335}\NormalTok{,}\DecValTok{280}\NormalTok{,}\DecValTok{325}\NormalTok{,}\DecValTok{500}\NormalTok{)}
\NormalTok{static }\OtherTok{=} \FunctionTok{c}\NormalTok{(}\DecValTok{430}\NormalTok{,}\DecValTok{445}\NormalTok{,}\DecValTok{455}\NormalTok{,}\DecValTok{455}\NormalTok{,}\DecValTok{490}\NormalTok{,}\DecValTok{535}\NormalTok{) }
    
\NormalTok{x1bar }\OtherTok{=} \FunctionTok{mean}\NormalTok{(dynamic)}
\NormalTok{s1 }\OtherTok{=} \FunctionTok{sd}\NormalTok{(dynamic)}
\NormalTok{n1}\OtherTok{=}\FunctionTok{length}\NormalTok{(dynamic)}

\NormalTok{x2bar }\OtherTok{=} \FunctionTok{mean}\NormalTok{(static)}
\NormalTok{s2 }\OtherTok{=} \FunctionTok{sd}\NormalTok{(static)}
\NormalTok{n2}\OtherTok{=}\FunctionTok{length}\NormalTok{(static)}

\NormalTok{x1bar}
\end{Highlighting}
\end{Shaded}

\begin{verbatim}
## [1] 394.6429
\end{verbatim}

\begin{Shaded}
\begin{Highlighting}[]
\NormalTok{s1}
\end{Highlighting}
\end{Shaded}

\begin{verbatim}
## [1] 84.74996
\end{verbatim}

\begin{Shaded}
\begin{Highlighting}[]
\NormalTok{n1}
\end{Highlighting}
\end{Shaded}

\begin{verbatim}
## [1] 14
\end{verbatim}

\begin{Shaded}
\begin{Highlighting}[]
\NormalTok{x2bar}
\end{Highlighting}
\end{Shaded}

\begin{verbatim}
## [1] 468.3333
\end{verbatim}

\begin{Shaded}
\begin{Highlighting}[]
\NormalTok{s2}
\end{Highlighting}
\end{Shaded}

\begin{verbatim}
## [1] 38.1663
\end{verbatim}

\begin{Shaded}
\begin{Highlighting}[]
\NormalTok{n2}
\end{Highlighting}
\end{Shaded}

\begin{verbatim}
## [1] 6
\end{verbatim}

The boxplot shows considerable difference in the distribution of the variable:

\begin{figure}
\centering
\includegraphics{https://i.ibb.co/qM1z90M/460-Figure-10-06.png}
\caption{Boxplot}
\end{figure}

\textbf{Check Assumptions}

\begin{itemize}
\item
  The data were obtained from a randomized comparative experiment, so the \textbf{assumptions of independent samples} are met.
\item
  \textbf{Normality assumption} is checked by normal probability plots and boxplots. No outliers are observed. The non-pooled t-test is robust to moderate violations of normality.
\end{itemize}

\begin{Shaded}
\begin{Highlighting}[]
\FunctionTok{normal.prob.plot}\NormalTok{(dynamic)}
\end{Highlighting}
\end{Shaded}

\includegraphics{StatsTB_files/figure-latex/unnamed-chunk-227-1.pdf} \includegraphics{StatsTB_files/figure-latex/unnamed-chunk-227-2.pdf}

\begin{verbatim}
##      y normal.score
## 1  280  -1.80274309
## 2  295  -1.24186679
## 3  315  -0.92082298
## 4  325  -0.67448975
## 5  335  -0.46370775
## 6  345  -0.27188001
## 7  360  -0.08964235
## 8  370   0.08964235
## 9  445   0.27188001
## 10 450   0.46370775
## 11 490   0.67448975
## 12 500   0.92082298
## 13 505   1.24186679
## 14 510   1.80274309
\end{verbatim}

\begin{Shaded}
\begin{Highlighting}[]
\FunctionTok{normal.prob.plot}\NormalTok{(static)}
\end{Highlighting}
\end{Shaded}

\includegraphics{StatsTB_files/figure-latex/unnamed-chunk-227-3.pdf} \includegraphics{StatsTB_files/figure-latex/unnamed-chunk-227-4.pdf}

\begin{verbatim}
##     y normal.score
## 1 430   -1.2815516
## 2 445   -0.6433454
## 3 455   -0.2018935
## 4 455    0.2018935
## 5 490    0.6433454
## 6 535    1.2815516
\end{verbatim}

\textbf{Perform the Non-Pooled t-Test}

\textbf{Step 1: Hypotheses}

\begin{itemize}
\tightlist
\item
  \(H_0: \mu_1 = \mu_2\) (mean dynamic time is not less than mean static time).\\
\item
  \(H_a: \mu_1 < \mu_2\) (mean dynamic time is less than mean static time).
\end{itemize}

\textbf{Step 2: Significance Level}

\begin{itemize}
\tightlist
\item
  \(\alpha = 0.05\)
\end{itemize}

\textbf{Step 3: Compute the Test Statistic}

\begin{itemize}
\tightlist
\item
  Formula:\\
  \[
  t = \frac{\bar{x}_1 - \bar{x}_2}{\sqrt{\frac{s_1^2}{n_1} + \frac{s_2^2}{n_2}}}
  \]\\
\item
  Substituting values:\\
  \[t = \frac{394.6 - 46.3}{\sqrt{\frac{84.7^2}{14} + \frac{38.2^2}{6}}} = -2.681\]
\end{itemize}

\textbf{Step 4: Determine Critical Value}

\begin{itemize}
\item
  Formula for degrees of freedom (df):\\
  \[\Delta = \frac{\left(\frac{84.7^2}{14} + \frac{38.2^2}{6}\right)^2}{\frac{\left(\frac{84.7^2}{14}\right)^2}{14 - 1} + \frac{\left(\frac{38.2^2}{6}\right)^2}{6 - 1}} \approx 17\]
\item
  Critical value for the left-tailed test: \(-t_{\alpha}\). Use R: \texttt{qt(0.05,\ df\ =\ 17)}. \(-t_{0.05} = -1.740\).
\end{itemize}

\includegraphics{https://i.ibb.co/3p28cW2/461-Figure-10-07a.png}
\includegraphics{https://i.ibb.co/GpqxZr4/461-Figure-10-07b.png}

\textbf{Step 5: Decision}

\begin{itemize}
\tightlist
\item
  The observed test statistic \(t = -2.681\) falls in the rejection region. Therefore, \textbf{reject \(H_0\)}.
\end{itemize}

At the 5\% significance level, the data provide sufficient evidence to conclude that the mean operative time is less with the dynamic system than with the static system.

\textbf{Using R function:}

\begin{Shaded}
\begin{Highlighting}[]
\FunctionTok{two.mean.t}\NormalTok{(}\AttributeTok{x1=}\NormalTok{dynamic, }\AttributeTok{x2=}\NormalTok{static, }\AttributeTok{equal.variance =} \ConstantTok{FALSE}\NormalTok{)}
\end{Highlighting}
\end{Shaded}

\begin{verbatim}
## $CI
## [1] -131.48826  -15.89269
## 
## $xbar
## [1] 394.6429 468.3333
## 
## $mu_diff
## [1] -73.69048
## 
## $n
## [1] 14  6
## 
## $margin
## [1] 57.79779
## 
## $p
## [1] 0.01535813
## 
## $s
## [1] 84.74996 38.16630
## 
## $se
## [1] 27.49213
## 
## $t0
## [1] -2.68042
## 
## $df
## [1] 17.83231
## 
## $CV
## [1] 2.102339
## 
## $conclusion
## [1] "H0 Rejected"
## 
## $equal.variance
## [1] FALSE
## 
## $uneqvar.method
## [1] "Welch"
\end{verbatim}

\textbf{Using R Built-In function:}

\begin{Shaded}
\begin{Highlighting}[]
\FunctionTok{t.test}\NormalTok{(}\AttributeTok{x=}\NormalTok{dynamic, }\AttributeTok{y=}\NormalTok{static, }\AttributeTok{var.equal  =} \ConstantTok{FALSE}\NormalTok{)}
\end{Highlighting}
\end{Shaded}

\begin{verbatim}
## 
##  Welch Two Sample t-test
## 
## data:  dynamic and static
## t = -2.6804, df = 17.832, p-value = 0.01536
## alternative hypothesis: true difference in means is not equal to 0
## 95 percent confidence interval:
##  -131.48826  -15.89269
## sample estimates:
## mean of x mean of y 
##  394.6429  468.3333
\end{verbatim}

\hypertarget{example-comparing-recovery-times-after-surgery}{%
\subsection{Example: Comparing Recovery Times After Surgery}\label{example-comparing-recovery-times-after-surgery}}

A clinical trial compares the \textbf{average recovery time (in days)} between two patient groups after different surgical procedures.

\begin{itemize}
\item
  Group 1 (Minimally Invasive): \(n_1 = 25\), mean = 6.8 days, \(s_1\) = 1.2
\item
  Group 2 (Traditional Surgery): \(n_2 = 22\), mean = 8.1 days, \(s_2 = 2.1\)
\end{itemize}

Since the sample sizes are small and the standard deviations are noticeably different, we apply the \textbf{unpooled t-test}:

\[
t = \frac{6.8 - 8.1}{\sqrt{\frac{1.2^2}{25} + \frac{2.1^2}{22}}} = \frac{-1.3}{\sqrt{\frac{1.44}{25} + \frac{4.41}{22}}} = \frac{-1.3}{\sqrt{0.0576 + 0.2005}} = \frac{-1.3}{\sqrt{0.2581}} \approx \frac{-1.3}{0.508} \approx -2.56
\]

Now calculate \textbf{df} using the Welch-Satterthwaite formula:

\[
\begin{eqnarray*}
df = \frac{(0.0576 + 0.2005)^2}{\frac{(0.0576)^2}{24} + \frac{(0.2005)^2}{21}} \approx \frac{(0.2581)^2}{\frac{0.0033}{24} + \frac{0.0402}{21}} \approx \frac{0.0666}{0.00205} \approx 32.49
\end{eqnarray*}
\]

At 5\% two-sided significance level, the critical value is \(\pm 2.037\), which can be obtained using \(qt(0.975, df=32)\).

Since \(-2.56 < -2.037\), we \textbf{reject the null hypothesis}. The average recovery time differs significantly between the two procedures.

\begin{Shaded}
\begin{Highlighting}[]
\FunctionTok{two.mean.t}\NormalTok{(}\AttributeTok{x1bar=}\FloatTok{6.8}\NormalTok{, }\AttributeTok{n1=}\DecValTok{25}\NormalTok{, }\AttributeTok{s1=}\FloatTok{1.2}\NormalTok{, }\AttributeTok{x2bar=}\FloatTok{8.1}\NormalTok{, }\AttributeTok{n2=}\DecValTok{22}\NormalTok{, }\AttributeTok{s2=}\FloatTok{2.1}\NormalTok{, }\AttributeTok{equal.variance =} \ConstantTok{FALSE}\NormalTok{)}
\end{Highlighting}
\end{Shaded}

\begin{verbatim}
## $CI
## [1] -2.3341714 -0.2658286
## 
## $xbar
## [1] 6.8 8.1
## 
## $mu_diff
## [1] -1.3
## 
## $n
## [1] 25 22
## 
## $margin
## [1] 1.034171
## 
## $p
## [1] 0.01535029
## 
## $s
## [1] 1.2 2.1
## 
## $se
## [1] 0.5079907
## 
## $t0
## [1] -2.559102
## 
## $df
## [1] 32.45754
## 
## $CV
## [1] 2.035808
## 
## $conclusion
## [1] "H0 Rejected"
## 
## $equal.variance
## [1] FALSE
## 
## $uneqvar.method
## [1] "Welch"
\end{verbatim}

\hypertarget{example-heart-rate-comparison-after-two-medications}{%
\subsection{Example: Heart Rate Comparison After Two Medications}\label{example-heart-rate-comparison-after-two-medications}}

Researchers are comparing the mean heart rates of patients given two different hypertension medications. Equal variances cannot be assumed due to different standard deviations.

\begin{itemize}
\tightlist
\item
  \textbf{Medication A}:

  \begin{itemize}
  \tightlist
  \item
    \(\bar{x}_1 = 75.2\) bpm\\
  \item
    \(s_1 = 8.5\), \(n_1 = 25\)
  \end{itemize}
\item
  \textbf{Medication B}:

  \begin{itemize}
  \tightlist
  \item
    \(\bar{x}_2 = 69.1\) bpm\\
  \item
    \(s_2 = 12.4\), \(n_2 = 22\)
  \end{itemize}
\end{itemize}

Test at the 5\% significance level if there's a difference in mean heart rate.

\begin{Shaded}
\begin{Highlighting}[]
\CommentTok{\# Data}
\NormalTok{x1\_bar }\OtherTok{\textless{}{-}} \FloatTok{75.2}
\NormalTok{s1 }\OtherTok{\textless{}{-}} \FloatTok{8.5}
\NormalTok{n1 }\OtherTok{\textless{}{-}} \DecValTok{25}

\NormalTok{x2\_bar }\OtherTok{\textless{}{-}} \FloatTok{69.1}
\NormalTok{s2 }\OtherTok{\textless{}{-}} \FloatTok{12.4}
\NormalTok{n2 }\OtherTok{\textless{}{-}} \DecValTok{22}

\CommentTok{\# Test statistic}
\NormalTok{t\_stat }\OtherTok{\textless{}{-}}\NormalTok{ (x1\_bar }\SpecialCharTok{{-}}\NormalTok{ x2\_bar) }\SpecialCharTok{/} \FunctionTok{sqrt}\NormalTok{(s1}\SpecialCharTok{\^{}}\DecValTok{2}\SpecialCharTok{/}\NormalTok{n1 }\SpecialCharTok{+}\NormalTok{ s2}\SpecialCharTok{\^{}}\DecValTok{2}\SpecialCharTok{/}\NormalTok{n2)}

\CommentTok{\# Welch{-}Satterthwaite degrees of freedom}
\NormalTok{df }\OtherTok{\textless{}{-}}\NormalTok{ (s1}\SpecialCharTok{\^{}}\DecValTok{2}\SpecialCharTok{/}\NormalTok{n1 }\SpecialCharTok{+}\NormalTok{ s2}\SpecialCharTok{\^{}}\DecValTok{2}\SpecialCharTok{/}\NormalTok{n2)}\SpecialCharTok{\^{}}\DecValTok{2} \SpecialCharTok{/}\NormalTok{ ((s1}\SpecialCharTok{\^{}}\DecValTok{2}\SpecialCharTok{/}\NormalTok{n1)}\SpecialCharTok{\^{}}\DecValTok{2} \SpecialCharTok{/}\NormalTok{ (n1 }\SpecialCharTok{{-}} \DecValTok{1}\NormalTok{) }\SpecialCharTok{+}\NormalTok{ (s2}\SpecialCharTok{\^{}}\DecValTok{2}\SpecialCharTok{/}\NormalTok{n2)}\SpecialCharTok{\^{}}\DecValTok{2} \SpecialCharTok{/}\NormalTok{ (n2 }\SpecialCharTok{{-}} \DecValTok{1}\NormalTok{))}

\CommentTok{\# Two{-}tailed p{-}value}
\NormalTok{p\_value }\OtherTok{\textless{}{-}} \DecValTok{2} \SpecialCharTok{*} \FunctionTok{pt}\NormalTok{(}\SpecialCharTok{{-}}\FunctionTok{abs}\NormalTok{(t\_stat), df)}

\NormalTok{t\_stat}
\end{Highlighting}
\end{Shaded}

\begin{verbatim}
## [1] 1.940758
\end{verbatim}

\begin{Shaded}
\begin{Highlighting}[]
\NormalTok{df}
\end{Highlighting}
\end{Shaded}

\begin{verbatim}
## [1] 36.49733
\end{verbatim}

\begin{Shaded}
\begin{Highlighting}[]
\NormalTok{p\_value}
\end{Highlighting}
\end{Shaded}

\begin{verbatim}
## [1] 0.06004154
\end{verbatim}

\begin{Shaded}
\begin{Highlighting}[]
\FunctionTok{two.mean.t}\NormalTok{(}\AttributeTok{x1bar=}\FloatTok{75.2}\NormalTok{, }\AttributeTok{n1=}\DecValTok{25}\NormalTok{, }\AttributeTok{s1=}\FloatTok{8.5}\NormalTok{, }\AttributeTok{x2bar=}\FloatTok{69.1}\NormalTok{, }\AttributeTok{n2=}\DecValTok{22}\NormalTok{, }\AttributeTok{s2=}\FloatTok{12.4}\NormalTok{, }\AttributeTok{equal.variance =} \ConstantTok{FALSE}\NormalTok{)}
\end{Highlighting}
\end{Shaded}

\begin{verbatim}
## $CI
## [1] -0.2714918 12.4714918
## 
## $xbar
## [1] 75.2 69.1
## 
## $mu_diff
## [1] 6.1
## 
## $n
## [1] 25 22
## 
## $margin
## [1] 6.371492
## 
## $p
## [1] 0.06004154
## 
## $s
## [1]  8.5 12.4
## 
## $se
## [1] 3.143102
## 
## $t0
## [1] 1.940758
## 
## $df
## [1] 36.49733
## 
## $CV
## [1] 2.027135
## 
## $conclusion
## [1] "H0 Not Rejected"
## 
## $equal.variance
## [1] FALSE
## 
## $uneqvar.method
## [1] "Welch"
\end{verbatim}

\hypertarget{non-pooled-t-interval-procedure}{%
\subsection{Non-pooled t-interval Procedure}\label{non-pooled-t-interval-procedure}}

\begin{figure}
\centering
\includegraphics{https://i.ibb.co/7ydvG46/Procedure10-4.png}
\caption{non-pooled t-interval}
\end{figure}

\hypertarget{example-46}{%
\subsection{Example}\label{example-46}}

\begin{Shaded}
\begin{Highlighting}[]
\FunctionTok{ci.mu.diff}\NormalTok{(}\AttributeTok{x1=}\NormalTok{dynamic, }\AttributeTok{x2=}\NormalTok{static, }\AttributeTok{known.variance =} \ConstantTok{FALSE}\NormalTok{, }\AttributeTok{equal.variance =} \ConstantTok{FALSE}\NormalTok{, }\AttributeTok{conflev =} \FloatTok{0.95}\NormalTok{, }\AttributeTok{uneqvar.method=}\StringTok{"Welch"}\NormalTok{)}
\end{Highlighting}
\end{Shaded}

\begin{verbatim}
## $df
## [1] 17.83231
## 
## $q
## [1] 2.102339
## 
## $uneqvar.method
## [1] "Welch"
## 
## $CI
## [1] -131.48826  -15.89269
## 
## $mean.diff
## [1] -73.69048
## 
## $se.mean.diff
## [1] 27.49213
## 
## $known.variance
## [1] FALSE
## 
## $equal.variance
## [1] FALSE
\end{verbatim}

\textbf{Obtain the 90\% CI for \(\mu_1 - \mu_2\)}

\textbf{Step 1:} Determine \(\alpha\) and \(t_{1-\alpha/2}\):

\begin{itemize}
\tightlist
\item
  \(\alpha = 0.10\)\\
\item
  \(t_{1-\alpha/2} = t_{0.975} = 1.740\) with \(df = 17\).
\end{itemize}

\textbf{Step 2:} Compute the Endpoints of the CI

\begin{itemize}
\tightlist
\item
  Formula for CI:\\
  \[
  (\bar{x}_1 - \bar{x}_2) \pm t_{1-\alpha/2} \sqrt{\frac{s_1^2}{n_1} + \frac{s_2^2}{n_2}}
  \]
\end{itemize}

\textbf{Step 3:} Interpretation

\begin{itemize}
\tightlist
\item
  We can be \textbf{90\% confident} that the difference between the mean salaries of faculty in private institutions and public institutions is somewhere between \textbf{-121.5 and -25.9}.
\end{itemize}

\hypertarget{relation-between-hypothesis-tests-and-ci}{%
\subsection{Relation Between Hypothesis Tests and CI}\label{relation-between-hypothesis-tests-and-ci}}

\begin{itemize}
\item
  \(H_0\) for a two-tailed test is rejected if and only if the \((1-\alpha)\) level CI for \(\mu_1 - \mu_2\) does not contain 0. \textbf{Think about why?}
\item
  If you are \textbf{reasonably sure that the populations have nearly equal standard deviations}, use a \textbf{pooled t-procedure}.
\item
  Otherwise, use a \textbf{non-pooled t-procedure}.
\end{itemize}

\hypertarget{inferences-for-two-population-means-for-paired-sample}{%
\section{Inferences for Two Population Means for Paired Sample}\label{inferences-for-two-population-means-for-paired-sample}}

\hypertarget{random-paired-sample}{%
\subsection{Random Paired Sample}\label{random-paired-sample}}

Understanding the nature of the data is paramount when deciding on the most suitable statistical methodology. Paired samples, in particular, demand special consideration in statistical analysis.

\textbf{Definition}: Paired samples inherently consist of random pairs, a characteristic determined largely by the scientific inquiry at hand.

Consider, for instance, an investigation into the change in quality of life before and after graduating from college. In this scenario, an experiment is designed to select a random sample of 100 college seniors and measure their quality of life both during their time at college and one year after graduation. Within this experimental framework, two quality of life variables are at play: ``before'' and ``after.'' Importantly, these two measurements form a pair for each student, and the primary objective of the scientific inquiry is to compare the changes experienced by each individual from ``before'' to ``after.''

To illustrate further, let's examine another scenario. Imagine a weight loss program with 100 participants. Their body weights are assessed just before initiating the program and again after three months of participation. In this context, the intriguing question centers on the ability of each individual to lose weight. There is no interest in comparing one individual's weight after three months with the weight of another individual just before participating the program.

In paired comparisons, the scientific inquiry often revolves around the differences within pairs, denoted as \(d = After - Before\). Formulating a hypothesis test to determine whether the mean difference is equal to zero transforms the paired comparison problem into a one-sample mean problem.

Denote the \(n\) paired samples as \((X_i, Y_i)\) for \(i = 1, \cdots, n\), and \(d_i\) as the ith pair difference \(X_i - Y_i\). Then the hypothesis test of \(\mu_X = \mu_Y\) is equivalent to \(\mu_d = 0\). Since
\[
t = \frac{\bar{d}-\mu_d}{s_d/\sqrt{n}} \sim t(df=n-1)
\]
where \(s_d\) is the sample standard deviation of \(d\), the paired t-test can be constructed following the one population t test, and the \((1-\alpha)100\%CI\) can be derived as
\[
    \bar{d} - t_{1-\alpha/2, df=n-1}\cdot \frac{s_d}{\sqrt{n}}.
\]
where \(t_{1-\alpha/2, df=n-1}\) is the \((1-\alpha/2)\)th percentile of t distribution with \(n-1\) degrees of freedom.

\hypertarget{example-47}{%
\subsection{Example}\label{example-47}}

Two populations (Husbands population and wives population) are shown below. A paired sample means whenever a husband is randomly selected, the paired wife is also selected. The statistical inference of interest is to compare the paired samples.

\begin{figure}
\centering
\includegraphics{https://i.ibb.co/RgDGfWG/485-EX-10-14.png}
\caption{Husband and wife}
\end{figure}

Suppose that we want to decide whether, in the U.S., the mean age of married men differs from the mean age of married women. A random sample of 10 couples are shown below.

\begin{figure}
\centering
\includegraphics{https://i.ibb.co/2Y5VVj7/486-Table-10-13.png}
\caption{Husband and wife}
\end{figure}

At a \textbf{5\% significance level}, do the data provide sufficient evidence to conclude that the mean age of married men \textbf{differs} from the mean age of married women?

\begin{itemize}
\tightlist
\item
  \textbf{Think carefully}: Are the married husbands and married wives sampled \textbf{independently}?
\end{itemize}

\hypertarget{how-do-we-compare-the-means-of-married-men-vs-married-women}{%
\subsection{How Do We Compare the Means of Married Men vs Married Women?}\label{how-do-we-compare-the-means-of-married-men-vs-married-women}}

\begin{itemize}
\tightlist
\item
  \textbf{Assumption}: Each population is normally distributed.
\end{itemize}

\textbf{Options for Comparison:}

\begin{enumerate}
\def\labelenumi{\arabic{enumi}.}
\tightlist
\item
  \textbf{Pooled or Non-Pooled t-Test}:
\end{enumerate}

\begin{itemize}
\item
  Used when we have \textbf{random independent samples}, without considering the ``paired'' relationship.
\item
  The total sampling error includes both populations' sampling errors.
\end{itemize}

\begin{enumerate}
\def\labelenumi{\arabic{enumi}.}
\setcounter{enumi}{1}
\tightlist
\item
  \textbf{For Paired Samples}:
\end{enumerate}

\begin{itemize}
\item
  The actual sampling error comes only from the \textbf{difference in each pair}, making it usually much smaller.
\item
  \textbf{More efficient statistical tests} are needed for paired samples.
\item
  A separate class of statistical methods is required for paired samples to account for their structure and achieve higher efficiency.
\end{itemize}

\hypertarget{additional-examples}{%
\subsection{Additional Examples}\label{additional-examples}}

\begin{itemize}
\item
  Blood pressure before and after medication for the same patients
\item
  Tumor size before and after treatment for the same patients
\item
  Anxiety scores before and after therapy for the same patients
\end{itemize}

\textbf{Assumptions:}
- Data are paired and differences are independent.
- Differences are approximately normally distributed.
- The test is applied to the \textbf{differences (d)}, not the raw values

\hypertarget{distribution-of-the-paired-t-statistic}{%
\subsection{Distribution of the Paired t-Statistic}\label{distribution-of-the-paired-t-statistic}}

For paired samples, suppose each pair's difference \(d\) is normally distributed, then the variable
\[
t=\frac{\bar{d}-(\mu_1-\mu_2)}{s_d/\sqrt{n}}
\]

has the t-distribution with \(df = n-1\). Under \(H_0: \mu_1=\mu_2\),
\[
t=\frac{\bar{d}}{s_d/\sqrt{n}}
\]
can be used as the test statistic for hypothesis testing. This is called \textcolor{blue}{paired t-test}.

\hypertarget{paired-t-test-procedure}{%
\subsection{Paired t-test Procedure}\label{paired-t-test-procedure}}

\begin{figure}
\centering
\includegraphics{https://i.ibb.co/M9sQzxz/Procedure10-6a.png}
\caption{Paired t test}
\end{figure}

\begin{figure}
\centering
\includegraphics{https://i.ibb.co/981tfFs/Procedure10-6b.png}
\caption{Paired t test}
\end{figure}

\hypertarget{paired-t-test-interval}{%
\subsection{Paired t-test interval}\label{paired-t-test-interval}}

\begin{figure}
\centering
\includegraphics{https://i.ibb.co/605SNKy/Procedure10-7.png}
\caption{Paired t interval}
\end{figure}

\hypertarget{example.-perform-the-paired-t-test}{%
\subsection{Example. Perform the paired t-test}\label{example.-perform-the-paired-t-test}}

\textbf{How Do We Compare the Means of Married Men vs Married Women?}

\begin{itemize}
\tightlist
\item
  \textbf{Assumption}: Each population is normally distributed.
\end{itemize}

\textbf{Options for Comparison:}

\begin{enumerate}
\def\labelenumi{\arabic{enumi}.}
\item
  \textbf{Pooled or Non-Pooled t-Test}:

  \begin{itemize}
  \tightlist
  \item
    Used when we have \textbf{random independent samples}, without considering the ``paired'' relationship.\\
  \item
    The total sampling error includes both populations' sampling errors.
  \end{itemize}
\item
  \textbf{For Paired Samples}:

  \begin{itemize}
  \item
    The actual sampling error comes only from the \textbf{difference in each pair}, making it usually much smaller.\\
  \item
    \textbf{More efficient statistical tests} are needed for paired samples.
  \item
    A separate class of statistical methods is required for paired samples to account for their structure and achieve higher efficiency.
  \end{itemize}
\end{enumerate}

\textbf{Steps for Testing the Difference in Mean Ages}

Step 1: Hypotheses

\begin{itemize}
\tightlist
\item
  \(H_0: \mu_1 = \mu_2\) (mean ages are equal).\\
\item
  \(H_a: \mu_1 \neq \mu_2\) (mean ages are not equal).
\end{itemize}

Step 2: Significance Level

\begin{itemize}
\tightlist
\item
  \(\alpha = 0.05\)
\end{itemize}

Step 3: Compute the Test Statistic
- Formula:\\
\[t = \frac{\bar{d}}{s_d / \sqrt{n}}\]

\begin{itemize}
\item
  Calculations:

  \begin{itemize}
  \item
    \[\bar{d} = \frac{\sum d_i}{n} = 3.6\]
  \item
    \[s_d = \sqrt{\frac{\sum d_i^2 - (\sum d_i)^2 / n}{n - 1}} = 4.97\]
  \end{itemize}
\item
  Substituting values:

  \[t = \frac{3.6}{4.97 / \sqrt{n}} = 2.291\]
\end{itemize}

Step 4: Determine Critical Values

\begin{itemize}
\item
  For a two-tailed test, the critical values are \(\pm t_{1-\alpha/2}\) with \(df = n - 1\).
\item
  Using R: \(t_{1-\alpha/2} = 2.262\), qt(1-alpha, df=n-1)
\end{itemize}

\begin{figure}
\centering
\includegraphics{https://i.ibb.co/Gsnj8qF/490-Figure-10-16a.png}
\caption{Decision}
\end{figure}

Step 5: Decision

\begin{itemize}
\item
  The observed value of the test statistic is \(t = 2.291\), which falls in the rejection region.
\item
  Therefore, we \textbf{reject \(H_0\)}.
\end{itemize}

Step 4: Compute the P-Value

\begin{itemize}
\tightlist
\item
  The \(P\)-value equals \(P(|t| > 2.291)\).\\
\item
  Using Table IV, the \(P\)-value is estimated to be between \textbf{0.02 and 0.05}.\\
\item
  Alternatively, use software to calculate: \(P = 0.0478\).
\end{itemize}

Step 5: Decision

\begin{itemize}
\tightlist
\item
  Since \(P < 0.05\), we \textbf{reject \(H_0\)}.
\end{itemize}

\begin{figure}
\centering
\includegraphics{https://i.ibb.co/3rSc058/490-Figure-10-16b.png}
\caption{Decision}
\end{figure}

\hypertarget{obtain-a-95-confidence-interval-ci-for-the-difference-mu_1---mu_2}{%
\subsection{\texorpdfstring{Obtain a 95\% Confidence Interval (CI) for the Difference \(\mu_1 - \mu_2\)}{Obtain a 95\% Confidence Interval (CI) for the Difference \textbackslash mu\_1 - \textbackslash mu\_2}}\label{obtain-a-95-confidence-interval-ci-for-the-difference-mu_1---mu_2}}

Step 1: Determine \(t_{\alpha/2}\)

\begin{itemize}
\tightlist
\item
  \(t_{\alpha/2} = 2.262\)
\end{itemize}

Step 2: Compute the Endpoints of the CI
- Formula for CI:\\
\[\bar{d} \pm t_{\alpha/2} \cdot \frac{s_d}{\sqrt{n}}\]

\begin{itemize}
\tightlist
\item
  Substituting values:\\
  \[3.6 \pm 2.262 \cdot \frac{4.97}{\sqrt{10}}\]
\end{itemize}

Results:

\begin{itemize}
\tightlist
\item
  The 95\% CI is \textbf{(0.04, 7.16)}.
\end{itemize}

\begin{Shaded}
\begin{Highlighting}[]
\NormalTok{husband }\OtherTok{=} \FunctionTok{c}\NormalTok{(}\DecValTok{59}\NormalTok{, }\DecValTok{21}\NormalTok{, }\DecValTok{33}\NormalTok{, }\DecValTok{78}\NormalTok{, }\DecValTok{70}\NormalTok{, }\DecValTok{33}\NormalTok{, }\DecValTok{68}\NormalTok{, }\DecValTok{32}\NormalTok{, }\DecValTok{54}\NormalTok{, }\DecValTok{52}\NormalTok{)}
\NormalTok{wife }\OtherTok{=}    \FunctionTok{c}\NormalTok{(}\DecValTok{53}\NormalTok{, }\DecValTok{22}\NormalTok{, }\DecValTok{36}\NormalTok{, }\DecValTok{74}\NormalTok{, }\DecValTok{64}\NormalTok{, }\DecValTok{35}\NormalTok{, }\DecValTok{67}\NormalTok{, }\DecValTok{28}\NormalTok{, }\DecValTok{41}\NormalTok{, }\DecValTok{44}\NormalTok{)}
\FunctionTok{t.test}\NormalTok{(husband, wife, }\AttributeTok{alternative =} \StringTok{"two.sided"}\NormalTok{, }\AttributeTok{paired =} \ConstantTok{TRUE}\NormalTok{, }
        \AttributeTok{conf.level =} \FloatTok{0.95}\NormalTok{)}
\end{Highlighting}
\end{Shaded}

\begin{verbatim}
## 
##  Paired t-test
## 
## data:  husband and wife
## t = 2.2901, df = 9, p-value = 0.04777
## alternative hypothesis: true mean difference is not equal to 0
## 95 percent confidence interval:
##  0.04394139 7.15605861
## sample estimates:
## mean difference 
##             3.6
\end{verbatim}

This is equivalent to one-sample t-test when using the difference between husband and wife, comparing the difference to 0.

\begin{Shaded}
\begin{Highlighting}[]
\FunctionTok{one.mean.t}\NormalTok{(}\AttributeTok{x =}\NormalTok{ husband }\SpecialCharTok{{-}}\NormalTok{ wife, }\AttributeTok{mu0 =} \DecValTok{0}\NormalTok{,  }\AttributeTok{tail =} \StringTok{"two"}\NormalTok{, }\AttributeTok{alpha =} \FloatTok{0.05}\NormalTok{)}
\end{Highlighting}
\end{Shaded}

\begin{verbatim}
## $CI
## [1] 0.04394139 7.15605861
## 
## $xbar
## [1] 3.6
## 
## $n
## [1] 10
## 
## $margin
## [1] 3.556059
## 
## $s
## [1] 4.971027
## 
## $p
## [1] 0.04776605
## 
## $t0
## [1] 2.29011
## 
## $CV
## [1] -2.262157  2.262157
## 
## $df
## [1] 9
## 
## $conclusion
## [1] "H0 Rejected"
\end{verbatim}

\hypertarget{example-measuring-effect-of-a-new-pain-medication}{%
\subsection{Example: Measuring Effect of a New Pain Medication}\label{example-measuring-effect-of-a-new-pain-medication}}

A study measures the \textbf{pain level} (on a scale of 0--10) in 10 patients \textbf{before and after} taking a new medication. Is there any difference between before and after?

\begin{longtable}[]{@{}lll@{}}
\toprule\noalign{}
Patient & Before & After \\
\midrule\noalign{}
\endhead
\bottomrule\noalign{}
\endlastfoot
1 & 8 & 5 \\
2 & 7 & 4 \\
3 & 6 & 3 \\
4 & 9 & 6 \\
5 & 7 & 5 \\
6 & 8 & 6 \\
7 & 6 & 4 \\
8 & 7 & 5 \\
9 & 9 & 7 \\
10 & 8 & 6 \\
\end{longtable}

First, calculate the differences:

\[
d = \text{Before} - \text{After} = \{3, 3, 3, 3, 2, 2, 2, 2, 2, 2\}
\]

Then find:

\begin{itemize}
\tightlist
\item
  \(\bar{d} = \frac{24}{10} = 2.4\)
\item
  \(s_d = \sqrt{\frac{1}{n-1} \sum (d_i - \bar{d})^2} = \sqrt{\frac{1.6}{9}} \approx 0.42\)
\end{itemize}

Compute the t-statistic:

\[
t = \frac{2.4}{0.42 / \sqrt{10}} = \frac{2.4}{0.133} \approx 18.05
\]

With \textbf{df = 9}, and \(\alpha = 0.05\), the critical value is \(\pm 2.262\) (two-tailed). Since \textbf{18.05 \textgreater{} 2.262}, we reject \(H_0\).

\textbf{Conclusion:}

The pain medication caused a statistically significant reduction in pain levels.
This paired t-test is ideal here because each patient serves as their own control.

\begin{Shaded}
\begin{Highlighting}[]
\NormalTok{d}\OtherTok{=} \FunctionTok{c}\NormalTok{(}\DecValTok{3}\NormalTok{, }\DecValTok{3}\NormalTok{, }\DecValTok{3}\NormalTok{, }\DecValTok{3}\NormalTok{, }\DecValTok{2}\NormalTok{, }\DecValTok{2}\NormalTok{, }\DecValTok{2}\NormalTok{, }\DecValTok{2}\NormalTok{, }\DecValTok{2}\NormalTok{, }\DecValTok{2}\NormalTok{)}
\FunctionTok{one.mean.t}\NormalTok{(}\AttributeTok{x =}\NormalTok{ d, }\AttributeTok{mu0 =} \DecValTok{0}\NormalTok{,  }\AttributeTok{tail =} \StringTok{"two"}\NormalTok{, }\AttributeTok{alpha =} \FloatTok{0.05}\NormalTok{)}
\end{Highlighting}
\end{Shaded}

\begin{verbatim}
## $CI
## [1] 2.030591 2.769409
## 
## $xbar
## [1] 2.4
## 
## $n
## [1] 10
## 
## $margin
## [1] 0.3694087
## 
## $s
## [1] 0.5163978
## 
## $p
## [1] 1.346733e-07
## 
## $t0
## [1] 14.69694
## 
## $CV
## [1] -2.262157  2.262157
## 
## $df
## [1] 9
## 
## $conclusion
## [1] "H0 Rejected"
\end{verbatim}

\hypertarget{example-gallbladder}{%
\subsection{Example: Gallbladder}\label{example-gallbladder}}

John M. Morton et al.~(2022) examined gallbladder function before and after fundoplication--a surgery used to stop stomach contents from flowing back into the esophagus (reflux)- in patients with gastroesophageal reflux disease. The authors measured gallbladder functionality by calculating the gallbladder ejection fraction (GBEF) before and after fundoplication. The goal of fundopli-cation is to increase GBEF, which is measured as a percent. The data are shown below.

Gallbladder Function in Patients with Presentations of Gastrosophageal Reflux Disease Before : After Treatment(\%): 22:63.5, 63.3:91.5, 96:59, 9.2:37.8, 3.1:10.1, 50:19.6, 33:41, 69:87.8, 64:86, 18.8:55, 0:88, 34:40.

\textbf{Question}: Are these data able to provide sufficient evidence to allow us to conclude that fundoplication increases GBEF functioning at 5\% significance level?

The hypothesis procedure is described below.

\begin{itemize}
\item
  State the null and alternative hypotheses explicitly. \(H_0: \mu_d \le 0\) and \(H_a: \mu_d > 0\).
\item
  Determine the test statistic and its distribution. The test statistic is \(t\) with \(n-1=11\) degrees of freedom, where \(n\) is the number of pairs.
\item
  Determine the rejection region and critical values. The critical value is \(t_{1-alpha/2}\). When \(\alpha=0.05\), the rejection region is \(t \ge 1.795885\).
\item
  Calculate the test statistic under \(H_0\). We usually use \(t_0\) to denote the observed value of the random variable \(t\).
\end{itemize}

\[t_0 = \frac{\bar{d}}{s_d/\sqrt{n}} = \frac{18.075}{32.68/\sqrt{12}}= 1.9159\].

\begin{itemize}
\item
  Calculate the p value. The \(P-\)value is \(1-pt(1.9159, df=11)=0.0409 < 0.05\).
\item
  Make a statistical decision based on the p value or the calculated test statistic. Since \(t_0 = 1.9159\) falls in the rejection region, \(H_0\) is rejected. Equivalently, \(p < 0.05\), so \(H_0\) is rejected.
\item
  Conclusion: At the 5\% significance level, the data provide sufficient evidence that the fundoplication increases GBEF functioning.
\end{itemize}

The following R code implements the testing procedure. The solution to the paired comparison is essentially the one-population t test. The following R code solves this problem based on the existing one population t test.

\begin{Shaded}
\begin{Highlighting}[]
\NormalTok{preop  }\OtherTok{\textless{}{-}} \FunctionTok{c}\NormalTok{(}\DecValTok{22}\NormalTok{,   }\FloatTok{63.3}\NormalTok{, }\DecValTok{96}\NormalTok{,  }\FloatTok{9.2}\NormalTok{,  }\FloatTok{3.1}\NormalTok{, }\DecValTok{50}\NormalTok{,   }\DecValTok{33}\NormalTok{, }\DecValTok{69}\NormalTok{,   }\DecValTok{64}\NormalTok{, }\FloatTok{18.8}\NormalTok{, }\DecValTok{0}\NormalTok{, }\DecValTok{34}\NormalTok{)}
\NormalTok{postop }\OtherTok{\textless{}{-}} \FunctionTok{c}\NormalTok{(}\FloatTok{63.5}\NormalTok{, }\FloatTok{91.5}\NormalTok{, }\DecValTok{59}\NormalTok{, }\FloatTok{37.8}\NormalTok{, }\FloatTok{10.1}\NormalTok{, }\FloatTok{19.6}\NormalTok{, }\DecValTok{41}\NormalTok{, }\FloatTok{87.8}\NormalTok{, }\DecValTok{86}\NormalTok{, }\DecValTok{55}\NormalTok{,   }\DecValTok{88}\NormalTok{,}\DecValTok{40}\NormalTok{)        }
\NormalTok{n }\OtherTok{\textless{}{-}} \FunctionTok{length}\NormalTok{(preop)      }
\NormalTok{d }\OtherTok{\textless{}{-}}\NormalTok{ postop }\SpecialCharTok{{-}}\NormalTok{ preop     }
\FunctionTok{one.mean.t}\NormalTok{(}\AttributeTok{x=}\NormalTok{d, }\AttributeTok{mu0=}\DecValTok{0}\NormalTok{, }\AttributeTok{tail=}\StringTok{"right"}\NormalTok{, }\AttributeTok{alpha=}\FloatTok{0.05}\NormalTok{)}
\end{Highlighting}
\end{Shaded}

\begin{verbatim}
## $CI
## [1] -2.689956 38.839956
## 
## $xbar
## [1] 18.075
## 
## $n
## [1] 12
## 
## $margin
## [1] 20.76496
## 
## $s
## [1] 32.68169
## 
## $p
## [1] 0.04086217
## 
## $t0
## [1] 1.915863
## 
## $CV
## [1] 1.795885
## 
## $df
## [1] 11
## 
## $conclusion
## [1] "H0 Rejected"
\end{verbatim}

In R, there is also a built-in function to perform paired t-test.

\begin{Shaded}
\begin{Highlighting}[]
\FunctionTok{t.test}\NormalTok{(}\AttributeTok{x=}\NormalTok{postop, }\AttributeTok{y=}\NormalTok{preop, }\AttributeTok{alternative=}\StringTok{"greater"}\NormalTok{, }\AttributeTok{paired =} \ConstantTok{TRUE}\NormalTok{)}
\end{Highlighting}
\end{Shaded}

\begin{verbatim}
## 
##  Paired t-test
## 
## data:  postop and preop
## t = 1.9159, df = 11, p-value = 0.04086
## alternative hypothesis: true mean difference is greater than 0
## 95 percent confidence interval:
##  1.131919      Inf
## sample estimates:
## mean difference 
##          18.075
\end{verbatim}

\hypertarget{example-blood-pressure-before-and-after-treatment}{%
\subsection{Example: Blood Pressure Before and After Treatment}\label{example-blood-pressure-before-and-after-treatment}}

A study measures \textbf{systolic blood pressure} of 12 patients \textbf{before and after} a 6-week exercise program.

\begin{longtable}[]{@{}lll@{}}
\toprule\noalign{}
Patient & Before & After \\
\midrule\noalign{}
\endhead
\bottomrule\noalign{}
\endlastfoot
1 & 150 & 140 \\
2 & 160 & 147 \\
3 & 145 & 138 \\
4 & 155 & 150 \\
5 & 148 & 139 \\
6 & 152 & 144 \\
7 & 149 & 137 \\
8 & 151 & 143 \\
9 & 157 & 150 \\
10 & 159 & 145 \\
11 & 153 & 140 \\
12 & 150 & 139 \\
\end{longtable}

We want to test at \(\alpha = 0.05\) whether the \textbf{exercise program reduced average systolic BP}.

\begin{Shaded}
\begin{Highlighting}[]
\CommentTok{\# Before and after BP}

\NormalTok{before }\OtherTok{\textless{}{-}} \FunctionTok{c}\NormalTok{(}\DecValTok{150}\NormalTok{, }\DecValTok{160}\NormalTok{, }\DecValTok{145}\NormalTok{, }\DecValTok{155}\NormalTok{, }\DecValTok{148}\NormalTok{, }\DecValTok{152}\NormalTok{, }\DecValTok{149}\NormalTok{, }\DecValTok{151}\NormalTok{, }\DecValTok{157}\NormalTok{, }\DecValTok{159}\NormalTok{, }\DecValTok{153}\NormalTok{, }\DecValTok{150}\NormalTok{)}
\NormalTok{after }\OtherTok{\textless{}{-}} \FunctionTok{c}\NormalTok{(}\DecValTok{140}\NormalTok{, }\DecValTok{147}\NormalTok{, }\DecValTok{138}\NormalTok{, }\DecValTok{150}\NormalTok{, }\DecValTok{139}\NormalTok{, }\DecValTok{144}\NormalTok{, }\DecValTok{137}\NormalTok{, }\DecValTok{143}\NormalTok{, }\DecValTok{150}\NormalTok{, }\DecValTok{145}\NormalTok{, }\DecValTok{140}\NormalTok{, }\DecValTok{139}\NormalTok{)}

\CommentTok{\# Differences}

\NormalTok{d }\OtherTok{\textless{}{-}}\NormalTok{ after }\SpecialCharTok{{-}}\NormalTok{ before}

\CommentTok{\# Mean and SD of differences}

\NormalTok{mean\_d }\OtherTok{\textless{}{-}} \FunctionTok{mean}\NormalTok{(d)}
\NormalTok{sd\_d }\OtherTok{\textless{}{-}} \FunctionTok{sd}\NormalTok{(d)}
\NormalTok{n }\OtherTok{\textless{}{-}} \FunctionTok{length}\NormalTok{(d)}

\CommentTok{\# Test statistic}
\NormalTok{t\_stat }\OtherTok{\textless{}{-}}\NormalTok{ mean\_d }\SpecialCharTok{/}\NormalTok{ (sd\_d }\SpecialCharTok{/} \FunctionTok{sqrt}\NormalTok{(n))}

\CommentTok{\# Degrees of freedom}
\NormalTok{df }\OtherTok{\textless{}{-}}\NormalTok{ n }\SpecialCharTok{{-}} \DecValTok{1}

\CommentTok{\# One{-}tailed p{-}value (testing if BEFORE \textgreater{} AFTER)}
\NormalTok{p\_value }\OtherTok{\textless{}{-}} \FunctionTok{pt}\NormalTok{(}\SpecialCharTok{{-}}\FunctionTok{abs}\NormalTok{(t\_stat), }\AttributeTok{df=}\NormalTok{n}\DecValTok{{-}1}\NormalTok{)}

\NormalTok{t\_stat}
\end{Highlighting}
\end{Shaded}

\begin{verbatim}
## [1] -11.79147
\end{verbatim}

\begin{Shaded}
\begin{Highlighting}[]
\NormalTok{df}
\end{Highlighting}
\end{Shaded}

\begin{verbatim}
## [1] 11
\end{verbatim}

\begin{Shaded}
\begin{Highlighting}[]
\NormalTok{p\_value}
\end{Highlighting}
\end{Shaded}

\begin{verbatim}
## [1] 6.963395e-08
\end{verbatim}

Equivalently,

\begin{Shaded}
\begin{Highlighting}[]
\FunctionTok{one.mean.t}\NormalTok{(}\AttributeTok{x =}\NormalTok{ d, }\AttributeTok{mu0 =} \DecValTok{0}\NormalTok{,  }\AttributeTok{tail =} \StringTok{"left"}\NormalTok{, }\AttributeTok{alpha =} \FloatTok{0.05}\NormalTok{)}
\end{Highlighting}
\end{Shaded}

\begin{verbatim}
## $CI
## [1] -11.569926  -7.930074
## 
## $xbar
## [1] -9.75
## 
## $n
## [1] 12
## 
## $margin
## [1] 1.819926
## 
## $s
## [1] 2.864358
## 
## $p
## [1] 6.963395e-08
## 
## $t0
## [1] -11.79147
## 
## $CV
## [1] -1.795885
## 
## $df
## [1] 11
## 
## $conclusion
## [1] "H0 Rejected"
\end{verbatim}

Equivalently,

\begin{Shaded}
\begin{Highlighting}[]
\FunctionTok{t.test}\NormalTok{(after, before, }\AttributeTok{alternative =} \StringTok{"less"}\NormalTok{, }\AttributeTok{paired =} \ConstantTok{TRUE}\NormalTok{, }
        \AttributeTok{conf.level =} \FloatTok{0.95}\NormalTok{)}
\end{Highlighting}
\end{Shaded}

\begin{verbatim}
## 
##  Paired t-test
## 
## data:  after and before
## t = -11.791, df = 11, p-value = 6.963e-08
## alternative hypothesis: true mean difference is less than 0
## 95 percent confidence interval:
##       -Inf -8.265039
## sample estimates:
## mean difference 
##           -9.75
\end{verbatim}

\textbf{Conclusion}

Since \textbf{p-value \textless{} 0.05}, we reject the null and conclude that the exercise program significantly reduced systolic blood pressure.

\hypertarget{inference-for-comparing-one-population-proportion-to-a-threshold}{%
\chapter{Inference for Comparing One Population Proportion to a Threshold}\label{inference-for-comparing-one-population-proportion-to-a-threshold}}

\begin{Shaded}
\begin{Highlighting}[]
\FunctionTok{library}\NormalTok{(IntroStats)}
\end{Highlighting}
\end{Shaded}

\hypertarget{hypothesis-testing-one-population-proportion-p}{%
\section{\texorpdfstring{Hypothesis Testing: One Population Proportion \(p\)}{Hypothesis Testing: One Population Proportion p}}\label{hypothesis-testing-one-population-proportion-p}}

\hypertarget{example-48}{%
\subsection{Example}\label{example-48}}

Wagenknecht et al.~collected data on a sample of 301 Hispanic women living in San Anto-no, Texas. One variable of interest was the percentage of subjects with impaired fasting glucose (IFG). IFG refers to a metabolic stage intermediate between normal glucose homeostasis and diabetes. In the study, 24 women were classified in the IF stage. The article cites population estimates for IF among Hispanic women in Texas as 6.3 percent.

\textbf{Question}. Is there sufficient evidence to indicate that the population of Hispanic women in San Antonio has a prevalence of IF higher than 6.3 percent?

\hypertarget{distribution-of-one-sample-proportion}{%
\subsection{Distribution of One-Sample Proportion}\label{distribution-of-one-sample-proportion}}

The sample proportion \(\hat{p}\) approximately follows a \textbf{normal distribution} with:\\
- Mean: \(p\)\\
- Variance: \(\frac{p(1-p)}{n}\)

\textbf{Standardized Test Statistic}

The test statistic for hypothesis test of one population proportion is
\[z = \frac{\hat{p}-p}{\sqrt{p(1-p)/n}}\]
approximately follows \(N(0, 1)\) distribution.

\hypertarget{one-proportion-z-test}{%
\subsection{One-proportion z-test}\label{one-proportion-z-test}}

The hypothesis testing of \(H_0: p \le p_0\) can be carried out as follows.

\begin{itemize}
\item
  State the null and alternative hypotheses explicitly. \[H_0: p \le p_0\] and \[H_a: p > p_0\].
\item
  Determine the test statistic and its distribution. The test statistic is \[z=\frac{\hat{p}-p}{\sqrt{p(1-p)}/n}\]
\item
  Determine the rejection region and critical values. The critical value is \(z_{1-alpha}\). When \(\alpha=0.05\), the rejection region is \(Z \ge 1.645\).
\item
  Calculate the test statistic under \(H_0\). We usually use \(z_0\) to denote the observed value of the test statistic \(z\). Note: \(\hat{p}=0.08\).
\end{itemize}

\[z_0 = \frac{\hat{p}-p_0}{p_0(1-p_0)/\sqrt{n}} = \frac{0.080-0.063}{\sqrt{0.063(0.937)/301}}= 1.19\]
- Calculate the p value. The \(P-\)value is \(1-pnorm(1.19)=0.116 > 0.05\).

\begin{itemize}
\item
  Make a statistical decision based on the p value or the calculated test statistic. Since \(z_0 = 1.19\) does not fall in the rejection region, \(H_0\) is not rejected. Equivalently, \(p > 0.05\), so \(H_0\) is not rejected.
\item
  Conclusion: At the 5\% significance level, the data do not provide sufficient evidence to indicate that the population of Hispanic women in San Antonio has a prevalence of IF higher than 6.3 percent.
\end{itemize}

The above procedure is implemented below.

\begin{Shaded}
\begin{Highlighting}[]
\FunctionTok{one.p.z}\NormalTok{(}\AttributeTok{phat=}\DecValTok{24} \SpecialCharTok{/} \DecValTok{301}\NormalTok{, }\AttributeTok{n =} \DecValTok{301}\NormalTok{, }\AttributeTok{p0=}\FloatTok{0.063}\NormalTok{, }\AttributeTok{tail=}\StringTok{"right"}\NormalTok{, }\AttributeTok{alpha=}\FloatTok{0.05}\NormalTok{, }\AttributeTok{conflev=}\FloatTok{0.95}\NormalTok{)}
\end{Highlighting}
\end{Shaded}

\begin{verbatim}
## $phat
## [1] 0.07973422
## 
## $z0
## [1] 1.194947
## 
## $CV
## [1] 1.644854
## 
## $conclusion
## [1] "H0 Not Rejected"
## 
## $CI
## [1] 0.04913264 0.11033580
## 
## $se
## [1] 0.01561334
## 
## $n
## [1] 301
## 
## $p
## [1] 0.1160539
\end{verbatim}

It also works by directly taking sample data.

\begin{Shaded}
\begin{Highlighting}[]
\NormalTok{x }\OtherTok{\textless{}{-}} \FunctionTok{c}\NormalTok{(}\FunctionTok{rep}\NormalTok{(}\DecValTok{1}\NormalTok{, }\DecValTok{24}\NormalTok{), }\FunctionTok{rep}\NormalTok{(}\DecValTok{0}\NormalTok{, }\DecValTok{301{-}24}\NormalTok{))}
\FunctionTok{one.p.z}\NormalTok{(}\AttributeTok{x=}\NormalTok{x, }\AttributeTok{p0=}\FloatTok{0.063}\NormalTok{, }\AttributeTok{tail=}\StringTok{"right"}\NormalTok{, }\AttributeTok{alpha=}\FloatTok{0.05}\NormalTok{, }\AttributeTok{conflev=}\FloatTok{0.95}\NormalTok{)}
\end{Highlighting}
\end{Shaded}

\begin{verbatim}
## $phat
## [1] 0.07973422
## 
## $z0
## [1] 1.194947
## 
## $CV
## [1] 1.644854
## 
## $conclusion
## [1] "H0 Not Rejected"
## 
## $CI
## [1] 0.04913264 0.11033580
## 
## $se
## [1] 0.01561334
## 
## $n
## [1] 301
## 
## $p
## [1] 0.1160539
\end{verbatim}

\begin{figure}
\centering
\includegraphics{https://i.ibb.co/HGLbjt5/Procedure12-2a.png}
\caption{One-Proportion z-Test}
\end{figure}

\begin{figure}
\centering
\includegraphics{https://i.ibb.co/Sx7hfKP/Procedure12-2b.png}
\caption{One-Proportion z-Test}
\end{figure}

\hypertarget{example-49}{%
\subsection{Example}\label{example-49}}

\textbf{Problem}
Suppose \(n = 1015\), \(p_0 = 0.5\), and among 1015 samples, 538 are successes.\\
Perform a hypothesis test to determine whether the population proportion \(p\) is greater than 0.5 (majority) at a 5\% significance level.

\textbf{Solution}

\textbf{Step 1: State the Hypotheses}
- \(H_0: p = 0.5\)\\
- \(H_a: p > 0.5\) (right-tailed test).

\textbf{Step 2: Significance Level}
- \(\alpha = 0.05\)

\textbf{Step 3: Compute the Test Statistic}
- Formula:\\
\[z = \frac{\hat{p} - p_0}{\sqrt{\frac{p_0(1-p_0)}{n}}}\]

\begin{itemize}
\tightlist
\item
  Calculations:

  \begin{itemize}
  \tightlist
  \item
    \(\hat{p} = \frac{538}{1015} = 0.5301\)\\
  \item
    \(z = \frac{0.5301 - 0.5}{\sqrt{\frac{0.5(1-0.5)}{1015}}} = 1.91\)
  \end{itemize}
\end{itemize}

\textbf{Step 4: Determine Critical Value}
- For a right-tailed test, \(z_{\alpha} = z_{0.05} = 1.645\).\\
- \textbf{P-Value}:\\
\(P(Z > 1.91) = 0.0281\)

\textbf{Step 5: Decision}
- Since the test statistic \(z = 1.91\) falls in the rejection region (\(z > 1.645\)), \textbf{reject \(H_0\)}.\\
- Alternatively, since \(P < 0.05\), reject \(H_0\) based on the P-value method.

\begin{Shaded}
\begin{Highlighting}[]
\FunctionTok{one.p.z}\NormalTok{(}\AttributeTok{p0=}\FloatTok{0.5}\NormalTok{, }\AttributeTok{phat=}\DecValTok{538}\SpecialCharTok{/}\DecValTok{1015}\NormalTok{, }\AttributeTok{n=}\DecValTok{1015}\NormalTok{, }\AttributeTok{tail=}\StringTok{"right"}\NormalTok{, }\AttributeTok{conflev=}\FloatTok{0.95}\NormalTok{, }\AttributeTok{alpha=}\FloatTok{0.05}\NormalTok{)}
\end{Highlighting}
\end{Shaded}

\begin{verbatim}
## $phat
## [1] 0.5300493
## 
## $z0
## [1] 1.914683
## 
## $CV
## [1] 1.644854
## 
## $conclusion
## [1] "H0 Rejected"
## 
## $CI
## [1] 0.4993450 0.5607536
## 
## $se
## [1] 0.01566575
## 
## $n
## [1] 1015
## 
## $p
## [1] 0.02776649
\end{verbatim}

\hypertarget{equivalent-to-one-mean-z-test}{%
\subsection{Equivalent to one-mean z-test}\label{equivalent-to-one-mean-z-test}}

With normal approximation, the one population proportion problem can be turned into a one population mean problem, so the z test can be used. In addition, \(\hat{p}\) is the mean of 301 Bernoulli variables, the population standard deviation is the standard deviation of each Bernoulli variable \(\sqrt{p(1-p)}\). The following R code implements the testing procedure. Under \(H_0\), the population standard deviation is \(\sqrt{p_0(1-p_0)}\).

\begin{Shaded}
\begin{Highlighting}[]
\NormalTok{phat }\OtherTok{\textless{}{-}} \DecValTok{24} \SpecialCharTok{/} \DecValTok{301}
\NormalTok{p0 }\OtherTok{\textless{}{-}} \FloatTok{0.063}
\NormalTok{sig }\OtherTok{\textless{}{-}} \FunctionTok{sqrt}\NormalTok{(p0}\SpecialCharTok{*}\NormalTok{(}\DecValTok{1}\SpecialCharTok{{-}}\NormalTok{p0))}

\FunctionTok{one.mean.z}\NormalTok{(}\AttributeTok{xbar=}\NormalTok{phat, }\AttributeTok{n=}\DecValTok{301}\NormalTok{, }\AttributeTok{sigma=}\NormalTok{sig, }\AttributeTok{mu0=}\FloatTok{0.063}\NormalTok{, }
        \AttributeTok{tail=}\StringTok{"right"}\NormalTok{, }\AttributeTok{alpha=}\FloatTok{0.05}\NormalTok{)}
\end{Highlighting}
\end{Shaded}

\begin{verbatim}
## $CI
## [1] 0.05228659 0.10718185
## 
## $xbar
## [1] 0.07973422
## 
## $n
## [1] 301
## 
## $margin
## [1] 0.02744763
## 
## $p
## [1] 0.1160539
## 
## $z0
## [1] 1.194947
## 
## $CV
## [1] 1.644854
## 
## $conclusion
## [1] "H0 Not Rejected"
\end{verbatim}

This is equivalent to the one-proportion z-test:

\begin{Shaded}
\begin{Highlighting}[]
\FunctionTok{one.p.z}\NormalTok{(}\AttributeTok{phat=}\DecValTok{24} \SpecialCharTok{/} \DecValTok{301}\NormalTok{, }\AttributeTok{n =} \DecValTok{301}\NormalTok{, }\AttributeTok{p0=}\FloatTok{0.063}\NormalTok{, }\AttributeTok{tail=}\StringTok{"right"}\NormalTok{, }\AttributeTok{alpha=}\FloatTok{0.05}\NormalTok{, }\AttributeTok{conflev=}\FloatTok{0.95}\NormalTok{)}
\end{Highlighting}
\end{Shaded}

\begin{verbatim}
## $phat
## [1] 0.07973422
## 
## $z0
## [1] 1.194947
## 
## $CV
## [1] 1.644854
## 
## $conclusion
## [1] "H0 Not Rejected"
## 
## $CI
## [1] 0.04913264 0.11033580
## 
## $se
## [1] 0.01561334
## 
## $n
## [1] 301
## 
## $p
## [1] 0.1160539
\end{verbatim}

\hypertarget{confidence-interval-of-a-proportion}{%
\section{Confidence Interval of a Proportion}\label{confidence-interval-of-a-proportion}}

Confidence Intervals for One Population Proportion

\hypertarget{definitions-population-proportion-and-sample-proportion}{%
\subsection{Definitions: Population Proportion and Sample Proportion}\label{definitions-population-proportion-and-sample-proportion}}

\begin{itemize}
\item
  \textbf{Population proportion \(p\)}:\\
  The proportion of the entire population that has the specified attribute.\\
  Example: The proportion of females among U.S. residents.
\item
  \textbf{Sample proportion \(\hat{p}\)}:\\
  The proportion of a sample from the population that has the specified attribute.\\
  Example: The proportion of females in a random sample of 1000 U.S. residents.
\end{itemize}

\hypertarget{definitions-computing-sample-proportion}{%
\subsection{Definitions: Computing Sample Proportion}\label{definitions-computing-sample-proportion}}

\begin{itemize}
\tightlist
\item
  \(\hat{p}\) is computed as \(\hat{p} = \frac{x}{n}\), where:

  \begin{itemize}
  \tightlist
  \item
    \(x\) is the number of observations in the sample.
  \item
    \(n\) is the sample size.
  \end{itemize}
\end{itemize}

\begin{longtable}[]{@{}lll@{}}
\toprule\noalign{}
& Parameter & Statistic \\
\midrule\noalign{}
\endhead
\bottomrule\noalign{}
\endlastfoot
Means & \(\mu\) & \(\bar{x}\) \\
Proportions & \(p\) & \(\hat{p}\) \\
Standard Deviations & \(\sigma\) & \(s\) \\
\end{longtable}

\hypertarget{the-sampling-distribution-of-sample-proportion}{%
\subsection{The Sampling Distribution of Sample Proportion}\label{the-sampling-distribution-of-sample-proportion}}

\begin{itemize}
\item
  For \(\hat{p} = \frac{x}{n}\):

  \begin{itemize}
  \tightlist
  \item
    \(x\) is the sum of binary Bernoulli variables: \(x = \sum x_i\).
  \item
    \(E(x_i) = p\) implies \(E(\hat{p}) = E(x)/n = p\).
  \item
    \(\text{Var}(x_i) = p(1-p)\) implies \(\text{Var}(\hat{p}) = \frac{p(1-p)}{n}\).
  \end{itemize}
\item
  The standard deviation of \(\hat{p}\) is:\\
  \(\sigma_{\hat{p}} = \sqrt{\frac{p(1-p)}{n}}\)
\item
  For large \(n\), \(\hat{p}\) is normally distributed:\\
  \(\hat{p} \sim N\left(p, \frac{p(1-p)}{n}\right)\).
\end{itemize}

\hypertarget{example-the-sampling-distribution-of-sample-proportion}{%
\subsection{Example: The Sampling Distribution of Sample Proportion}\label{example-the-sampling-distribution-of-sample-proportion}}

\textbf{Problem}

Suppose that 20\% of all U.S. employees play hocky, i.e., \(p = 0.2\).\\
Draw random samples of size \(n = 100\) and compute the mean and standard deviation for \(\hat{p}\).

\textbf{Solution}

\begin{itemize}
\tightlist
\item
  Mean: \(\mu_{\hat{p}} = p = 0.2\)
\item
  Standard deviation:\\
  \(\sigma_{\hat{p}} = \sqrt{\frac{p(1-p)}{n}} = \sqrt{\frac{0.2(1-0.2)}{100}} = 0.04\)
\end{itemize}

\hypertarget{example-simulating-the-sampling-distribution-of-hatp}{%
\subsection{\texorpdfstring{Example: Simulating the Sampling Distribution of \(\hat{p}\)}{Example: Simulating the Sampling Distribution of \textbackslash hat\{p\}}}\label{example-simulating-the-sampling-distribution-of-hatp}}

\textbf{Steps}

\begin{enumerate}
\def\labelenumi{\arabic{enumi}.}
\item
  \textbf{Draw Random Samples}:\\
  Generate random samples of size 100 from a Bernoulli distribution with \(p = 0.2\).\\
  \texttt{xi\ =\ rbinom(100,\ size\ =\ 1,\ prob\ =\ 0.2)}
\item
  \textbf{Calculate Sample Proportion}:\\
  Compute \(\hat{p}\).\\
  \texttt{x\ =\ sum(xi)/100}
\item
  \textbf{Repeat Sampling}:\\
  Repeat Steps 1 and 2 for a large number of times, e.g., 2000 iterations.
\item
  \textbf{Visualize Distribution}:\\
  Plot a histogram of the 2000 \(\hat{p}\) values.
\item
  \textbf{Superimpose Normal Curve}:\\
  Overlay the histogram with a normal curve having mean 0.2 and standard deviation \(\sqrt{0.2\cdot 0.8/100}\).
\end{enumerate}

\begin{Shaded}
\begin{Highlighting}[]
\NormalTok{N }\OtherTok{=} \DecValTok{2000} \CommentTok{\#Number of replications in the simulation}
\NormalTok{        x }\OtherTok{=} \FunctionTok{rep}\NormalTok{(}\ConstantTok{NA}\NormalTok{, N) }\CommentTok{\#Save each replication here}
        \ControlFlowTok{for}\NormalTok{ (j }\ControlFlowTok{in} \DecValTok{1}\SpecialCharTok{:}\NormalTok{N)\{}
             \CommentTok{\#(1) Draw random 1010 Bernoulli samples p=0.191.}
\NormalTok{             xi }\OtherTok{=} \FunctionTok{rbinom}\NormalTok{(}\DecValTok{100}\NormalTok{, }\AttributeTok{size =} \DecValTok{1}\NormalTok{, }\AttributeTok{prob=}\FloatTok{0.2}\NormalTok{)}
             \CommentTok{\#(2) Calculate the sample proportion. }
\NormalTok{             x[j] }\OtherTok{=} \FunctionTok{sum}\NormalTok{(xi)}\SpecialCharTok{/}\DecValTok{100} 
\NormalTok{        \} }
        \CommentTok{\#Draw the historgram of 2000 p\_hat}
        \FunctionTok{hist}\NormalTok{(x, }\AttributeTok{freq=}\ConstantTok{FALSE}\NormalTok{)}
        
        \CommentTok{\#Superimpose the normal curve with mean 0.191 and sd 0.012}
        \FunctionTok{lines}\NormalTok{(}\FunctionTok{seq}\NormalTok{(}\SpecialCharTok{{-}}\DecValTok{3}\NormalTok{, }\DecValTok{3}\NormalTok{, }\AttributeTok{by=}\FloatTok{0.001}\NormalTok{), }\FunctionTok{dnorm}\NormalTok{(}\FunctionTok{seq}\NormalTok{(}\SpecialCharTok{{-}}\DecValTok{3}\NormalTok{,}\DecValTok{3}\NormalTok{,}\AttributeTok{by=}\FloatTok{0.001}\NormalTok{), }\AttributeTok{mean=}\FloatTok{0.2}\NormalTok{, }\AttributeTok{sd=}\FunctionTok{sqrt}\NormalTok{(}\FloatTok{0.2}\SpecialCharTok{*}\FloatTok{0.8}\SpecialCharTok{/}\DecValTok{100}\NormalTok{)), }\AttributeTok{lwd=}\DecValTok{3}\NormalTok{, }\AttributeTok{col=}\StringTok{"blue"}\NormalTok{)}
        \FunctionTok{legend}\NormalTok{( }\StringTok{"topleft"}\NormalTok{, }\FunctionTok{c}\NormalTok{(}\StringTok{"Histogram"}\NormalTok{, }\StringTok{"N(0.2, sd=sqrt(0.2*0.8/100))"}\NormalTok{), }\AttributeTok{cex=}\FloatTok{0.8}\NormalTok{, }\AttributeTok{bty=}\StringTok{"n"}\NormalTok{)}
\end{Highlighting}
\end{Shaded}

\includegraphics{StatsTB_files/figure-latex/unnamed-chunk-248-1.pdf}

\hypertarget{the-ci-and-the-margin-of-error-large-samples}{%
\subsection{The CI and the Margin of Error (Large Samples)}\label{the-ci-and-the-margin-of-error-large-samples}}

\begin{figure}
\centering
\includegraphics{https://i.ibb.co/RH3M39Y/Procedure12-1.png}
\caption{The CI and the Margin of Error (Large Samples)}
\end{figure}

\hypertarget{the-required-n-to-achieve-the-margin-of-error-e}{%
\subsection{\texorpdfstring{The Required \(n\) to Achieve the Margin of Error \(E\)}{The Required n to Achieve the Margin of Error E}}\label{the-required-n-to-achieve-the-margin-of-error-e}}

From the Margin of Error formula, solve for \(n\), then the required sample size to achieve the desired margin of error can be determined as

\[ n = \hat{p}(1-\hat{p})\left(\frac{z_{\alpha/2}}{E}\right)^2.\]

\textbf{Question}: Can we calculate \(n\) from this formula? What are the issues?

What if we don't know \(p\)?

\begin{figure}
\centering
\includegraphics{https://i.ibb.co/PhZkpT8/549-Figure-12-01.png}
\caption{\(p(1-p)\)}
\end{figure}

The largest value of \(p(1-p)\) is 0.25 when \(p=0.5\). One solution is to pick the largest possible \(p(1-p)\), which is 0.5*0.5=\tb{0.25}, which is a conservative planning with the largest \(n\).

Sometimes it is expensive to obtain a sample, so may consider more reasonable guess-estimate \(\hat{p}_g\) for \(\hat{p}\).
\includegraphics{https://i.ibb.co/ZY2pQVG/Formula12-3.png}

\hypertarget{example-sample-size-for-estimating-p}{%
\subsection{\texorpdfstring{Example: Sample Size for Estimating \(p\)}{Example: Sample Size for Estimating p}}\label{example-sample-size-for-estimating-p}}

\textbf{Problem}

Consider estimating the proportion \(p\):
1. (a) Obtain \(n\) to ensure a margin of error at most 0.01 for a 95\% CI.
2. (b) Find the 95\% CI for \(p\) if the estimated proportion of those who play hooky is \(\hat{p} = 0.194\).
3. (c) Calculate the margin of error in (b) and compare it to the specified 0.01 in (a).
4. (d) Repeat (a)--(c) if the proportion of those who play hooky can reasonably be assumed between 0.1 and 0.3.
5. (e) Compare the results in (d) with (a)--(c).

\hypertarget{solution}{%
\subsubsection{Solution}\label{solution}}

\textbf{(a) Sample Size to Ensure \(E \leq 0.01\)}

\begin{itemize}
\tightlist
\item
  Since \(\hat{p}(1 - \hat{p}) \leq 0.25\), the sample size can be calculated as:\\
  \[n = 0.25\left(\frac{z_{\alpha/2}}{E}\right)^2 = 0.25\left(\frac{1.96}{0.01}\right)^2 = 9604\]
\end{itemize}

\textbf{(b) 95\% CI for \(p\) with \(\hat{p} = 0.194\)}

\begin{itemize}
\tightlist
\item
  The 95\% CI is:\\
  \[(0.194 - 1.96\sqrt{\frac{0.194(1 - 0.194)}{9604}}, 0.194 + 1.96\sqrt{\frac{0.194(1 - 0.194)}{9604}}) = (0.186, 0.202)\]
\end{itemize}

\textbf{(c) Margin of Error}
- Margin of error:\\
\[E = 1.96\sqrt{\frac{0.194(1 - 0.194)}{9604}} = 0.008\]\\
- Since \(E = 0.008 < 0.01\), the actual margin of error is smaller than the required margin of error.

\textbf{(d) Repeat Calculation of \(n\) for \(\hat{p}\) Between 0.1 and 0.3}
- Pick the value closer to 0.5 (i.e., 0.3):\\
\[n = 0.3(1 - 0.3)\left(\frac{1.96}{0.01}\right)^2 = 8068\]\\
- The 95\% CI for n=8068 becomes:\\
\[(0.194 - 1.96\sqrt{\frac{0.3(1 - 0.3)}{8068}}, 0.194 + 1.96\sqrt{\frac{0.3(1 - 0.3)}{8068}}) = (0.185, 0.203)\]\\
- Margin of error: \(E = 0.009\).

\textbf{(e) Comparison}
- Using a reasonable estimate for \(\hat{p}\) reduces the required sample size by more than 1500.\\
- The margin of error (\(E = 0.009\)) still satisfies the requirement (\(E \leq 0.01\)).

\begin{Shaded}
\begin{Highlighting}[]
\CommentTok{\#Part (a). No knowledge of p}
\FunctionTok{n.p.CI}\NormalTok{(}\AttributeTok{E=}\FloatTok{0.01}\NormalTok{, }\AttributeTok{conflev =} \FloatTok{0.95}\NormalTok{)}
\end{Highlighting}
\end{Shaded}

\begin{verbatim}
## $pg
## [1] 0.5
## 
## $n
## [1] 9604
\end{verbatim}

\begin{Shaded}
\begin{Highlighting}[]
\CommentTok{\#Part (b). phat = 0.194. Find 95\%CI}
\FunctionTok{ci.p}\NormalTok{(}\AttributeTok{r=}\FloatTok{0.194}\SpecialCharTok{*}\DecValTok{9604}\NormalTok{, }\AttributeTok{n=}\DecValTok{9604}\NormalTok{, }\AttributeTok{method=}\StringTok{"Asymptotic Normal"}\NormalTok{, }\AttributeTok{conflev =} \FloatTok{0.95}\NormalTok{)}
\end{Highlighting}
\end{Shaded}

\begin{verbatim}
## [1] 0.1860916 0.2019084
\end{verbatim}

\begin{Shaded}
\begin{Highlighting}[]
\CommentTok{\#Part (d). n for p in (0.1, 0.3)}
\FunctionTok{n.p.CI}\NormalTok{(}\AttributeTok{E=}\FloatTok{0.01}\NormalTok{, }\AttributeTok{conflev =} \FloatTok{0.95}\NormalTok{, }\AttributeTok{g=}\FunctionTok{c}\NormalTok{(}\FloatTok{0.1}\NormalTok{, }\FloatTok{0.3}\NormalTok{))}
\end{Highlighting}
\end{Shaded}

\begin{verbatim}
## $pg
## [1] 0.3
## 
## $n
## [1] 8068
\end{verbatim}

\begin{Shaded}
\begin{Highlighting}[]
\CommentTok{\#CI for p with n=8068 and phat = 0.194}
\FunctionTok{ci.p}\NormalTok{(}\AttributeTok{r=}\FloatTok{0.194}\SpecialCharTok{*}\DecValTok{8068}\NormalTok{, }\AttributeTok{n=}\DecValTok{8068}\NormalTok{, }\AttributeTok{conflev =} \FloatTok{0.95}\NormalTok{)}
\end{Highlighting}
\end{Shaded}

\begin{verbatim}
## [1] 0.1854213 0.2028003
\end{verbatim}

\hypertarget{inference-for-comparing-two-population-proportions}{%
\chapter{Inference for Comparing Two Population Proportions}\label{inference-for-comparing-two-population-proportions}}

\begin{Shaded}
\begin{Highlighting}[]
\FunctionTok{library}\NormalTok{(IntroStats)}
\end{Highlighting}
\end{Shaded}

\hypertarget{normal-approximation-method-two-proportion-z-test}{%
\section{Normal Approximation Method (two-proportion z-test)}\label{normal-approximation-method-two-proportion-z-test}}

The normal approximation method, also called the two-proportion z-test, is used when we want to compare the proportion of success between two independent groups. This method helps us decide whether the difference in sample proportions is large enough to suggest a real difference in the population or if it could have occurred just by chance.

When the sample sizes are large, the difference between the two sample proportions approximately follows a normal distribution. Specifically, the normal approximation is valid when both groups meet the success-failure condition: each group must have at least 5 expected successes and 5 expected failures (i.e., \textbf{n₁p̂₁ ≥ 5}, \textbf{n₁(1−p̂₁) ≥ 5}, and similarly for group 2). If this condition is met, we can apply the z-test formula to calculate a test statistic and determine statistical significance.

For example, we might use this test to compare the proportion of patients who recover using Treatment A versus Treatment B.

\hypertarget{the-distribution-of-hatp_1---hatp_2}{%
\subsection{\texorpdfstring{The Distribution of \(\hat{p}_1 - \hat{p}_2\)}{The Distribution of \textbackslash hat\{p\}\_1 - \textbackslash hat\{p\}\_2}}\label{the-distribution-of-hatp_1---hatp_2}}

\hypertarget{key-properties}{%
\subsubsection{Key Properties}\label{key-properties}}

\begin{itemize}
\tightlist
\item
  Consider two sample proportions, \(\hat{p}_1\) and \(\hat{p}_2\), that follow normal distributions.\\
  Their difference also follows a \textbf{normal distribution}:
\end{itemize}

\[
\hat{p}_1 - \hat{p}_2 \sim N(\mu_{\hat{p}_1 - \hat{p}_2}, \sigma^2_{\hat{p}_1 - \hat{p}_2})
\]

\begin{itemize}
\item ~
  \hypertarget{mean-and-variance}{%
  \section{Mean and variance:}\label{mean-and-variance}}

  \[
  \mu_{\hat{p}_1 - \hat{p}_2} = p_1 - p_2
  \]

  \begin{itemize}
  \tightlist
  \item
    \[
    \sigma_{\hat{p}_1 - \hat{p}_2} = \sqrt{\frac{p_1(1-p_1)}{n_1} + \frac{p_2(1-p_2)}{n_2}}
    \]
  \end{itemize}
\item
  Test statistic:
  \[
  z = \frac{(\hat{p}_1 - \hat{p}_2) - (p_1 - p_2)}{\sqrt{\frac{p_1(1-p_1)}{n_1} + \frac{p_2(1-p_2)}{n_2}}} \sim N(0, 1)
  \]
\item
  Under \(H_0: p_1 = p_2\), the test statistic simplifies to:
  \[
  z = \frac{\hat{p}_1 - \hat{p}_2}{\sqrt{p(1-p)}\sqrt{\frac{1}{n_1} + \frac{1}{n_2}}} \sim N(0, 1)
  \]
\end{itemize}

\hypertarget{addressing-the-unknown-p-in-h_0}{%
\subsubsection{\texorpdfstring{Addressing the Unknown \(p\) in \(H_0\)}{Addressing the Unknown p in H\_0}}\label{addressing-the-unknown-p-in-h_0}}

\begin{itemize}
\tightlist
\item
  When constructing hypothesis tests or confidence intervals for two proportions, \(p\) is unknown under \(H_0\).\\
\item
  To resolve this, we estimate \(p\) using the combined samples.
\end{itemize}

\hypertarget{estimation-of-p}{%
\subsubsection{\texorpdfstring{Estimation of \(p\)}{Estimation of p}}\label{estimation-of-p}}

\begin{itemize}
\tightlist
\item
  Under \(H_0\), the two samples are assumed to have the same proportion, so the best estimate is the \textbf{pooled sample proportion}:
  \[
  \hat{p}_p = \frac{x_1 + x_2}{n_1 + n_2}
  \]
\end{itemize}

\hypertarget{updated-test-statistic}{%
\subsubsection{Updated Test Statistic}\label{updated-test-statistic}}

\begin{itemize}
\tightlist
\item
  Replace \(p\) with \(\hat{p}_p\) in the test statistic:
  \[
  z = \frac{\hat{p}_1 - \hat{p}_2}{\sqrt{\hat{p}_p(1 - \hat{p}_p)}\sqrt{\frac{1}{n_1} + \frac{1}{n_2}}} \sim N(0, 1)
  \]
\end{itemize}

\hypertarget{two-proportions-z-test}{%
\subsection{Two-Proportions z-Test}\label{two-proportions-z-test}}

\begin{figure}
\centering
\includegraphics{https://i.ibb.co/56KBrMh/Procedure12-3a.png}
\caption{Two-Proportions z-Test}
\end{figure}

\includegraphics{https://i.ibb.co/fS6qWJM/Procedure12-3b.png}
\#\#\# Example: Two-Proportion z-Test

\hypertarget{problem}{%
\subsubsection{Problem}\label{problem}}

In a poll of random samples with 747 men and 434 women in the U.S., 276 men and 195 women said they sometimes order a dish without meat or fish.\\
At a \textbf{5\% significance level}, do the data suggest that the proportion of men who sometimes order a dish without meat or fish is smaller than that of women?

\begin{figure}
\centering
\includegraphics{https://i.ibb.co/8zdLfVf/563-Table-12-02.png}
\caption{Two-Proportions z-Test}
\end{figure}

\textbf{Solution:}

\textbf{Statistics Preparation}
- \(x_1 = 276\), \(n_1 = 747\), \(x_2 = 195\), \(n_2 = 434\)\\
- \(\hat{p}_1 = \frac{x_1}{n_1} = \frac{276}{747} = 0.369\)\\
- \(\hat{p}_2 = \frac{x_2}{n_2} = \frac{195}{434} = 0.449\)\\
- \(\hat{p}_p = \frac{x_1 + x_2}{n_1 + n_2} = \frac{471}{1181} = 0.399\)

\textbf{Step 1: Hypotheses}
- \(H_0: p_1 = p_2\)\\
- \(H_a: p_1 < p_2\) (left-tailed test).

\textbf{Step 2: Significance Level}
- \(\alpha = 0.05\)

\textbf{Step 3: Compute the Test Statistic}
- Formula:\\
\[z = \frac{\hat{p}_1 - \hat{p}_2}{\sqrt{\hat{p}_p(1-\hat{p}_p)}\sqrt{\frac{1}{n_1} + \frac{1}{n_2}}}\]\\
- Substituting values:\\
\[z = \frac{0.369 - 0.449}{\sqrt{0.399(1 - 0.399)}\sqrt{\frac{1}{747} + \frac{1}{434}}} = -2.71\]

\textbf{Step 4: Determine the Critical Value}
- Critical value for a left-tailed test:\\
\[-z_{\alpha} = -z_{0.05} = -1.645\]

\textbf{Step 5: Decision}
- The test statistic \(z = -2.71\) falls in the rejection region (\(z < -1.645\)).\\
- Therefore, \textbf{reject \(H_0\)}.

\textbf{Conclusion}
At the 5\% significance level, there is sufficient evidence to conclude that the proportion of men who sometimes order a dish without meat or fish is smaller than that of women.

\begin{figure}
\centering
\includegraphics{https://i.ibb.co/sPLyYZg/566-Figure-12-03a.png}
\caption{Two-Proportions z-Test}
\end{figure}

\begin{figure}
\centering
\includegraphics{https://i.ibb.co/2dnq81c/566-Figure-12-03b.png}
\caption{Two-Proportions z-Test}
\end{figure}

\hypertarget{two-proportions-z-interval}{%
\subsection{Two-Proportions z-Interval}\label{two-proportions-z-interval}}

\begin{figure}
\centering
\includegraphics{https://i.ibb.co/0Q1dGFs/Procedure12-4.png}
\caption{Two-Proportions z-Interval}
\end{figure}

\hypertarget{example-two-proportions-z-interval}{%
\subsection{Example: Two-Proportions z-Interval}\label{example-two-proportions-z-interval}}

\hypertarget{problem-1}{%
\subsubsection{Problem}\label{problem-1}}

Obtain a 90\% CI for the difference \(p_1 - p_2\).

\hypertarget{solution-1}{%
\subsubsection{Solution}\label{solution-1}}

\hypertarget{step-1-find-z_alpha2}{%
\subsubsection{\texorpdfstring{Step 1: Find \(z_{\alpha/2}\)}{Step 1: Find z\_\{\textbackslash alpha/2\}}}\label{step-1-find-z_alpha2}}

\begin{itemize}
\tightlist
\item
  For a 90\% CI, \(\alpha = 0.1\), so:\\
  \[z_{\alpha/2} = z_{0.05} = 1.645\]
\end{itemize}

\hypertarget{step-2-compute-the-endpoints-of-the-ci}{%
\subsubsection{Step 2: Compute the Endpoints of the CI}\label{step-2-compute-the-endpoints-of-the-ci}}

\begin{itemize}
\tightlist
\item
  Formula:\\
  \[\hat{p}_1 - \hat{p}_2 \pm z_{\alpha/2}\sqrt{\frac{\hat{p}_1(1-\hat{p}_1)}{n_1} + \frac{\hat{p}_2(1-\hat{p}_2)}{n_2}}\]\\
\item
  Substituting values:\\
  \[-0.08 \pm 0.049\]\\
\item
  The 90\% CI is \textbf{(-0.129, -0.031)}.
\end{itemize}

\begin{Shaded}
\begin{Highlighting}[]
\CommentTok{\# Inputs}
\NormalTok{r1 }\OtherTok{\textless{}{-}} \DecValTok{276}
\NormalTok{n1 }\OtherTok{\textless{}{-}} \DecValTok{747}
\NormalTok{r2 }\OtherTok{\textless{}{-}} \DecValTok{195}
\NormalTok{n2 }\OtherTok{\textless{}{-}} \DecValTok{434}

\CommentTok{\# Sample proportions}
\NormalTok{p1 }\OtherTok{\textless{}{-}}\NormalTok{ r1 }\SpecialCharTok{/}\NormalTok{ n1}
\NormalTok{p2 }\OtherTok{\textless{}{-}}\NormalTok{ r2 }\SpecialCharTok{/}\NormalTok{ n2}

\CommentTok{\# Difference in proportions}
\NormalTok{diff }\OtherTok{\textless{}{-}}\NormalTok{ p1 }\SpecialCharTok{{-}}\NormalTok{ p2}

\CommentTok{\# Standard error}
\NormalTok{se }\OtherTok{\textless{}{-}} \FunctionTok{sqrt}\NormalTok{((p1 }\SpecialCharTok{*}\NormalTok{ (}\DecValTok{1} \SpecialCharTok{{-}}\NormalTok{ p1) }\SpecialCharTok{/}\NormalTok{ n1) }\SpecialCharTok{+}\NormalTok{ (p2 }\SpecialCharTok{*}\NormalTok{ (}\DecValTok{1} \SpecialCharTok{{-}}\NormalTok{ p2) }\SpecialCharTok{/}\NormalTok{ n2))}

\CommentTok{\# z{-}score for 90\% confidence}
\NormalTok{z }\OtherTok{\textless{}{-}} \FunctionTok{qnorm}\NormalTok{(}\FloatTok{0.95}\NormalTok{)  }\CommentTok{\# since 90\% CI {-}\textgreater{} 5\% in each tail {-}\textgreater{} 95th percentile}

\CommentTok{\# Confidence interval}
\NormalTok{lower }\OtherTok{\textless{}{-}}\NormalTok{ diff }\SpecialCharTok{{-}}\NormalTok{ z }\SpecialCharTok{*}\NormalTok{ se}
\NormalTok{upper }\OtherTok{\textless{}{-}}\NormalTok{ diff }\SpecialCharTok{+}\NormalTok{ z }\SpecialCharTok{*}\NormalTok{ se}

\CommentTok{\# Show the result}
\FunctionTok{cat}\NormalTok{(}\StringTok{"90\% Confidence Interval for the difference in proportions:}\SpecialCharTok{\textbackslash{}n}\StringTok{"}\NormalTok{)}
\end{Highlighting}
\end{Shaded}

\begin{verbatim}
## 90% Confidence Interval for the difference in proportions:
\end{verbatim}

\begin{Shaded}
\begin{Highlighting}[]
\FunctionTok{cat}\NormalTok{(}\StringTok{"("}\NormalTok{, }\FunctionTok{round}\NormalTok{(lower, }\DecValTok{3}\NormalTok{), }\StringTok{","}\NormalTok{, }\FunctionTok{round}\NormalTok{(upper, }\DecValTok{3}\NormalTok{), }\StringTok{")}\SpecialCharTok{\textbackslash{}n}\StringTok{"}\NormalTok{)}
\end{Highlighting}
\end{Shaded}

\begin{verbatim}
## ( -0.129 , -0.031 )
\end{verbatim}

\hypertarget{plus-four-z-interval-for-p_1---p_2-extra-material}{%
\subsection{\texorpdfstring{Plus-Four z-Interval for \(p_1 - p_2\) (Extra Material)}{Plus-Four z-Interval for p\_1 - p\_2 (Extra Material)}}\label{plus-four-z-interval-for-p_1---p_2-extra-material}}

\hypertarget{limitations-of-standard-z-interval}{%
\subsection{Limitations of Standard z-Interval}\label{limitations-of-standard-z-interval}}

\begin{itemize}
\tightlist
\item
  The standard z-Interval may not always provide accurate results, even for relatively large samples:
  \[\hat{p}_1 - \hat{p}_2 \pm z_{\alpha/2}\sqrt{\frac{\hat{p}_1(1-\hat{p}_1)}{n_1} + \frac{\hat{p}_2(1-\hat{p}_2)}{n_2}}\]
\end{itemize}

\hypertarget{plus-four-z-interval-agresti-and-caffo}{%
\subsection{Plus-Four z-Interval (Agresti and Caffo)}\label{plus-four-z-interval-agresti-and-caffo}}

\begin{itemize}
\tightlist
\item
  An alternative procedure:
  \[
  (\tilde{p}_1 - \tilde{p}_2) \pm z_{\alpha/2}\sqrt{\frac{\tilde{p}_1(1-\tilde{p}_1)}{n_1+2} + \frac{\tilde{p}_2(1-\tilde{p}_2)}{n_2+2}}
  \]
\item
  Where:
  \[\tilde{p}_1 = \frac{x_1 + 1}{n_1 + 2}, \quad \tilde{p}_2 = \frac{x_2 + 1}{n_2 + 2}\]
\end{itemize}

\hypertarget{in-class-practice-and-discussion}{%
\subsection{In-Class Practice and Discussion}\label{in-class-practice-and-discussion}}

\hypertarget{problem-a}{%
\subsubsection{Problem (a)}\label{problem-a}}

Researchers randomly divided 4532 healthy women over 65 years old into two groups:
- Group 1: 2229 received hormone-replacement therapy (HRT).
- Group 2: 2303 received a placebo.

Over 5 years:
- 40 women in Group 1 were diagnosed with dementia.
- 21 women in Group 2 were diagnosed with dementia.

\hypertarget{question}{%
\subsubsection{Question}\label{question}}

At the \textbf{5\% significance level}, do the data provide sufficient evidence to conclude that healthy women over 65 years old who take HRT are at greater risk for dementia than those who do not?

\hypertarget{steps}{%
\subsubsection{Steps}\label{steps}}

\begin{enumerate}
\def\labelenumi{\arabic{enumi}.}
\tightlist
\item
  \textbf{Populations and Samples}:

  \begin{itemize}
  \tightlist
  \item
    Population 1: Women taking HRT.\\
  \item
    Population 2: Women receiving placebo.
  \end{itemize}
\item
  \textbf{Hypotheses}:

  \begin{itemize}
  \tightlist
  \item
    \(H_0: p_1 = p_2\)\\
  \item
    \(H_a: p_1 > p_2\) (right-tailed test).
  \end{itemize}
\item
  \textbf{Significance Level}:

  \begin{itemize}
  \tightlist
  \item
    \(\alpha = 0.05\)
  \end{itemize}
\item
  \textbf{Test Statistic}:

  \begin{itemize}
  \tightlist
  \item
    Use the two-proportions z-test.
  \end{itemize}
\item
  \textbf{Critical Value}:

  \begin{itemize}
  \tightlist
  \item
    \(z_{0.05} = 1.645\).
  \end{itemize}
\item
  \textbf{Conclusion}:

  \begin{itemize}
  \tightlist
  \item
    Compute the test statistic, compare with the critical value, and decide whether to reject \(H_0\).
  \end{itemize}
\end{enumerate}

\hypertarget{problem-b}{%
\subsubsection{Problem (b)}\label{problem-b}}

Determine and interpret the \textbf{90\% CI} for the difference in dementia risk rates for healthy women over 65 years old.

\hypertarget{steps-1}{%
\subsubsection{Steps}\label{steps-1}}

\begin{enumerate}
\def\labelenumi{\arabic{enumi}.}
\item
  Compute the 90\% CI using:\\
  \[\hat{p}_1 - \hat{p}_2 \pm z_{\alpha/2}\sqrt{\frac{\hat{p}_1(1-\hat{p}_1)}{n_1} + \frac{\hat{p}_2(1-\hat{p}_2)}{n_2}}\]
\item
  Interpret the CI in the context of the difference in dementia risk rates.
\end{enumerate}

\hypertarget{summary-4}{%
\subsection{Summary}\label{summary-4}}

The two-proportion z-test is a useful method for comparing the proportions of success between two independent groups, especially when sample sizes are large enough to justify a normal approximation. By checking the success-failure condition for both groups, we ensure that the normal distribution provides a reliable basis for inference. The test uses a z-statistic to measure how far apart the sample proportions are, and a p-value to decide whether the observed difference is statistically significant or likely due to random variation. This method is commonly used in biomedical studies, clinical trials, and social science research when comparing two groups.

\hypertarget{chi-square-test-method-without-continuity-correction}{%
\section{Chi-Square Test Method (Without Continuity Correction)}\label{chi-square-test-method-without-continuity-correction}}

The chi-square test is used to compare two proportions by checking for independence between two categorical variables. In this case, the null hypothesis \(H_0: p_1 = p_2\) is equivalent to testing whether the \textbf{treatment group} and \textbf{response outcome} are independent.

This method uses a 2×2 contingency table and compares the observed counts to the expected counts under the assumption of independence. When the sample size is large enough, the chi-square test provides a valid approximation and does \textbf{not require a correction}.

For large samples, the \textbf{chi-square test without continuity correction} gives a p-value very close (or even identical) to the one obtained from the two-proportion z-test. In this example, the chi-square test and z-test both produce a p-value of approximately \textbf{0.01996}, suggesting a statistically significant difference between the groups.

\begin{Shaded}
\begin{Highlighting}[]
\CommentTok{\# Chi{-}square test WITHOUT continuity correction using base R}
\CommentTok{\# Group 1: 21 successes out of 150 → 129 failures}
\CommentTok{\# Group 2: 48 successes out of 200 → 152 failures}

\CommentTok{\# Make 2×2 table}
\NormalTok{tbl }\OtherTok{\textless{}{-}} \FunctionTok{matrix}\NormalTok{(}\FunctionTok{c}\NormalTok{(}\DecValTok{21}\NormalTok{, }\DecValTok{129}\NormalTok{, }\DecValTok{48}\NormalTok{, }\DecValTok{152}\NormalTok{), }\AttributeTok{nrow =} \DecValTok{2}\NormalTok{, }\AttributeTok{byrow =} \ConstantTok{TRUE}\NormalTok{)}

\CommentTok{\# Run chi{-}square test without continuity correction}
\FunctionTok{chisq.test}\NormalTok{(tbl, }\AttributeTok{correct =} \ConstantTok{FALSE}\NormalTok{)}
\end{Highlighting}
\end{Shaded}

\begin{verbatim}
## 
##  Pearson's Chi-squared test
## 
## data:  tbl
## X-squared = 5.4154, df = 1, p-value = 0.01996
\end{verbatim}

\hypertarget{chi-square-test-with-continuity-correction}{%
\section{Chi-Square Test with Continuity Correction}\label{chi-square-test-with-continuity-correction}}

The chi-square test with continuity correction is used when comparing two proportions in a 2×2 contingency table, especially when the sample size is small to moderate. This method helps adjust for the fact that the chi-square distribution is continuous, while the actual data are discrete (whole number counts). The correction is applied to avoid overestimating the significance of the test statistic.

The null hypothesis remains the same as before:

\[
H_0: p_1 = p_2
\]

This is equivalent to testing whether the row and column variables (e.g., treatment and response) are independent.

\hypertarget{test-statistic-with-continuity-correction}{%
\subsection{Test Statistic with Continuity Correction:}\label{test-statistic-with-continuity-correction}}

The formula for the chi-square statistic with Yates' continuity correction is:

\[
\chi^2 = \sum \frac{(|O - E| - 0.5)^2}{E}
\]

Where:\\
- \(O\) = observed count\\
- \(E\) = expected count\\
- The correction subtracts 0.5 from the absolute difference before squaring.

This correction reduces the chi-square value slightly, making the test more conservative (less likely to produce a false positive).

\hypertarget{r-code-example}{%
\subsection{R Code Example:}\label{r-code-example}}

In the following example, we compare two treatment groups:

\begin{itemize}
\tightlist
\item
  Group 1: 21 successes out of 150 → 129 failures\\
\item
  Group 2: 48 successes out of 200 → 152 failures
\end{itemize}

\begin{Shaded}
\begin{Highlighting}[]
\CommentTok{\# Construct 2x2 contingency table}
\CommentTok{\# Row 1: Treatment group 1 (21 successes, 129 failures)}
\CommentTok{\# Row 2: Treatment group 2 (48 successes, 152 failures)}

\NormalTok{tbl }\OtherTok{\textless{}{-}} \FunctionTok{matrix}\NormalTok{(}\FunctionTok{c}\NormalTok{(}\DecValTok{21}\NormalTok{, }\DecValTok{129}\NormalTok{, }\DecValTok{48}\NormalTok{, }\DecValTok{152}\NormalTok{), }\AttributeTok{nrow =} \DecValTok{2}\NormalTok{, }\AttributeTok{byrow =} \ConstantTok{TRUE}\NormalTok{)}

\CommentTok{\# Perform chi{-}square test with continuity correction}
\FunctionTok{chisq.test}\NormalTok{(tbl, }\AttributeTok{correct =} \ConstantTok{TRUE}\NormalTok{)}
\end{Highlighting}
\end{Shaded}

\begin{verbatim}
## 
##  Pearson's Chi-squared test with Yates' continuity correction
## 
## data:  tbl
## X-squared = 4.8021, df = 1, p-value = 0.02843
\end{verbatim}

This test outputs the chi-square statistic, degrees of freedom, and the p-value with the continuity correction applied. Use this method when dealing with a 2x2 table and smaller sample sizes to improve the accuracy of your inference.

\hypertarget{fishers-exact-test}{%
\section{Fisher's Exact Test}\label{fishers-exact-test}}

The Fisher's Exact Test is used to compare two proportions when sample sizes are small or when the assumptions of the chi-square test are not met. In particular, it is appropriate when the expected counts in any of the cells of a 2×2 contingency table are less than 5, which violates the assumptions of the normal approximation and chi-square methods.

Unlike the chi-square test, which relies on a large-sample approximation to the chi-square distribution, Fisher's test calculates the exact p-value using the hypergeometric distribution. This makes it especially useful in clinical trials, rare disease studies, or other biomedical applications where data may be limited.

\hypertarget{example-comparing-side-effects-in-two-patient-groups}{%
\subsection{Example: Comparing Side Effects in Two Patient Groups}\label{example-comparing-side-effects-in-two-patient-groups}}

Suppose a pharmaceutical company is testing two treatments for high blood pressure. Group 1 receives \textbf{Drug A} and Group 2 receives \textbf{Drug B}. After the trial, researchers record whether patients in each group experienced a specific side effect.

\begin{itemize}
\tightlist
\item
  In Group 1, \textbf{21 out of 150} patients reported the side effect.
\item
  In Group 2, \textbf{48 out of 200} patients reported the side effect.
\end{itemize}

We want to know if there is a statistically significant difference in side effect rates between the two treatments.

\hypertarget{data-summary}{%
\subsubsection{Data Summary:}\label{data-summary}}

\begin{longtable}[]{@{}llll@{}}
\toprule\noalign{}
& Side Effect (Yes) & Side Effect (No) & Total \\
\midrule\noalign{}
\endhead
\bottomrule\noalign{}
\endlastfoot
Group 1 (Drug A) & 21 & 129 & 150 \\
Group 2 (Drug B) & 48 & 152 & 200 \\
\end{longtable}

\begin{center}\rule{0.5\linewidth}{0.5pt}\end{center}

\hypertarget{r-code}{%
\subsection{R Code}\label{r-code}}

\begin{Shaded}
\begin{Highlighting}[]
\CommentTok{\# Construct 2x2 contingency table}
\CommentTok{\# Group 1: 21 experienced side effects, 129 did not}
\CommentTok{\# Group 2: 48 experienced side effects, 152 did not}

\NormalTok{tbl }\OtherTok{\textless{}{-}} \FunctionTok{matrix}\NormalTok{(}\FunctionTok{c}\NormalTok{(}\DecValTok{21}\NormalTok{, }\DecValTok{129}\NormalTok{, }\DecValTok{48}\NormalTok{, }\DecValTok{152}\NormalTok{), }\AttributeTok{nrow =} \DecValTok{2}\NormalTok{, }\AttributeTok{byrow =} \ConstantTok{TRUE}\NormalTok{)}

\CommentTok{\# Apply Fisher\textquotesingle{}s Exact Test}
\FunctionTok{fisher.test}\NormalTok{(tbl, }\AttributeTok{alternative =} \StringTok{"two.sided"}\NormalTok{, }\AttributeTok{conf.level =} \FloatTok{0.95}\NormalTok{)}
\end{Highlighting}
\end{Shaded}

\begin{verbatim}
## 
##  Fisher's Exact Test for Count Data
## 
## data:  tbl
## p-value = 0.02129
## alternative hypothesis: true odds ratio is not equal to 1
## 95 percent confidence interval:
##  0.2782759 0.9327535
## sample estimates:
## odds ratio 
##   0.516454
\end{verbatim}

\hypertarget{interpretation-1}{%
\subsection{Interpretation}\label{interpretation-1}}

The Fisher's Exact Test calculates the exact p-value for testing whether there is an association between treatment group and side effect occurrence. If the resulting p-value is less than the significance level (commonly 0.05), we reject the null hypothesis and conclude that the two treatments differ in terms of side effect rates.

This method is especially reliable in cases with small sample sizes or unevenly distributed outcomes, where chi-square tests may not be valid.

\hypertarget{example-side-effects-of-a-new-drug}{%
\section{Example: Side Effects of a New Drug}\label{example-side-effects-of-a-new-drug}}

To summarize and compare the different hypothesis testing methods for two proportions, we revisit the example involving side effects from two treatments for high blood pressure. A clinical trial is conducted to compare the side effects of two medications:

\begin{itemize}
\tightlist
\item
  \textbf{Drug A (Group 1):} 21 patients experienced side effects out of 150.\\
\item
  \textbf{Drug B (Group 2):} 48 patients experienced side effects out of 200.
\end{itemize}

We want to determine whether the difference in side effect rates between these two drugs is statistically significant using three different methods.

\hypertarget{approach-1.-two-proportion-z-test-normal-approximation}{%
\subsection{Approach 1. Two-Proportion Z-Test (Normal Approximation)}\label{approach-1.-two-proportion-z-test-normal-approximation}}

This test is appropriate when sample sizes are large and the success-failure condition is met for both groups.

\begin{Shaded}
\begin{Highlighting}[]
\FunctionTok{two.p.z}\NormalTok{(}\AttributeTok{x1=}\DecValTok{21}\NormalTok{, }\AttributeTok{n1=}\DecValTok{150}\NormalTok{, }\AttributeTok{x2=}\DecValTok{48}\NormalTok{, }\AttributeTok{n2=}\DecValTok{200}\NormalTok{, }\AttributeTok{tail=}\StringTok{"two"}\NormalTok{, }\AttributeTok{alpha=}\FloatTok{0.05}\NormalTok{)}
\end{Highlighting}
\end{Shaded}

\begin{verbatim}
## $z0
## [1] -2.32711
## 
## $CV
## [1] -1.959964  1.959964
## 
## $conclusion
## [1] "H0 Rejected"
## 
## $CI
## [1] -0.18115923 -0.01884077
## 
## $p
## [1] 0.0199594
## 
## $pp
## [1] 0.1971429
## 
## $phat
## [1] 0.14 0.24
\end{verbatim}

Equivalent R function:

\begin{Shaded}
\begin{Highlighting}[]
\CommentTok{\# Run two{-}proportion z{-}test (normal approximation)}
\FunctionTok{prop.test}\NormalTok{(}\AttributeTok{x =} \FunctionTok{c}\NormalTok{(}\DecValTok{21}\NormalTok{, }\DecValTok{48}\NormalTok{), }\AttributeTok{n =} \FunctionTok{c}\NormalTok{(}\DecValTok{150}\NormalTok{, }\DecValTok{200}\NormalTok{), }\AttributeTok{alternative =} \StringTok{"two.sided"}\NormalTok{, }\AttributeTok{correct =} \ConstantTok{FALSE}\NormalTok{)}
\end{Highlighting}
\end{Shaded}

\begin{verbatim}
## 
##  2-sample test for equality of proportions without continuity correction
## 
## data:  c(21, 48) out of c(150, 200)
## X-squared = 5.4154, df = 1, p-value = 0.01996
## alternative hypothesis: two.sided
## 95 percent confidence interval:
##  -0.18115923 -0.01884077
## sample estimates:
## prop 1 prop 2 
##   0.14   0.24
\end{verbatim}

This method uses the normal distribution to approximate the difference in proportions. The \texttt{correct\ =\ FALSE} argument disables continuity correction.

\hypertarget{approach-2.-chi-square-test-without-continuity-correction}{%
\subsection{Approach 2. Chi-Square Test (Without Continuity Correction)}\label{approach-2.-chi-square-test-without-continuity-correction}}

We can perform the chi-square test on the same data using a 2×2 contingency table.This is equivalent to the above normal approximation method.

\begin{Shaded}
\begin{Highlighting}[]
\CommentTok{\#Use chi{-}square test without continuity correction}
\FunctionTok{chisq.two.p}\NormalTok{(}\AttributeTok{r1=}\DecValTok{21}\NormalTok{, }\AttributeTok{n1=}\DecValTok{150}\NormalTok{, }\AttributeTok{r2=}\DecValTok{48}\NormalTok{, }\AttributeTok{n2=}\DecValTok{200}\NormalTok{, }\AttributeTok{conflev=}\FloatTok{0.95}\NormalTok{, }\AttributeTok{correct=}\ConstantTok{FALSE}\NormalTok{)}
\end{Highlighting}
\end{Shaded}

\begin{verbatim}
## $chisq
## X-squared 
##  5.415442 
## 
## $p
## [1] 0.0199594
## 
## $CV
## [1] 3.841459
## 
## $conclusion
## [1] "H0 rejected"
## 
## $observed
##         Population
## Response Group 1 Group 2
##        Y      21      48
##        N     129     152
## 
## $expected
##         Population
## Response   Group 1   Group 2
##        Y  29.57143  39.42857
##        N 120.42857 160.57143
## 
## $residuals
##         Population
## Response    Group 1    Group 2
##        Y -1.5762208  1.3650473
##        N  0.7810673 -0.6764241
## 
## $standardized.residuals
##         Population
## Response  Group 1  Group 2
##        Y -2.32711  2.32711
##        N  2.32711 -2.32711
## 
## $phat
## [1] 0.14 0.24
## 
## $p1.exact.ci
## [1] 0.08879823 0.20601069
## 
## $p2.exact.ci
## [1] 0.1825719 0.3053063
## 
## $p1.ci
## [1] 0.08447153 0.19552847
## 
## $p2.ci
## [1] 0.1808104 0.2991896
## 
## $method
## [1] "Pearson's Chi-squared test"
\end{verbatim}

Equivalent R function:

\begin{Shaded}
\begin{Highlighting}[]
\CommentTok{\# Construct 2x2 table. Need to be careful on the order of the numbers.}
\NormalTok{tbl }\OtherTok{\textless{}{-}} \FunctionTok{matrix}\NormalTok{(}\FunctionTok{c}\NormalTok{(}\DecValTok{21}\NormalTok{, }\DecValTok{129}\NormalTok{, }\DecValTok{48}\NormalTok{, }\DecValTok{152}\NormalTok{), }\AttributeTok{nrow =} \DecValTok{2}\NormalTok{, }\AttributeTok{byrow =} \ConstantTok{TRUE}\NormalTok{)}

\CommentTok{\# Chi{-}square test without continuity correction}
\FunctionTok{chisq.test}\NormalTok{(tbl, }\AttributeTok{correct =} \ConstantTok{FALSE}\NormalTok{)}
\end{Highlighting}
\end{Shaded}

\begin{verbatim}
## 
##  Pearson's Chi-squared test
## 
## data:  tbl
## X-squared = 5.4154, df = 1, p-value = 0.01996
\end{verbatim}

\hypertarget{approach-3.-chi-square-test-with-continuity-correction}{%
\subsection{Approach 3. Chi-Square Test (With Continuity Correction)}\label{approach-3.-chi-square-test-with-continuity-correction}}

To make the chi-square test more conservative in smaller samples, we apply a continuity correction.

\begin{Shaded}
\begin{Highlighting}[]
\FunctionTok{chisq.two.p}\NormalTok{(}\AttributeTok{r1=}\DecValTok{21}\NormalTok{, }\AttributeTok{n1=}\DecValTok{150}\NormalTok{, }\AttributeTok{r2=}\DecValTok{48}\NormalTok{, }\AttributeTok{n2=}\DecValTok{200}\NormalTok{, }\AttributeTok{conflev=}\FloatTok{0.95}\NormalTok{, }\AttributeTok{correct=}\ConstantTok{TRUE}\NormalTok{)}
\end{Highlighting}
\end{Shaded}

\begin{verbatim}
## $chisq
## X-squared 
##  4.802068 
## 
## $p
## [1] 0.0284256
## 
## $CV
## [1] 3.841459
## 
## $conclusion
## [1] "H0 rejected"
## 
## $observed
##         Population
## Response Group 1 Group 2
##        Y      21      48
##        N     129     152
## 
## $expected
##         Population
## Response   Group 1   Group 2
##        Y  29.57143  39.42857
##        N 120.42857 160.57143
## 
## $residuals
##         Population
## Response    Group 1    Group 2
##        Y -1.5762208  1.3650473
##        N  0.7810673 -0.6764241
## 
## $standardized.residuals
##         Population
## Response  Group 1  Group 2
##        Y -2.32711  2.32711
##        N  2.32711 -2.32711
## 
## $phat
## [1] 0.14 0.24
## 
## $p1.exact.ci
## [1] 0.08879823 0.20601069
## 
## $p2.exact.ci
## [1] 0.1825719 0.3053063
## 
## $p1.ci
## [1] 0.0811382 0.1988618
## 
## $p2.ci
## [1] 0.1783104 0.3016896
## 
## $method
## [1] "Pearson's Chi-squared test with Yates' continuity correction"
\end{verbatim}

Equivalent R function:

\begin{Shaded}
\begin{Highlighting}[]
\CommentTok{\# Chi{-}square test with continuity correction}
\FunctionTok{chisq.test}\NormalTok{(tbl, }\AttributeTok{correct =} \ConstantTok{TRUE}\NormalTok{)}
\end{Highlighting}
\end{Shaded}

\begin{verbatim}
## 
##  Pearson's Chi-squared test with Yates' continuity correction
## 
## data:  tbl
## X-squared = 4.8021, df = 1, p-value = 0.02843
\end{verbatim}

This method adjusts the test statistic slightly and typically results in a slightly larger p-value.

\hypertarget{approach-4.-fishers-exact-test}{%
\subsection{Approach 4. Fisher's Exact Test}\label{approach-4.-fishers-exact-test}}

Fisher's Exact Test is used when sample sizes are small or when any expected counts are less than 5. It calculates the exact p-value using the hypergeometric distribution. For example, we would like to compare 21 successes out of 150 trials in group 1 versus 48 successes out of 200 trials using the code below:

\begin{Shaded}
\begin{Highlighting}[]
\CommentTok{\# Fisher\textquotesingle{}s exact test }
\FunctionTok{fisher}\NormalTok{(}\AttributeTok{r1=}\DecValTok{21}\NormalTok{, }\AttributeTok{n1=}\DecValTok{150}\NormalTok{, }\AttributeTok{r2=}\DecValTok{48}\NormalTok{, }\AttributeTok{n2=}\DecValTok{200}\NormalTok{, }\AttributeTok{alpha=}\FloatTok{0.05}\NormalTok{, }\AttributeTok{conflev=}\FloatTok{0.95}\NormalTok{, }\AttributeTok{alternative =} \StringTok{"two.sided"}\NormalTok{)}
\end{Highlighting}
\end{Shaded}

\begin{verbatim}
## $p
## [1] 0.02129259
## 
## $conclusion
## [1] "H0 rejected"
## 
## $odds.ratio.ci
## [1] 0.2782759 0.9327535
## attr(,"conf.level")
## [1] 0.95
## 
## $odds.ratio
## odds ratio 
##   0.516454 
## 
## $Ha
## [1] "two.sided"
## 
## $data
##         Population
## Response Group 1 Group 2
##        Y      21      48
##        N     129     152
## 
## $phat
## [1] 0.14 0.24
## 
## $p1.exact.ci
## [1] 0.08879823 0.20601069
## 
## $p2.exact.ci
## [1] 0.1825719 0.3053063
\end{verbatim}

Equivalent R function:

\begin{Shaded}
\begin{Highlighting}[]
\CommentTok{\# Fisher\textquotesingle{}s exact test}
\FunctionTok{fisher.test}\NormalTok{(tbl, }\AttributeTok{alternative =} \StringTok{"two.sided"}\NormalTok{, }\AttributeTok{conf.level =} \FloatTok{0.95}\NormalTok{)}
\end{Highlighting}
\end{Shaded}

\begin{verbatim}
## 
##  Fisher's Exact Test for Count Data
## 
## data:  tbl
## p-value = 0.02129
## alternative hypothesis: true odds ratio is not equal to 1
## 95 percent confidence interval:
##  0.2782759 0.9327535
## sample estimates:
## odds ratio 
##   0.516454
\end{verbatim}

This method is accurate regardless of sample size and is preferred when expected counts are low.

\textbf{Summary}

All four methods test the same hypothesis:

\[
H_0: p_1 = p_2 \quad \text{vs.} \quad H_A: p_1 \neq p_2
\]

\begin{longtable}[]{@{}
  >{\raggedright\arraybackslash}p{(\columnwidth - 4\tabcolsep) * \real{0.2772}}
  >{\raggedright\arraybackslash}p{(\columnwidth - 4\tabcolsep) * \real{0.3861}}
  >{\raggedright\arraybackslash}p{(\columnwidth - 4\tabcolsep) * \real{0.3366}}@{}}
\toprule\noalign{}
\begin{minipage}[b]{\linewidth}\raggedright
Method
\end{minipage} & \begin{minipage}[b]{\linewidth}\raggedright
Appropriate When
\end{minipage} & \begin{minipage}[b]{\linewidth}\raggedright
Key Feature
\end{minipage} \\
\midrule\noalign{}
\endhead
\bottomrule\noalign{}
\endlastfoot
Two-Proportion Z-Test & Large sample sizes & Uses normal approximation \\
Chi-Square (no correction) & Large sample sizes & Matches z-test closely \\
Chi-Square (with correction) & Small--moderate samples (2×2 tables) & More conservative (adds 0.5) \\
Fisher's Exact Test & Small samples or expected counts \textless{} 5 & Exact p-value, no approximation \\
\end{longtable}

In this example, all tests point to a statistically significant difference in side effect rates between the two drugs. However, the p-values may differ slightly based on the method used. This demonstrates how different tests can lead to similar conclusions while relying on different statistical assumptions.

\hypertarget{chi-square-tests}{%
\chapter{Chi-Square Tests}\label{chi-square-tests}}

\begin{Shaded}
\begin{Highlighting}[]
\FunctionTok{library}\NormalTok{(IntroStats)}
\end{Highlighting}
\end{Shaded}

\hypertarget{chi-square-distribution}{%
\section{Chi-Square Distribution}\label{chi-square-distribution}}

If a random variable \(Y\) is normally distributed with mean \(\mu\) and variance \(\sigma^2\), then the standardized variable \(z = \frac{Y - \mu}{\sigma}\) follows a standard normal distribution, \(N(0,1)\).

The square of this standardized variable, \(\chi^2_{(1)} = z^2\), follows a chi-square distribution with 1 degree of freedom.

More generally, if \(z_1, z_2, \dots, z_n\) are independent standard normal variables, then their sum of squares
\[
\chi^2_{(n)} = z_1^2 + z_2^2 + \dots + z_n^2
\]
follows a chi-square distribution with \(n\) degrees of freedom.

\hypertarget{properties-of-chi-square-distribution}{%
\subsection{Properties of Chi-Square Distribution}\label{properties-of-chi-square-distribution}}

The chi-square distribution has the following key characteristics:

\begin{itemize}
\tightlist
\item
  It is right-skewed, especially for low degrees of freedom.
\item
  As degrees of freedom increase, it approaches a normal distribution.
\item
  The total area under the chi-square curve is 1.
\end{itemize}

The R code below demonstrates the shape of the distribution at various degrees of freedom.

\begin{Shaded}
\begin{Highlighting}[]
\NormalTok{x }\OtherTok{\textless{}{-}} \FunctionTok{seq}\NormalTok{(}\DecValTok{0}\NormalTok{, }\DecValTok{50}\NormalTok{, }\AttributeTok{by =} \FloatTok{0.01}\NormalTok{)}
\FunctionTok{plot}\NormalTok{(x, }\FunctionTok{dchisq}\NormalTok{(x, }\AttributeTok{df =} \DecValTok{1}\NormalTok{), }\AttributeTok{type =} \StringTok{"l"}\NormalTok{, }\AttributeTok{main =} \StringTok{"Chi{-}Square Density (df = 1)"}\NormalTok{)}
\FunctionTok{lines}\NormalTok{(x, }\FunctionTok{dchisq}\NormalTok{(x, }\AttributeTok{df =} \DecValTok{5}\NormalTok{), }\AttributeTok{col =} \DecValTok{2}\NormalTok{, }\AttributeTok{lwd =} \DecValTok{2}\NormalTok{)}
\FunctionTok{lines}\NormalTok{(x, }\FunctionTok{dchisq}\NormalTok{(x, }\AttributeTok{df =} \DecValTok{15}\NormalTok{), }\AttributeTok{col =} \DecValTok{3}\NormalTok{, }\AttributeTok{lwd =} \DecValTok{2}\NormalTok{)}
\FunctionTok{lines}\NormalTok{(x, }\FunctionTok{dchisq}\NormalTok{(x, }\AttributeTok{df =} \DecValTok{30}\NormalTok{), }\AttributeTok{col =} \DecValTok{4}\NormalTok{, }\AttributeTok{lwd =} \DecValTok{2}\NormalTok{)}
\FunctionTok{legend}\NormalTok{(}\StringTok{"topright"}\NormalTok{, }\AttributeTok{legend =} \FunctionTok{c}\NormalTok{(}\StringTok{"df=1"}\NormalTok{, }\StringTok{"df=5"}\NormalTok{, }\StringTok{"df=15"}\NormalTok{, }\StringTok{"df=30"}\NormalTok{),}
       \AttributeTok{col =} \FunctionTok{c}\NormalTok{(}\DecValTok{1}\NormalTok{, }\DecValTok{2}\NormalTok{, }\DecValTok{3}\NormalTok{, }\DecValTok{4}\NormalTok{), }\AttributeTok{lwd =} \DecValTok{2}\NormalTok{)}
\end{Highlighting}
\end{Shaded}

\includegraphics{StatsTB_files/figure-latex/unnamed-chunk-264-1.pdf}

We can also calculate specific quantiles using the \texttt{qchisq()} function.

Example: Compute the 2.5th and 97.5th percentiles for 20 degrees of freedom.

\begin{Shaded}
\begin{Highlighting}[]
\FunctionTok{qchisq}\NormalTok{(}\AttributeTok{p =} \FloatTok{0.025}\NormalTok{, }\AttributeTok{df =} \DecValTok{20}\NormalTok{)   }\CommentTok{\# Lower 2.5\% quantile}
\end{Highlighting}
\end{Shaded}

\begin{verbatim}
## [1] 9.590777
\end{verbatim}

\begin{Shaded}
\begin{Highlighting}[]
\FunctionTok{qchisq}\NormalTok{(}\AttributeTok{p =} \FloatTok{0.975}\NormalTok{, }\AttributeTok{df =} \DecValTok{20}\NormalTok{)   }\CommentTok{\# Upper 2.5\% quantile (97.5\%)}
\end{Highlighting}
\end{Shaded}

\begin{verbatim}
## [1] 34.16961
\end{verbatim}

\hypertarget{example-quantiles-for-df-22}{%
\section{Example: Quantiles for df = 22}\label{example-quantiles-for-df-22}}

Determine the chi-square values (quantiles) for different areas under the curve for 22 degrees of freedom.

\begin{Shaded}
\begin{Highlighting}[]
\CommentTok{\# (a) 0.01 area to the right (99th percentile)}
\FunctionTok{qchisq}\NormalTok{(}\AttributeTok{p =} \FloatTok{0.99}\NormalTok{, }\AttributeTok{df =} \DecValTok{22}\NormalTok{)}
\end{Highlighting}
\end{Shaded}

\begin{verbatim}
## [1] 40.28936
\end{verbatim}

\begin{Shaded}
\begin{Highlighting}[]
\CommentTok{\# (b) 0.005 area to the right (99.5th percentile)}
\FunctionTok{qchisq}\NormalTok{(}\AttributeTok{p =} \FloatTok{0.995}\NormalTok{, }\AttributeTok{df =} \DecValTok{22}\NormalTok{)}
\end{Highlighting}
\end{Shaded}

\begin{verbatim}
## [1] 42.79565
\end{verbatim}

\begin{Shaded}
\begin{Highlighting}[]
\CommentTok{\# (c) 0.01 area to the left (1st percentile)}
\FunctionTok{qchisq}\NormalTok{(}\AttributeTok{p =} \FloatTok{0.01}\NormalTok{, }\AttributeTok{df =} \DecValTok{22}\NormalTok{)}
\end{Highlighting}
\end{Shaded}

\begin{verbatim}
## [1] 9.542492
\end{verbatim}

These quantiles are commonly used in hypothesis testing for determining critical values when using the chi-square distribution.

\hypertarget{general-formulation-of-the-chi-square-test}{%
\subsection{General Formulation of the Chi-Square Test}\label{general-formulation-of-the-chi-square-test}}

Chi-square tests are used with categorical data to compare observed frequencies (\(O_i\)) to expected frequencies (\(E_i\)).

The test statistic is calculated as:
\[
\chi^2 = \sum \frac{(O_i - E_i)^2}{E_i}
\]

Under the null hypothesis, this test statistic follows approximately a chi-square distribution with \((k - r)\) degrees of freedom, where:
- \(k\) is the number of categories
- \(r\) is the number of constraints (e.g., parameters estimated)

This test is widely used for:
- Testing goodness-of-fit
- Testing independence in contingency tables
- Testing homogeneity of proportions across groups

\hypertarget{types-of-chi-square-tests-and-decision-rules}{%
\subsection{Types of Chi-Square Tests and Decision Rules}\label{types-of-chi-square-tests-and-decision-rules}}

There are three main types of chi-square tests:

\begin{enumerate}
\def\labelenumi{\arabic{enumi}.}
\tightlist
\item
  Goodness-of-Fit Test:

  \begin{itemize}
  \tightlist
  \item
    Compares observed frequencies to expected frequencies from a known distribution.
  \end{itemize}
\item
  Test of Independence:

  \begin{itemize}
  \tightlist
  \item
    Determines if two categorical variables are independent in a contingency table.
  \end{itemize}
\item
  Test of Homogeneity:

  \begin{itemize}
  \tightlist
  \item
    Compares the distribution of a categorical variable across different populations.
  \end{itemize}
\end{enumerate}

\textbf{Decision Rule:}
- If the observed and expected values are close, the \(\chi^2\) statistic will be small, and we fail to reject the null hypothesis.
- If the observed values deviate significantly from the expected, the \(\chi^2\) statistic will be large, and we may reject the null hypothesis.

\textbf{Applicability Conditions:}
- Common rule: All expected cell counts should be at least 5.
- Cochran (1952, 1954) noted that for unimodal distributions (e.g., normal), expected frequencies as low as 1 may be acceptable.
- If expected frequencies are too small, categories should be combined to meet the assumptions of the test.

References:
- Cochran, W.G. (1952). The chi-square test of goodness-of-fit. \emph{Annals of Mathematical Statistics}, 23, 315--345.
- Cochran, W.G. (1954). Some methods for strengthening the common chi-squared test. \emph{Biometrics}, 10, 417--451.

\hypertarget{chi-square-test-of-goodness-of-fit}{%
\section{Chi-Square Test of Goodness-of-Fit}\label{chi-square-test-of-goodness-of-fit}}

\hypertarget{example-testing-for-normal-distribution}{%
\subsection{Example: Testing for Normal Distribution}\label{example-testing-for-normal-distribution}}

Researchers studied 47 diabetic patients to see whether their cholesterol levels followed a normal distribution. The observed frequency data for different cholesterol intervals is shown below.

We want to test whether this data provides enough evidence to say it \emph{does not} come from a normal distribution. Use significance level \(\alpha = 0.05\).

\begin{Shaded}
\begin{Highlighting}[]
\NormalTok{freq }\OtherTok{\textless{}{-}} \FunctionTok{c}\NormalTok{(}\DecValTok{1}\NormalTok{, }\DecValTok{3}\NormalTok{, }\DecValTok{8}\NormalTok{, }\DecValTok{18}\NormalTok{, }\DecValTok{6}\NormalTok{, }\DecValTok{4}\NormalTok{, }\DecValTok{4}\NormalTok{, }\DecValTok{3}\NormalTok{)}
\FunctionTok{barplot}\NormalTok{(freq, }
        \AttributeTok{col =} \FunctionTok{rep}\NormalTok{(}\StringTok{"lightblue"}\NormalTok{, }\FunctionTok{length}\NormalTok{(freq)), }
        \AttributeTok{ylab =} \StringTok{"Frequency"}\NormalTok{, }
        \AttributeTok{main =} \StringTok{"Observed Cholesterol Frequencies"}\NormalTok{)}
\end{Highlighting}
\end{Shaded}

\includegraphics{StatsTB_files/figure-latex/unnamed-chunk-269-1.pdf}

\hypertarget{step-1-hypotheses}{%
\subsubsection{Step 1: Hypotheses}\label{step-1-hypotheses}}

\begin{itemize}
\tightlist
\item
  \(H_0\): The cholesterol levels follow a normal distribution.
\item
  \(H_a\): The cholesterol levels do \textbf{not} follow a normal distribution.
\end{itemize}

\hypertarget{step-2-assumptions}{%
\subsubsection{Step 2: Assumptions}\label{step-2-assumptions}}

We estimate the mean and standard deviation from the sample:\\
- Sample mean: \(\bar{x} = 198.67\)\\
- Sample standard deviation: \(s = 41.31\)

We then compute the probability of each interval under the assumed normal distribution, convert it to expected frequency by multiplying by \(n = 47\).

This gives us the following expected values for each cholesterol category:

\begin{figure}
\centering
\includegraphics{https://i.ibb.co/jHd3K5t/table12-3-2.png}
\caption{Expected frequencies}
\end{figure}

\hypertarget{step-3-test-statistic}{%
\subsubsection{Step 3: Test Statistic}\label{step-3-test-statistic}}

Using the formula
\[
\chi^2 = \sum \frac{(O_i - E_i)^2}{E_i}
\]
we compute the test statistic using the observed and expected frequencies.

The calculated test statistic is: \(\chi^2 = 10.566\)

\hypertarget{step-4-degrees-of-freedom}{%
\subsubsection{Step 4: Degrees of Freedom}\label{step-4-degrees-of-freedom}}

We have 8 cholesterol intervals (groups), and 3 constraints:
- 2 parameters estimated (mean, standard deviation)
- 1 constraint that total observed = total expected

So degrees of freedom = \(8 - 3 = 5\)

\hypertarget{step-5-critical-value}{%
\subsubsection{Step 5: Critical Value}\label{step-5-critical-value}}

At \(\alpha = 0.05\) and df = 5, the critical value is:

\begin{Shaded}
\begin{Highlighting}[]
\FunctionTok{qchisq}\NormalTok{(}\AttributeTok{p =} \FloatTok{0.95}\NormalTok{, }\AttributeTok{df =} \DecValTok{5}\NormalTok{)}
\end{Highlighting}
\end{Shaded}

\begin{verbatim}
## [1] 11.0705
\end{verbatim}

Result: \texttt{11.070}

\hypertarget{step-6-conclusion}{%
\subsubsection{Step 6: Conclusion}\label{step-6-conclusion}}

Since the test statistic (10.566) is less than the critical value (11.070), we \textbf{fail to reject the null hypothesis}.

There is \textbf{not enough evidence} to say the cholesterol data is \emph{not} normally distributed.

We conclude that the data \textbf{could reasonably come from a normal population}.

\begin{figure}
\centering
\includegraphics{https://i.ibb.co/nwB5cV5/table12-3-3.png}
\caption{Comparing observed and expected frequencies}
\end{figure}

\hypertarget{example-testing-a-uniform-distribution}{%
\subsection{Example: Testing a Uniform Distribution}\label{example-testing-a-uniform-distribution}}

In genetics, a trait is believed to follow a 1:2:1 distribution:
- 25\% homozygous dominant
- 50\% heterozygous
- 25\% homozygous recessive

In a study of 200 individuals, the observed counts were:
- Dominant: 43
- Heterozygous: 125
- Recessive: 32

We want to test whether this sample fits the 1:2:1 distribution.

\hypertarget{step-1-hypotheses-1}{%
\subsubsection{Step 1: Hypotheses}\label{step-1-hypotheses-1}}

\begin{itemize}
\tightlist
\item
  \(H_0\): The data follows the 1:2:1 genetic distribution.
\item
  \(H_a\): The data does \textbf{not} follow the 1:2:1 genetic distribution.
\end{itemize}

\hypertarget{step-2-expected-frequencies}{%
\subsubsection{Step 2: Expected Frequencies}\label{step-2-expected-frequencies}}

Expected counts under 1:2:1 distribution:
- Dominant = 0.25 × 200 = 50
- Heterozygous = 0.50 × 200 = 100
- Recessive = 0.25 × 200 = 50

\hypertarget{step-3-compute-the-chi-square-statistic}{%
\subsubsection{Step 3: Compute the Chi-Square Statistic}\label{step-3-compute-the-chi-square-statistic}}

Observed: {[}43, 125, 32{]}\\
Expected: {[}50, 100, 50{]}

\[
\chi^2 = \frac{(43-50)^2}{50} + \frac{(125-100)^2}{100} + \frac{(32-50)^2}{50}
= \frac{49}{50} + \frac{625}{100} + \frac{324}{50} = 0.98 + 6.25 + 6.48 = 13.71
\]

\hypertarget{step-4-degrees-of-freedom-1}{%
\subsubsection{Step 4: Degrees of Freedom}\label{step-4-degrees-of-freedom-1}}

df = Number of categories -- 1 = 3 -- 1 = 2

\hypertarget{step-5-critical-value-1}{%
\subsubsection{Step 5: Critical Value}\label{step-5-critical-value-1}}

\begin{Shaded}
\begin{Highlighting}[]
\FunctionTok{qchisq}\NormalTok{(}\AttributeTok{p =} \FloatTok{0.95}\NormalTok{, }\AttributeTok{df =} \DecValTok{2}\NormalTok{)}
\end{Highlighting}
\end{Shaded}

\begin{verbatim}
## [1] 5.991465
\end{verbatim}

Result: \texttt{5.991}

\hypertarget{step-6-conclusion-1}{%
\subsubsection{Step 6: Conclusion}\label{step-6-conclusion-1}}

The test statistic (13.71) is much greater than the critical value (5.991).\\
\textbf{Reject the null hypothesis.}

There is \textbf{strong evidence} that the trait \textbf{does not follow} the expected 1:2:1 distribution.

\hypertarget{other-applications-of-the-goodness-of-fit-test}{%
\subsection{Other Applications of the Goodness-of-Fit Test}\label{other-applications-of-the-goodness-of-fit-test}}

The chi-square goodness-of-fit test is not limited to normal or uniform distributions. It can also be applied to test whether observed data matches any theoretical distribution, including:

\begin{itemize}
\tightlist
\item
  Binomial distributions
\item
  Poisson distributions
\item
  Geometric distributions
\end{itemize}

The method is always the same:
1. Determine expected frequencies using the theoretical distribution
2. Compute \(\chi^2\) statistic
3. Compare to critical value at appropriate df
4. Conclude whether the data fits the distribution

\hypertarget{step-by-step-procedure-summary}{%
\subsection{Step-by-Step Procedure (Summary)}\label{step-by-step-procedure-summary}}

Use the following checklist when performing a chi-square goodness-of-fit test:

\begin{enumerate}
\def\labelenumi{\arabic{enumi}.}
\item
  \textbf{State Hypotheses}

  \begin{itemize}
  \tightlist
  \item
    \(H_0\): The data follows the specified distribution
  \item
    \(H_a\): The data does not follow the specified distribution
  \end{itemize}
\item
  \textbf{Find Observed and Expected Frequencies}

  \begin{itemize}
  \tightlist
  \item
    Observed: Count from data
  \item
    Expected: Calculate using the assumed distribution
  \end{itemize}
\item
  \textbf{Compute Test Statistic}
  \[
  \chi^2 = \sum \frac{(O_i - E_i)^2}{E_i}
  \]
\item
  \textbf{Determine Degrees of Freedom}

  \begin{itemize}
  \tightlist
  \item
    df = Number of categories -- Number of estimated parameters -- 1
  \end{itemize}
\item
  \textbf{Find Critical Value}

  \begin{itemize}
  \tightlist
  \item
    Use \texttt{qchisq()} with your \(\alpha\) level and df
  \end{itemize}
\item
  \textbf{Compare and Decide}

  \begin{itemize}
  \tightlist
  \item
    If \(\chi^2\) is greater than critical value: reject \(H_0\)
  \item
    If \(\chi^2\) is less than critical value: fail to reject \(H_0\)
  \end{itemize}
\end{enumerate}

\includegraphics{https://i.ibb.co/XXcjsZ6/Procedure13-1a.png}
\includegraphics{https://i.ibb.co/885Jm5y/Procedure13-1b.png}

\hypertarget{chi-square-test-of-independence}{%
\section{Chi-Square Test of Independence}\label{chi-square-test-of-independence}}

\hypertarget{example-race-and-preconceptional-use-of-folic-acid}{%
\subsection{Example: Race and Preconceptional Use of Folic Acid}\label{example-race-and-preconceptional-use-of-folic-acid}}

In 1992, the U.S. Public Health Service and the CDC recommended that women of childbearing age take 400 µg of folic acid daily to reduce neural tube birth defects.

To evaluate whether race is associated with folic acid use, a study was conducted involving 693 pregnant women who called a teratology information service.

The researchers recorded whether each woman took folic acid before pregnancy and their race (White, Black, Other). We want to test whether \textbf{folic acid use and race are independent}.

\hypertarget{step-1-observed-frequencies}{%
\subsubsection{Step 1: Observed Frequencies}\label{step-1-observed-frequencies}}

\begin{Shaded}
\begin{Highlighting}[]
\CommentTok{\# Contingency table of observed counts}
\NormalTok{M }\OtherTok{\textless{}{-}} \FunctionTok{as.table}\NormalTok{(}\FunctionTok{rbind}\NormalTok{(}\FunctionTok{c}\NormalTok{(}\DecValTok{260}\NormalTok{, }\DecValTok{299}\NormalTok{), }
                    \FunctionTok{c}\NormalTok{(}\DecValTok{15}\NormalTok{, }\DecValTok{41}\NormalTok{), }
                    \FunctionTok{c}\NormalTok{(}\DecValTok{7}\NormalTok{, }\DecValTok{14}\NormalTok{)))}
\FunctionTok{dimnames}\NormalTok{(M) }\OtherTok{\textless{}{-}} \FunctionTok{list}\NormalTok{(}\AttributeTok{Race =} \FunctionTok{c}\NormalTok{(}\StringTok{"White"}\NormalTok{, }\StringTok{"Black"}\NormalTok{, }\StringTok{"Other"}\NormalTok{),}
                    \AttributeTok{Folic =} \FunctionTok{c}\NormalTok{(}\StringTok{"Yes"}\NormalTok{, }\StringTok{"No"}\NormalTok{))}

\CommentTok{\# View the contingency table}
\NormalTok{M}
\end{Highlighting}
\end{Shaded}

\begin{verbatim}
##        Folic
## Race    Yes  No
##   White 260 299
##   Black  15  41
##   Other   7  14
\end{verbatim}

This table shows counts of women by race and whether they used folic acid before conception.

We will now test for independence using a chi-square test.

\begin{figure}
\centering
\includegraphics{https://i.ibb.co/M1HTpMm/table12-4-3.png}
\caption{Observed Frequencies Table}
\end{figure}

\hypertarget{step-2-hypotheses}{%
\subsubsection{Step 2: Hypotheses}\label{step-2-hypotheses}}

\begin{itemize}
\tightlist
\item
  \(H_0\): Race and folic acid use are independent.\\
\item
  \(H_a\): Race and folic acid use are associated.
\end{itemize}

\hypertarget{step-3-compute-expected-frequencies}{%
\subsubsection{Step 3: Compute Expected Frequencies}\label{step-3-compute-expected-frequencies}}

\begin{Shaded}
\begin{Highlighting}[]
\NormalTok{Xsq }\OtherTok{\textless{}{-}} \FunctionTok{chisq.test}\NormalTok{(M)   }\CommentTok{\# Run test}
\NormalTok{Xsq}\SpecialCharTok{$}\NormalTok{expected           }\CommentTok{\# View expected frequencies under independence}
\end{Highlighting}
\end{Shaded}

\begin{verbatim}
##        Folic
## Race           Yes        No
##   White 247.858491 311.14151
##   Black  24.830189  31.16981
##   Other   9.311321  11.68868
\end{verbatim}

Expected frequencies are calculated using:

\[
E_{r,c} = \frac{(row\ total) \times (column\ total)}{n}
\]

Example: For White women who used folic acid,
\[
E_{1,1} = \frac{559 \times 282}{636} = 247.86
\]

\hypertarget{step-4-test-statistic-and-degrees-of-freedom}{%
\subsubsection{Step 4: Test Statistic and Degrees of Freedom}\label{step-4-test-statistic-and-degrees-of-freedom}}

\begin{Shaded}
\begin{Highlighting}[]
\NormalTok{Xsq}\SpecialCharTok{$}\NormalTok{statistic   }\CommentTok{\# Chi{-}square value}
\end{Highlighting}
\end{Shaded}

\begin{verbatim}
## X-squared 
##  9.091262
\end{verbatim}

\begin{Shaded}
\begin{Highlighting}[]
\NormalTok{Xsq}\SpecialCharTok{$}\NormalTok{parameter   }\CommentTok{\# Degrees of freedom}
\end{Highlighting}
\end{Shaded}

\begin{verbatim}
## df 
##  2
\end{verbatim}

Test statistic: \(\chi^2 = 9.0896\)\\
Degrees of freedom: \((3 - 1)(2 - 1) = 2\)

\hypertarget{step-5-critical-value-2}{%
\subsubsection{Step 5: Critical Value}\label{step-5-critical-value-2}}

\begin{Shaded}
\begin{Highlighting}[]
\FunctionTok{qchisq}\NormalTok{(}\AttributeTok{p =} \FloatTok{0.95}\NormalTok{, }\AttributeTok{df =} \DecValTok{2}\NormalTok{)}
\end{Highlighting}
\end{Shaded}

\begin{verbatim}
## [1] 5.991465
\end{verbatim}

Result: \texttt{5.991}

\hypertarget{step-6-conclusion-2}{%
\subsubsection{Step 6: Conclusion}\label{step-6-conclusion-2}}

Since the test statistic (9.09) \textgreater{} critical value (5.991),\\
we \textbf{reject the null hypothesis}.

There is \textbf{significant evidence} of an association between race and folic acid use before pregnancy.

\begin{figure}
\centering
\includegraphics{https://i.ibb.co/TM6j2tb/table12-4-4.png}
\caption{Test Results Summary}
\end{figure}

\hypertarget{yatess-correction-for-22-tables}{%
\subsection{Yates's Correction for 2×2 Tables}\label{yatess-correction-for-22-tables}}

When using a 2×2 contingency table with small sample sizes, the chi-square test may \textbf{overestimate significance}. To correct this, Yates (1934) introduced a continuity correction.

The corrected formula is:

\[
\chi^2_{corrected} = \frac{n(|ad - bc| - 0.5n)^2}{(a + c)(b + d)(a + b)(c + d)}
\]

This correction reduces the absolute value of the difference between observed and expected, making the test more conservative.

\begin{Shaded}
\begin{Highlighting}[]
\CommentTok{\# With Yates correction (default = TRUE)}
\FunctionTok{chisq.test}\NormalTok{(M, }\AttributeTok{correct =} \ConstantTok{TRUE}\NormalTok{)}
\end{Highlighting}
\end{Shaded}

\begin{verbatim}
## 
##  Pearson's Chi-squared test
## 
## data:  M
## X-squared = 9.0913, df = 2, p-value = 0.01061
\end{verbatim}

\begin{Shaded}
\begin{Highlighting}[]
\CommentTok{\# Without correction}
\FunctionTok{chisq.test}\NormalTok{(M, }\AttributeTok{correct =} \ConstantTok{FALSE}\NormalTok{)}
\end{Highlighting}
\end{Shaded}

\begin{verbatim}
## 
##  Pearson's Chi-squared test
## 
## data:  M
## X-squared = 9.0913, df = 2, p-value = 0.01061
\end{verbatim}

In R, the correction is applied by default for 2×2 tables. For larger tables like 3×2 (our current case), the correction is not used.

Use the correction only when \textbf{both} conditions are met:
- It's a 2×2 table
- Some expected frequencies are small (e.g., \textless{} 5)

\hypertarget{step-by-step-procedure-summary-1}{%
\subsection{Step-by-Step Procedure (Summary)}\label{step-by-step-procedure-summary-1}}

Use the following procedure when conducting a chi-square test of independence:

\begin{enumerate}
\def\labelenumi{\arabic{enumi}.}
\tightlist
\item
  \textbf{State Hypotheses}

  \begin{itemize}
  \tightlist
  \item
    \(H_0\): The row and column variables are independent.
  \item
    \(H_a\): There is an association between them.
  \end{itemize}
\item
  \textbf{Create the Contingency Table}

  \begin{itemize}
  \tightlist
  \item
    Record observed counts across categories.
  \end{itemize}
\item
  \textbf{Calculate Expected Counts}

  \begin{itemize}
  \tightlist
  \item
    Use: \(E_{r,c} = \frac{(row\ total)(column\ total)}{n}\)
  \end{itemize}
\item
  \textbf{Compute Chi-Square Statistic}

  \begin{itemize}
  \tightlist
  \item
    Use: \(\chi^2 = \sum \frac{(O_i - E_i)^2}{E_i}\)
  \end{itemize}
\item
  \textbf{Degrees of Freedom}

  \begin{itemize}
  \tightlist
  \item
    \((\text{#rows} - 1)(\text{#columns} - 1)\)
  \end{itemize}
\item
  \textbf{Compare to Critical Value}

  \begin{itemize}
  \tightlist
  \item
    Use \texttt{qchisq(p,\ df)} to get critical value.
  \end{itemize}
\item
  \textbf{Make a Decision}

  \begin{itemize}
  \tightlist
  \item
    If \(\chi^2 >\) critical value → reject \(H_0\)
  \item
    Otherwise, fail to reject \(H_0\)
  \end{itemize}
\end{enumerate}

\includegraphics{https://i.ibb.co/nfh45Rm/Procedure13-2a.png}\\
\includegraphics{https://i.ibb.co/vV9MSbn/Procedure13-2b.png}

\hypertarget{chi-square-test-of-homogeneity}{%
\section{Chi-square Test of Homogeneity}\label{chi-square-test-of-homogeneity}}

\hypertarget{case-study-are-migraine-rates-different-in-people-with-narcolepsy}{%
\subsection{Case Study: Are Migraine Rates Different in People with Narcolepsy?}\label{case-study-are-migraine-rates-different-in-people-with-narcolepsy}}

Suppose a neurologist wants to know whether migraine headaches occur at different rates in people with narcolepsy compared to healthy individuals. A sample of 96 people with narcolepsy and 96 healthy controls were surveyed.

We want to test whether migraine frequency is the same across the two groups --- in other words, whether the groups are homogeneous regarding migraine experience.

\hypertarget{hypotheses}{%
\subsubsection{Hypotheses}\label{hypotheses}}

\begin{itemize}
\tightlist
\item
  \(H_0\): The two groups have the same migraine distribution (homogeneous).
\item
  \(H_A\): The groups differ in migraine rates (not homogeneous).
\end{itemize}

\hypertarget{observed-data}{%
\subsubsection{Observed Data}\label{observed-data}}

\begin{longtable}[]{@{}llll@{}}
\toprule\noalign{}
Group & Migraine & No Migraine & Total \\
\midrule\noalign{}
\endhead
\bottomrule\noalign{}
\endlastfoot
Narcolepsy & 35 & 61 & 96 \\
Healthy & 32 & 64 & 96 \\
\textbf{Total} & 67 & 125 & 192 \\
\end{longtable}

\begin{Shaded}
\begin{Highlighting}[]
\NormalTok{migraine\_tbl }\OtherTok{\textless{}{-}} \FunctionTok{as.table}\NormalTok{(}\FunctionTok{rbind}\NormalTok{(}\FunctionTok{c}\NormalTok{(}\DecValTok{35}\NormalTok{, }\DecValTok{61}\NormalTok{),}
                               \FunctionTok{c}\NormalTok{(}\DecValTok{32}\NormalTok{, }\DecValTok{64}\NormalTok{)))}
\FunctionTok{dimnames}\NormalTok{(migraine\_tbl) }\OtherTok{\textless{}{-}} \FunctionTok{list}\NormalTok{(}\AttributeTok{Group =} \FunctionTok{c}\NormalTok{(}\StringTok{"Narcolepsy"}\NormalTok{, }\StringTok{"Healthy"}\NormalTok{),}
                               \AttributeTok{Migraine =} \FunctionTok{c}\NormalTok{(}\StringTok{"Yes"}\NormalTok{, }\StringTok{"No"}\NormalTok{))}
\FunctionTok{chisq.test}\NormalTok{(migraine\_tbl, }\AttributeTok{correct =} \ConstantTok{FALSE}\NormalTok{)}
\end{Highlighting}
\end{Shaded}

\begin{verbatim}
## 
##  Pearson's Chi-squared test
## 
## data:  migraine_tbl
## X-squared = 0.20633, df = 1, p-value = 0.6497
\end{verbatim}

\hypertarget{result-interpretation}{%
\subsubsection{Result \& Interpretation}\label{result-interpretation}}

The Chi-square test returns a statistic of 0.126 with 1 degree of freedom. At the 5\% level, the cutoff is 3.841. Since 0.126 is much smaller than 3.841, we do not reject the null hypothesis. This means we have no strong evidence that migraine patterns are different in the two groups --- their distributions appear similar.

\hypertarget{example-comparing-two-independent-proportions-with-a-medical-test}{%
\subsection{Example: Comparing Two Independent Proportions with a Medical Test}\label{example-comparing-two-independent-proportions-with-a-medical-test}}

Let's say we're testing whether a new diagnostic test shows different positive rates across two hospitals.

\begin{itemize}
\tightlist
\item
  \textbf{Hospital A:} 100 patients tested, 60 positive\\
\item
  \textbf{Hospital B:} 120 patients tested, 48 positive
\end{itemize}

We want to compare whether the proportion of positives is the same in both places.

\hypertarget{hypotheses-1}{%
\subsubsection{Hypotheses}\label{hypotheses-1}}

\begin{itemize}
\tightlist
\item
  \(H_0\): The proportion of positives is the same in both hospitals.
\item
  \(H_A\): The proportions are different.
\end{itemize}

Instead of doing separate tests for each proportion, we compare them simultaneously.

\hypertarget{contingency-table}{%
\subsubsection{Contingency Table}\label{contingency-table}}

\begin{longtable}[]{@{}llll@{}}
\toprule\noalign{}
Hospital & Positive & Negative & Total \\
\midrule\noalign{}
\endhead
\bottomrule\noalign{}
\endlastfoot
A & 60 & 40 & 100 \\
B & 48 & 72 & 120 \\
\textbf{Total} & 108 & 112 & 220 \\
\end{longtable}

\begin{Shaded}
\begin{Highlighting}[]
\NormalTok{test\_tbl }\OtherTok{\textless{}{-}} \FunctionTok{as.table}\NormalTok{(}\FunctionTok{rbind}\NormalTok{(}\FunctionTok{c}\NormalTok{(}\DecValTok{60}\NormalTok{, }\DecValTok{40}\NormalTok{),}
                           \FunctionTok{c}\NormalTok{(}\DecValTok{48}\NormalTok{, }\DecValTok{72}\NormalTok{)))}
\FunctionTok{dimnames}\NormalTok{(test\_tbl) }\OtherTok{\textless{}{-}} \FunctionTok{list}\NormalTok{(}\AttributeTok{Hospital =} \FunctionTok{c}\NormalTok{(}\StringTok{"A"}\NormalTok{, }\StringTok{"B"}\NormalTok{),}
                           \AttributeTok{Result =} \FunctionTok{c}\NormalTok{(}\StringTok{"Positive"}\NormalTok{, }\StringTok{"Negative"}\NormalTok{))}

\FunctionTok{chisq.test}\NormalTok{(test\_tbl, }\AttributeTok{correct =} \ConstantTok{FALSE}\NormalTok{)}
\end{Highlighting}
\end{Shaded}

\begin{verbatim}
## 
##  Pearson's Chi-squared test
## 
## data:  test_tbl
## X-squared = 8.7302, df = 1, p-value = 0.00313
\end{verbatim}

\begin{Shaded}
\begin{Highlighting}[]
\FunctionTok{chisq.test}\NormalTok{(test\_tbl, }\AttributeTok{correct =} \ConstantTok{TRUE}\NormalTok{)}
\end{Highlighting}
\end{Shaded}

\begin{verbatim}
## 
##  Pearson's Chi-squared test with Yates' continuity correction
## 
## data:  test_tbl
## X-squared = 7.9482, df = 1, p-value = 0.004813
\end{verbatim}

\hypertarget{why-this-works}{%
\subsubsection{Why This Works}\label{why-this-works}}

The Chi-square test is comparing the difference between observed and expected counts under the assumption that the two proportions are the same. This test is mathematically equivalent to a z-test for proportions --- in fact, the Chi-square statistic is just the square of the z-score.

\hypertarget{interpretation-2}{%
\subsubsection{Interpretation}\label{interpretation-2}}

The test statistic is around 8.73. This is larger than the critical value (3.841), so we reject the null hypothesis. This means there's a statistically significant difference in test positivity between the two hospitals.

\hypertarget{mcnemars-test-rash-before-and-after-treatment-in-the-same-patients}{%
\subsection{McNemar's Test: Rash Before and After Treatment in the Same Patients}\label{mcnemars-test-rash-before-and-after-treatment-in-the-same-patients}}

Let's shift focus to a paired comparison. In this case, we measure the same group of patients before and after giving them a treatment for a skin rash.

\hypertarget{study-setup}{%
\subsubsection{Study Setup}\label{study-setup}}

\begin{itemize}
\tightlist
\item
  100 patients were checked at Time 1 (before treatment) and again at Time 2 (after treatment).
\item
  We only care about changes in rash status (yes/no), not severity.
\end{itemize}

\hypertarget{data}{%
\subsubsection{Data}\label{data}}

\begin{longtable}[]{@{}llll@{}}
\toprule\noalign{}
& Rash After & No Rash After & Total \\
\midrule\noalign{}
\endhead
\bottomrule\noalign{}
\endlastfoot
Rash Before & 38 & 12 & 50 \\
No Rash Before & 5 & 45 & 50 \\
\textbf{Total} & 43 & 57 & 100 \\
\end{longtable}

Here, we only focus on discordant pairs:
- 12 improved (rash → no rash)
- 5 worsened (no rash → rash)

McNemar's test uses:

\[
\chi^2 = \frac{(b - c)^2}{b + c}
\]

Where:
- \(b\) = count of patients who improved (12)
- \(c\) = count who worsened (5)

So:

\[
\chi^2 = \frac{(12 - 5)^2}{12 + 5} = \frac{49}{17} \approx 2.88
\]

\begin{Shaded}
\begin{Highlighting}[]
\NormalTok{rash\_tbl }\OtherTok{\textless{}{-}} \FunctionTok{as.table}\NormalTok{(}\FunctionTok{rbind}\NormalTok{(}\FunctionTok{c}\NormalTok{(}\DecValTok{38}\NormalTok{, }\DecValTok{12}\NormalTok{),}
                           \FunctionTok{c}\NormalTok{(}\DecValTok{5}\NormalTok{, }\DecValTok{45}\NormalTok{)))}
\FunctionTok{dimnames}\NormalTok{(rash\_tbl) }\OtherTok{\textless{}{-}} \FunctionTok{list}\NormalTok{(}\AttributeTok{Time1 =} \FunctionTok{c}\NormalTok{(}\StringTok{"Rash"}\NormalTok{, }\StringTok{"No Rash"}\NormalTok{),}
                           \AttributeTok{Time2 =} \FunctionTok{c}\NormalTok{(}\StringTok{"Rash"}\NormalTok{, }\StringTok{"No Rash"}\NormalTok{))}

\FunctionTok{mcnemar.test}\NormalTok{(rash\_tbl, }\AttributeTok{correct =} \ConstantTok{FALSE}\NormalTok{)}
\end{Highlighting}
\end{Shaded}

\begin{verbatim}
## 
##  McNemar's Chi-squared test
## 
## data:  rash_tbl
## McNemar's chi-squared = 2.8824, df = 1, p-value = 0.08956
\end{verbatim}

\begin{Shaded}
\begin{Highlighting}[]
\FunctionTok{mcnemar.test}\NormalTok{(rash\_tbl, }\AttributeTok{correct =} \ConstantTok{TRUE}\NormalTok{)}
\end{Highlighting}
\end{Shaded}

\begin{verbatim}
## 
##  McNemar's Chi-squared test with continuity correction
## 
## data:  rash_tbl
## McNemar's chi-squared = 2.1176, df = 1, p-value = 0.1456
\end{verbatim}

\hypertarget{interpretation-3}{%
\subsubsection{Interpretation}\label{interpretation-3}}

The test statistic is 2.88 with 1 degree of freedom. This is less than the cutoff of 3.841, so we do not reject the null hypothesis. That means the treatment didn't cause a statistically significant change in rash status --- though more people improved than worsened, the difference could just be due to chance in this sample.

\hypertarget{procedure-for-conducting-a-chi-square-test-of-homogeneity}{%
\subsection{Procedure for Conducting a Chi-square Test of Homogeneity}\label{procedure-for-conducting-a-chi-square-test-of-homogeneity}}

\includegraphics{https://i.ibb.co/wM9rQPZ/Procedure13-3a.png}\\
\includegraphics{https://i.ibb.co/rvQqGmq/Procedure13-3b.png}

\hypertarget{example-infection-type-vs-antibiotic-response}{%
\section{Example: Infection Type vs Antibiotic Response}\label{example-infection-type-vs-antibiotic-response}}

\begin{Shaded}
\begin{Highlighting}[]
\CommentTok{\# Biomedical Example: Infection Type vs Antibiotic Response}

\CommentTok{\# A hospital wants to assess whether the effectiveness of three antibiotics (A, B, C)}
\CommentTok{\# depends on the type of infection: respiratory, urinary, or skin infection.}
\CommentTok{\# The observed data shows the number of patients who responded positively}
\CommentTok{\# to each antibiotic by infection type.}

\CommentTok{\# Construct the contingency table:}
\NormalTok{infection\_data }\OtherTok{\textless{}{-}} \FunctionTok{matrix}\NormalTok{(}\FunctionTok{c}\NormalTok{(}\DecValTok{30}\NormalTok{, }\DecValTok{20}\NormalTok{, }\DecValTok{10}\NormalTok{,   }\CommentTok{\# Antibiotic A}
                           \DecValTok{25}\NormalTok{, }\DecValTok{30}\NormalTok{, }\DecValTok{15}\NormalTok{,   }\CommentTok{\# Antibiotic B}
                           \DecValTok{20}\NormalTok{, }\DecValTok{25}\NormalTok{, }\DecValTok{25}\NormalTok{),  }\CommentTok{\# Antibiotic C}
                         \AttributeTok{nrow =} \DecValTok{3}\NormalTok{, }\AttributeTok{byrow =} \ConstantTok{TRUE}\NormalTok{)}

\FunctionTok{colnames}\NormalTok{(infection\_data) }\OtherTok{\textless{}{-}} \FunctionTok{c}\NormalTok{(}\StringTok{"Respiratory"}\NormalTok{, }\StringTok{"Urinary"}\NormalTok{, }\StringTok{"Skin"}\NormalTok{)}
\FunctionTok{rownames}\NormalTok{(infection\_data) }\OtherTok{\textless{}{-}} \FunctionTok{c}\NormalTok{(}\StringTok{"Antibiotic A"}\NormalTok{, }\StringTok{"Antibiotic B"}\NormalTok{, }\StringTok{"Antibiotic C"}\NormalTok{)}

\NormalTok{infection\_data }\OtherTok{\textless{}{-}} \FunctionTok{as.table}\NormalTok{(infection\_data)}
\NormalTok{infection\_data}
\end{Highlighting}
\end{Shaded}

\begin{verbatim}
##              Respiratory Urinary Skin
## Antibiotic A          30      20   10
## Antibiotic B          25      30   15
## Antibiotic C          20      25   25
\end{verbatim}

\hypertarget{hypotheses-2}{%
\subsection{Hypotheses}\label{hypotheses-2}}

\begin{Shaded}
\begin{Highlighting}[]
\CommentTok{\# Null hypothesis (H0): The response to antibiotics is independent of infection type.}
\CommentTok{\# Alternative hypothesis (Ha): The response to antibiotics depends on the infection type.}
\end{Highlighting}
\end{Shaded}

\hypertarget{chi-square-test-of-independence-1}{%
\subsection{Chi-Square Test of Independence}\label{chi-square-test-of-independence-1}}

\begin{Shaded}
\begin{Highlighting}[]
\CommentTok{\# Perform the chi{-}square test of independence:}
\NormalTok{test\_result }\OtherTok{\textless{}{-}} \FunctionTok{chisq.test}\NormalTok{(infection\_data)}

\CommentTok{\# Print observed and expected counts:}
\NormalTok{test\_result}\SpecialCharTok{$}\NormalTok{observed}
\end{Highlighting}
\end{Shaded}

\begin{verbatim}
##              Respiratory Urinary Skin
## Antibiotic A          30      20   10
## Antibiotic B          25      30   15
## Antibiotic C          20      25   25
\end{verbatim}

\begin{Shaded}
\begin{Highlighting}[]
\NormalTok{test\_result}\SpecialCharTok{$}\NormalTok{expected}
\end{Highlighting}
\end{Shaded}

\begin{verbatim}
##              Respiratory Urinary Skin
## Antibiotic A       22.50   22.50 15.0
## Antibiotic B       26.25   26.25 17.5
## Antibiotic C       26.25   26.25 17.5
\end{verbatim}

\hypertarget{interpretation-4}{%
\subsection{Interpretation}\label{interpretation-4}}

\begin{Shaded}
\begin{Highlighting}[]
\CommentTok{\# Print test statistic, degrees of freedom, and p{-}value:}
\NormalTok{test\_result}
\end{Highlighting}
\end{Shaded}

\begin{verbatim}
## 
##  Pearson's Chi-squared test
## 
## data:  infection_data
## X-squared = 10.159, df = 4, p-value = 0.03784
\end{verbatim}

\begin{Shaded}
\begin{Highlighting}[]
\CommentTok{\# Interpretation:}
\CommentTok{\# The Chi{-}square test statistic is compared to a critical value from the Chi{-}square distribution,}
\CommentTok{\# or we use the p{-}value approach.}

\CommentTok{\# If p{-}value \textless{} 0.05, we reject the null hypothesis and conclude}
\CommentTok{\# that there is a significant association between infection type and antibiotic response.}
\end{Highlighting}
\end{Shaded}

\hypertarget{visualizing-the-results}{%
\subsection{Visualizing the Results}\label{visualizing-the-results}}

\begin{Shaded}
\begin{Highlighting}[]
\CommentTok{\# Create a mosaic plot to visualize the relationship:}
\FunctionTok{mosaicplot}\NormalTok{(infection\_data,}
           \AttributeTok{main =} \StringTok{"Antibiotic Response by Infection Type"}\NormalTok{,}
           \AttributeTok{color =} \ConstantTok{TRUE}\NormalTok{,}
           \AttributeTok{xlab =} \StringTok{"Infection Type"}\NormalTok{,}
           \AttributeTok{ylab =} \StringTok{"Antibiotic"}\NormalTok{,}
           \AttributeTok{las =} \DecValTok{1}\NormalTok{)}
\end{Highlighting}
\end{Shaded}

\includegraphics{StatsTB_files/figure-latex/unnamed-chunk-285-1.pdf}

\hypertarget{conclusion}{%
\subsection{Conclusion}\label{conclusion}}

Suppose the p-value from the test is 0.037.Since 0.037 \textless{} 0.05, we reject the null hypothesis.

Conclusion: There is a statistically significant association between infection type and antibiotic response. Certain antibiotics may be more effective for specific types of infections, and treatment plans should consider this interaction.

\hypertarget{one-way-anova}{%
\chapter{One-Way ANOVA}\label{one-way-anova}}

\begin{Shaded}
\begin{Highlighting}[]
\FunctionTok{library}\NormalTok{(IntroStats)}
\end{Highlighting}
\end{Shaded}

\hypertarget{introduction-to-anova}{%
\section{Introduction to ANOVA}\label{introduction-to-anova}}

The statistical technique known as Analysis of Variance (ANOVA) was primarily developed by the pioneering statistician R.A. Fisher. ANOVA plays a critical role in experimental data analysis, particularly when researchers aim to compare the means of three or more populations simultaneously.

When the goal is to compare the means of two populations, several methods are typically employed. A pooled t-test, used when the population variances are assumed equal (\(\sigma_1 = \sigma_2\)). An unpooled (Welch's) t-test, applied when the variances are not assumed equal (\(\sigma_1 \ne \sigma_2\)). A Wilcoxon Rank-Sum test, a nonparametric alternative when normality cannot be assumed for the populations.

However, when the comparison involves more than two population means, these pairwise methods are insufficient. Instead, ANOVA provides a framework to test the following hypotheses:

\[H_0: \mu_1 = \mu_2 = \cdots = \mu_k\]\\
\[H_a: \text{At least one } \mu_j \text{ differs from the others}.\]

In such cases, ANOVA helps determine whether any of the group means are statistically significantly different from each other, without inflating the Type I error rate that would result from conducting multiple pairwise comparisons.

\hypertarget{why-do-variances-matter-when-comparing-means}{%
\subsection{Why Do Variances Matter When Comparing Means?}\label{why-do-variances-matter-when-comparing-means}}

\begin{figure}
\centering
\includegraphics{https://i.ibb.co/p0ggLw7/ANOVA-illustration.png}
\caption{Figure. Analysis of variances is essential for comparing means. The top figure is unlikely to indicate difference in \(\mu_1\) and \(\mu_2\), because it is unclear whether the difference is due to the difference in population means (\(\mu_1, \mu_2\)) or to the variation within the populations.}
\end{figure}

When we compare the means of two groups, simply observing that the sample means are different isn't enough to claim a meaningful difference exists between the populations. The observed gap between \(\bar{x}_1\) and \(\bar{x}_2\) might reflect a true difference in the population means (\(\mu_1 \ne \mu_2\)), or it might just be due to natural variability within the data.

This idea is important in statistical inference: we want to know if the difference between the sample means is big enough, compared to the variability in the data, to suggest that it's not just due to random chance.If the variability within each group is high, even a moderate difference in sample means might not be convincing. But if the variability is low, that same difference might be strong evidence of a true effect.

This reasoning is captured in the pooled two-sample \emph{t}-test, where the test statistic is:
\[
t = \frac{\bar{x}_1 - \bar{x}_2}{s_p \sqrt{\frac{1}{n_1} + \frac{1}{n_2}}}
\]
Here:
- \(\bar{x}_1\) and \(\bar{x}_2\) are the sample means,\\
- \(s_p\) is the pooled standard deviation,\\
- \(n_1\) and \(n_2\) are the sample sizes.

The \emph{t}-value tells us how large the difference in sample means is \textbf{relative to} the pooled variation, adjusted for sample size. If this ratio is large, it indicates strong evidence that the population means are different --- and we would reject the null hypothesis. But if the ratio is small, the difference may just be due to random chance, and we fail to reject \(H_0\).

This is why variation within groups must always be taken into account when comparing means.

\hypertarget{key-assumptions-of-anova}{%
\section{Key Assumptions of ANOVA}\label{key-assumptions-of-anova}}

To ensure the validity of ANOVA results, the following assumptions must be satisfied:

\begin{enumerate}
\def\labelenumi{\arabic{enumi}.}
\tightlist
\item
  \textbf{Independence}

  \begin{itemize}
  \tightlist
  \item
    Observations within and across groups must be independent of one another.
  \item
    This is typically ensured through proper experimental design, such as random assignment.
  \end{itemize}
\item
  \textbf{Normality}

  \begin{itemize}
  \tightlist
  \item
    The distribution of the residuals (errors) within each group should be approximately normal.
  \item
    This assumption becomes less critical with larger sample sizes due to the Central Limit Theorem.
  \end{itemize}
\item
  \textbf{Homogeneity of Variance (Homoscedasticity)}

  \begin{itemize}
  \tightlist
  \item
    All groups should have approximately equal variances.
  \item
    This can be checked using tests like Levene's test or by examining residual plots.
  \end{itemize}
\end{enumerate}

Violations of these assumptions can affect the accuracy of p-values and the validity of conclusions drawn from ANOVA. When assumptions are not met, alternative methods like the Welch's ANOVA (for unequal variances) or nonparametric tests (e.g., Kruskal-Wallis test) may be more appropriate.

\hypertarget{one--way-anova-procedure}{%
\section{One- Way ANOVA Procedure}\label{one--way-anova-procedure}}

\begin{enumerate}
\def\labelenumi{\arabic{enumi}.}
\item
  \textbf{Description of Data}\\
  Identify the response (dependent) variable and the factor (independent variable) under consideration. Determine how many groups or treatment levels are being compared and ensure the data is appropriately structured (e.g., each observation is linked to a specific group).
\item
  \textbf{Assumptions}\\
  ANOVA requires several key assumptions:

  \begin{itemize}
  \tightlist
  \item
    \textbf{Independence}: Observations are independently and randomly sampled.
  \item
    \textbf{Normality}: The residuals (errors) in each group are approximately normally distributed.
  \item
    \textbf{Homogeneity of Variances}: The variances across the groups are roughly equal (also known as homoscedasticity). This can be checked using Levene's Test or Bartlett's Test.
  \end{itemize}
\item
  \textbf{Hypotheses}\\
  Let \(\mu_1, \mu_2, \dots, \mu_k\) represent the population means of the \(k\) groups.

  \begin{itemize}
  \tightlist
  \item
    Null hypothesis (\(H_0\)): \(\mu_1 = \mu_2 = \dots = \mu_k\) (all group means are equal)
  \item
    Alternative hypothesis (\(H_a\)): At least one group mean is different
  \end{itemize}
\item
  \textbf{Test Statistic}\\
  The ANOVA test statistic is the \textbf{F-ratio}:
  \[
  F = \frac{\text{Mean Square Between Groups (MSB)}}{\text{Mean Square Within Groups (MSW)}}
  \]
\item
  \textbf{Distribution of Test Statistic}\\
  Under the null hypothesis, the F-statistic follows an \textbf{F-distribution} with:

  \begin{itemize}
  \tightlist
  \item
    Degrees of freedom numerator: \(df_1 = k - 1\)
  \item
    Degrees of freedom denominator: \(df_2 = N - k\)\\
    where \(k\) is the number of groups and \(N\) is the total number of observations.
  \end{itemize}
\item
  \textbf{Decision Rule}

  \begin{itemize}
  \tightlist
  \item
    Choose a significance level \(\alpha\) (commonly 0.05).
  \item
    Reject \(H_0\) if the calculated F-statistic is greater than the critical value from the F-distribution, or if the \textbf{p-value \textless{} α}.
  \end{itemize}
\item
  \textbf{Calculation of Test Statistic}

  \begin{itemize}
  \tightlist
  \item
    Compute the group means and overall mean.
  \item
    Calculate the \textbf{Sum of Squares Between (SSB)} and \textbf{Sum of Squares Within (SSW)}.
  \item
    Compute:
    \[
    MSB = \frac{SSB}{k - 1}, \quad MSW = \frac{SSW}{N - k}
    \]
    \[
    F = \frac{MSB}{MSW}
    \]
  \end{itemize}
\item
  \textbf{Statistical Decision}\\
  Based on the F-statistic and the critical value (or p-value), decide whether or not to reject the null hypothesis.
\item
  \textbf{Conclusion}\\
  Interpret the results in the context of the problem. For example:\\
  \emph{``There is sufficient evidence at the 5\% significance level to conclude that at least one group mean is significantly different.''}
\item
  \textbf{Determination of p-value}\\
\end{enumerate}

\begin{itemize}
\tightlist
\item
  The p-value is the probability of observing an F-statistic as extreme as the one calculated, assuming \(H_0\) is true.
\item
  Use R (or software) to compute it:
\end{itemize}

\begin{verbatim}
     summary(aov(response ~ group, data = dataset))
\end{verbatim}

\begin{itemize}
\tightlist
\item
  If \textbf{p-value \textless{} α}, reject the null hypothesis.
\end{itemize}

\hypertarget{anova-table}{%
\section{ANOVA table}\label{anova-table}}

\hypertarget{one-way-anova-definition}{%
\subsection{One-Way ANOVA Definition:}\label{one-way-anova-definition}}

Suppose we want to compare the effectiveness of four different drugs. We randomly assign 16 patients into four groups, where each group receives one of the drugs. Since patients are assigned completely at random, this setup is known as a completely randomized design, and it is analyzed using a one-way ANOVA.

One-way ANOVA (Analysis of Variance) is the most basic type of ANOVA model, used when there is only one factor or variable of interest---typically a treatment group or category. It generalizes the pooled t-test to situations where there are more than two groups.

In this setup:
- There are \(k\) treatment groups.
- Each group \(j\) contains \(n_j\) observations.
- The outcome or response for the \(i\)th subject in group \(j\) is denoted as \(X_{ij}\).

The one-way ANOVA model can be written in the form of a linear model:

\[
X_{ij} = \mu + \tau_j + \epsilon_{ij} = \mu_j + \epsilon_{ij}
\]

Where:
- \(\mu\) is the grand mean, or the average across all groups.
- \(\tau_j\) is the treatment effect for group \(j\), which reflects how much the mean of group \(j\) (\(\mu_j\)) deviates from the grand mean (\(\mu\)), so \(\tau_j = \mu_j - \mu\).
- \(\epsilon_{ij}\) is the random error term, representing individual variation within each group.

This model assumes that:
- All errors \(\epsilon_{ij}\) are independent and normally distributed with mean 0 and constant variance.
- The only systematic difference between groups is due to the treatment effect \(\tau_j\).

The goal of one-way ANOVA is to test whether all group means are equal, which corresponds to the null hypothesis:

\[
H_0: \tau_1 = \tau_2 = \cdots = \tau_k = 0 \quad \text{(i.e., all group means are the same)}
\]

If the null hypothesis is rejected, we conclude that at least one group mean differs significantly from the others.

\hypertarget{description-of-data-the-anova-table}{%
\subsection{Description of Data \& The ANOVA table}\label{description-of-data-the-anova-table}}

\begin{figure}
\centering
\includegraphics{https://i.ibb.co/DRv99Ls/completely-randomized-design.png}
\caption{Figure 18.4.2a Completely randomized design}
\end{figure}

Let \(x_{ij}\) be the ith observation resulting from the jth treatment, where \(i= 1, 2, \cdots, n_j\).

\begin{figure}
\centering
\includegraphics{https://i.ibb.co/NnPFBsB/one-way-ANOVA.png}
\caption{Figure 18.4.2b Table of sample values for the completely randomized design}
\end{figure}

\begin{enumerate}
\def\labelenumi{\arabic{enumi}.}
\item
  Sample sum of the jth group:
  \[T_{.j} = \sum_{i=1}^{n_j} x_{ij}\]
\item
  Sample mean of the jth group
  \[\bar{x}_{.j} = T_{.j}/n_j\]
\item
  Grand sample sum
  \[T_{..}= \sum_{j=1}^k T_{.j} = \sum_{j=1}^k\sum_{i=1}^{n_j}x_{ij}\]
\item
  Grand sample mean\\
  \[\bar{x}_{..} = T_{..}/N \]
\item
  Total sum of squares
  \[SST = \sum_{j=1}^k\sum_{i=1}^{n_j}(x_{ij}-\bar{x}_{..})^2\]
\item
  Within groups sum of squares
  \[SSE = \sum_{j=1}^k\sum_{i=1}^{n_j}(x_{ij}-\bar{x}_{.j})^2\]
\item
  Between groups sum of squares
  \[SSTr = \sum_{j=1}^k n_j (\bar{x}_{.j}-\bar{x}_{..})^2\]
\end{enumerate}

\hypertarget{anova-identity}{%
\subsection{ANOVA Identity}\label{anova-identity}}

\begin{figure}
\centering
\includegraphics{https://i.ibb.co/d5h42vs/ANOVA-identity.png}
\caption{Figure 18.4.3 ANOVA Identity}
\end{figure}

\[SST = SSTr + SSE\]

Below is the ANOVA ttable for reference:

\begin{figure}
\centering
\includegraphics{https://i.ibb.co/hB9wC51/ANOVA-table.png}
\caption{Figure 18.4.3 ANOVA Table}
\end{figure}

\hypertarget{model-checking}{%
\section{Model Checking}\label{model-checking}}

Before we can trust the results from a one-way ANOVA, we need to check whether its core model assumptions hold. This process is called model checking. ANOVA relies on several key assumptions about the data structure and variability. If these are not met, the conclusions drawn from ANOVA may be invalid.

Here are the assumptions we aim to check:

\begin{enumerate}
\def\labelenumi{\arabic{enumi}.}
\item
  \textbf{Group Independence}\\
  The \(k\) treatment groups must be independent. That is, observations in one group should not influence those in another. This is typically ensured by proper randomization in the study design.
\item
  \textbf{Normality of the Response Variable}\\
  For each group, the response variable \(X_{ij}\) is assumed to follow a normal distribution:\\
  \[
  X_{ij} \sim N(\mu_j, \sigma^2) = N(\mu + \tau_j, \sigma^2)
  \]
\item
  \textbf{Equal Variances Across Groups (Homoscedasticity)}\\
  All treatment groups should have approximately the same variability. In other words,\\
  \[
  \sigma_1^2 = \cdots = \sigma_k^2 = \sigma^2
  \]
\item
  \textbf{Random and Normally Distributed Errors}\\
  The residuals (errors) from the model are assumed to be normally distributed and independent, with a mean of zero and constant variance:\\
  \[
  \epsilon_{ij} \sim N(0, \sigma^2), \quad \sum \tau_j = 0
  \]
\end{enumerate}

\hypertarget{visualizing-the-assumptions}{%
\subsection{Visualizing the Assumptions}\label{visualizing-the-assumptions}}

The following plot demonstrates a situation where three treatment groups have different means but equal standard deviations. This visual supports the equal variance assumption required by the ANOVA model --- which is something we look for when checking if the model fits the data.

\begin{Shaded}
\begin{Highlighting}[]
\NormalTok{x }\OtherTok{\textless{}{-}} \FunctionTok{seq}\NormalTok{(}\SpecialCharTok{{-}}\DecValTok{10}\NormalTok{, }\DecValTok{20}\NormalTok{, }\AttributeTok{by=}\FloatTok{0.01}\NormalTok{)}
\NormalTok{d1 }\OtherTok{\textless{}{-}} \FunctionTok{dnorm}\NormalTok{(x, }\AttributeTok{mean=}\DecValTok{3}\NormalTok{, }\AttributeTok{sd=}\DecValTok{2}\NormalTok{)}
\NormalTok{d2 }\OtherTok{\textless{}{-}} \FunctionTok{dnorm}\NormalTok{(x, }\AttributeTok{mean=}\DecValTok{6}\NormalTok{, }\AttributeTok{sd=}\DecValTok{2}\NormalTok{)}
\NormalTok{d3 }\OtherTok{\textless{}{-}} \FunctionTok{dnorm}\NormalTok{(x, }\AttributeTok{mean=}\DecValTok{9}\NormalTok{, }\AttributeTok{sd=}\DecValTok{2}\NormalTok{)}

\FunctionTok{plot}\NormalTok{(x, d1, }\AttributeTok{type =} \StringTok{"n"}\NormalTok{, }\AttributeTok{ylab=}\StringTok{""}\NormalTok{, }\AttributeTok{xlab=}\StringTok{""}\NormalTok{, }
     \AttributeTok{main=}\StringTok{"3 Populations with Same Standard Deviation (ANOVA Model Assumption)"}\NormalTok{)}
\FunctionTok{lines}\NormalTok{(x, d1, }\AttributeTok{col=}\StringTok{"green"}\NormalTok{, }\AttributeTok{lwd=}\DecValTok{3}\NormalTok{)}
\FunctionTok{lines}\NormalTok{(x, d2, }\AttributeTok{col=}\StringTok{"red"}\NormalTok{, }\AttributeTok{lwd=}\DecValTok{3}\NormalTok{)}
\FunctionTok{lines}\NormalTok{(x, d3, }\AttributeTok{col=}\StringTok{"blue"}\NormalTok{, }\AttributeTok{lwd=}\DecValTok{3}\NormalTok{)}
\end{Highlighting}
\end{Shaded}

\includegraphics{StatsTB_files/figure-latex/unnamed-chunk-287-1.pdf}

This figure shows that while each population has a different average, the spread is identical, which aligns with the equal variance assumption we check in model diagnostics.

\hypertarget{f-test-statistic}{%
\subsection{F-test statistic}\label{f-test-statistic}}

\[
\text{Variance Ratio (V.R.)} = \frac{\text{Among groups mean square}}{\text{Within groups mean square}} = \frac{MSTr}{MSE} = \frac{SSTr / (k-1) }{SSE / (N-k)}
\]
where
\[MSE = SSE / (N-k) = \frac{\sum_{j=1}^k\sum_{i=1}^{n_j}(x_{ij}-\bar{x}_{.j})^2}{k-1} \\
        MSTr = SSTr / (k-1) = \frac{\sum_{j=1}^k n_j (x_{.j}-\bar{x}_{..})^2}{k-1}\]

When \(H_0\) is true, \(V.R.\) tends to be 1 and the two mean squares are equal. When V.R. much greater than 1, the mean square difference observed among groups cannot be explained by the within groups mean square (i.e.~variability). So we tend to reject \(H_0\).

Under \(H_0\),\\
\[F = \frac{MSTr}{MSE} \sim F(k-1, N-k)\]

There are two types of degrees of freedom: numerator and denominator, in \(F\) distribution.

\hypertarget{summary-5}{%
\subsection{Summary}\label{summary-5}}

Although ANOVA assumes normality, equal variance, and independence, we never blindly accept these. Instead, we check them using diagnostic tools such as:

\begin{itemize}
\tightlist
\item
  Residual plots (random scatter around zero)
\item
  Q-Q plots (check for normality)
\item
  Levene's or Bartlett's test (for equal variance)
\end{itemize}

This model checking step helps ensure that the results of our F-test are trustworthy and based on valid model behavior.

\hypertarget{equal-variance-check-levenes-test}{%
\subsection{Equal Variance Check: Levene's Test}\label{equal-variance-check-levenes-test}}

One key assumption of ANOVA is that all treatment groups have the same variance. We check this using Levene's Test. If the p-value is small (typically less than 0.05), we reject the assumption of equal variances.

\begin{Shaded}
\begin{Highlighting}[]
\CommentTok{\# Load required package}
\FunctionTok{library}\NormalTok{(car)}

\CommentTok{\# Create sample data for 3 groups}
\NormalTok{group }\OtherTok{\textless{}{-}} \FunctionTok{factor}\NormalTok{(}\FunctionTok{rep}\NormalTok{(}\FunctionTok{c}\NormalTok{(}\StringTok{"A"}\NormalTok{, }\StringTok{"B"}\NormalTok{, }\StringTok{"C"}\NormalTok{), }\AttributeTok{each =} \DecValTok{10}\NormalTok{))}
\NormalTok{values }\OtherTok{\textless{}{-}} \FunctionTok{c}\NormalTok{(}
  \FunctionTok{rnorm}\NormalTok{(}\DecValTok{10}\NormalTok{, }\AttributeTok{mean =} \DecValTok{5}\NormalTok{, }\AttributeTok{sd =} \DecValTok{1}\NormalTok{),   }\CommentTok{\# Group A}
  \FunctionTok{rnorm}\NormalTok{(}\DecValTok{10}\NormalTok{, }\AttributeTok{mean =} \DecValTok{5}\NormalTok{, }\AttributeTok{sd =} \DecValTok{1}\NormalTok{),   }\CommentTok{\# Group B}
  \FunctionTok{rnorm}\NormalTok{(}\DecValTok{10}\NormalTok{, }\AttributeTok{mean =} \DecValTok{5}\NormalTok{, }\AttributeTok{sd =} \DecValTok{3}\NormalTok{)    }\CommentTok{\# Group C (larger variance)}
\NormalTok{)}

\CommentTok{\# Run Levene’s Test}
\FunctionTok{leveneTest}\NormalTok{(values }\SpecialCharTok{\textasciitilde{}}\NormalTok{ group)}
\end{Highlighting}
\end{Shaded}

\begin{verbatim}
## Levene's Test for Homogeneity of Variance (center = median)
##       Df F value Pr(>F)
## group  2  1.6474 0.2113
##       27
\end{verbatim}

\textbf{Interpretation:}\\
- Look at the p-value in the output.\\
- If p \textless{} 0.05, it suggests that variances are significantly different across groups → ANOVA assumption violated.\\
- If p ≥ 0.05, then equal variance assumption is reasonable.

\textbf{Formula for Levene's Test}:
Levene's Test works by comparing the absolute deviations of each observation from its group mean (or median):

\[
W = \frac{(N - k)}{(k - 1)} \cdot \frac{\sum_{j=1}^k n_j (Z_{.j} - Z_{..})^2}{\sum_{j=1}^k \sum_{i=1}^{n_j} (Z_{ij} - Z_{.j})^2}
\]

Where:
- \(Z_{ij} = |X_{ij} - \tilde{X}_j|\), the absolute deviation from the group mean or median\\
- \(Z_{.j}\) is the mean of deviations in group \(j\)\\
- \(Z_{..}\) is the grand mean of all deviations\\
- \(N\) is the total sample size, and \(k\) is the number of groups

This test statistic follows an approximate \textbf{F-distribution} with \(k - 1\) and \(N - k\) degrees of freedom.

\hypertarget{normality-of-residuals}{%
\subsection{Normality of Residuals}\label{normality-of-residuals}}

Another assumption is that the residuals (errors) from the ANOVA model follow a normal distribution. We use visual tools like Q-Q plots and histograms to check this.

\begin{Shaded}
\begin{Highlighting}[]
\CommentTok{\# Fit an ANOVA model}
\NormalTok{model }\OtherTok{\textless{}{-}} \FunctionTok{aov}\NormalTok{(values }\SpecialCharTok{\textasciitilde{}}\NormalTok{ group)}

\CommentTok{\# Get residuals from the model}
\NormalTok{res }\OtherTok{\textless{}{-}} \FunctionTok{residuals}\NormalTok{(model)}

\CommentTok{\# Q{-}Q plot to check normality}
\FunctionTok{qqnorm}\NormalTok{(res)}
\FunctionTok{qqline}\NormalTok{(res, }\AttributeTok{col =} \StringTok{"red"}\NormalTok{)}
\end{Highlighting}
\end{Shaded}

\includegraphics{StatsTB_files/figure-latex/unnamed-chunk-289-1.pdf}

\begin{Shaded}
\begin{Highlighting}[]
\CommentTok{\# Histogram of residuals}
\FunctionTok{hist}\NormalTok{(res, }\AttributeTok{main =} \StringTok{"Histogram of Residuals"}\NormalTok{, }\AttributeTok{col =} \StringTok{"lightblue"}\NormalTok{, }\AttributeTok{xlab =} \StringTok{"Residuals"}\NormalTok{)}
\end{Highlighting}
\end{Shaded}

\includegraphics{StatsTB_files/figure-latex/unnamed-chunk-289-2.pdf}

\textbf{Interpretation:}\\
- In the Q-Q plot, if points fall roughly along the straight line → residuals are close to normal.\\
- The histogram should look roughly bell-shaped if residuals are normal.\\
- If there is heavy skew or large departures from the Q-Q line, consider a transformation or a nonparametric method.

\hypertarget{f-test-in-anova}{%
\section{F-Test in ANOVA}\label{f-test-in-anova}}

The F-test is the main tool used in ANOVA to decide whether there are any statistically significant differences among group means. It compares the variability between groups (caused by the treatment effect) to the variability within groups (caused by random noise or natural differences).

\hypertarget{properties-of-the-f-distribution}{%
\subsection{Properties of the F Distribution}\label{properties-of-the-f-distribution}}

The F-distribution is the sampling distribution used to calculate the test statistic in ANOVA. Here are a few important things to know:

\begin{itemize}
\tightlist
\item
  The shape of the F-distribution is right-skewed, especially when sample sizes are small.
\item
  As both degrees of freedom (numerator and denominator) get larger, the distribution becomes less skewed and more bell-shaped.
\item
  The F-distribution is always positive because it's based on a ratio of variances.
\end{itemize}

\begin{Shaded}
\begin{Highlighting}[]
\CommentTok{\# Draw two F{-}distributions with different degrees of freedom}
\CommentTok{\# df1 = between{-}group df = k {-} 1}
\CommentTok{\# df2 = within{-}group df = N {-} k}

\FunctionTok{draw.F.dist}\NormalTok{(}\AttributeTok{df1 =} \DecValTok{3}\NormalTok{, }\AttributeTok{df2 =} \DecValTok{100}\NormalTok{)  }\CommentTok{\# smaller sample size}
\end{Highlighting}
\end{Shaded}

\includegraphics{StatsTB_files/figure-latex/unnamed-chunk-290-1.pdf}

\begin{Shaded}
\begin{Highlighting}[]
\FunctionTok{draw.F.dist}\NormalTok{(}\AttributeTok{df1 =} \DecValTok{3}\NormalTok{, }\AttributeTok{df2 =} \DecValTok{500}\NormalTok{)  }\CommentTok{\# larger sample size}
\end{Highlighting}
\end{Shaded}

\includegraphics{StatsTB_files/figure-latex/unnamed-chunk-290-2.pdf}

These plots show how the F-distribution changes with sample size. With more data, the curve smooths out and becomes less skewed.

\hypertarget{critical-value-and-p-value-in-hypothesis-testing}{%
\subsection{Critical Value and p-value in Hypothesis Testing}\label{critical-value-and-p-value-in-hypothesis-testing}}

In ANOVA, we calculate the F-statistic and compare it to a critical value from the F-distribution, or use a p-value to make our decision.

\hypertarget{critical-value}{%
\subsubsection{Critical Value:}\label{critical-value}}

The critical value is the cutoff point beyond which we would reject the null hypothesis (no difference between group means). For example, the 95th percentile of an \(F(3, 100)\) distribution is:

\begin{Shaded}
\begin{Highlighting}[]
\FunctionTok{qf}\NormalTok{(}\AttributeTok{p =} \FloatTok{0.95}\NormalTok{, }\AttributeTok{df1 =} \DecValTok{3}\NormalTok{, }\AttributeTok{df2 =} \DecValTok{100}\NormalTok{)}
\end{Highlighting}
\end{Shaded}

\begin{verbatim}
## [1] 2.695534
\end{verbatim}

If your observed F-statistic is greater than this critical value, the result is considered statistically significant at the 0.05 level.

\hypertarget{p-value-1}{%
\subsubsection{p-value:}\label{p-value-1}}

The p-value tells us how extreme our observed F-statistic is. It's calculated as the probability of getting an F-value as large or larger than what we observed, assuming the null hypothesis is true:

\[
p = P(F > F_0)
\]

Where \(F_0\) is the F-statistic from your ANOVA output.

A \textbf{small p-value (usually \textless{} 0.05)} means the observed differences between group means are unlikely to be due to chance alone --- we would reject the null hypothesis.

\hypertarget{summary-6}{%
\subsection{Summary}\label{summary-6}}

The F-test is the core of ANOVA. By comparing between-group and within-group variation using the F-distribution, we can test whether at least one group mean is significantly different. We use:
- Critical values from the F-distribution as decision cutoffs.
- p-values to measure how extreme the test result is.

Understanding the shape and logic of the F-distribution helps interpret ANOVA results correctly.

\hypertarget{simultaneous-confidence-intervals}{%
\section{Simultaneous Confidence Intervals}\label{simultaneous-confidence-intervals}}

Once we reject the global null hypothesis in ANOVA (i.e., we conclude that not all group means are equal), we often want to explore which specific groups are different from each other. This leads us to pairwise comparisons between all groups.

However, doing multiple comparisons increases the chance of making a Type I error --- finding a difference when there isn't one --- just by chance. To avoid this, we use simultaneous confidence intervals to control the overall error rate across all comparisons.

\hypertarget{what-does-simultaneous-mean}{%
\subsection{What Does ``Simultaneous'' Mean?}\label{what-does-simultaneous-mean}}

Instead of building one confidence interval at a time, simultaneous confidence intervals are created so that all intervals are valid together with a high probability --- usually 95\%.

If we're comparing \(k\) groups, there are \(\frac{k(k-1)}{2}\) possible pairwise comparisons. Each interval gives us a range of values for the difference between two group means.

The goal is to construct a family of confidence intervals:
\[
\{(L_1, U_1), (L_2, U_2), \dots, (L_k, U_k)\}
\]
so that the overall confidence level is at least \(1 - \alpha\), such as 95\%.

This is called family-wise confidence level, and it ensures we don't make too many false discoveries just by testing many group pairs.

\hypertarget{why-this-matters-in-anova}{%
\subsection{Why This Matters in ANOVA}\label{why-this-matters-in-anova}}

In the selenium example with 4 types of meat (VEN, SQU, RRB, NRB), the null hypothesis was:
\[
H_0: \mu_{VEN} = \mu_{SQU} = \mu_{RRB} = \mu_{NRB}
\]

After rejecting this null hypothesis using the F-test, we want to ask:
- Is VEN different from SQU?
- Is RRB different from NRB?
- And so on\ldots{}

There are 6 total pairwise comparisons. If we test them all at the usual 95\% level without adjustment, we inflate our chances of making a mistake. That's why we use simultaneous confidence intervals to keep the family-wise error rate under control.

\hypertarget{visualizing-pairwise-comparisons}{%
\subsection{Visualizing Pairwise Comparisons}\label{visualizing-pairwise-comparisons}}

Once we compute simultaneous confidence intervals, it's helpful to visualize them to better understand which group differences are statistically significant. These intervals show a range of likely values for the difference between two group means.

A key feature to look for is whether the interval includes 0:

\begin{itemize}
\tightlist
\item
  If it does not include 0, we conclude that the two group means are significantly different.
\item
  If it includes 0, we conclude that there is no significant difference between those two groups.
\end{itemize}

In ANOVA, one of the most commonly used methods for this is Tukey's Honestly Significant Difference (HSD) test, which provides both:
- Adjusted confidence intervals for each pairwise comparison, and
- A visual plot of those intervals.

We'll explore the Tukey's method and its output in detail in the next section.

\hypertarget{tukeys-hsd-multiple-comparisons}{%
\section{Tukey's HSD \& Multiple Comparisons}\label{tukeys-hsd-multiple-comparisons}}

Tukey's \textbf{Honestly significance difference} (HSD) method is frequently used in ANOVA analysis handling the multiplicity issue. In this method, the critical value for pairwise comparison \(HSD_{ij}\) and the simultaneous \((1-\alpha)100\%\) confidence interval for \(\mu_i - \mu_j\) are below.

\[
HSD_{ij} = q_{\alpha, k, N-k} \sqrt{\frac{MSE}{2}\left(\frac{1}{n_i}+\frac{1}{n_j}\right)}
\]
\[
\bar{x}_i - \bar{x}_j \pm q_{\alpha, k, N-k} \sqrt{\frac{MSE}{2}\left(\frac{1}{n_i}+\frac{1}{n_j}\right)}
\]

where \(\alpha\) is the significance level and \(q\) follows a distribution, called \textbf{studentized range statistic}.

The R code to find \(q_{\alpha}(\kappa, \nu)\) is below. For example, 4 groups, and total sample size \(N=144\) has 95\% quantile of studentized range distribution is 3.677.

\begin{Shaded}
\begin{Highlighting}[]
\FunctionTok{qtukey}\NormalTok{(}\DecValTok{1}\FloatTok{{-}0.05}\NormalTok{, }\AttributeTok{nmeans=}\DecValTok{4}\NormalTok{, }\AttributeTok{df=}\DecValTok{144{-}4}\NormalTok{)}
\end{Highlighting}
\end{Shaded}

\begin{verbatim}
## [1] 3.677176
\end{verbatim}

If the absolute difference between two sample means \(|\bar{x}_{.i}-\bar{x}_{.j}| > HSD_{ij}\), also equivalently if the above CI excludes 0, then conclude their difference is significant.

For example, for \(\alpha=0.05\), the pairwise comparison \(H_0: \mu_{VEN} = \mu_{RRB}\) has \(HSD = 8.68\), while their sample mean difference is \(3.208 < 8.68\), so do not reject \(H_0\) and we cannot conclude there is a significance difference.

For the null hypothesis \(H_0: \mu_{VEN} = \mu_{SQU}\), on the other hand, \(HSD = 10.04\) and the sample mean difference is \(17.37 > 10.04\), so their pairwise \(H_0\) is rejected.

Figure below shows the HSD simultaneous CIs for all 6 pairwise comparisons. All CIs except for RRB vs VEN exclude 0.

\hypertarget{tukey-plot-visualization}{%
\subsection{Tukey Plot Visualization}\label{tukey-plot-visualization}}

\begin{Shaded}
\begin{Highlighting}[]
\CommentTok{\# Create dataset manually}
\NormalTok{selenium\_data }\OtherTok{\textless{}{-}} \FunctionTok{data.frame}\NormalTok{(}
  \AttributeTok{mean =} \FunctionTok{c}\NormalTok{(}\FloatTok{18.5}\NormalTok{, }\FloatTok{21.2}\NormalTok{, }\FloatTok{22.7}\NormalTok{, }\FloatTok{25.3}\NormalTok{, }\FloatTok{19.8}\NormalTok{, }\FloatTok{24.1}\NormalTok{, }\FloatTok{17.4}\NormalTok{, }\FloatTok{23.3}\NormalTok{, }\FloatTok{26.5}\NormalTok{, }\FloatTok{20.1}\NormalTok{, }\FloatTok{21.6}\NormalTok{, }\FloatTok{24.9}\NormalTok{),}
  \AttributeTok{group =} \FunctionTok{factor}\NormalTok{(}\FunctionTok{c}\NormalTok{(}\StringTok{"VEN"}\NormalTok{, }\StringTok{"VEN"}\NormalTok{, }\StringTok{"VEN"}\NormalTok{, }\StringTok{"RRB"}\NormalTok{, }\StringTok{"RRB"}\NormalTok{, }\StringTok{"RRB"}\NormalTok{, }\StringTok{"SQU"}\NormalTok{, }\StringTok{"SQU"}\NormalTok{, }\StringTok{"SQU"}\NormalTok{, }\StringTok{"OTH"}\NormalTok{, }\StringTok{"OTH"}\NormalTok{, }\StringTok{"OTH"}\NormalTok{))}
\NormalTok{)}
\end{Highlighting}
\end{Shaded}

\begin{Shaded}
\begin{Highlighting}[]
\NormalTok{selenium }\OtherTok{\textless{}{-}} \FunctionTok{aov}\NormalTok{(mean }\SpecialCharTok{\textasciitilde{}}\NormalTok{ group, }\AttributeTok{data =}\NormalTok{ selenium\_data)}
\NormalTok{tukey }\OtherTok{\textless{}{-}} \FunctionTok{TukeyHSD}\NormalTok{(selenium, }\AttributeTok{ordered =} \ConstantTok{TRUE}\NormalTok{, }\AttributeTok{conf.level =} \FloatTok{0.95}\NormalTok{)}
\NormalTok{tukey}
\end{Highlighting}
\end{Shaded}

\begin{verbatim}
##   Tukey multiple comparisons of means
##     95% family-wise confidence level
##     factor levels have been ordered
## 
## Fit: aov(formula = mean ~ group, data = selenium_data)
## 
## $group
##              diff       lwr       upr     p adj
## OTH-VEN 1.4000000 -6.892519  9.692519 0.9465107
## SQU-VEN 1.6000000 -6.692519  9.892519 0.9235063
## RRB-VEN 2.2666667 -6.025852 10.559185 0.8174996
## SQU-OTH 0.2000000 -8.092519  8.492519 0.9998201
## RRB-OTH 0.8666667 -7.425852  9.159185 0.9861308
## RRB-SQU 0.6666667 -7.625852  8.959185 0.9935396
\end{verbatim}

\begin{Shaded}
\begin{Highlighting}[]
\FunctionTok{plot}\NormalTok{(tukey)}
\end{Highlighting}
\end{Shaded}

\includegraphics{StatsTB_files/figure-latex/unnamed-chunk-294-1.pdf}

\hypertarget{bonferroni-correction}{%
\subsection{Bonferroni Correction}\label{bonferroni-correction}}

Bonferroni Correction is a post-hoc method used after ANOVA to compare group means while controlling the overall Type I error rate. When performing multiple comparisons, the chance of finding a false positive increases. Bonferroni adjusts for this by dividing the significance level (alpha) by the number of comparisons.

For example, if alpha = 0.05 and there are 6 comparisons, each test is evaluated at 0.05 / 6 = 0.0083. This makes it harder to find significant results, reducing the risk of false positives. Bonferroni is easy to apply and works well when the number of comparisons is small, but it can be overly conservative when many tests are performed.

\begin{Shaded}
\begin{Highlighting}[]
\CommentTok{\# Bonferroni correction is used to control the familywise Type I error rate when making multiple comparisons.}
\CommentTok{\# Instead of using the standard significance level (e.g., 0.05) for each test,}
\CommentTok{\# Bonferroni adjusts the threshold by dividing alpha by the number of comparisons.}

\CommentTok{\# For example:}
\NormalTok{alpha }\OtherTok{\textless{}{-}} \FloatTok{0.05}
\NormalTok{k }\OtherTok{\textless{}{-}} \DecValTok{6}
\NormalTok{bonf\_alpha }\OtherTok{\textless{}{-}}\NormalTok{ alpha }\SpecialCharTok{/}\NormalTok{ k}
\NormalTok{bonf\_alpha}
\end{Highlighting}
\end{Shaded}

\begin{verbatim}
## [1] 0.008333333
\end{verbatim}

\begin{Shaded}
\begin{Highlighting}[]
\CommentTok{\# Biomedical Example:}
\CommentTok{\# A clinical study compares wound healing times (in days) across four treatments: A, B, C, and D.}

\NormalTok{healing\_data }\OtherTok{\textless{}{-}} \FunctionTok{data.frame}\NormalTok{(}
  \AttributeTok{time =} \FunctionTok{c}\NormalTok{(}\DecValTok{10}\NormalTok{, }\DecValTok{11}\NormalTok{, }\DecValTok{9}\NormalTok{, }\DecValTok{13}\NormalTok{, }\DecValTok{14}\NormalTok{, }\DecValTok{12}\NormalTok{, }\DecValTok{16}\NormalTok{, }\DecValTok{17}\NormalTok{, }\DecValTok{15}\NormalTok{, }\DecValTok{8}\NormalTok{, }\DecValTok{7}\NormalTok{, }\DecValTok{9}\NormalTok{, }\DecValTok{12}\NormalTok{, }\DecValTok{13}\NormalTok{, }\DecValTok{11}\NormalTok{, }\DecValTok{14}\NormalTok{),}
  \AttributeTok{treatment =} \FunctionTok{factor}\NormalTok{(}\FunctionTok{c}\NormalTok{(}\StringTok{"A"}\NormalTok{, }\StringTok{"A"}\NormalTok{, }\StringTok{"A"}\NormalTok{, }\StringTok{"B"}\NormalTok{, }\StringTok{"B"}\NormalTok{, }\StringTok{"B"}\NormalTok{, }\StringTok{"C"}\NormalTok{, }\StringTok{"C"}\NormalTok{, }\StringTok{"C"}\NormalTok{, }\StringTok{"D"}\NormalTok{, }\StringTok{"D"}\NormalTok{, }\StringTok{"D"}\NormalTok{, }\StringTok{"A"}\NormalTok{, }\StringTok{"B"}\NormalTok{, }\StringTok{"C"}\NormalTok{, }\StringTok{"D"}\NormalTok{))}
\NormalTok{)}
\end{Highlighting}
\end{Shaded}

\begin{Shaded}
\begin{Highlighting}[]
\CommentTok{\# Step 1: Perform ANOVA to test for any group differences}
\NormalTok{anova\_model }\OtherTok{\textless{}{-}} \FunctionTok{aov}\NormalTok{(time }\SpecialCharTok{\textasciitilde{}}\NormalTok{ treatment, }\AttributeTok{data =}\NormalTok{ healing\_data)}
\FunctionTok{summary}\NormalTok{(anova\_model)}
\end{Highlighting}
\end{Shaded}

\begin{verbatim}
##             Df Sum Sq Mean Sq F value Pr(>F)  
## treatment    3  68.19  22.729   4.806 0.0201 *
## Residuals   12  56.75   4.729                 
## ---
## Signif. codes:  0 '***' 0.001 '**' 0.01 '*' 0.05 '.' 0.1 ' ' 1
\end{verbatim}

\begin{Shaded}
\begin{Highlighting}[]
\CommentTok{\# Step 2: Apply Bonferroni{-}adjusted pairwise t{-}tests}
\FunctionTok{pairwise.t.test}\NormalTok{(healing\_data}\SpecialCharTok{$}\NormalTok{time, healing\_data}\SpecialCharTok{$}\NormalTok{treatment, }\AttributeTok{p.adjust.method =} \StringTok{"bonferroni"}\NormalTok{)}
\end{Highlighting}
\end{Shaded}

\begin{verbatim}
## 
##  Pairwise comparisons using t tests with pooled SD 
## 
## data:  healing_data$time and healing_data$treatment 
## 
##   A     B     C    
## B 0.780 -     -    
## C 0.103 1.000 -    
## D 1.000 0.252 0.031
## 
## P value adjustment method: bonferroni
\end{verbatim}

\begin{Shaded}
\begin{Highlighting}[]
\CommentTok{\# Interpretation:}
\CommentTok{\# Each pairwise comparison (e.g., A vs B, A vs C) is evaluated using an adjusted p{-}value.}
\CommentTok{\# If the adjusted p{-}value is below 0.05, the difference in healing times is statistically significant.}
\CommentTok{\# Bonferroni correction reduces the risk of false positives but makes significance harder to achieve.}
\end{Highlighting}
\end{Shaded}

\hypertarget{scheffuxe9s-test}{%
\subsection{Scheffé's Test}\label{scheffuxe9s-test}}

Scheffé's test is a post-hoc method used after running an ANOVA, especially when we want to compare more than just simple pairwise group differences. It is one of the most flexible methods because it allows for any contrast between groups, not just two at a time. However, this flexibility comes at a cost: Scheffé's test is more conservative, meaning it's harder to find statistically significant results.

Unlike Tukey or Bonferroni, Scheffé does not adjust alpha, but instead adjusts the critical F-value used to test group differences. This makes it ideal when researchers want to test custom hypotheses, like comparing the average of two groups against a third.

Below is a biomedical example using wound healing times.

\begin{Shaded}
\begin{Highlighting}[]
\CommentTok{\# Load the dataset: healing times (in days) for four treatments A, B, C, D}
\NormalTok{healing\_data }\OtherTok{\textless{}{-}} \FunctionTok{data.frame}\NormalTok{(}
  \AttributeTok{time =} \FunctionTok{c}\NormalTok{(}\DecValTok{10}\NormalTok{, }\DecValTok{11}\NormalTok{, }\DecValTok{9}\NormalTok{, }\DecValTok{12}\NormalTok{, }\DecValTok{13}\NormalTok{, }\DecValTok{14}\NormalTok{, }\DecValTok{16}\NormalTok{, }\DecValTok{15}\NormalTok{, }\DecValTok{17}\NormalTok{, }\DecValTok{8}\NormalTok{, }\DecValTok{7}\NormalTok{, }\DecValTok{9}\NormalTok{),}
  \AttributeTok{treatment =} \FunctionTok{factor}\NormalTok{(}\FunctionTok{c}\NormalTok{(}\StringTok{"A"}\NormalTok{, }\StringTok{"A"}\NormalTok{, }\StringTok{"A"}\NormalTok{, }\StringTok{"B"}\NormalTok{, }\StringTok{"B"}\NormalTok{, }\StringTok{"B"}\NormalTok{, }\StringTok{"C"}\NormalTok{, }\StringTok{"C"}\NormalTok{, }\StringTok{"C"}\NormalTok{, }\StringTok{"D"}\NormalTok{, }\StringTok{"D"}\NormalTok{, }\StringTok{"D"}\NormalTok{))}
\NormalTok{)}
\end{Highlighting}
\end{Shaded}

\begin{Shaded}
\begin{Highlighting}[]
\CommentTok{\# Step 1: Perform one{-}way ANOVA to check for overall group differences}
\NormalTok{anova\_model }\OtherTok{\textless{}{-}} \FunctionTok{aov}\NormalTok{(time }\SpecialCharTok{\textasciitilde{}}\NormalTok{ treatment, }\AttributeTok{data =}\NormalTok{ healing\_data)}
\FunctionTok{summary}\NormalTok{(anova\_model)}
\end{Highlighting}
\end{Shaded}

\begin{verbatim}
##             Df Sum Sq Mean Sq F value   Pr(>F)    
## treatment    3  110.2   36.75   36.75 5.01e-05 ***
## Residuals    8    8.0    1.00                     
## ---
## Signif. codes:  0 '***' 0.001 '**' 0.01 '*' 0.05 '.' 0.1 ' ' 1
\end{verbatim}

\begin{Shaded}
\begin{Highlighting}[]
\CommentTok{\# Step 2: Run Scheffé’s test using the agricolae package}
\ControlFlowTok{if}\NormalTok{ (}\SpecialCharTok{!}\FunctionTok{require}\NormalTok{(agricolae)) \{}
  \FunctionTok{install.packages}\NormalTok{(}\StringTok{"agricolae"}\NormalTok{)}
  \FunctionTok{library}\NormalTok{(agricolae)}
\NormalTok{\} }\ControlFlowTok{else}\NormalTok{ \{}
  \FunctionTok{library}\NormalTok{(agricolae)}
\NormalTok{\}}

\CommentTok{\# Apply the Scheffé test}
\NormalTok{scheffe\_result }\OtherTok{\textless{}{-}} \FunctionTok{scheffe.test}\NormalTok{(anova\_model, }\StringTok{"treatment"}\NormalTok{, }\AttributeTok{group =} \ConstantTok{TRUE}\NormalTok{)}
\FunctionTok{print}\NormalTok{(scheffe\_result)}
\end{Highlighting}
\end{Shaded}

\begin{verbatim}
## $statistics
##   MSerror Df        F  Mean       CV  Scheffe CriticalDifference
##         1  8 4.066181 11.75 8.510638 3.492641           2.851729
## 
## $parameters
##      test    name.t ntr alpha
##   Scheffe treatment   4  0.05
## 
## $means
##   time std r        se Min Max  Q25 Q50  Q75
## A   10   1 3 0.5773503   9  11  9.5  10 10.5
## B   13   1 3 0.5773503  12  14 12.5  13 13.5
## C   16   1 3 0.5773503  15  17 15.5  16 16.5
## D    8   1 3 0.5773503   7   9  7.5   8  8.5
## 
## $comparison
## NULL
## 
## $groups
##   time groups
## C   16      a
## B   13      b
## A   10      c
## D    8      c
## 
## attr(,"class")
## [1] "group"
\end{verbatim}

\begin{Shaded}
\begin{Highlighting}[]
\CommentTok{\# Interpretation:}
\CommentTok{\# The output shows which group means are statistically different based on Scheffé\textquotesingle{}s method.}
\CommentTok{\# If two treatments do not share a letter (e.g., A vs C), it means their healing times differ significantly.}
\CommentTok{\# Because Scheffé is conservative, any significant result is considered strong evidence of a real difference.}
\CommentTok{\# This test is especially useful when comparing more complex combinations of groups beyond simple pairs.}
\end{Highlighting}
\end{Shaded}

\hypertarget{biomedical-example-comparing-wound-healing-times-with-three-topical-treatments}{%
\subsection{Biomedical Example: Comparing Wound Healing Times with Three Topical Treatments}\label{biomedical-example-comparing-wound-healing-times-with-three-topical-treatments}}

In this example, a clinical study investigates how long it takes for wounds to heal using three different topical treatments: A, B, and C. The outcome variable is healing time in days.

We will first perform a one-way ANOVA to determine if there's an overall difference in healing time between the groups. If significant, we will follow up with three post-hoc tests: Tukey's HSD, Bonferroni Correction, and Scheffé's Test.

\begin{Shaded}
\begin{Highlighting}[]
\CommentTok{\# Step 1: Create the dataset}

\NormalTok{healing\_data }\OtherTok{\textless{}{-}} \FunctionTok{data.frame}\NormalTok{(}
  \AttributeTok{time =} \FunctionTok{c}\NormalTok{(}\DecValTok{10}\NormalTok{, }\DecValTok{11}\NormalTok{, }\DecValTok{9}\NormalTok{, }\DecValTok{12}\NormalTok{, }\DecValTok{13}\NormalTok{, }\DecValTok{14}\NormalTok{, }\DecValTok{16}\NormalTok{, }\DecValTok{15}\NormalTok{, }\DecValTok{17}\NormalTok{),}
  \AttributeTok{treatment =} \FunctionTok{factor}\NormalTok{(}\FunctionTok{c}\NormalTok{(}\StringTok{"A"}\NormalTok{, }\StringTok{"A"}\NormalTok{, }\StringTok{"A"}\NormalTok{, }\StringTok{"B"}\NormalTok{, }\StringTok{"B"}\NormalTok{, }\StringTok{"B"}\NormalTok{, }\StringTok{"C"}\NormalTok{, }\StringTok{"C"}\NormalTok{, }\StringTok{"C"}\NormalTok{))}
\NormalTok{)}
\end{Highlighting}
\end{Shaded}

This dataset includes wound healing times for 3 patients in each of the 3 treatment groups.
We will use ANOVA to test for a significant difference in average healing time across the treatments.

\begin{Shaded}
\begin{Highlighting}[]
\CommentTok{\# Step 2: Run one{-}way ANOVA}

\NormalTok{anova\_model }\OtherTok{\textless{}{-}} \FunctionTok{aov}\NormalTok{(time }\SpecialCharTok{\textasciitilde{}}\NormalTok{ treatment, }\AttributeTok{data =}\NormalTok{ healing\_data)}
\FunctionTok{summary}\NormalTok{(anova\_model)}
\end{Highlighting}
\end{Shaded}

\begin{verbatim}
##             Df Sum Sq Mean Sq F value Pr(>F)   
## treatment    2     54      27      27  0.001 **
## Residuals    6      6       1                  
## ---
## Signif. codes:  0 '***' 0.001 '**' 0.01 '*' 0.05 '.' 0.1 ' ' 1
\end{verbatim}

If the p-value from ANOVA is less than 0.05, we reject the null hypothesis and conclude that at least one group differs in mean healing time. We then proceed with post-hoc tests to find out which groups differ.

\begin{Shaded}
\begin{Highlighting}[]
\CommentTok{\# Step 3: Tukey\textquotesingle{}s HSD Test}

\NormalTok{tukey\_result }\OtherTok{\textless{}{-}} \FunctionTok{TukeyHSD}\NormalTok{(anova\_model)}
\NormalTok{tukey\_result}
\end{Highlighting}
\end{Shaded}

\begin{verbatim}
##   Tukey multiple comparisons of means
##     95% family-wise confidence level
## 
## Fit: aov(formula = time ~ treatment, data = healing_data)
## 
## $treatment
##     diff       lwr      upr     p adj
## B-A    3 0.4947644 5.505236 0.0242291
## C-A    6 3.4947644 8.505236 0.0007942
## C-B    3 0.4947644 5.505236 0.0242291
\end{verbatim}

Tukey's test compares all group pairs and adjusts for multiple comparisons. If the adjusted p-value is less than 0.05, the difference between those two groups is considered statistically significant.

\begin{Shaded}
\begin{Highlighting}[]
\CommentTok{\# Step 4: Bonferroni{-}adjusted pairwise t{-}tests}

\FunctionTok{pairwise.t.test}\NormalTok{(healing\_data}\SpecialCharTok{$}\NormalTok{time, healing\_data}\SpecialCharTok{$}\NormalTok{treatment, }\AttributeTok{p.adjust.method =} \StringTok{"bonferroni"}\NormalTok{)}
\end{Highlighting}
\end{Shaded}

\begin{verbatim}
## 
##  Pairwise comparisons using t tests with pooled SD 
## 
## data:  healing_data$time and healing_data$treatment 
## 
##   A       B      
## B 0.03121 -      
## C 0.00097 0.03121
## 
## P value adjustment method: bonferroni
\end{verbatim}

Bonferroni correction divides the alpha level (e.g., 0.05) by the number of comparisons.
This makes it harder to find significant differences, but it lowers the risk of false positives.

\begin{Shaded}
\begin{Highlighting}[]
\CommentTok{\# Step 5: Scheffé’s Test (using agricolae package)}

\ControlFlowTok{if}\NormalTok{ (}\SpecialCharTok{!}\FunctionTok{require}\NormalTok{(agricolae)) \{}
  \FunctionTok{install.packages}\NormalTok{(}\StringTok{"agricolae"}\NormalTok{)}
  \FunctionTok{library}\NormalTok{(agricolae)}
\NormalTok{\} }\ControlFlowTok{else}\NormalTok{ \{}
  \FunctionTok{library}\NormalTok{(agricolae)}
\NormalTok{\}}

\NormalTok{scheffe\_result }\OtherTok{\textless{}{-}} \FunctionTok{scheffe.test}\NormalTok{(anova\_model, }\StringTok{"treatment"}\NormalTok{, }\AttributeTok{group =} \ConstantTok{TRUE}\NormalTok{)}
\NormalTok{scheffe\_result}
\end{Highlighting}
\end{Shaded}

\begin{verbatim}
## $statistics
##   MSerror Df        F Mean       CV  Scheffe CriticalDifference
##         1  6 5.143253   13 7.692308 3.207258           2.618715
## 
## $parameters
##      test    name.t ntr alpha
##   Scheffe treatment   3  0.05
## 
## $means
##   time std r        se Min Max  Q25 Q50  Q75
## A   10   1 3 0.5773503   9  11  9.5  10 10.5
## B   13   1 3 0.5773503  12  14 12.5  13 13.5
## C   16   1 3 0.5773503  15  17 15.5  16 16.5
## 
## $comparison
## NULL
## 
## $groups
##   time groups
## C   16      a
## B   13      b
## A   10      c
## 
## attr(,"class")
## [1] "group"
\end{verbatim}

Scheffé's Test is a flexible post-hoc method that allows for all kinds of group contrasts, not just pairwise. It is the most conservative method, meaning it's the strictest when declaring significance.
If two groups do not share a letter group in the output, their means are significantly different.

Final Summary:

\begin{itemize}
\tightlist
\item
  Tukey: Great for all pairwise comparisons, balances Type I error and power.
\item
  Bonferroni: Very strict, good when few comparisons are made.
\item
  Scheffé: Most flexible, best for testing custom group contrasts, but least powerful.
\end{itemize}

Choose the test that fits the scenario accordingly.

\hypertarget{two-way-anova}{%
\chapter{Two-Way ANOVA}\label{two-way-anova}}

\begin{Shaded}
\begin{Highlighting}[]
\FunctionTok{library}\NormalTok{(IntroStats)}
\end{Highlighting}
\end{Shaded}

In this chapter, we explore several types of Two-way ANOVA designs that help us analyze the effects of two categorical variables on a continuous outcome. These include the Randomized Complete Block Design, designs with replication, repeated measures models, two-factor repeated measures, and factorial designs. We use biomedical examples and R code to illustrate how these techniques are applied in real-world data analysis.

\hypertarget{the-randomized-complete-block-design}{%
\section{The Randomized Complete Block Design}\label{the-randomized-complete-block-design}}

\hypertarget{overview}{%
\subsection{Overview}\label{overview}}

The Randomized Complete Block Design (RCBD) is a classic example of a two-way ANOVA where one factor (e.g., treatment) is the primary focus, and the second factor (the block, such as age group or patient group) is used to reduce variability. In this design, each treatment is applied exactly once within each block.

\begin{figure}
\centering
\includegraphics{https://i.ibb.co/4g7364s/2-ANOVA-RCBD.png}
\caption{Two-way ANOVA Design}
\end{figure}

\hypertarget{anova-table-structure}{%
\subsection{ANOVA Table Structure}\label{anova-table-structure}}

Each cell in the design represents one observation---no replicates. The two-way ANOVA table breaks the variation into components from the block effect, treatment effect, and residual (error).

\begin{figure}
\centering
\includegraphics{https://i.ibb.co/m51w5yT/2-ANOVA-RCBD-table.png}
\caption{Two-way ANOVA Table}
\end{figure}

\hypertarget{biomedical-example-teaching-method-and-age-group}{%
\subsection{Biomedical Example: Teaching Method and Age Group}\label{biomedical-example-teaching-method-and-age-group}}

Suppose a physical therapist wants to compare the effectiveness of three teaching methods for training patients on a medical device. Since age may impact learning, five age groups are used as blocks. One patient per method is assigned within each age group.

\begin{Shaded}
\begin{Highlighting}[]
\FunctionTok{library}\NormalTok{(knitr)}
\NormalTok{data }\OtherTok{\textless{}{-}} \FunctionTok{data.frame}\NormalTok{(}
  \AttributeTok{Age =} \FunctionTok{c}\NormalTok{(}\StringTok{"Under 20"}\NormalTok{, }\StringTok{"20{-}29"}\NormalTok{, }\StringTok{"30{-}39"}\NormalTok{, }\StringTok{"40{-}49"}\NormalTok{, }\StringTok{"50 and over"}\NormalTok{),}
  \AttributeTok{Method\_A =} \FunctionTok{c}\NormalTok{(}\DecValTok{7}\NormalTok{, }\DecValTok{8}\NormalTok{, }\DecValTok{9}\NormalTok{, }\DecValTok{10}\NormalTok{, }\DecValTok{11}\NormalTok{),}
  \AttributeTok{Method\_B =} \FunctionTok{c}\NormalTok{(}\DecValTok{9}\NormalTok{, }\DecValTok{9}\NormalTok{, }\DecValTok{9}\NormalTok{, }\DecValTok{9}\NormalTok{, }\DecValTok{12}\NormalTok{),}
  \AttributeTok{Method\_C =} \FunctionTok{c}\NormalTok{(}\DecValTok{10}\NormalTok{, }\DecValTok{10}\NormalTok{, }\DecValTok{12}\NormalTok{, }\DecValTok{12}\NormalTok{, }\DecValTok{14}\NormalTok{)}
\NormalTok{)}
\FunctionTok{kable}\NormalTok{(data, }\AttributeTok{caption =} \StringTok{"Time (in minutes) to learn device by age group and method"}\NormalTok{)}
\end{Highlighting}
\end{Shaded}

\begin{table}

\caption{\label{tab:unnamed-chunk-310}Time (in minutes) to learn device by age group and method}
\centering
\begin{tabular}[t]{l|r|r|r}
\hline
Age & Method\_A & Method\_B & Method\_C\\
\hline
Under 20 & 7 & 9 & 10\\
\hline
20-29 & 8 & 9 & 10\\
\hline
30-39 & 9 & 9 & 12\\
\hline
40-49 & 10 & 9 & 12\\
\hline
50 and over & 11 & 12 & 14\\
\hline
\end{tabular}
\end{table}

The statistical model is:

\[x_{ij} = \mu + \beta_i + \tau_j + \epsilon_{ij}\]

Where:
- \(\mu\) is the overall mean
- \(\beta_i\) is the effect of the \(i\)-th block (age group)
- \(\tau_j\) is the effect of the \(j\)-th method
- \(\epsilon_{ij}\) is the random error for patient \(ij\)

We test:
- \(H_0\): All methods have the same mean effect.
- \(H_a\): At least one method differs.

This design helps control for age-related variability while isolating the effect of teaching method.

\hypertarget{fitting-the-model-in-r}{%
\subsection{Fitting the Model in R}\label{fitting-the-model-in-r}}

\begin{Shaded}
\begin{Highlighting}[]
\NormalTok{x }\OtherTok{\textless{}{-}} \FunctionTok{c}\NormalTok{(}\DecValTok{7}\NormalTok{,}\DecValTok{8}\NormalTok{,}\DecValTok{9}\NormalTok{,}\DecValTok{10}\NormalTok{,}\DecValTok{11}\NormalTok{, }\DecValTok{9}\NormalTok{,}\DecValTok{9}\NormalTok{,}\DecValTok{9}\NormalTok{,}\DecValTok{9}\NormalTok{,}\DecValTok{12}\NormalTok{, }\DecValTok{10}\NormalTok{,}\DecValTok{10}\NormalTok{,}\DecValTok{12}\NormalTok{,}\DecValTok{12}\NormalTok{,}\DecValTok{14}\NormalTok{)}
\NormalTok{age }\OtherTok{\textless{}{-}} \FunctionTok{rep}\NormalTok{(}\FunctionTok{c}\NormalTok{(}\StringTok{"Under 20"}\NormalTok{, }\StringTok{"20{-}29"}\NormalTok{, }\StringTok{"30{-}39"}\NormalTok{, }\StringTok{"40{-}49"}\NormalTok{, }\StringTok{"50 and over"}\NormalTok{), }\DecValTok{3}\NormalTok{)}
\NormalTok{method }\OtherTok{\textless{}{-}} \FunctionTok{rep}\NormalTok{(}\FunctionTok{c}\NormalTok{(}\StringTok{"A"}\NormalTok{, }\StringTok{"B"}\NormalTok{, }\StringTok{"C"}\NormalTok{), }\AttributeTok{each=}\DecValTok{5}\NormalTok{)}
\NormalTok{mydata }\OtherTok{\textless{}{-}} \FunctionTok{data.frame}\NormalTok{(}\AttributeTok{x =}\NormalTok{ x, }\AttributeTok{age =}\NormalTok{ age, }\AttributeTok{method =}\NormalTok{ method)}
\NormalTok{block.anova }\OtherTok{\textless{}{-}} \FunctionTok{aov}\NormalTok{(x }\SpecialCharTok{\textasciitilde{}}\NormalTok{ age }\SpecialCharTok{+}\NormalTok{ method, }\AttributeTok{data =}\NormalTok{ mydata)}
\FunctionTok{summary}\NormalTok{(block.anova)}
\end{Highlighting}
\end{Shaded}

\begin{verbatim}
##             Df Sum Sq Mean Sq F value   Pr(>F)    
## age          4 24.933   6.233   14.38 0.001002 ** 
## method       2 18.533   9.267   21.39 0.000617 ***
## Residuals    8  3.467   0.433                     
## ---
## Signif. codes:  0 '***' 0.001 '**' 0.01 '*' 0.05 '.' 0.1 ' ' 1
\end{verbatim}

\hypertarget{model-diagnostics}{%
\subsection{Model Diagnostics}\label{model-diagnostics}}

We assess residuals to validate ANOVA assumptions.

\begin{Shaded}
\begin{Highlighting}[]
\FunctionTok{plot}\NormalTok{(block.anova)}
\end{Highlighting}
\end{Shaded}

\includegraphics{StatsTB_files/figure-latex/unnamed-chunk-312-1.pdf} \includegraphics{StatsTB_files/figure-latex/unnamed-chunk-312-2.pdf} \includegraphics{StatsTB_files/figure-latex/unnamed-chunk-312-3.pdf} \includegraphics{StatsTB_files/figure-latex/unnamed-chunk-312-4.pdf}

From the diagnostic plots, the residuals appear approximately normally distributed with constant variance, satisfying model assumptions.

\hypertarget{tukey-hsd-pairwise-comparison}{%
\subsection{Tukey HSD Pairwise Comparison}\label{tukey-hsd-pairwise-comparison}}

To explore pairwise differences between methods:

\begin{Shaded}
\begin{Highlighting}[]
\NormalTok{tukey }\OtherTok{\textless{}{-}} \FunctionTok{TukeyHSD}\NormalTok{(block.anova, }\AttributeTok{which=}\StringTok{"method"}\NormalTok{, }\AttributeTok{ordered=}\ConstantTok{TRUE}\NormalTok{, }\AttributeTok{conf.level =} \FloatTok{0.95}\NormalTok{)}
\NormalTok{tukey}
\end{Highlighting}
\end{Shaded}

\begin{verbatim}
##   Tukey multiple comparisons of means
##     95% family-wise confidence level
##     factor levels have been ordered
## 
## Fit: aov(formula = x ~ age + method, data = mydata)
## 
## $method
##     diff        lwr      upr     p adj
## B-A  0.6 -0.5896489 1.789649 0.3666717
## C-A  2.6  1.4103511 3.789649 0.0006358
## C-B  2.0  0.8103511 3.189649 0.0034083
\end{verbatim}

\begin{Shaded}
\begin{Highlighting}[]
\FunctionTok{plot}\NormalTok{(tukey)}
\end{Highlighting}
\end{Shaded}

\includegraphics{StatsTB_files/figure-latex/unnamed-chunk-313-1.pdf}

This comparison shows which methods differ significantly after adjusting for multiple testing.

\hypertarget{additional-biomedical-example-pain-management-protocols}{%
\subsection{Additional Biomedical Example: Pain Management Protocols}\label{additional-biomedical-example-pain-management-protocols}}

Imagine comparing three postoperative pain management protocols: standard care, oral medication, and nerve block. Patients are grouped by type of surgery (e.g., knee, hip, shoulder), and one patient per treatment is assigned per surgery type.

This is also a RCBD where surgery type is the block and treatment is the factor of interest. The analysis follows the same steps and structure described above.

\hypertarget{the-randomized-complete-block-design-with-replicates}{%
\section{The Randomized Complete Block Design with Replicates}\label{the-randomized-complete-block-design-with-replicates}}

\hypertarget{overview-1}{%
\subsection{Overview}\label{overview-1}}

In this version of RCBD, multiple patients are assigned to each treatment-block combination, allowing us to evaluate not only main effects but also interaction effects.

\begin{figure}
\centering
\includegraphics{https://i.ibb.co/xFZhHKN/2-ANOVA-RCBD-replicates.png}
\caption{RCBD with Replicates}
\end{figure}

\hypertarget{anova-table-with-interaction}{%
\subsection{ANOVA Table with Interaction}\label{anova-table-with-interaction}}

\begin{figure}
\centering
\includegraphics{https://i.ibb.co/gDxVq0S/2-ANOVA-RCBD-replicates-table.png}
\caption{RCBD Replicate Table}
\end{figure}

The addition of replication introduces a block-by-treatment interaction term. If significant, it implies treatment effects vary by block.

You would use a similar R formula but include an interaction term like \texttt{aov(response\ \textasciitilde{}\ block\ *\ treatment)}.

\hypertarget{the-repeated-measures-design}{%
\section{The Repeated Measures Design}\label{the-repeated-measures-design}}

\hypertarget{overview-2}{%
\subsection{Overview}\label{overview-2}}

Repeated measures designs involve taking multiple measurements on the same subject. For example, physical function scores may be recorded at baseline, 1 month, 3 months, and 6 months. Since these repeated observations are on the same individuals, they are correlated and should not be treated as independent.

\hypertarget{data-visualization}{%
\subsection{Data Visualization}\label{data-visualization}}

\begin{Shaded}
\begin{Highlighting}[]
\NormalTok{x }\OtherTok{\textless{}{-}} \FunctionTok{c}\NormalTok{(}\DecValTok{80}\NormalTok{,}\DecValTok{95}\NormalTok{,}\DecValTok{65}\NormalTok{,}\DecValTok{50}\NormalTok{,}\DecValTok{60}\NormalTok{,}\DecValTok{70}\NormalTok{,}\DecValTok{80}\NormalTok{,}\DecValTok{70}\NormalTok{,}\DecValTok{80}\NormalTok{,}\DecValTok{65}\NormalTok{,}\DecValTok{60}\NormalTok{,}\DecValTok{50}\NormalTok{,}\DecValTok{50}\NormalTok{,}\DecValTok{85}\NormalTok{,}\DecValTok{50}\NormalTok{,}\DecValTok{15}\NormalTok{,}\DecValTok{10}\NormalTok{,}\DecValTok{80}\NormalTok{,}
       \DecValTok{60}\NormalTok{,}\DecValTok{90}\NormalTok{,}\DecValTok{55}\NormalTok{,}\DecValTok{45}\NormalTok{,}\DecValTok{75}\NormalTok{,}\DecValTok{70}\NormalTok{,}\DecValTok{80}\NormalTok{,}\DecValTok{60}\NormalTok{,}\DecValTok{80}\NormalTok{,}\DecValTok{30}\NormalTok{,}\DecValTok{70}\NormalTok{,}\DecValTok{50}\NormalTok{,}\DecValTok{65}\NormalTok{,}\DecValTok{45}\NormalTok{,}\DecValTok{65}\NormalTok{,}\DecValTok{30}\NormalTok{,}\DecValTok{15}\NormalTok{,}\DecValTok{85}\NormalTok{,}
       \DecValTok{95}\NormalTok{,}\DecValTok{95}\NormalTok{,}\DecValTok{50}\NormalTok{,}\DecValTok{70}\NormalTok{,}\DecValTok{80}\NormalTok{,}\DecValTok{75}\NormalTok{,}\DecValTok{85}\NormalTok{,}\DecValTok{75}\NormalTok{,}\DecValTok{70}\NormalTok{,}\DecValTok{45}\NormalTok{,}\DecValTok{95}\NormalTok{,}\DecValTok{70}\NormalTok{,}\DecValTok{80}\NormalTok{,}\DecValTok{85}\NormalTok{,}\DecValTok{90}\NormalTok{,}\DecValTok{20}\NormalTok{,}\DecValTok{55}\NormalTok{,}\DecValTok{90}\NormalTok{,}
       \DecValTok{100}\NormalTok{,}\DecValTok{95}\NormalTok{,}\DecValTok{45}\NormalTok{,}\DecValTok{70}\NormalTok{,}\DecValTok{85}\NormalTok{,}\DecValTok{70}\NormalTok{,}\DecValTok{80}\NormalTok{,}\DecValTok{65}\NormalTok{,}\DecValTok{65}\NormalTok{,}\DecValTok{60}\NormalTok{,}\DecValTok{80}\NormalTok{,}\DecValTok{60}\NormalTok{,}\DecValTok{65}\NormalTok{,}\DecValTok{80}\NormalTok{,}\DecValTok{70}\NormalTok{,}\DecValTok{25}\NormalTok{,}\DecValTok{75}\NormalTok{,}\DecValTok{70}\NormalTok{)}
\NormalTok{subj }\OtherTok{\textless{}{-}} \FunctionTok{factor}\NormalTok{(}\FunctionTok{rep}\NormalTok{(}\DecValTok{1}\SpecialCharTok{:}\DecValTok{18}\NormalTok{, }\DecValTok{4}\NormalTok{))}
\NormalTok{time }\OtherTok{\textless{}{-}} \FunctionTok{factor}\NormalTok{(}\FunctionTok{rep}\NormalTok{(}\FunctionTok{c}\NormalTok{(}\StringTok{"baseline"}\NormalTok{, }\StringTok{"M1"}\NormalTok{, }\StringTok{"M3"}\NormalTok{, }\StringTok{"M6"}\NormalTok{), }\AttributeTok{each=}\DecValTok{18}\NormalTok{))}
\NormalTok{data }\OtherTok{\textless{}{-}} \FunctionTok{data.frame}\NormalTok{(subj, time, x)}
\FunctionTok{library}\NormalTok{(ggpubr)}
\NormalTok{bxp }\OtherTok{\textless{}{-}} \FunctionTok{ggboxplot}\NormalTok{(data, }\AttributeTok{x =} \StringTok{"time"}\NormalTok{, }\AttributeTok{y =} \StringTok{"x"}\NormalTok{, }\AttributeTok{add =} \StringTok{"point"}\NormalTok{)}
\NormalTok{bxp}
\end{Highlighting}
\end{Shaded}

\includegraphics{StatsTB_files/figure-latex/unnamed-chunk-314-1.pdf}

\hypertarget{fit-the-repeated-measures-anova}{%
\subsection{Fit the Repeated Measures ANOVA}\label{fit-the-repeated-measures-anova}}

\begin{Shaded}
\begin{Highlighting}[]
\NormalTok{repeated.anova }\OtherTok{\textless{}{-}} \FunctionTok{aov}\NormalTok{(x }\SpecialCharTok{\textasciitilde{}}\NormalTok{ time }\SpecialCharTok{+} \FunctionTok{Error}\NormalTok{(subj}\SpecialCharTok{/}\NormalTok{time), }\AttributeTok{data =}\NormalTok{ data)}
\FunctionTok{summary}\NormalTok{(repeated.anova)}
\end{Highlighting}
\end{Shaded}

\begin{verbatim}
## 
## Error: subj
##           Df Sum Sq Mean Sq F value Pr(>F)
## Residuals 17  20238    1190               
## 
## Error: subj:time
##           Df Sum Sq Mean Sq F value  Pr(>F)   
## time       3   2396   798.6   5.501 0.00237 **
## Residuals 51   7404   145.2                   
## ---
## Signif. codes:  0 '***' 0.001 '**' 0.01 '*' 0.05 '.' 0.1 ' ' 1
\end{verbatim}

\hypertarget{model-diagnostics-and-normality-checks}{%
\subsection{Model Diagnostics and Normality Checks}\label{model-diagnostics-and-normality-checks}}

\begin{Shaded}
\begin{Highlighting}[]
\FunctionTok{library}\NormalTok{(rstatix)}
\NormalTok{data }\SpecialCharTok{\%\textgreater{}\%} \FunctionTok{group\_by}\NormalTok{(time) }\SpecialCharTok{\%\textgreater{}\%} \FunctionTok{shapiro\_test}\NormalTok{(x)}
\end{Highlighting}
\end{Shaded}

\begin{verbatim}
## # A tibble: 4 x 4
##   time     variable statistic      p
##   <fct>    <chr>        <dbl>  <dbl>
## 1 baseline x            0.900 0.0568
## 2 M1       x            0.960 0.611 
## 3 M3       x            0.884 0.0303
## 4 M6       x            0.935 0.235
\end{verbatim}

\begin{Shaded}
\begin{Highlighting}[]
\FunctionTok{ggqqplot}\NormalTok{(data, }\StringTok{"x"}\NormalTok{, }\AttributeTok{facet.by =} \StringTok{"time"}\NormalTok{)}
\end{Highlighting}
\end{Shaded}

\includegraphics{StatsTB_files/figure-latex/unnamed-chunk-316-1.pdf}

If Shapiro-Wilk p-values are \textgreater{} 0.05 and QQ-plots are approximately straight, assumptions are met.

\hypertarget{adjusted-models-with-anova_test}{%
\subsection{\texorpdfstring{Adjusted Models with \texttt{anova\_test()}}{Adjusted Models with anova\_test()}}\label{adjusted-models-with-anova_test}}

\begin{Shaded}
\begin{Highlighting}[]
\NormalTok{res.aov }\OtherTok{\textless{}{-}} \FunctionTok{anova\_test}\NormalTok{(}\AttributeTok{data =}\NormalTok{ data, }\AttributeTok{dv =}\NormalTok{ x, }\AttributeTok{wid =}\NormalTok{ subj, }\AttributeTok{within =}\NormalTok{ time)}
\FunctionTok{get\_anova\_table}\NormalTok{(res.aov)}
\end{Highlighting}
\end{Shaded}

\begin{verbatim}
## ANOVA Table (type III tests)
## 
##   Effect  DFn   DFd     F     p p<.05  ges
## 1   time 2.22 37.68 5.501 0.006     * 0.08
\end{verbatim}

\hypertarget{pairwise-comparisons-multiple-corrections}{%
\subsection{Pairwise Comparisons (Multiple Corrections)}\label{pairwise-comparisons-multiple-corrections}}

\begin{Shaded}
\begin{Highlighting}[]
\CommentTok{\# Bonferroni}
\NormalTok{pwc }\OtherTok{\textless{}{-}}\NormalTok{ data }\SpecialCharTok{\%\textgreater{}\%} \FunctionTok{pairwise\_t\_test}\NormalTok{(x }\SpecialCharTok{\textasciitilde{}}\NormalTok{ time, }\AttributeTok{paired =} \ConstantTok{TRUE}\NormalTok{, }\AttributeTok{p.adjust.method =} \StringTok{"bonferroni"}\NormalTok{)}
\NormalTok{pwc}
\end{Highlighting}
\end{Shaded}

\begin{verbatim}
## # A tibble: 6 x 10
##   .y.   group1   group2    n1    n2 statistic    df     p p.adj p.adj.signif
## * <chr> <chr>    <chr>  <int> <int>     <dbl> <dbl> <dbl> <dbl> <chr>       
## 1 x     baseline M1        18    18     0.658    17 0.519 1     ns          
## 2 x     baseline M3        18    18    -2.72     17 0.015 0.088 ns          
## 3 x     baseline M6        18    18    -1.75     17 0.099 0.592 ns          
## 4 x     M1       M3        18    18    -3.86     17 0.001 0.007 **          
## 5 x     M1       M6        18    18    -2.22     17 0.04  0.24  ns          
## 6 x     M3       M6        18    18     1.40     17 0.18  1     ns
\end{verbatim}

Repeat using \texttt{hommel}, \texttt{hochberg} if desired.

\hypertarget{visualization-of-p-values}{%
\subsection{Visualization of P-values}\label{visualization-of-p-values}}

\begin{Shaded}
\begin{Highlighting}[]
\NormalTok{pwc }\OtherTok{\textless{}{-}}\NormalTok{ pwc }\SpecialCharTok{\%\textgreater{}\%} \FunctionTok{add\_xy\_position}\NormalTok{(}\AttributeTok{x =} \StringTok{"time"}\NormalTok{)}
\NormalTok{bxp }\SpecialCharTok{+} \FunctionTok{stat\_pvalue\_manual}\NormalTok{(pwc) }\SpecialCharTok{+}
  \FunctionTok{labs}\NormalTok{(}\AttributeTok{subtitle =} \FunctionTok{get\_test\_label}\NormalTok{(res.aov, }\AttributeTok{detailed =} \ConstantTok{TRUE}\NormalTok{),}
       \AttributeTok{caption =} \FunctionTok{get\_pwc\_label}\NormalTok{(pwc))}
\end{Highlighting}
\end{Shaded}

\includegraphics{StatsTB_files/figure-latex/unnamed-chunk-319-1.pdf}

\hypertarget{two-factor-repeated-measures-design}{%
\section{Two-Factor Repeated Measures Design}\label{two-factor-repeated-measures-design}}

\hypertarget{overview-3}{%
\subsection{Overview}\label{overview-3}}

This design examines the effect of \textbf{two within-subjects factors} on a continuous response variable. A common biomedical application is evaluating how a treatment (e.g., new drug vs placebo) affects subjects \textbf{over multiple time points}. Both factors --- treatment and time --- are repeated measures here.

\hypertarget{setup-and-visualization}{%
\subsection{Setup and Visualization}\label{setup-and-visualization}}

Imagine a study where 18 subjects are randomly assigned to either a \textbf{new drug} or a \textbf{placebo}. Their physical function is measured at \textbf{baseline}, \textbf{month 1}, \textbf{month 3}, and \textbf{month 6}. We want to see:
- Does physical function differ over time?
- Is there a treatment effect?
- Is there a \textbf{time × treatment interaction}?

\begin{Shaded}
\begin{Highlighting}[]
\NormalTok{treatment }\OtherTok{\textless{}{-}} \FunctionTok{rep}\NormalTok{(}\StringTok{"New Drug"}\NormalTok{, }\FunctionTok{length}\NormalTok{(subj))}
\NormalTok{treatment[subj }\SpecialCharTok{\%in\%} \DecValTok{1}\SpecialCharTok{:}\DecValTok{9}\NormalTok{] }\OtherTok{\textless{}{-}} \StringTok{"Control"}
\NormalTok{data}\SpecialCharTok{$}\NormalTok{treatment }\OtherTok{\textless{}{-}} \FunctionTok{factor}\NormalTok{(treatment)}
\end{Highlighting}
\end{Shaded}

We add the treatment assignment and visualize scores by time and treatment:

\begin{Shaded}
\begin{Highlighting}[]
\NormalTok{bxp }\OtherTok{\textless{}{-}} \FunctionTok{ggboxplot}\NormalTok{(}
\NormalTok{  data, }\AttributeTok{x =} \StringTok{"time"}\NormalTok{, }\AttributeTok{y =} \StringTok{"x"}\NormalTok{,}
  \AttributeTok{color =} \StringTok{"treatment"}\NormalTok{, }\AttributeTok{palette =} \StringTok{"jco"}
\NormalTok{)}
\NormalTok{bxp}
\end{Highlighting}
\end{Shaded}

\includegraphics{StatsTB_files/figure-latex/unnamed-chunk-321-1.pdf}

\hypertarget{normality-check}{%
\subsection{Normality Check}\label{normality-check}}

Before running ANOVA, we check that residuals are approximately normal:

\begin{Shaded}
\begin{Highlighting}[]
\NormalTok{data }\SpecialCharTok{\%\textgreater{}\%}
  \FunctionTok{group\_by}\NormalTok{(treatment, time) }\SpecialCharTok{\%\textgreater{}\%}
  \FunctionTok{shapiro\_test}\NormalTok{(x)}
\end{Highlighting}
\end{Shaded}

\begin{verbatim}
## # A tibble: 8 x 5
##   time     treatment variable statistic      p
##   <fct>    <fct>     <chr>        <dbl>  <dbl>
## 1 baseline Control   x            0.969 0.883 
## 2 M1       Control   x            0.969 0.887 
## 3 M3       Control   x            0.930 0.486 
## 4 M6       Control   x            0.961 0.811 
## 5 baseline New Drug  x            0.911 0.322 
## 6 M1       New Drug  x            0.963 0.832 
## 7 M3       New Drug  x            0.883 0.169 
## 8 M6       New Drug  x            0.796 0.0182
\end{verbatim}

\begin{Shaded}
\begin{Highlighting}[]
\FunctionTok{ggqqplot}\NormalTok{(data, }\StringTok{"x"}\NormalTok{, }\AttributeTok{ggtheme =} \FunctionTok{theme\_bw}\NormalTok{()) }\SpecialCharTok{+}
  \FunctionTok{facet\_grid}\NormalTok{(time }\SpecialCharTok{\textasciitilde{}}\NormalTok{ treatment, }\AttributeTok{labeller =} \StringTok{"label\_both"}\NormalTok{)}
\end{Highlighting}
\end{Shaded}

\includegraphics{StatsTB_files/figure-latex/unnamed-chunk-322-1.pdf}

If most p-values \textgreater{} 0.05 and QQ-plots show straight lines, normality is assumed.

\hypertarget{two-factor-repeated-measures-anova}{%
\subsection{Two-Factor Repeated Measures ANOVA}\label{two-factor-repeated-measures-anova}}

Now we fit the full model with interaction:

\begin{Shaded}
\begin{Highlighting}[]
\NormalTok{anova2 }\OtherTok{\textless{}{-}} \FunctionTok{aov}\NormalTok{(x }\SpecialCharTok{\textasciitilde{}}\NormalTok{ treatment }\SpecialCharTok{+}\NormalTok{ time }\SpecialCharTok{+}\NormalTok{ treatment}\SpecialCharTok{:}\NormalTok{time }\SpecialCharTok{+} \FunctionTok{Error}\NormalTok{(subj}\SpecialCharTok{/}\NormalTok{time), }\AttributeTok{data =}\NormalTok{ data)}
\FunctionTok{summary}\NormalTok{(anova2)}
\end{Highlighting}
\end{Shaded}

\begin{verbatim}
## 
## Error: subj
##           Df Sum Sq Mean Sq F value Pr(>F)  
## treatment  1   3472    3472   3.314 0.0875 .
## Residuals 16  16765    1048                 
## ---
## Signif. codes:  0 '***' 0.001 '**' 0.01 '*' 0.05 '.' 0.1 ' ' 1
## 
## Error: subj:time
##                Df Sum Sq Mean Sq F value  Pr(>F)   
## time            3   2396   798.6   5.581 0.00229 **
## treatment:time  3    536   178.7   1.249 0.30246   
## Residuals      48   6868   143.1                   
## ---
## Signif. codes:  0 '***' 0.001 '**' 0.01 '*' 0.05 '.' 0.1 ' ' 1
\end{verbatim}

Interpretation:
- \textbf{Treatment effect} tells if there's an overall difference between drug and placebo.
- \textbf{Time effect} shows whether average scores change over time.
- \textbf{Interaction} tells if treatment effect depends on time.

If the interaction is \textbf{not significant}, we may drop it:

\begin{Shaded}
\begin{Highlighting}[]
\NormalTok{anova3 }\OtherTok{\textless{}{-}} \FunctionTok{aov}\NormalTok{(x }\SpecialCharTok{\textasciitilde{}}\NormalTok{ treatment }\SpecialCharTok{+}\NormalTok{ time }\SpecialCharTok{+} \FunctionTok{Error}\NormalTok{(subj}\SpecialCharTok{/}\NormalTok{time), }\AttributeTok{data =}\NormalTok{ data)}
\FunctionTok{summary}\NormalTok{(anova3)}
\end{Highlighting}
\end{Shaded}

\begin{verbatim}
## 
## Error: subj
##           Df Sum Sq Mean Sq F value Pr(>F)  
## treatment  1   3472    3472   3.314 0.0875 .
## Residuals 16  16765    1048                 
## ---
## Signif. codes:  0 '***' 0.001 '**' 0.01 '*' 0.05 '.' 0.1 ' ' 1
## 
## Error: subj:time
##           Df Sum Sq Mean Sq F value  Pr(>F)   
## time       3   2396   798.6   5.501 0.00237 **
## Residuals 51   7404   145.2                   
## ---
## Signif. codes:  0 '***' 0.001 '**' 0.01 '*' 0.05 '.' 0.1 ' ' 1
\end{verbatim}

\hypertarget{pairwise-comparisons}{%
\subsection{Pairwise Comparisons}\label{pairwise-comparisons}}

If no interaction, we can interpret the main effects separately.

\hypertarget{comparing-treatments-unpaired}{%
\subsubsection{Comparing Treatments (Unpaired)}\label{comparing-treatments-unpaired}}

\begin{Shaded}
\begin{Highlighting}[]
\NormalTok{pwc.treatment }\OtherTok{\textless{}{-}}\NormalTok{ data }\SpecialCharTok{\%\textgreater{}\%}
  \FunctionTok{pairwise\_t\_test}\NormalTok{(x }\SpecialCharTok{\textasciitilde{}}\NormalTok{ treatment, }\AttributeTok{paired =} \ConstantTok{FALSE}\NormalTok{, }\AttributeTok{p.adjust.method =} \StringTok{"bonferroni"}\NormalTok{)}
\NormalTok{pwc.treatment}
\end{Highlighting}
\end{Shaded}

\begin{verbatim}
## # A tibble: 1 x 9
##   .y.   group1  group2      n1    n2       p p.signif   p.adj p.adj.signif
## * <chr> <chr>   <chr>    <int> <int>   <dbl> <chr>      <dbl> <chr>       
## 1 x     Control New Drug    36    36 0.00348 **       0.00348 **
\end{verbatim}

\hypertarget{comparing-time-points-paired}{%
\subsubsection{Comparing Time Points (Paired)}\label{comparing-time-points-paired}}

\begin{Shaded}
\begin{Highlighting}[]
\NormalTok{pwc.time }\OtherTok{\textless{}{-}}\NormalTok{ data }\SpecialCharTok{\%\textgreater{}\%}
  \FunctionTok{pairwise\_t\_test}\NormalTok{(x }\SpecialCharTok{\textasciitilde{}}\NormalTok{ time, }\AttributeTok{paired =} \ConstantTok{TRUE}\NormalTok{, }\AttributeTok{p.adjust.method =} \StringTok{"bonferroni"}\NormalTok{)}
\NormalTok{pwc.time}
\end{Highlighting}
\end{Shaded}

\begin{verbatim}
## # A tibble: 6 x 10
##   .y.   group1   group2    n1    n2 statistic    df     p p.adj p.adj.signif
## * <chr> <chr>    <chr>  <int> <int>     <dbl> <dbl> <dbl> <dbl> <chr>       
## 1 x     baseline M1        18    18     0.658    17 0.519 1     ns          
## 2 x     baseline M3        18    18    -2.72     17 0.015 0.088 ns          
## 3 x     baseline M6        18    18    -1.75     17 0.099 0.592 ns          
## 4 x     M1       M3        18    18    -3.86     17 0.001 0.007 **          
## 5 x     M1       M6        18    18    -2.22     17 0.04  0.24  ns          
## 6 x     M3       M6        18    18     1.40     17 0.18  1     ns
\end{verbatim}

\hypertarget{visualization-with-adjusted-p-values}{%
\subsection{Visualization with Adjusted P-values}\label{visualization-with-adjusted-p-values}}

\begin{Shaded}
\begin{Highlighting}[]
\NormalTok{pwc.time }\OtherTok{\textless{}{-}}\NormalTok{ pwc.time }\SpecialCharTok{\%\textgreater{}\%} \FunctionTok{add\_xy\_position}\NormalTok{(}\AttributeTok{x =} \StringTok{"time"}\NormalTok{)}
\NormalTok{bxp }\SpecialCharTok{+}
  \FunctionTok{stat\_pvalue\_manual}\NormalTok{(pwc.time, }\AttributeTok{tip.length =} \DecValTok{0}\NormalTok{, }\AttributeTok{hide.ns =} \ConstantTok{TRUE}\NormalTok{) }\SpecialCharTok{+}
  \FunctionTok{labs}\NormalTok{(}
    \AttributeTok{subtitle =} \FunctionTok{get\_test\_label}\NormalTok{(res.aov, }\AttributeTok{detailed =} \ConstantTok{TRUE}\NormalTok{),}
    \AttributeTok{caption =} \FunctionTok{get\_pwc\_label}\NormalTok{(pwc.time)}
\NormalTok{  )}
\end{Highlighting}
\end{Shaded}

\includegraphics{StatsTB_files/figure-latex/unnamed-chunk-327-1.pdf}

\hypertarget{advanced-pairwise-analysis}{%
\subsection{Advanced Pairwise Analysis}\label{advanced-pairwise-analysis}}

To evaluate \textbf{treatment differences at each time point}:

\begin{Shaded}
\begin{Highlighting}[]
\NormalTok{one.way }\OtherTok{\textless{}{-}}\NormalTok{ data }\SpecialCharTok{\%\textgreater{}\%}
  \FunctionTok{group\_by}\NormalTok{(time) }\SpecialCharTok{\%\textgreater{}\%}
  \FunctionTok{anova\_test}\NormalTok{(}\AttributeTok{dv =}\NormalTok{ x, }\AttributeTok{wid =}\NormalTok{ subj, }\AttributeTok{between =}\NormalTok{ treatment) }\SpecialCharTok{\%\textgreater{}\%}
  \FunctionTok{get\_anova\_table}\NormalTok{() }\SpecialCharTok{\%\textgreater{}\%}
  \FunctionTok{adjust\_pvalue}\NormalTok{(}\AttributeTok{method =} \StringTok{"bonferroni"}\NormalTok{)}
\NormalTok{one.way}
\end{Highlighting}
\end{Shaded}

\begin{verbatim}
## # A tibble: 4 x 9
##   time     Effect      DFn   DFd     F     p `p<.05`   ges p.adj
##   <fct>    <chr>     <dbl> <dbl> <dbl> <dbl> <chr>   <dbl> <dbl>
## 1 baseline treatment     1    16 4.57  0.048 "*"     0.222 0.192
## 2 M1       treatment     1    16 3.94  0.065 ""      0.198 0.26 
## 3 M3       treatment     1    16 0.564 0.463 ""      0.034 1    
## 4 M6       treatment     1    16 1.58  0.226 ""      0.09  0.904
\end{verbatim}

And to compare \textbf{time effects within each treatment group}:

\begin{Shaded}
\begin{Highlighting}[]
\NormalTok{one.way2 }\OtherTok{\textless{}{-}}\NormalTok{ data }\SpecialCharTok{\%\textgreater{}\%}
  \FunctionTok{group\_by}\NormalTok{(treatment) }\SpecialCharTok{\%\textgreater{}\%}
  \FunctionTok{anova\_test}\NormalTok{(}\AttributeTok{dv =}\NormalTok{ x, }\AttributeTok{wid =}\NormalTok{ subj, }\AttributeTok{within =}\NormalTok{ time) }\SpecialCharTok{\%\textgreater{}\%}
  \FunctionTok{get\_anova\_table}\NormalTok{() }\SpecialCharTok{\%\textgreater{}\%}
  \FunctionTok{adjust\_pvalue}\NormalTok{(}\AttributeTok{method =} \StringTok{"hommel"}\NormalTok{)}
\NormalTok{one.way2}
\end{Highlighting}
\end{Shaded}

\begin{verbatim}
## # A tibble: 2 x 9
##   treatment Effect   DFn   DFd     F     p `p<.05`   ges p.adj
##   <fct>     <chr>  <dbl> <dbl> <dbl> <dbl> <chr>   <dbl> <dbl>
## 1 Control   time     1.5  12.0  1.58 0.243 ""      0.054 0.243
## 2 New Drug  time     3    24    4.18 0.016 "*"     0.132 0.032
\end{verbatim}

\hypertarget{the-factorial-design}{%
\section{The Factorial Design}\label{the-factorial-design}}

\hypertarget{overview-4}{%
\subsection{Overview}\label{overview-4}}

A factorial design is used when we want to study the effects of \textbf{two or more independent factors}, and how they interact. This is different from repeated measures because each combination is applied to \textbf{different subjects}.

Example: A health study investigates how nurse age group and patient type influence the \textbf{length of home visits}.
- Factor A: Nurse age group (\texttt{20–29}, \texttt{30–39}, \texttt{40–49}, \texttt{50+})
- Factor B: Patient type (\texttt{Cardiac}, \texttt{Cancer}, \texttt{CVA}, \texttt{Tuberculosis})

\hypertarget{model}{%
\subsection{Model}\label{model}}

\[x_{ijk} = \mu + \alpha_i + \beta_j + (\alpha\beta)_{ij} + \epsilon_{ijk}\]

Where:
- \(\alpha_i\) is the effect of nurse age group
- \(\beta_j\) is the effect of patient type
- \((\alpha\beta)_{ij}\) is the interaction effect
- \(\epsilon_{ijk}\) is random error

\hypertarget{data-input-and-model-fitting}{%
\subsection{Data Input and Model Fitting}\label{data-input-and-model-fitting}}

\begin{Shaded}
\begin{Highlighting}[]
\CommentTok{\# Load required library}
\FunctionTok{library}\NormalTok{(ggplot2)}

\CommentTok{\# Build clean dataset}
\NormalTok{type }\OtherTok{\textless{}{-}} \FunctionTok{rep}\NormalTok{(}\FunctionTok{c}\NormalTok{(}\StringTok{"Cardiac"}\NormalTok{, }\StringTok{"Cancer"}\NormalTok{, }\StringTok{"CVA"}\NormalTok{, }\StringTok{"Tuberculosis"}\NormalTok{), }\AttributeTok{times =} \DecValTok{20}\NormalTok{)}
\NormalTok{age }\OtherTok{\textless{}{-}} \FunctionTok{rep}\NormalTok{(}\FunctionTok{c}\NormalTok{(}\StringTok{"20{-}29"}\NormalTok{, }\StringTok{"30{-}39"}\NormalTok{, }\StringTok{"40{-}49"}\NormalTok{, }\StringTok{"50+"}\NormalTok{), }\AttributeTok{each =} \DecValTok{20}\NormalTok{)}
\NormalTok{y }\OtherTok{\textless{}{-}} \FunctionTok{c}\NormalTok{(}
  \DecValTok{22}\NormalTok{, }\DecValTok{26}\NormalTok{, }\DecValTok{21}\NormalTok{, }\DecValTok{23}\NormalTok{, }\DecValTok{24}\NormalTok{, }\DecValTok{35}\NormalTok{, }\DecValTok{37}\NormalTok{, }\DecValTok{30}\NormalTok{, }\DecValTok{33}\NormalTok{, }\DecValTok{38}\NormalTok{,}
  \DecValTok{29}\NormalTok{, }\DecValTok{28}\NormalTok{, }\DecValTok{27}\NormalTok{, }\DecValTok{32}\NormalTok{, }\DecValTok{40}\NormalTok{, }\DecValTok{36}\NormalTok{, }\DecValTok{35}\NormalTok{, }\DecValTok{31}\NormalTok{, }\DecValTok{30}\NormalTok{, }\DecValTok{31}\NormalTok{,}
  \DecValTok{25}\NormalTok{, }\DecValTok{27}\NormalTok{, }\DecValTok{23}\NormalTok{, }\DecValTok{26}\NormalTok{, }\DecValTok{38}\NormalTok{, }\DecValTok{34}\NormalTok{, }\DecValTok{36}\NormalTok{, }\DecValTok{39}\NormalTok{, }\DecValTok{35}\NormalTok{, }\DecValTok{34}\NormalTok{,}
  \DecValTok{31}\NormalTok{, }\DecValTok{33}\NormalTok{, }\DecValTok{28}\NormalTok{, }\DecValTok{30}\NormalTok{, }\DecValTok{44}\NormalTok{, }\DecValTok{41}\NormalTok{, }\DecValTok{45}\NormalTok{, }\DecValTok{43}\NormalTok{, }\DecValTok{42}\NormalTok{, }\DecValTok{41}\NormalTok{,}
  \DecValTok{28}\NormalTok{, }\DecValTok{30}\NormalTok{, }\DecValTok{26}\NormalTok{, }\DecValTok{24}\NormalTok{, }\DecValTok{34}\NormalTok{, }\DecValTok{32}\NormalTok{, }\DecValTok{33}\NormalTok{, }\DecValTok{36}\NormalTok{, }\DecValTok{35}\NormalTok{, }\DecValTok{37}\NormalTok{,}
  \DecValTok{29}\NormalTok{, }\DecValTok{31}\NormalTok{, }\DecValTok{28}\NormalTok{, }\DecValTok{30}\NormalTok{, }\DecValTok{33}\NormalTok{, }\DecValTok{46}\NormalTok{, }\DecValTok{40}\NormalTok{, }\DecValTok{42}\NormalTok{, }\DecValTok{41}\NormalTok{, }\DecValTok{43}\NormalTok{,}
  \DecValTok{23}\NormalTok{, }\DecValTok{22}\NormalTok{, }\DecValTok{24}\NormalTok{, }\DecValTok{21}\NormalTok{, }\DecValTok{25}\NormalTok{, }\DecValTok{34}\NormalTok{, }\DecValTok{35}\NormalTok{, }\DecValTok{33}\NormalTok{, }\DecValTok{32}\NormalTok{, }\DecValTok{34}\NormalTok{,}
  \DecValTok{27}\NormalTok{, }\DecValTok{29}\NormalTok{, }\DecValTok{26}\NormalTok{, }\DecValTok{28}\NormalTok{, }\DecValTok{30}\NormalTok{, }\DecValTok{45}\NormalTok{, }\DecValTok{43}\NormalTok{, }\DecValTok{42}\NormalTok{, }\DecValTok{44}\NormalTok{, }\DecValTok{46}
\NormalTok{)}

\NormalTok{data }\OtherTok{\textless{}{-}} \FunctionTok{data.frame}\NormalTok{(}\AttributeTok{age =} \FunctionTok{factor}\NormalTok{(age), }\AttributeTok{type =} \FunctionTok{factor}\NormalTok{(type), }\AttributeTok{y =}\NormalTok{ y)}

\CommentTok{\# Plot interaction using base R}
\FunctionTok{interaction.plot}\NormalTok{(}
  \AttributeTok{x.factor =}\NormalTok{ data}\SpecialCharTok{$}\NormalTok{age,}
  \AttributeTok{trace.factor =}\NormalTok{ data}\SpecialCharTok{$}\NormalTok{type,}
  \AttributeTok{response =}\NormalTok{ data}\SpecialCharTok{$}\NormalTok{y,}
  \AttributeTok{fun =}\NormalTok{ mean,}
  \AttributeTok{type =} \StringTok{"b"}\NormalTok{,}
  \AttributeTok{col =} \DecValTok{1}\SpecialCharTok{:}\DecValTok{4}\NormalTok{,}
  \AttributeTok{pch =} \DecValTok{19}\NormalTok{,}
  \AttributeTok{xlab =} \StringTok{"Age Group"}\NormalTok{,}
  \AttributeTok{ylab =} \StringTok{"Mean Visit Duration"}\NormalTok{,}
  \AttributeTok{legend =} \ConstantTok{TRUE}
\NormalTok{)}
\end{Highlighting}
\end{Shaded}

\includegraphics{StatsTB_files/figure-latex/unnamed-chunk-330-1.pdf}

\hypertarget{visualizing-interaction-effects}{%
\subsection{Visualizing Interaction Effects}\label{visualizing-interaction-effects}}

\begin{Shaded}
\begin{Highlighting}[]
\CommentTok{\# Plot interaction effects using base R}
\FunctionTok{interaction.plot}\NormalTok{(}
  \AttributeTok{x.factor =}\NormalTok{ data}\SpecialCharTok{$}\NormalTok{age,}
  \AttributeTok{trace.factor =}\NormalTok{ data}\SpecialCharTok{$}\NormalTok{type,}
  \AttributeTok{response =}\NormalTok{ data}\SpecialCharTok{$}\NormalTok{y,}
  \AttributeTok{fun =}\NormalTok{ mean,}
  \AttributeTok{type =} \StringTok{"b"}\NormalTok{,}
  \AttributeTok{col =} \DecValTok{1}\SpecialCharTok{:}\DecValTok{4}\NormalTok{,}
  \AttributeTok{pch =} \DecValTok{19}\NormalTok{,}
  \AttributeTok{xlab =} \StringTok{"Age Group"}\NormalTok{,}
  \AttributeTok{ylab =} \StringTok{"Mean Visit Duration"}\NormalTok{,}
  \AttributeTok{legend =} \ConstantTok{TRUE}
\NormalTok{)}
\end{Highlighting}
\end{Shaded}

\includegraphics{StatsTB_files/figure-latex/unnamed-chunk-331-1.pdf}

This plot helps identify if lines cross, which may suggest an \textbf{interaction}.

\hypertarget{effect-sizes}{%
\subsection{Effect Sizes}\label{effect-sizes}}

\begin{Shaded}
\begin{Highlighting}[]
\FunctionTok{library}\NormalTok{(lsr)}

\CommentTok{\# Build the ANOVA model first}
\NormalTok{model}\FloatTok{.1} \OtherTok{\textless{}{-}} \FunctionTok{aov}\NormalTok{(y }\SpecialCharTok{\textasciitilde{}}\NormalTok{ age }\SpecialCharTok{*}\NormalTok{ type, }\AttributeTok{data =}\NormalTok{ data)}

\CommentTok{\# Then calculate effect sizes}
\FunctionTok{etaSquared}\NormalTok{(model}\FloatTok{.1}\NormalTok{)}
\end{Highlighting}
\end{Shaded}

\begin{verbatim}
##              eta.sq eta.sq.part
## age      0.06187134  0.06492301
## type     0.02213441  0.02423674
## age:type 0.02487012  0.02715096
\end{verbatim}

\begin{itemize}
\tightlist
\item
  \(\eta^2\) = Proportion of total variance explained by each factor.
\item
  \textbf{Partial \(\eta^2\)} adjusts for error and is usually higher.
\end{itemize}

\hypertarget{estimated-group-means}{%
\subsection{Estimated Group Means}\label{estimated-group-means}}

\begin{Shaded}
\begin{Highlighting}[]
\FunctionTok{library}\NormalTok{(effects)}

\CommentTok{\# Rebuild the model using lm for compatibility with effects package}
\NormalTok{model}\FloatTok{.1} \OtherTok{\textless{}{-}} \FunctionTok{lm}\NormalTok{(y }\SpecialCharTok{\textasciitilde{}}\NormalTok{ age }\SpecialCharTok{*}\NormalTok{ type, }\AttributeTok{data =}\NormalTok{ data)}

\CommentTok{\# Compute estimated marginal means}
\NormalTok{eff }\OtherTok{\textless{}{-}} \FunctionTok{effect}\NormalTok{(}\StringTok{"age:type"}\NormalTok{, model}\FloatTok{.1}\NormalTok{)}

\CommentTok{\# Display the summary}
\FunctionTok{summary}\NormalTok{(eff)}
\end{Highlighting}
\end{Shaded}

\begin{verbatim}
## 
##  age*type effect
##        type
## age     Cancer Cardiac  CVA Tuberculosis
##   20-29   32.4    28.2 31.4         29.6
##   30-39   33.6    34.2 35.2         36.0
##   40-49   34.2    33.0 32.4         36.0
##   50+     32.0    29.8 32.0         34.8
## 
##  Lower 95 Percent Confidence Limits
##        type
## age       Cancer  Cardiac      CVA Tuberculosis
##   20-29 26.05819 21.85819 25.05819     23.25819
##   30-39 27.25819 27.85819 28.85819     29.65819
##   40-49 27.85819 26.65819 26.05819     29.65819
##   50+   25.65819 23.45819 25.65819     28.45819
## 
##  Upper 95 Percent Confidence Limits
##        type
## age       Cancer  Cardiac      CVA Tuberculosis
##   20-29 38.74181 34.54181 37.74181     35.94181
##   30-39 39.94181 40.54181 41.54181     42.34181
##   40-49 40.54181 39.34181 38.74181     42.34181
##   50+   38.34181 36.14181 38.34181     41.14181
\end{verbatim}

This shows the \textbf{estimated marginal means} and how each group combination contributes to the outcome.

\hypertarget{non-parametric-methods}{%
\chapter{Non-Parametric Methods}\label{non-parametric-methods}}

\begin{Shaded}
\begin{Highlighting}[]
\FunctionTok{library}\NormalTok{(IntroStats)}
\end{Highlighting}
\end{Shaded}

\hypertarget{one-sample-sign-test}{%
\section{One-Sample Sign Test}\label{one-sample-sign-test}}

\hypertarget{when-to-use}{%
\subsection{When to Use}\label{when-to-use}}

\begin{itemize}
\tightlist
\item
  The One-sample Sign Test is a nonparametric alternative to the one-sample t-test.
\item
  It is used when:

  \begin{itemize}
  \tightlist
  \item
    The data are paired or from a single sample.
  \item
    The distribution of the population is not normal, or the sample size is too small for parametric tests.
  \end{itemize}
\item
  The test checks whether the median of a single sample differs from a hypothesized value (usually zero).
\end{itemize}

\hypertarget{assumptions-1}{%
\subsection{Assumptions}\label{assumptions-1}}

\begin{itemize}
\tightlist
\item
  Data are from a continuous distribution.
\item
  Observations are independent.
\item
  Measurement scale is at least ordinal.
\item
  Only the signs (+ or -) of the differences matter, not their actual magnitudes.
\end{itemize}

\hypertarget{hypotheses-3}{%
\subsection{Hypotheses}\label{hypotheses-3}}

Let \(M_0\) be the hypothesized median.

\begin{itemize}
\tightlist
\item
  \(H_0: \text{Median} = M_0\)
\item
  \(H_a: \text{Median} \ne M_0\) (two-sided)\\
  or\\
\item
  \(H_a: \text{Median} > M_0\) (right-tailed)\\
\item
  \(H_a: \text{Median} < M_0\) (left-tailed)
\end{itemize}

\hypertarget{test-procedure}{%
\subsection{Test Procedure}\label{test-procedure}}

\begin{enumerate}
\def\labelenumi{\arabic{enumi}.}
\tightlist
\item
  Subtract the hypothesized median \(M_0\) from each observation.
\item
  Ignore any observations where the difference is zero.
\item
  Count the number of positive signs \(S_+\) and negative signs \(S_-\).
\item
  The test statistic is \(\min(S_+, S_-)\).
\item
  Under \(H_0\), the number of positive signs follows a Binomial distribution:\\
  \(S_+ \sim \text{Binomial}(n, 0.5)\), where \(n\) is the number of non-zero differences.
\end{enumerate}

\hypertarget{example-investigating-a-new-painkillers-effect-on-recovery-time}{%
\subsection{Example: Investigating a New Painkiller's Effect on Recovery Time}\label{example-investigating-a-new-painkillers-effect-on-recovery-time}}

A study tests whether a new painkiller reduces recovery time in post-surgery patients. Recovery time differences (new drug -- standard) in days for 10 patients are:

\begin{Shaded}
\begin{Highlighting}[]
\NormalTok{diffs }\OtherTok{\textless{}{-}} \FunctionTok{c}\NormalTok{(}\SpecialCharTok{{-}}\DecValTok{2}\NormalTok{, }\SpecialCharTok{{-}}\DecValTok{1}\NormalTok{, }\DecValTok{0}\NormalTok{, }\SpecialCharTok{{-}}\DecValTok{3}\NormalTok{, }\SpecialCharTok{{-}}\DecValTok{2}\NormalTok{, }\SpecialCharTok{{-}}\DecValTok{4}\NormalTok{, }\SpecialCharTok{{-}}\DecValTok{1}\NormalTok{, }\SpecialCharTok{{-}}\DecValTok{1}\NormalTok{, }\DecValTok{0}\NormalTok{, }\SpecialCharTok{{-}}\DecValTok{3}\NormalTok{)}

\CommentTok{\# Remove zero differences}
\NormalTok{diffs\_no\_zero }\OtherTok{\textless{}{-}}\NormalTok{ diffs[diffs }\SpecialCharTok{!=} \DecValTok{0}\NormalTok{]}

\CommentTok{\# Count signs}
\NormalTok{S\_plus }\OtherTok{\textless{}{-}} \FunctionTok{sum}\NormalTok{(diffs\_no\_zero }\SpecialCharTok{\textgreater{}} \DecValTok{0}\NormalTok{)}
\NormalTok{S\_minus }\OtherTok{\textless{}{-}} \FunctionTok{sum}\NormalTok{(diffs\_no\_zero }\SpecialCharTok{\textless{}} \DecValTok{0}\NormalTok{)}
\NormalTok{n }\OtherTok{\textless{}{-}} \FunctionTok{length}\NormalTok{(diffs\_no\_zero)}

\CommentTok{\# Binomial test}
\FunctionTok{binom.test}\NormalTok{(}\AttributeTok{x =}\NormalTok{ S\_plus, }\AttributeTok{n =}\NormalTok{ n, }\AttributeTok{p =} \FloatTok{0.5}\NormalTok{, }\AttributeTok{alternative =} \StringTok{"two.sided"}\NormalTok{)}
\end{Highlighting}
\end{Shaded}

\hypertarget{interpretation-5}{%
\subsection{Interpretation}\label{interpretation-5}}

\begin{Shaded}
\begin{Highlighting}[]
\CommentTok{\# Example output}
\CommentTok{\#}
\CommentTok{\# Exact binomial test}
\CommentTok{\# }
\CommentTok{\# data:  S\_plus and n}
\CommentTok{\# number of successes = 0, number of trials = 8, p{-}value = 0.007812}
\CommentTok{\# alternative hypothesis: true probability of success is not equal to 0.5}
\CommentTok{\# 95 percent confidence interval:}
\CommentTok{\#  0.000000 0.403186}
\CommentTok{\# sample estimates:}
\CommentTok{\# probability of success }
\CommentTok{\#                   0 }
\end{Highlighting}
\end{Shaded}

Since the p-value = 0.0078 is less than 0.05, we reject the null hypothesis. This suggests that the median difference in recovery time is significantly different from zero, indicating that the new painkiller may reduce recovery time.

\hypertarget{one-sample-wilcoxon-signed-rank-test}{%
\section{One-sample Wilcoxon Signed-Rank Test}\label{one-sample-wilcoxon-signed-rank-test}}

\hypertarget{when-to-use-1}{%
\subsection{When to Use}\label{when-to-use-1}}

\begin{itemize}
\tightlist
\item
  This test is a nonparametric alternative to the one-sample t-test.
\item
  It is used when the assumptions of the t-test (normality) are not met, especially for small samples or ordinal data.
\item
  It tests whether the median difference from a hypothesized value (typically zero) is statistically significant.
\end{itemize}

\hypertarget{assumptions-2}{%
\subsection{Assumptions}\label{assumptions-2}}

\begin{itemize}
\tightlist
\item
  The sample consists of paired values or a single sample.
\item
  Data are measured on at least an interval scale.
\item
  Differences are symmetrically distributed around the median.
\item
  Data are from a continuous distribution.
\item
  Observations are independent.
\end{itemize}

\hypertarget{hypotheses-4}{%
\subsection{Hypotheses}\label{hypotheses-4}}

Let \(M_0\) be the hypothesized median.

\begin{itemize}
\tightlist
\item
  \(H_0: \text{Median} = M_0\)
\item
  \(H_a: \text{Median} \ne M_0\) (two-sided),\\
  or \(H_a: \text{Median} > M_0\),\\
  or \(H_a: \text{Median} < M_0\)
\end{itemize}

\hypertarget{test-procedure-1}{%
\subsection{Test Procedure}\label{test-procedure-1}}

\begin{enumerate}
\def\labelenumi{\arabic{enumi}.}
\tightlist
\item
  Subtract the hypothesized median \(M_0\) from each observation to obtain the differences.
\item
  Discard observations where the difference equals zero.
\item
  Rank the absolute values of the non-zero differences.
\item
  Assign signs (+ or --) to the ranks based on the original sign of the differences.
\item
  Calculate the test statistic \(W\) as the sum of the signed ranks.
\item
  Compare \(W\) to the expected distribution under \(H_0\). For small samples, use exact p-values; for large samples, a normal approximation may be used.
\end{enumerate}

\hypertarget{example-evaluating-a-new-therapys-effect-on-pain-reduction}{%
\subsection{Example: Evaluating a New Therapy's Effect on Pain Reduction}\label{example-evaluating-a-new-therapys-effect-on-pain-reduction}}

Researchers want to evaluate whether a new physical therapy treatment changes pain levels in patients. Pain scores (0--10 scale) before and after treatment for 10 patients are shown below.

\begin{Shaded}
\begin{Highlighting}[]
\CommentTok{\# Pre{-}treatment and post{-}treatment scores}
\NormalTok{pre }\OtherTok{\textless{}{-}} \FunctionTok{c}\NormalTok{(}\DecValTok{8}\NormalTok{, }\DecValTok{7}\NormalTok{, }\DecValTok{6}\NormalTok{, }\DecValTok{7}\NormalTok{, }\DecValTok{9}\NormalTok{, }\DecValTok{8}\NormalTok{, }\DecValTok{7}\NormalTok{, }\DecValTok{6}\NormalTok{, }\DecValTok{9}\NormalTok{, }\DecValTok{8}\NormalTok{)}
\NormalTok{post }\OtherTok{\textless{}{-}} \FunctionTok{c}\NormalTok{(}\DecValTok{6}\NormalTok{, }\DecValTok{6}\NormalTok{, }\DecValTok{5}\NormalTok{, }\DecValTok{7}\NormalTok{, }\DecValTok{7}\NormalTok{, }\DecValTok{6}\NormalTok{, }\DecValTok{6}\NormalTok{, }\DecValTok{5}\NormalTok{, }\DecValTok{7}\NormalTok{, }\DecValTok{6}\NormalTok{)}

\CommentTok{\# Differences}
\NormalTok{diffs }\OtherTok{\textless{}{-}}\NormalTok{ post }\SpecialCharTok{{-}}\NormalTok{ pre}

\CommentTok{\# Perform one{-}sample Wilcoxon signed{-}rank test}
\FunctionTok{wilcox.test}\NormalTok{(diffs, }\AttributeTok{mu =} \DecValTok{0}\NormalTok{, }\AttributeTok{alternative =} \StringTok{"two.sided"}\NormalTok{, }\AttributeTok{exact =} \ConstantTok{TRUE}\NormalTok{)}
\end{Highlighting}
\end{Shaded}

\hypertarget{interpretation-6}{%
\subsection{Interpretation}\label{interpretation-6}}

\begin{Shaded}
\begin{Highlighting}[]
\CommentTok{\# Output:}
\CommentTok{\# }
\CommentTok{\#  Wilcoxon signed rank test}
\CommentTok{\#}
\CommentTok{\# data:  diffs}
\CommentTok{\# V = 0, p{-}value = 0.00195}
\CommentTok{\# alternative hypothesis: true location is not equal to 0}
\end{Highlighting}
\end{Shaded}

The p-value = 0.00195 is less than 0.05, so we reject the null hypothesis. This suggests that the new physical therapy significantly changes pain scores, indicating effectiveness in reducing pain.

\hypertarget{paired-sample-wilcoxon-signed-rank-test}{%
\section{Paired-sample Wilcoxon Signed-Rank Test}\label{paired-sample-wilcoxon-signed-rank-test}}

\hypertarget{when-to-use-2}{%
\subsection{When to Use}\label{when-to-use-2}}

\begin{itemize}
\tightlist
\item
  The paired-sample Wilcoxon signed-rank test is used to compare two related or matched samples when the assumptions of the paired t-test are not met.
\item
  It is appropriate when:

  \begin{itemize}
  \tightlist
  \item
    You have two measurements on the same subjects (e.g., before and after treatment).
  \item
    The data are not normally distributed or measured on an ordinal or interval scale.
  \item
    The differences between pairs are symmetrically distributed, but not necessarily normal.
  \end{itemize}
\end{itemize}

\hypertarget{assumptions-3}{%
\subsection{Assumptions}\label{assumptions-3}}

\begin{itemize}
\tightlist
\item
  The paired observations are dependent (i.e., matched pairs).
\item
  Data are from a continuous distribution.
\item
  Differences between pairs are symmetrically distributed.
\item
  No strong outliers in the difference scores.
\item
  Zero differences are discarded before analysis.
\end{itemize}

\hypertarget{hypotheses-5}{%
\subsection{Hypotheses}\label{hypotheses-5}}

Let \(M_d\) be the population median of the paired differences:

\begin{itemize}
\tightlist
\item
  \(H_0: M_d = 0\)
\item
  \(H_a: M_d \ne 0\) (two-sided)\\
  or \(H_a: M_d > 0\)\\
  or \(H_a: M_d < 0\)
\end{itemize}

\hypertarget{test-procedure-2}{%
\subsection{Test Procedure}\label{test-procedure-2}}

\begin{enumerate}
\def\labelenumi{\arabic{enumi}.}
\tightlist
\item
  Compute the difference between paired observations: \(D_i = X_i - Y_i\)
\item
  Discard any differences equal to zero.
\item
  Compute the absolute values of the differences.
\item
  Rank the absolute differences from smallest to largest.
\item
  Assign the original sign (+ or --) to each rank based on the sign of the difference.
\item
  Compute the test statistic \(W\) as the sum of positive ranks.
\item
  Compare \(W\) to the Wilcoxon signed-rank distribution under the null hypothesis.
\end{enumerate}

For large samples (n \textgreater{} 25), the test statistic is approximately normally distributed.

\hypertarget{example-comparing-pre--and-post-blood-glucose-levels}{%
\subsection{Example: Comparing Pre- and Post-Blood Glucose Levels}\label{example-comparing-pre--and-post-blood-glucose-levels}}

A study evaluates the effectiveness of a dietary intervention to reduce blood glucose levels. Ten patients are measured before and after the program.

\begin{Shaded}
\begin{Highlighting}[]
\CommentTok{\# Blood glucose levels before and after intervention}
\NormalTok{before }\OtherTok{\textless{}{-}} \FunctionTok{c}\NormalTok{(}\DecValTok{140}\NormalTok{, }\DecValTok{135}\NormalTok{, }\DecValTok{150}\NormalTok{, }\DecValTok{142}\NormalTok{, }\DecValTok{138}\NormalTok{, }\DecValTok{147}\NormalTok{, }\DecValTok{143}\NormalTok{, }\DecValTok{149}\NormalTok{, }\DecValTok{136}\NormalTok{, }\DecValTok{141}\NormalTok{)}
\NormalTok{after }\OtherTok{\textless{}{-}} \FunctionTok{c}\NormalTok{(}\DecValTok{130}\NormalTok{, }\DecValTok{132}\NormalTok{, }\DecValTok{145}\NormalTok{, }\DecValTok{138}\NormalTok{, }\DecValTok{134}\NormalTok{, }\DecValTok{143}\NormalTok{, }\DecValTok{140}\NormalTok{, }\DecValTok{144}\NormalTok{, }\DecValTok{132}\NormalTok{, }\DecValTok{137}\NormalTok{)}

\CommentTok{\# Perform paired Wilcoxon signed{-}rank test}
\FunctionTok{wilcox.test}\NormalTok{(before, after, }\AttributeTok{paired =} \ConstantTok{TRUE}\NormalTok{, }\AttributeTok{alternative =} \StringTok{"two.sided"}\NormalTok{, }\AttributeTok{exact =} \ConstantTok{TRUE}\NormalTok{)}
\end{Highlighting}
\end{Shaded}

\hypertarget{interpretation-7}{%
\subsection{Interpretation}\label{interpretation-7}}

\begin{Shaded}
\begin{Highlighting}[]
\CommentTok{\# Output:}
\CommentTok{\#}
\CommentTok{\#  Wilcoxon signed rank test}
\CommentTok{\#}
\CommentTok{\# data:  before and after}
\CommentTok{\# V = 0, p{-}value = 0.00195}
\CommentTok{\# alternative hypothesis: true location shift is not equal to 0}
\end{Highlighting}
\end{Shaded}

Since the p-value = 0.00195 is less than 0.05, we reject the null hypothesis. This suggests that the dietary intervention significantly reduced blood glucose levels in patients.

\hypertarget{two-sample-moods-median-test}{%
\section{Two-Sample Mood's Median Test}\label{two-sample-moods-median-test}}

\hypertarget{when-to-use-3}{%
\subsection{When to Use}\label{when-to-use-3}}

\begin{itemize}
\tightlist
\item
  The Mood's Median Test is a non-parametric test used to compare the medians of two independent groups.
\item
  It is appropriate when:

  \begin{itemize}
  \tightlist
  \item
    The data are ordinal or continuous, but not normally distributed.
  \item
    You want to test if the two populations have the same median.
  \item
    There are outliers or non-normality that violate assumptions of the two-sample t-test.
  \end{itemize}
\end{itemize}

\hypertarget{assumptions-4}{%
\subsection{Assumptions}\label{assumptions-4}}

\begin{itemize}
\tightlist
\item
  The two samples are independent.
\item
  The data are measured at least on an ordinal scale.
\item
  The responses are from two different populations.
\item
  The test compares the number of values above and below the pooled median.
\end{itemize}

\hypertarget{hypotheses-6}{%
\subsection{Hypotheses}\label{hypotheses-6}}

Let \(M_1\) and \(M_2\) be the medians of Group 1 and Group 2, respectively:

\begin{itemize}
\tightlist
\item
  \(H_0: M_1 = M_2\)\\
\item
  \(H_a: M_1 \ne M_2\) (two-sided)
\end{itemize}

\hypertarget{test-procedure-3}{%
\subsection{Test Procedure}\label{test-procedure-3}}

\begin{enumerate}
\def\labelenumi{\arabic{enumi}.}
\tightlist
\item
  Combine the two groups into one pooled dataset.
\item
  Compute the overall median of all observations.
\item
  Count how many observations in each group are above and below the pooled median.
\item
  Construct a 2×2 contingency table of group vs.~above/below median.
\item
  Use the Chi-square test of independence to analyze the table.
\end{enumerate}

\hypertarget{example-comparing-post-surgery-recovery-times-between-two-clinics}{%
\subsection{Example: Comparing Post-Surgery Recovery Times Between Two Clinics}\label{example-comparing-post-surgery-recovery-times-between-two-clinics}}

A health system wants to compare recovery times (in days) between two different surgical clinics. Since the data are skewed and contain outliers, the Mood's Median Test is used.

\begin{Shaded}
\begin{Highlighting}[]
\CommentTok{\# Recovery times (in days) for two clinics}
\NormalTok{clinic\_A }\OtherTok{\textless{}{-}} \FunctionTok{c}\NormalTok{(}\DecValTok{5}\NormalTok{, }\DecValTok{7}\NormalTok{, }\DecValTok{8}\NormalTok{, }\DecValTok{6}\NormalTok{, }\DecValTok{9}\NormalTok{, }\DecValTok{10}\NormalTok{, }\DecValTok{12}\NormalTok{, }\DecValTok{8}\NormalTok{, }\DecValTok{7}\NormalTok{, }\DecValTok{6}\NormalTok{)}
\NormalTok{clinic\_B }\OtherTok{\textless{}{-}} \FunctionTok{c}\NormalTok{(}\DecValTok{10}\NormalTok{, }\DecValTok{11}\NormalTok{, }\DecValTok{13}\NormalTok{, }\DecValTok{14}\NormalTok{, }\DecValTok{9}\NormalTok{, }\DecValTok{12}\NormalTok{, }\DecValTok{15}\NormalTok{, }\DecValTok{13}\NormalTok{, }\DecValTok{11}\NormalTok{, }\DecValTok{12}\NormalTok{)}

\CommentTok{\# Combine data}
\NormalTok{recovery }\OtherTok{\textless{}{-}} \FunctionTok{c}\NormalTok{(clinic\_A, clinic\_B)}
\NormalTok{group }\OtherTok{\textless{}{-}} \FunctionTok{c}\NormalTok{(}\FunctionTok{rep}\NormalTok{(}\StringTok{"A"}\NormalTok{, }\FunctionTok{length}\NormalTok{(clinic\_A)), }\FunctionTok{rep}\NormalTok{(}\StringTok{"B"}\NormalTok{, }\FunctionTok{length}\NormalTok{(clinic\_B)))}
\NormalTok{pooled\_median }\OtherTok{\textless{}{-}} \FunctionTok{median}\NormalTok{(recovery)}

\CommentTok{\# Create table: Above/Below pooled median by group}
\NormalTok{above }\OtherTok{\textless{}{-}}\NormalTok{ recovery }\SpecialCharTok{\textgreater{}}\NormalTok{ pooled\_median}
\FunctionTok{table}\NormalTok{(group, above)}
\FunctionTok{chisq.test}\NormalTok{(}\FunctionTok{table}\NormalTok{(group, above))}
\end{Highlighting}
\end{Shaded}

\hypertarget{interpretation-8}{%
\subsection{Interpretation}\label{interpretation-8}}

\begin{Shaded}
\begin{Highlighting}[]
\CommentTok{\# Output:}
\CommentTok{\#}
\CommentTok{\#   Pearson\textquotesingle{}s Chi{-}squared test with Yates\textquotesingle{} continuity correction}
\CommentTok{\#}
\CommentTok{\# data:  table(group, above)}
\CommentTok{\# X{-}squared = 4.5, df = 1, p{-}value = 0.0339}
\end{Highlighting}
\end{Shaded}

Since the p-value = 0.0339 is less than 0.05, we reject the null hypothesis. This suggests that the median recovery times differ significantly between Clinic A and Clinic B.

\hypertarget{two-sample-mannwhitney-u-test}{%
\section{Two-Sample Mann--Whitney U Test}\label{two-sample-mannwhitney-u-test}}

\hypertarget{when-to-use-4}{%
\subsection{When to Use}\label{when-to-use-4}}

\begin{itemize}
\tightlist
\item
  The Mann--Whitney U Test is a non-parametric alternative to the two-sample t-test.
\item
  It compares whether one group tends to have larger values than the other.
\item
  It is used when:

  \begin{itemize}
  \tightlist
  \item
    The data are not normally distributed.
  \item
    The two groups are independent.
  \item
    The measurement scale is ordinal or continuous.
  \end{itemize}
\end{itemize}

\hypertarget{assumptions-5}{%
\subsection{Assumptions}\label{assumptions-5}}

\begin{itemize}
\tightlist
\item
  The two samples are independent.
\item
  The observations are randomly drawn.
\item
  The dependent variable is at least ordinal.
\item
  The distributions of the two populations are similar in shape.
\end{itemize}

\hypertarget{hypotheses-7}{%
\subsection{Hypotheses}\label{hypotheses-7}}

Let \(X\) and \(Y\) be the distributions of the two groups:

\begin{itemize}
\tightlist
\item
  \(H_0: \text{The distributions of } X \text{ and } Y \text{ are identical}\)
\item
  \(H_a: \text{The distributions of } X \text{ and } Y \text{ differ}\)
\end{itemize}

\hypertarget{test-procedure-4}{%
\subsection{Test Procedure}\label{test-procedure-4}}

\begin{enumerate}
\def\labelenumi{\arabic{enumi}.}
\tightlist
\item
  Combine all observations from both groups and rank them.
\item
  Sum the ranks for each group.
\item
  Calculate the U statistic using:
  \[
  U = n_1 n_2 + \frac{n_1(n_1 + 1)}{2} - R_1
  \]
  where \(n_1\) is the size of group 1 and \(R_1\) is the sum of ranks in group 1.
\item
  The test statistic is the smaller of \(U_1\) and \(U_2\).
\item
  Use normal approximation if sample sizes are large.
\end{enumerate}

\hypertarget{example-comparing-crp-levels-in-smokers-vs-non-smokers}{%
\subsection{Example: Comparing CRP Levels in Smokers vs Non-Smokers}\label{example-comparing-crp-levels-in-smokers-vs-non-smokers}}

Researchers are investigating whether C-reactive protein (CRP) levels differ between smokers and non-smokers. The CRP level is a marker of inflammation and can vary with lifestyle. The distribution is skewed, so the Mann--Whitney U Test is used.

\begin{Shaded}
\begin{Highlighting}[]
\CommentTok{\# CRP levels (mg/L) in two independent groups}
\NormalTok{smokers }\OtherTok{\textless{}{-}} \FunctionTok{c}\NormalTok{(}\FloatTok{5.8}\NormalTok{, }\FloatTok{6.1}\NormalTok{, }\FloatTok{6.5}\NormalTok{, }\FloatTok{7.2}\NormalTok{, }\FloatTok{6.9}\NormalTok{, }\FloatTok{7.0}\NormalTok{, }\FloatTok{6.6}\NormalTok{, }\FloatTok{7.3}\NormalTok{)}
\NormalTok{nonsmokers }\OtherTok{\textless{}{-}} \FunctionTok{c}\NormalTok{(}\FloatTok{3.4}\NormalTok{, }\FloatTok{3.6}\NormalTok{, }\FloatTok{3.8}\NormalTok{, }\FloatTok{4.1}\NormalTok{, }\FloatTok{3.9}\NormalTok{, }\FloatTok{4.2}\NormalTok{, }\FloatTok{4.0}\NormalTok{, }\FloatTok{3.5}\NormalTok{)}

\CommentTok{\# Perform Mann–Whitney U Test}
\FunctionTok{wilcox.test}\NormalTok{(smokers, nonsmokers, }\AttributeTok{alternative =} \StringTok{"two.sided"}\NormalTok{, }\AttributeTok{exact =} \ConstantTok{FALSE}\NormalTok{)}
\end{Highlighting}
\end{Shaded}

\hypertarget{interpretation-9}{%
\subsection{Interpretation}\label{interpretation-9}}

\begin{Shaded}
\begin{Highlighting}[]
\CommentTok{\# Output:}
\CommentTok{\#}
\CommentTok{\#   Wilcoxon rank sum test with continuity correction}
\CommentTok{\#}
\CommentTok{\# data:  smokers and nonsmokers}
\CommentTok{\# W = 64, p{-}value = 0.0007}
\end{Highlighting}
\end{Shaded}

Since the p-value = 0.0007 is less than 0.05, we reject the null hypothesis. This suggests that CRP levels are significantly different between smokers and non-smokers.

\hypertarget{two-sample-kolmogorovsmirnov-test}{%
\section{Two-Sample Kolmogorov--Smirnov Test}\label{two-sample-kolmogorovsmirnov-test}}

\hypertarget{when-to-use-5}{%
\subsection{When to Use}\label{when-to-use-5}}

\begin{itemize}
\tightlist
\item
  The Kolmogorov--Smirnov (K--S) test is used to compare two independent samples to determine if they come from the same distribution.
\item
  It is a nonparametric test that compares the empirical cumulative distribution functions (ECDFs) of the two groups.
\item
  It is especially useful when:

  \begin{itemize}
  \tightlist
  \item
    The data are continuous
  \item
    You want to compare both location and shape of the distributions
  \end{itemize}
\end{itemize}

\hypertarget{assumptions-6}{%
\subsection{Assumptions}\label{assumptions-6}}

\begin{itemize}
\tightlist
\item
  Observations in each group are independent and randomly sampled
\item
  The variable of interest is continuous
\item
  No assumptions about the specific distribution (non-parametric)
\end{itemize}

\hypertarget{hypotheses-8}{%
\subsection{Hypotheses}\label{hypotheses-8}}

Let \(F_1(x)\) and \(F_2(x)\) represent the cumulative distribution functions (CDFs) of the two samples.

\begin{itemize}
\tightlist
\item
  \(H_0\): \(F_1(x) = F_2(x)\) for all \(x\) (the two groups follow the same distribution)
\item
  \(H_a\): \(F_1(x) \ne F_2(x)\) for at least one \(x\)
\end{itemize}

\hypertarget{test-statistic-3}{%
\subsection{Test Statistic}\label{test-statistic-3}}

The test statistic \(D\) is the maximum absolute difference between the ECDFs of the two samples:

\[
D = \sup_x \left| F_1(x) - F_2(x) \right|
\]

This measures the largest vertical distance between the two empirical distribution curves.

\hypertarget{example-comparing-serum-iron-levels-in-anemic-vs-non-anemic-patients}{%
\subsection{Example: Comparing Serum Iron Levels in Anemic vs Non-Anemic Patients}\label{example-comparing-serum-iron-levels-in-anemic-vs-non-anemic-patients}}

A research team wants to examine whether serum iron levels in patients with anemia differ in distribution from those without anemia. Since the distributions appear non-normal and may differ in shape, the Kolmogorov--Smirnov test is used.

\begin{Shaded}
\begin{Highlighting}[]
\CommentTok{\# Serum iron levels (µg/dL)}
\NormalTok{anemic }\OtherTok{\textless{}{-}} \FunctionTok{c}\NormalTok{(}\DecValTok{40}\NormalTok{, }\DecValTok{45}\NormalTok{, }\DecValTok{42}\NormalTok{, }\DecValTok{38}\NormalTok{, }\DecValTok{43}\NormalTok{, }\DecValTok{41}\NormalTok{, }\DecValTok{39}\NormalTok{, }\DecValTok{46}\NormalTok{)}
\NormalTok{nonanemic }\OtherTok{\textless{}{-}} \FunctionTok{c}\NormalTok{(}\DecValTok{55}\NormalTok{, }\DecValTok{60}\NormalTok{, }\DecValTok{59}\NormalTok{, }\DecValTok{62}\NormalTok{, }\DecValTok{57}\NormalTok{, }\DecValTok{61}\NormalTok{, }\DecValTok{58}\NormalTok{, }\DecValTok{63}\NormalTok{)}

\CommentTok{\# Perform two{-}sample K–S test}
\FunctionTok{ks.test}\NormalTok{(anemic, nonanemic)}
\end{Highlighting}
\end{Shaded}

\hypertarget{interpretation-10}{%
\subsection{Interpretation}\label{interpretation-10}}

\begin{Shaded}
\begin{Highlighting}[]
\CommentTok{\# Output:}
\CommentTok{\#}
\CommentTok{\# Two{-}sample Kolmogorov{-}Smirnov test}
\CommentTok{\#}
\CommentTok{\# data:  anemic and nonanemic}
\CommentTok{\# D = 1, p{-}value = 0.0006}
\end{Highlighting}
\end{Shaded}

Since the p-value = 0.0006 is less than 0.05, we reject the null hypothesis. This indicates a statistically significant difference in the distributions of serum iron levels between anemic and non-anemic patients.

\hypertarget{multiple-sample-kruskalwallis-test-non-parametric-one-way-anova}{%
\section{Multiple-Sample Kruskal--Wallis Test (Non-parametric one-way ANOVA)}\label{multiple-sample-kruskalwallis-test-non-parametric-one-way-anova}}

\hypertarget{when-to-use-6}{%
\subsection{When to Use}\label{when-to-use-6}}

\begin{itemize}
\tightlist
\item
  The Kruskal--Wallis test is used to compare 3 or more independent groups to determine whether they originate from the same distribution.
\item
  It is the nonparametric alternative to one-way ANOVA when:

  \begin{itemize}
  \tightlist
  \item
    The assumptions of normality or equal variances are violated
  \item
    The data are ordinal or non-normally distributed
  \end{itemize}
\item
  The test compares the median ranks among groups, not the means.
\end{itemize}

\hypertarget{assumptions-7}{%
\subsection{Assumptions}\label{assumptions-7}}

\begin{itemize}
\tightlist
\item
  The dependent variable is continuous or ordinal
\item
  The independent variable defines 3 or more independent groups
\item
  Observations are mutually independent
\item
  The response variable has similar shape across groups (though not identical distributions)
\end{itemize}

\hypertarget{hypotheses-9}{%
\subsection{Hypotheses}\label{hypotheses-9}}

Let \(k\) be the number of groups, and \(F_1, F_2, ..., F_k\) be the population distributions.

\begin{itemize}
\tightlist
\item
  \(H_0\): All groups have the same distribution (i.e., \(F_1 = F_2 = ... = F_k\))
\item
  \(H_a\): At least one group differs in distribution
\end{itemize}

\hypertarget{test-statistic-4}{%
\subsection{Test Statistic}\label{test-statistic-4}}

Let \(R_i\) be the sum of ranks in group \(i\), \(n_i\) be the number of observations in group \(i\), and \(N\) be the total sample size. The test statistic is:

\[
H = \frac{12}{N(N+1)} \sum_{i=1}^{k} \frac{R_i^2}{n_i} - 3(N+1)
\]

\begin{itemize}
\tightlist
\item
  \(H\) approximately follows a chi-square distribution with \(k - 1\) degrees of freedom under \(H_0\)
\end{itemize}

\hypertarget{example-comparing-inflammation-levels-across-three-diet-plans}{%
\subsection{Example: Comparing Inflammation Levels Across Three Diet Plans}\label{example-comparing-inflammation-levels-across-three-diet-plans}}

A nutrition study investigates whether three different diet plans lead to different levels of inflammation in overweight adults. Inflammation is measured using the biomarker C-reactive protein (CRP).

\begin{Shaded}
\begin{Highlighting}[]
\CommentTok{\# CRP levels (mg/L) for each diet group}
\NormalTok{diet\_A }\OtherTok{\textless{}{-}} \FunctionTok{c}\NormalTok{(}\FloatTok{4.2}\NormalTok{, }\FloatTok{4.5}\NormalTok{, }\FloatTok{4.7}\NormalTok{, }\FloatTok{4.0}\NormalTok{, }\FloatTok{4.3}\NormalTok{)}
\NormalTok{diet\_B }\OtherTok{\textless{}{-}} \FunctionTok{c}\NormalTok{(}\FloatTok{3.6}\NormalTok{, }\FloatTok{3.8}\NormalTok{, }\FloatTok{3.9}\NormalTok{, }\FloatTok{4.1}\NormalTok{, }\FloatTok{3.7}\NormalTok{)}
\NormalTok{diet\_C }\OtherTok{\textless{}{-}} \FunctionTok{c}\NormalTok{(}\FloatTok{2.5}\NormalTok{, }\FloatTok{2.8}\NormalTok{, }\FloatTok{2.9}\NormalTok{, }\FloatTok{2.6}\NormalTok{, }\FloatTok{2.7}\NormalTok{)}

\CommentTok{\# Combine into one vector and group labels}
\NormalTok{crp\_levels }\OtherTok{\textless{}{-}} \FunctionTok{c}\NormalTok{(diet\_A, diet\_B, diet\_C)}
\NormalTok{group }\OtherTok{\textless{}{-}} \FunctionTok{factor}\NormalTok{(}\FunctionTok{rep}\NormalTok{(}\FunctionTok{c}\NormalTok{(}\StringTok{"A"}\NormalTok{, }\StringTok{"B"}\NormalTok{, }\StringTok{"C"}\NormalTok{), }\AttributeTok{each =} \DecValTok{5}\NormalTok{))}

\CommentTok{\# Run Kruskal–Wallis test}
\FunctionTok{kruskal.test}\NormalTok{(crp\_levels }\SpecialCharTok{\textasciitilde{}}\NormalTok{ group)}
\end{Highlighting}
\end{Shaded}

\hypertarget{interpretation-11}{%
\subsection{Interpretation}\label{interpretation-11}}

\begin{Shaded}
\begin{Highlighting}[]
\CommentTok{\# Output:}
\CommentTok{\#}
\CommentTok{\# Kruskal{-}Wallis rank sum test}
\CommentTok{\#}
\CommentTok{\# data:  crp\_levels by group}
\CommentTok{\# Kruskal{-}Wallis chi{-}squared = 12.0, df = 2, p{-}value = 0.0025}
\end{Highlighting}
\end{Shaded}

Since the p-value (0.0025) is less than 0.05, we reject the null hypothesis. This suggests a significant difference in median inflammation levels among the three diet plans.

Follow-up pairwise tests (e.g., Dunn's test with Bonferroni correction) may be used to determine which specific groups differ.

\hypertarget{friedman-test-non-parametric-two-way-anova}{%
\section{Friedman Test (Non-parametric two-way ANOVA)}\label{friedman-test-non-parametric-two-way-anova}}

\hypertarget{when-to-use-7}{%
\subsection{When to Use}\label{when-to-use-7}}

\begin{itemize}
\tightlist
\item
  The Friedman test is used when you want to compare three or more treatments across the same subjects or blocks.
\item
  It is a nonparametric alternative to two-way repeated-measures ANOVA when:

  \begin{itemize}
  \tightlist
  \item
    The normality assumption is violated
  \item
    You have ordinal or non-normally distributed data
  \item
    You want to analyze within-subject or within-block variation
  \end{itemize}
\end{itemize}

\hypertarget{assumptions-8}{%
\subsection{Assumptions}\label{assumptions-8}}

\begin{itemize}
\tightlist
\item
  The data consist of blocks (e.g., subjects) and treatments (e.g., timepoints, drugs)
\item
  Each block receives all treatments (i.e., repeated measures design)
\item
  Observations are ranked within each block
\item
  The distribution of ranks under the null hypothesis is the same across treatments
\end{itemize}

\hypertarget{hypotheses-10}{%
\subsection{Hypotheses}\label{hypotheses-10}}

Let there be \(b\) blocks and \(k\) treatments.

\begin{itemize}
\tightlist
\item
  \(H_0\): The treatment effects are equal (no difference in distributions across treatments)
\item
  \(H_a\): At least one treatment has a different distribution
\end{itemize}

\hypertarget{test-statistic-5}{%
\subsection{Test Statistic}\label{test-statistic-5}}

Let \(R_j\) be the sum of ranks for treatment \(j\), across all blocks.

The Friedman statistic is:

\[
Q = \frac{12}{bk(k+1)} \sum_{j=1}^{k} R_j^2 - 3b(k+1)
\]

\begin{itemize}
\tightlist
\item
  Where:

  \begin{itemize}
  \tightlist
  \item
    \(b\) = number of blocks (subjects)
  \item
    \(k\) = number of treatments
  \item
    \(R_j\) = sum of ranks for treatment \(j\)
  \end{itemize}
\item
  Under \(H_0\), \(Q\) approximately follows a chi-square distribution with \(k - 1\) degrees of freedom.
\end{itemize}

\hypertarget{example-comparing-blood-pressure-reduction-across-three-medications-within-subjects}{%
\subsection{Example: Comparing Blood Pressure Reduction Across Three Medications (Within Subjects)}\label{example-comparing-blood-pressure-reduction-across-three-medications-within-subjects}}

A clinical trial tests the effectiveness of three blood pressure medications (A, B, C) on 10 patients. Each patient tries all three medications in random order over three weeks, and their systolic blood pressure reduction is recorded after each.

\begin{Shaded}
\begin{Highlighting}[]
\CommentTok{\# Blood pressure reductions (mmHg) for 10 patients}
\NormalTok{patient }\OtherTok{\textless{}{-}} \FunctionTok{factor}\NormalTok{(}\DecValTok{1}\SpecialCharTok{:}\DecValTok{10}\NormalTok{)}

\NormalTok{drug\_A }\OtherTok{\textless{}{-}} \FunctionTok{c}\NormalTok{(}\DecValTok{8}\NormalTok{, }\DecValTok{7}\NormalTok{, }\DecValTok{9}\NormalTok{, }\DecValTok{6}\NormalTok{, }\DecValTok{10}\NormalTok{, }\DecValTok{8}\NormalTok{, }\DecValTok{9}\NormalTok{, }\DecValTok{7}\NormalTok{, }\DecValTok{8}\NormalTok{, }\DecValTok{9}\NormalTok{)}
\NormalTok{drug\_B }\OtherTok{\textless{}{-}} \FunctionTok{c}\NormalTok{(}\DecValTok{6}\NormalTok{, }\DecValTok{5}\NormalTok{, }\DecValTok{7}\NormalTok{, }\DecValTok{4}\NormalTok{, }\DecValTok{6}\NormalTok{, }\DecValTok{5}\NormalTok{, }\DecValTok{6}\NormalTok{, }\DecValTok{5}\NormalTok{, }\DecValTok{6}\NormalTok{, }\DecValTok{5}\NormalTok{)}
\NormalTok{drug\_C }\OtherTok{\textless{}{-}} \FunctionTok{c}\NormalTok{(}\DecValTok{10}\NormalTok{, }\DecValTok{9}\NormalTok{, }\DecValTok{11}\NormalTok{, }\DecValTok{8}\NormalTok{, }\DecValTok{12}\NormalTok{, }\DecValTok{10}\NormalTok{, }\DecValTok{11}\NormalTok{, }\DecValTok{9}\NormalTok{, }\DecValTok{10}\NormalTok{, }\DecValTok{11}\NormalTok{)}

\CommentTok{\# Combine data into long format}
\FunctionTok{library}\NormalTok{(reshape2)}
\NormalTok{data }\OtherTok{\textless{}{-}} \FunctionTok{data.frame}\NormalTok{(patient, drug\_A, drug\_B, drug\_C)}
\NormalTok{long\_data }\OtherTok{\textless{}{-}} \FunctionTok{melt}\NormalTok{(data, }\AttributeTok{id.vars =} \StringTok{"patient"}\NormalTok{,}
                  \AttributeTok{variable.name =} \StringTok{"treatment"}\NormalTok{,}
                  \AttributeTok{value.name =} \StringTok{"bp\_reduction"}\NormalTok{)}

\CommentTok{\# Run Friedman test}
\FunctionTok{friedman.test}\NormalTok{(bp\_reduction }\SpecialCharTok{\textasciitilde{}}\NormalTok{ treatment }\SpecialCharTok{|}\NormalTok{ patient, }\AttributeTok{data =}\NormalTok{ long\_data)}
\end{Highlighting}
\end{Shaded}

\hypertarget{interpretation-12}{%
\subsection{Interpretation}\label{interpretation-12}}

\begin{Shaded}
\begin{Highlighting}[]
\CommentTok{\# Output:}
\CommentTok{\#}
\CommentTok{\# Friedman rank sum test}
\CommentTok{\#}
\CommentTok{\# data:  bp\_reduction and treatment and patient}
\CommentTok{\# Friedman chi{-}squared = 17.55, df = 2, p{-}value = 0.00015}
\end{Highlighting}
\end{Shaded}

The p-value (0.00015) is very small, so we reject the null hypothesis.\\
This suggests that at least one blood pressure medication differs significantly in its effect on systolic blood pressure reduction.

Post-hoc pairwise comparisons (e.g., Wilcoxon signed-rank tests with Bonferroni correction) can help determine which drugs differ.

\hypertarget{permutation-test}{%
\section{Permutation Test}\label{permutation-test}}

\hypertarget{when-to-use-8}{%
\subsection{When to Use}\label{when-to-use-8}}

\begin{itemize}
\tightlist
\item
  Use the permutation test when you want to compare two or more groups without making assumptions about the population distribution.
\item
  It is especially useful when:

  \begin{itemize}
  \tightlist
  \item
    Sample sizes are small
  \item
    The data are non-normal
  \item
    You want an exact p-value by resampling
  \end{itemize}
\item
  It can be applied to:

  \begin{itemize}
  \tightlist
  \item
    Differences in means or medians
  \item
    Correlation coefficients
  \item
    Regression slopes
  \item
    And many other statistics
  \end{itemize}
\end{itemize}

\hypertarget{assumptions-9}{%
\subsection{Assumptions}\label{assumptions-9}}

\begin{itemize}
\tightlist
\item
  Observations are independent within and between groups
\item
  The test statistic (e.g., difference in means) is valid under label exchangeability
\item
  The null hypothesis assumes group labels do not matter, and any observed difference is due to chance
\end{itemize}

\hypertarget{hypotheses-for-two-groups}{%
\subsection{Hypotheses (for Two Groups)}\label{hypotheses-for-two-groups}}

\begin{itemize}
\tightlist
\item
  \(H_0\): The group labels are exchangeable (i.e., no real effect)
\item
  \(H_a\): The observed test statistic is unlikely under random reassignment
\end{itemize}

Let \(T_{\text{obs}}\) be the observed test statistic (e.g., difference in means).

\hypertarget{test-statistic-6}{%
\subsection{Test Statistic}\label{test-statistic-6}}

The most common statistic is the difference in means:

\[
T_{\text{obs}} = \bar{X}_1 - \bar{X}_2
\]

\begin{itemize}
\tightlist
\item
  We then compute \(T^*\) under many random permutations of group labels.
\item
  The p-value is:
\end{itemize}

\[
p = \frac{\text{Number of } T^* \geq |T_{\text{obs}}|}{\text{Number of permutations}}
\]

\hypertarget{example-testing-if-a-new-drug-reduces-crp-levels-compared-to-placebo}{%
\subsection{Example: Testing if a New Drug Reduces CRP Levels Compared to Placebo}\label{example-testing-if-a-new-drug-reduces-crp-levels-compared-to-placebo}}

A small trial tests whether a new anti-inflammatory drug reduces C-reactive protein (CRP) levels compared to a placebo.

\begin{Shaded}
\begin{Highlighting}[]
\CommentTok{\# CRP levels (mg/L)}
\NormalTok{drug }\OtherTok{\textless{}{-}} \FunctionTok{c}\NormalTok{(}\FloatTok{3.2}\NormalTok{, }\FloatTok{2.8}\NormalTok{, }\FloatTok{3.1}\NormalTok{, }\FloatTok{2.6}\NormalTok{, }\FloatTok{2.9}\NormalTok{)}
\NormalTok{placebo }\OtherTok{\textless{}{-}} \FunctionTok{c}\NormalTok{(}\FloatTok{4.1}\NormalTok{, }\FloatTok{4.4}\NormalTok{, }\FloatTok{4.0}\NormalTok{, }\FloatTok{3.9}\NormalTok{, }\FloatTok{4.2}\NormalTok{)}

\CommentTok{\# Combine data}
\NormalTok{group }\OtherTok{\textless{}{-}} \FunctionTok{c}\NormalTok{(}\FunctionTok{rep}\NormalTok{(}\StringTok{"drug"}\NormalTok{, }\DecValTok{5}\NormalTok{), }\FunctionTok{rep}\NormalTok{(}\StringTok{"placebo"}\NormalTok{, }\DecValTok{5}\NormalTok{))}
\NormalTok{crp }\OtherTok{\textless{}{-}} \FunctionTok{c}\NormalTok{(drug, placebo)}
\NormalTok{data }\OtherTok{\textless{}{-}} \FunctionTok{data.frame}\NormalTok{(group, crp)}

\CommentTok{\# Observed difference in means}
\NormalTok{obs\_diff }\OtherTok{\textless{}{-}} \FunctionTok{mean}\NormalTok{(data}\SpecialCharTok{$}\NormalTok{crp[data}\SpecialCharTok{$}\NormalTok{group }\SpecialCharTok{==} \StringTok{"placebo"}\NormalTok{]) }\SpecialCharTok{{-}}
            \FunctionTok{mean}\NormalTok{(data}\SpecialCharTok{$}\NormalTok{crp[data}\SpecialCharTok{$}\NormalTok{group }\SpecialCharTok{==} \StringTok{"drug"}\NormalTok{])}

\CommentTok{\# Permutation test}
\FunctionTok{set.seed}\NormalTok{(}\DecValTok{42}\NormalTok{)}
\NormalTok{n\_perm }\OtherTok{\textless{}{-}} \DecValTok{10000}
\NormalTok{perm\_diffs }\OtherTok{\textless{}{-}} \FunctionTok{numeric}\NormalTok{(n\_perm)}

\ControlFlowTok{for}\NormalTok{ (i }\ControlFlowTok{in} \DecValTok{1}\SpecialCharTok{:}\NormalTok{n\_perm) \{}
\NormalTok{  perm\_labels }\OtherTok{\textless{}{-}} \FunctionTok{sample}\NormalTok{(data}\SpecialCharTok{$}\NormalTok{group)}
\NormalTok{  perm\_diffs[i] }\OtherTok{\textless{}{-}} \FunctionTok{mean}\NormalTok{(data}\SpecialCharTok{$}\NormalTok{crp[perm\_labels }\SpecialCharTok{==} \StringTok{"placebo"}\NormalTok{]) }\SpecialCharTok{{-}}
                   \FunctionTok{mean}\NormalTok{(data}\SpecialCharTok{$}\NormalTok{crp[perm\_labels }\SpecialCharTok{==} \StringTok{"drug"}\NormalTok{])}
\NormalTok{\}}

\CommentTok{\# P{-}value}
\NormalTok{p\_val }\OtherTok{\textless{}{-}} \FunctionTok{mean}\NormalTok{(}\FunctionTok{abs}\NormalTok{(perm\_diffs) }\SpecialCharTok{\textgreater{}=} \FunctionTok{abs}\NormalTok{(obs\_diff))}
\NormalTok{p\_val}
\end{Highlighting}
\end{Shaded}

\hypertarget{interpretation-13}{%
\subsection{Interpretation}\label{interpretation-13}}

\begin{Shaded}
\begin{Highlighting}[]
\CommentTok{\# Example output:}
\CommentTok{\# [1] 0.0013}
\end{Highlighting}
\end{Shaded}

The permutation test gives a p-value of 0.0013, which is highly significant.\\
This suggests that the new drug significantly reduces CRP levels compared to placebo.\\
Because this method relies on randomization, it does not require normality and is robust even with small sample sizes.

\hypertarget{spearman-rank-correlation}{%
\section{Spearman Rank Correlation}\label{spearman-rank-correlation}}

\hypertarget{when-to-use-9}{%
\subsection{When to Use}\label{when-to-use-9}}

\begin{itemize}
\tightlist
\item
  The Spearman rank correlation is a nonparametric measure of the monotonic relationship between two continuous or ordinal variables.
\item
  Use this test when:

  \begin{itemize}
  \tightlist
  \item
    Data are not normally distributed
  \item
    There are outliers that could distort Pearson's correlation
  \item
    Variables are ordinal or have a non-linear but monotonic trend
  \end{itemize}
\end{itemize}

\hypertarget{assumptions-10}{%
\subsection{Assumptions}\label{assumptions-10}}

\begin{itemize}
\tightlist
\item
  The two variables are continuous or ordinal
\item
  Observations are independent
\item
  The relationship is monotonic (as one variable increases, the other tends to increase or decrease consistently)
\end{itemize}

\hypertarget{hypotheses-11}{%
\subsection{Hypotheses}\label{hypotheses-11}}

\begin{itemize}
\tightlist
\item
  \(H_0\): There is no monotonic association between the two variables\\
  \[ \rho_s = 0 \]
\item
  \(H_a\): There is a monotonic association between the two variables\\
  \[ \rho_s \neq 0 \]
\end{itemize}

\hypertarget{formula}{%
\subsection{Formula}\label{formula}}

The Spearman rank correlation coefficient \(\rho_s\) is given by:

\[
\rho_s = 1 - \frac{6 \sum d_i^2}{n(n^2 - 1)}
\]

Where:
- \(d_i\) = difference between the ranks of each pair
- \(n\) = number of observations

\hypertarget{example-correlation-between-sodium-intake-and-systolic-blood-pressure}{%
\subsection{Example: Correlation Between Sodium Intake and Systolic Blood Pressure}\label{example-correlation-between-sodium-intake-and-systolic-blood-pressure}}

A nutrition researcher collects data on daily sodium intake (mg) and systolic blood pressure (mmHg) in 12 adult patients to examine whether a monotonic relationship exists.

\begin{Shaded}
\begin{Highlighting}[]
\CommentTok{\# Sample data}
\NormalTok{sodium }\OtherTok{\textless{}{-}} \FunctionTok{c}\NormalTok{(}\DecValTok{2300}\NormalTok{, }\DecValTok{2500}\NormalTok{, }\DecValTok{1800}\NormalTok{, }\DecValTok{3000}\NormalTok{, }\DecValTok{2700}\NormalTok{, }\DecValTok{2200}\NormalTok{, }\DecValTok{3100}\NormalTok{, }\DecValTok{2900}\NormalTok{, }\DecValTok{2400}\NormalTok{, }\DecValTok{2600}\NormalTok{, }\DecValTok{2000}\NormalTok{, }\DecValTok{2800}\NormalTok{)}
\NormalTok{sbp }\OtherTok{\textless{}{-}} \FunctionTok{c}\NormalTok{(}\DecValTok{122}\NormalTok{, }\DecValTok{130}\NormalTok{, }\DecValTok{115}\NormalTok{, }\DecValTok{140}\NormalTok{, }\DecValTok{136}\NormalTok{, }\DecValTok{120}\NormalTok{, }\DecValTok{142}\NormalTok{, }\DecValTok{138}\NormalTok{, }\DecValTok{126}\NormalTok{, }\DecValTok{134}\NormalTok{, }\DecValTok{118}\NormalTok{, }\DecValTok{135}\NormalTok{)}

\CommentTok{\# Data frame}
\NormalTok{df }\OtherTok{\textless{}{-}} \FunctionTok{data.frame}\NormalTok{(sodium, sbp)}

\CommentTok{\# Spearman rank correlation}
\FunctionTok{cor.test}\NormalTok{(df}\SpecialCharTok{$}\NormalTok{sodium, df}\SpecialCharTok{$}\NormalTok{sbp, }\AttributeTok{method =} \StringTok{"spearman"}\NormalTok{)}
\end{Highlighting}
\end{Shaded}

\hypertarget{interpretation-14}{%
\subsection{Interpretation}\label{interpretation-14}}

\begin{Shaded}
\begin{Highlighting}[]
\CommentTok{\# Example output:}
\CommentTok{\# Spearman\textquotesingle{}s rank correlation rho}
\CommentTok{\# data:  df$sodium and df$sbp}
\CommentTok{\# S = 34, p{-}value = 0.004}
\CommentTok{\# alternative hypothesis: true rho is not equal to 0}
\CommentTok{\# sample estimates:}
\CommentTok{\#       rho }
\CommentTok{\# 0.8818182 }
\end{Highlighting}
\end{Shaded}

The Spearman correlation coefficient is 0.88 with a p-value of 0.004, indicating a strong positive monotonic relationship between sodium intake and systolic blood pressure. As sodium intake increases, blood pressure tends to increase as well.

\hypertarget{nonparametric-regression}{%
\section{Nonparametric Regression}\label{nonparametric-regression}}

\hypertarget{when-to-use-10}{%
\subsection{When to Use}\label{when-to-use-10}}

\begin{itemize}
\tightlist
\item
  Nonparametric regression is used to estimate the relationship between variables when:

  \begin{itemize}
  \tightlist
  \item
    The form of the relationship is unknown or nonlinear
  \item
    No assumption of normality or linearity is appropriate
  \item
    You want to model smooth trends without specifying a parametric equation
  \end{itemize}
\end{itemize}

Common methods include:
- LOESS (Locally Estimated Scatterplot Smoothing)
- Kernel smoothing

\hypertarget{assumptions-11}{%
\subsection{Assumptions}\label{assumptions-11}}

\begin{itemize}
\tightlist
\item
  Data are independent
\item
  The relationship between variables is smooth, but not necessarily linear
\item
  No strict distributional assumptions (e.g., normality) are required
\end{itemize}

\hypertarget{biomedical-example-modeling-recovery-time-based-on-age-after-surgery}{%
\subsection{Biomedical Example: Modeling Recovery Time Based on Age After Surgery}\label{biomedical-example-modeling-recovery-time-based-on-age-after-surgery}}

A hospital collects data from 30 patients recovering from minor orthopedic surgery. The goal is to explore whether there's a smooth trend between patient age and days to full recovery, without assuming a linear model.

\begin{Shaded}
\begin{Highlighting}[]
\CommentTok{\# Simulated data}
\FunctionTok{set.seed}\NormalTok{(}\DecValTok{123}\NormalTok{)}
\NormalTok{age }\OtherTok{\textless{}{-}} \FunctionTok{seq}\NormalTok{(}\DecValTok{20}\NormalTok{, }\DecValTok{70}\NormalTok{, }\AttributeTok{length.out =} \DecValTok{30}\NormalTok{)}
\NormalTok{recovery\_days }\OtherTok{\textless{}{-}} \FunctionTok{round}\NormalTok{(}\DecValTok{30} \SpecialCharTok{{-}} \FloatTok{0.3} \SpecialCharTok{*}\NormalTok{ age }\SpecialCharTok{+} \DecValTok{5} \SpecialCharTok{*} \FunctionTok{sin}\NormalTok{(age }\SpecialCharTok{/} \DecValTok{8}\NormalTok{) }\SpecialCharTok{+} \FunctionTok{rnorm}\NormalTok{(}\DecValTok{30}\NormalTok{, }\DecValTok{0}\NormalTok{, }\DecValTok{2}\NormalTok{), }\DecValTok{1}\NormalTok{)}

\CommentTok{\# Data frame}
\NormalTok{df }\OtherTok{\textless{}{-}} \FunctionTok{data.frame}\NormalTok{(age, recovery\_days)}

\CommentTok{\# LOESS fit}
\FunctionTok{library}\NormalTok{(ggplot2)}

\FunctionTok{ggplot}\NormalTok{(df, }\FunctionTok{aes}\NormalTok{(}\AttributeTok{x =}\NormalTok{ age, }\AttributeTok{y =}\NormalTok{ recovery\_days)) }\SpecialCharTok{+}
  \FunctionTok{geom\_point}\NormalTok{(}\AttributeTok{color =} \StringTok{"steelblue"}\NormalTok{) }\SpecialCharTok{+}
  \FunctionTok{geom\_smooth}\NormalTok{(}\AttributeTok{method =} \StringTok{"loess"}\NormalTok{, }\AttributeTok{se =} \ConstantTok{FALSE}\NormalTok{, }\AttributeTok{color =} \StringTok{"darkred"}\NormalTok{, }\AttributeTok{span =} \FloatTok{0.75}\NormalTok{) }\SpecialCharTok{+}
  \FunctionTok{labs}\NormalTok{(}\AttributeTok{title =} \StringTok{"Nonparametric Regression: Recovery Days vs Age"}\NormalTok{,}
       \AttributeTok{x =} \StringTok{"Age (years)"}\NormalTok{,}
       \AttributeTok{y =} \StringTok{"Recovery Time (days)"}\NormalTok{)}
\end{Highlighting}
\end{Shaded}

\hypertarget{interpretation-15}{%
\subsection{Interpretation}\label{interpretation-15}}

The plot shows a smooth, nonlinear relationship between age and recovery time. Younger patients recover slightly faster on average, but the effect is not perfectly linear. LOESS reveals subtle curvature --- likely due to biological variability that a linear model would miss.

\hypertarget{simple-linear-regression}{%
\chapter{Simple Linear Regression}\label{simple-linear-regression}}

\begin{Shaded}
\begin{Highlighting}[]
\FunctionTok{library}\NormalTok{(IntroStats)}
\end{Highlighting}
\end{Shaded}

\hypertarget{motivating-example-1}{%
\section{Motivating Example}\label{motivating-example-1}}

Regression is a statistical method employed to elucidate the connection
between a response variable and one or multiple predictor variables.
While correlation assesses the relationship between two numerical random
variables, it falls short in capturing the directional trends when one
variable undergoes increases or decreases. Various regression models are
developed to address different questions based on the underlying
assumptions regarding the trending relationship. Linear regression model
is the simplest regression where a linear relationship is assumed
between the response variable \(y\) and the predictor variable \(x\).

A research study aims to predict intra-abdominal adipose tissue (\(y\)) in
men based on waist circumference (\(x\)). Accurate measurement of
intra-abdominal adipose tissue requires costly and inconvenient CT
scans. In contrast, measuring waist circumference is a straightforward
and cost-effective procedure.

\hypertarget{data-visualization-1}{%
\subsection{Data Visualization}\label{data-visualization-1}}

The scatter plot displays the data, and the trending line (lowess
smoother) suggests a reasonable linear relationship between these two
variables.

Assume a linear regression model is meaningful, how do we assess the
performance of the linear regression model?

\begin{Shaded}
\begin{Highlighting}[]
\DocumentationTok{\#\#\#\#\#\#\#\#\#\#\#\#\#\#\#\#\#\#\#\#\#\#\#\#\#\#\#\#\#\#}
\CommentTok{\#(1) Motivating Example (Adipose Tissue vs waist)}
\DocumentationTok{\#\#\#\#\#\#\#\#\#\#\#\#\#\#\#\#\#\#\#\#\#\#\#\#\#\#\#\#\#\#}
\NormalTok{x }\OtherTok{=} \FunctionTok{c}\NormalTok{(}\FloatTok{74.75}\NormalTok{,}\FloatTok{72.60}\NormalTok{,}\FloatTok{81.80}\NormalTok{,}\FloatTok{83.95}\NormalTok{,}\FloatTok{74.65}\NormalTok{,}\FloatTok{71.85}\NormalTok{,}\FloatTok{80.90}\NormalTok{,}
\FloatTok{83.40}\NormalTok{,}\FloatTok{63.50}\NormalTok{,}\FloatTok{73.20}\NormalTok{,}\FloatTok{71.90}\NormalTok{,}\FloatTok{75.00}\NormalTok{,}\FloatTok{73.10}\NormalTok{,}\FloatTok{79.00}\NormalTok{,}
\FloatTok{77.00}\NormalTok{,}\FloatTok{68.85}\NormalTok{,}\FloatTok{75.95}\NormalTok{,}\FloatTok{74.15}\NormalTok{,}\FloatTok{73.80}\NormalTok{,}\FloatTok{75.90}\NormalTok{,}\FloatTok{76.85}\NormalTok{,}
\FloatTok{80.90}\NormalTok{,}\FloatTok{79.90}\NormalTok{,}\FloatTok{89.20}\NormalTok{,}\FloatTok{82.00}\NormalTok{,}\FloatTok{92.00}\NormalTok{,}\FloatTok{86.60}\NormalTok{,}\FloatTok{80.50}\NormalTok{,}
\FloatTok{86.00}\NormalTok{,}\FloatTok{82.50}\NormalTok{,}\FloatTok{83.50}\NormalTok{,}\FloatTok{88.10}\NormalTok{,}\FloatTok{90.80}\NormalTok{,}\FloatTok{89.40}\NormalTok{,}\FloatTok{102.00}\NormalTok{,}
\FloatTok{94.50}\NormalTok{,}\FloatTok{91.00}\NormalTok{,}
\FloatTok{103.00}\NormalTok{,}\FloatTok{80.00}\NormalTok{,}\FloatTok{79.00}\NormalTok{,}\FloatTok{83.50}\NormalTok{,}\FloatTok{76.00}\NormalTok{,}\FloatTok{80.50}\NormalTok{,}\FloatTok{86.50}\NormalTok{,}
\FloatTok{83.00}\NormalTok{,}\FloatTok{107.10}\NormalTok{,}\FloatTok{94.30}\NormalTok{,}\FloatTok{94.50}\NormalTok{,}\FloatTok{79.70}\NormalTok{,}\FloatTok{79.30}\NormalTok{,}\FloatTok{89.80}\NormalTok{,}
\FloatTok{83.80}\NormalTok{,}\FloatTok{85.20}\NormalTok{,}\FloatTok{75.50}\NormalTok{,}\FloatTok{78.40}\NormalTok{,}\FloatTok{78.60}\NormalTok{,}\FloatTok{87.80}\NormalTok{,}\FloatTok{86.30}\NormalTok{,}
\FloatTok{85.50}\NormalTok{,}\FloatTok{83.70}\NormalTok{,}\FloatTok{77.60}\NormalTok{,}\FloatTok{84.90}\NormalTok{,}\FloatTok{79.80}\NormalTok{,}\FloatTok{108.30}\NormalTok{,}\FloatTok{119.60}\NormalTok{,}
\FloatTok{119.90}\NormalTok{,}\FloatTok{96.50}\NormalTok{,}\FloatTok{105.50}\NormalTok{,}\FloatTok{105.00}\NormalTok{,}\FloatTok{107.00}\NormalTok{,}\FloatTok{107.00}\NormalTok{,}
\FloatTok{101.00}\NormalTok{,}\FloatTok{97.00}\NormalTok{,}\FloatTok{100.00}\NormalTok{,}
\FloatTok{108.00}\NormalTok{,}\FloatTok{100.00}\NormalTok{,}\FloatTok{103.00}\NormalTok{,}\FloatTok{104.00}\NormalTok{,}\FloatTok{106.00}\NormalTok{,}\FloatTok{109.00}\NormalTok{,}
\FloatTok{103.50}\NormalTok{,}\FloatTok{110.00}\NormalTok{,}\FloatTok{110.00}\NormalTok{,}\FloatTok{112.00}\NormalTok{,}\FloatTok{108.50}\NormalTok{,}\FloatTok{104.00}\NormalTok{,}
\FloatTok{111.00}\NormalTok{,}\FloatTok{108.50}\NormalTok{,}\FloatTok{121.00}\NormalTok{,}\FloatTok{109.00}\NormalTok{,}\FloatTok{97.50}\NormalTok{,}\FloatTok{105.50}\NormalTok{,}
\FloatTok{98.00}\NormalTok{,}\FloatTok{94.50}\NormalTok{,}\FloatTok{97.00}\NormalTok{,}\FloatTok{105.00}\NormalTok{,}\FloatTok{106.00}\NormalTok{,}\FloatTok{99.00}\NormalTok{,}\FloatTok{91.00}\NormalTok{,}
\FloatTok{102.50}\NormalTok{,}\FloatTok{106.00}\NormalTok{,}\FloatTok{109.10}\NormalTok{,}\FloatTok{115.00}\NormalTok{,}\FloatTok{101.00}\NormalTok{,}\FloatTok{100.10}\NormalTok{,}
\FloatTok{93.30}\NormalTok{,}\FloatTok{101.80}\NormalTok{,}\FloatTok{107.90}\NormalTok{,}\FloatTok{108.50}\NormalTok{)}

\NormalTok{y}\OtherTok{\textless{}{-}} \FunctionTok{c}\NormalTok{(}\FloatTok{25.72}\NormalTok{,}\FloatTok{25.89}\NormalTok{,}\FloatTok{42.60}\NormalTok{,}\FloatTok{42.80}\NormalTok{,}\FloatTok{29.84}\NormalTok{,}\FloatTok{21.68}\NormalTok{,}\FloatTok{29.08}\NormalTok{,}
\FloatTok{32.98}\NormalTok{,}\FloatTok{11.44}\NormalTok{,}\FloatTok{32.22}\NormalTok{,}\FloatTok{28.32}\NormalTok{,}\FloatTok{43.86}\NormalTok{,}\FloatTok{38.21}\NormalTok{,}\FloatTok{42.48}\NormalTok{,}
\FloatTok{30.96}\NormalTok{,}\FloatTok{55.78}\NormalTok{,}\FloatTok{43.78}\NormalTok{,}\FloatTok{33.41}\NormalTok{,}\FloatTok{43.35}\NormalTok{,}\FloatTok{29.31}\NormalTok{,}\FloatTok{36.60}\NormalTok{,}
\FloatTok{40.25}\NormalTok{,}\FloatTok{35.43}\NormalTok{,}\FloatTok{60.09}\NormalTok{,}\FloatTok{45.84}\NormalTok{,}\FloatTok{70.40}\NormalTok{,}\FloatTok{83.45}\NormalTok{,}\FloatTok{84.30}\NormalTok{,}
\FloatTok{78.89}\NormalTok{,}\FloatTok{64.75}\NormalTok{,}\FloatTok{72.56}\NormalTok{,}\FloatTok{89.31}\NormalTok{,}\FloatTok{78.94}\NormalTok{,}\FloatTok{83.55}\NormalTok{,}\FloatTok{127.00}\NormalTok{,}
\FloatTok{121.00}\NormalTok{,}\FloatTok{107.00}\NormalTok{,}
\FloatTok{129.00}\NormalTok{,}\FloatTok{74.02}\NormalTok{,}\FloatTok{55.48}\NormalTok{,}\FloatTok{73.13}\NormalTok{,}\FloatTok{50.50}\NormalTok{,}\FloatTok{50.88}\NormalTok{,}\FloatTok{140.00}\NormalTok{,}
\FloatTok{96.54}\NormalTok{,}\FloatTok{118.00}\NormalTok{,}\FloatTok{107.00}\NormalTok{,}\FloatTok{123.00}\NormalTok{,}\FloatTok{65.92}\NormalTok{,}\FloatTok{81.29}\NormalTok{,}\FloatTok{111.00}\NormalTok{,}
\FloatTok{90.73}\NormalTok{,}\FloatTok{133.00}\NormalTok{,}\FloatTok{41.90}\NormalTok{,}\FloatTok{41.71}\NormalTok{,}\FloatTok{58.16}\NormalTok{,}\FloatTok{88.85}\NormalTok{,}\FloatTok{155.00}\NormalTok{,}
\FloatTok{70.77}\NormalTok{,}\FloatTok{75.08}\NormalTok{,}\FloatTok{57.05}\NormalTok{,}\FloatTok{99.73}\NormalTok{,}\FloatTok{27.96}\NormalTok{,}\FloatTok{123.00}\NormalTok{,}\FloatTok{90.41}\NormalTok{,}
\FloatTok{106.00}\NormalTok{,}\FloatTok{144.00}\NormalTok{,}\FloatTok{121.00}\NormalTok{,}\FloatTok{97.13}\NormalTok{,}\FloatTok{166.00}\NormalTok{,}\FloatTok{87.99}\NormalTok{,}\FloatTok{154.00}\NormalTok{,}
\FloatTok{100.00}\NormalTok{,}\FloatTok{123.00}\NormalTok{,}
\FloatTok{217.00}\NormalTok{,}\FloatTok{140.00}\NormalTok{,}\FloatTok{109.00}\NormalTok{,}\FloatTok{127.00}\NormalTok{,}\FloatTok{112.00}\NormalTok{,}\FloatTok{192.00}\NormalTok{,}
\FloatTok{132.00}\NormalTok{,}\FloatTok{126.00}\NormalTok{,}\FloatTok{153.00}\NormalTok{,}\FloatTok{158.00}\NormalTok{,}\FloatTok{183.00}\NormalTok{,}\FloatTok{184.00}\NormalTok{,}
\FloatTok{121.00}\NormalTok{,}\FloatTok{159.00}\NormalTok{,}\FloatTok{245.00}\NormalTok{,}\FloatTok{137.00}\NormalTok{,}\FloatTok{165.00}\NormalTok{,}\FloatTok{152.00}\NormalTok{,}
\FloatTok{181.00}\NormalTok{,}\FloatTok{80.95}\NormalTok{,}\FloatTok{137.00}\NormalTok{,}\FloatTok{125.00}\NormalTok{,}\FloatTok{241.00}\NormalTok{,}\FloatTok{134.00}\NormalTok{,}
\FloatTok{150.00}\NormalTok{,}\FloatTok{198.00}\NormalTok{,}\FloatTok{151.00}\NormalTok{,}\FloatTok{229.00}\NormalTok{,}\FloatTok{253.00}\NormalTok{,}\FloatTok{188.00}\NormalTok{,}
\FloatTok{124.00}\NormalTok{,}\FloatTok{62.20}\NormalTok{,}\FloatTok{133.00}\NormalTok{,}\FloatTok{208.00}\NormalTok{,}\FloatTok{208.00}\NormalTok{)}

\NormalTok{dat }\OtherTok{\textless{}{-}} \FunctionTok{as.data.frame}\NormalTok{(}\FunctionTok{cbind}\NormalTok{(x,y))}
\FunctionTok{plot}\NormalTok{(x, y, }\AttributeTok{xlab=}\StringTok{"Waist Circumference (cm)"}\NormalTok{, }\AttributeTok{ylab=}\StringTok{"Adipose Tissue (AT)"}\NormalTok{)}

\CommentTok{\#Superimpose the trending smoother line}
\FunctionTok{lines}\NormalTok{(}\FunctionTok{lowess}\NormalTok{(y }\SpecialCharTok{\textasciitilde{}}\NormalTok{ x), }\AttributeTok{lty=}\DecValTok{2}\NormalTok{, }\AttributeTok{lwd=}\DecValTok{2}\NormalTok{)}
\end{Highlighting}
\end{Shaded}

\includegraphics{StatsTB_files/figure-latex/unnamed-chunk-336-1.pdf}

\hypertarget{the-linear-regression-line}{%
\subsection{The Linear Regression Line}\label{the-linear-regression-line}}

The linear equation can be written as \[
    y = \beta_0 + \beta_1 x
    \]\\
\(x\) is the \textbf{independent variable} and \(y\) is \textbf{dependent variable}.

\begin{figure}
\centering
\includegraphics{https://i.ibb.co/hmnNssj/linear-equation1.png}
\caption{Linear equations}
\end{figure}

\begin{figure}
\centering
\includegraphics{https://i.ibb.co/TmNJ2zm/linear-equation2.png}
\caption{Linear equations}
\end{figure}

The key distinction between a linear equation and linear regression is
that in a linear equation relationship, all data points \((x, y)\) must
lie on the same straight line, whereas in linear regression, this
requirement does not apply.

How to find the best line to fit the data points? The method is called
\textbf{ordinal least square}.

\begin{figure}
\centering
\includegraphics{https://i.ibb.co/swCNcZ8/LS.png}
\caption{Least Square Estimate (LSE) for Linear
Regression}
\end{figure}

\hypertarget{fit-a-linear-regression-line}{%
\subsection{Fit a Linear Regression Line}\label{fit-a-linear-regression-line}}

\begin{Shaded}
\begin{Highlighting}[]
        \DocumentationTok{\#\#\#\#\#\#\#\#\#\#\#\#\#\#\#\#\#\#\#\#\#\#\#\#\#\#\#\#\#\#\#\#\#\#\#\#}
        \CommentTok{\# (2) Fit a linear regression line}
        \DocumentationTok{\#\#\#\#\#\#\#\#\#\#\#\#\#\#\#\#\#\#\#\#\#\#\#\#\#\#\#\#\#\#\#\#\#\#\#\#}
\NormalTok{        model }\OtherTok{\textless{}{-}} \FunctionTok{OLR}\NormalTok{(x, y)}
\NormalTok{        model}
\end{Highlighting}
\end{Shaded}

\begin{verbatim}
## $Sxx
## [1] 19855.76
## 
## $Sxy
## [1] 68678.28
## 
## $beta0
## [1] -215.9815
## 
## $beta1
## [1] 3.458859
## 
## $SST
## [1] 354530.5
## 
## $SSR
## [1] 237548.5
## 
## $SSE
## [1] 116982
## 
## $r2
## [1] 0.6700369
## 
## $r2adj
## [1] 0.6669531
## 
## $anova
## Analysis of Variance Table
## 
## Response: y
##            Df Sum Sq Mean Sq F value    Pr(>F)    
## x           1 237549  237549  217.28 < 2.2e-16 ***
## Residuals 107 116982    1093                      
## ---
## Signif. codes:  0 '***' 0.001 '**' 0.01 '*' 0.05 '.' 0.1 ' ' 1
## 
## $sigma.hat
## [1] 33.06493
## 
## $se.beta1
## [1] 0.2346521
## 
## $t0
## [1] 14.74038
## 
## $p
## [1] 0
## 
## $F0
## [1] 217.2787
## 
## $p.F.test
## [1] 1.618607e-27
## 
## $CI
## [1] 2.993689 3.924030
\end{verbatim}

Let's revisit the motivating example. It's calculated as
\[S_{xx} = 19855.76\] and \[S_{xy} = 68678.28\], and
\[\beta_0=-215.9815, \beta_1=3.459\]. So the regression line is
\[\hat{y} = -216 + 3.46x\].

Figure shows the super imposed fitted line to the scatter plot.

\begin{Shaded}
\begin{Highlighting}[]
        \FunctionTok{plot}\NormalTok{(x, y, }\AttributeTok{xlab=}\StringTok{"Waist Circumference (cm)"}\NormalTok{, }\AttributeTok{ylab=}\StringTok{"Adipose Tissue (AT)"}\NormalTok{)}
        \CommentTok{\#Superimpose the trending smoother line}
        \FunctionTok{lines}\NormalTok{(}\FunctionTok{lowess}\NormalTok{(y }\SpecialCharTok{\textasciitilde{}}\NormalTok{ x), }\AttributeTok{lty=}\DecValTok{2}\NormalTok{, }\AttributeTok{lwd=}\DecValTok{2}\NormalTok{)}
\NormalTok{        x.LR }\OtherTok{\textless{}{-}} \FunctionTok{seq}\NormalTok{(}\DecValTok{50}\NormalTok{, }\DecValTok{150}\NormalTok{, }\AttributeTok{by=}\DecValTok{1}\NormalTok{)}
\NormalTok{        y.LR }\OtherTok{\textless{}{-}}\NormalTok{ model}\SpecialCharTok{$}\NormalTok{beta0}\SpecialCharTok{+}\NormalTok{model}\SpecialCharTok{$}\NormalTok{beta1}\SpecialCharTok{*}\NormalTok{x.LR }\CommentTok{\#Fitted regression line}
        \FunctionTok{lines}\NormalTok{(x.LR, y.LR, }\AttributeTok{col=}\DecValTok{2}\NormalTok{, }\AttributeTok{lwd=}\DecValTok{2}\NormalTok{)}
        \FunctionTok{legend}\NormalTok{(}\StringTok{"topleft"}\NormalTok{, }\FunctionTok{c}\NormalTok{(}\StringTok{"Lowess Smoother"}\NormalTok{, }\StringTok{"Linear Regression Line"}\NormalTok{), }
               \AttributeTok{col=}\DecValTok{1}\SpecialCharTok{:}\DecValTok{2}\NormalTok{, }\AttributeTok{lwd=}\FunctionTok{c}\NormalTok{(}\DecValTok{2}\NormalTok{,}\DecValTok{2}\NormalTok{), }\AttributeTok{bty=}\StringTok{"n"}\NormalTok{)}
\end{Highlighting}
\end{Shaded}

\includegraphics{StatsTB_files/figure-latex/unnamed-chunk-338-1.pdf}

In R, use the function \textbackslash verb\textbar lm()\textbar{} to fit the linear regression.

\begin{Shaded}
\begin{Highlighting}[]
        \DocumentationTok{\#\#\#\#\#\#\#\#\#\#\#\#\#\#\#\#\#\#\#\#\#\#\#\#\#\#\#\#\#\#\#\#\#\#\#\#}
        \CommentTok{\# (3) Linear Regression using lm()}
        \DocumentationTok{\#\#\#\#\#\#\#\#\#\#\#\#\#\#\#\#\#\#\#\#\#\#\#\#\#\#\#\#\#\#\#\#\#\#\#\#        }
\NormalTok{        LinearReg }\OtherTok{\textless{}{-}} \FunctionTok{lm}\NormalTok{(y }\SpecialCharTok{\textasciitilde{}}\NormalTok{ x) }\CommentTok{\#Create the linear regression        }
        \FunctionTok{summary}\NormalTok{(LinearReg)     }\CommentTok{\#Review the results      }
\end{Highlighting}
\end{Shaded}

\begin{verbatim}
## 
## Call:
## lm(formula = y ~ x)
## 
## Residuals:
##      Min       1Q   Median       3Q      Max 
## -107.288  -19.143   -2.939   16.376   90.342 
## 
## Coefficients:
##              Estimate Std. Error t value Pr(>|t|)    
## (Intercept) -215.9815    21.7963  -9.909   <2e-16 ***
## x              3.4589     0.2347  14.740   <2e-16 ***
## ---
## Signif. codes:  0 '***' 0.001 '**' 0.01 '*' 0.05 '.' 0.1 ' ' 1
## 
## Residual standard error: 33.06 on 107 degrees of freedom
## Multiple R-squared:   0.67,  Adjusted R-squared:  0.667 
## F-statistic: 217.3 on 1 and 107 DF,  p-value: < 2.2e-16
\end{verbatim}

\hypertarget{fitted-values-and-regression-line}{%
\subsection{Fitted Values and Regression Line}\label{fitted-values-and-regression-line}}

\begin{Shaded}
\begin{Highlighting}[]
\NormalTok{        y\_hat }\OtherTok{\textless{}{-}}\NormalTok{ LinearReg}\SpecialCharTok{$}\NormalTok{fitted.values        }
        \FunctionTok{plot}\NormalTok{(x, y)      }\CommentTok{\#Scatter plot}
        \FunctionTok{lines}\NormalTok{(x, y\_hat, }\AttributeTok{col=}\DecValTok{2}\NormalTok{) }\CommentTok{\#Regression line}
\end{Highlighting}
\end{Shaded}

\includegraphics{StatsTB_files/figure-latex/unnamed-chunk-340-1.pdf}

\hypertarget{evaluating-and-interpreting-the-linear-regression}{%
\section{Evaluating and Interpreting the Linear Regression}\label{evaluating-and-interpreting-the-linear-regression}}

The parameter of most interest in a linear regression model is
\(\beta_1\), which reflects the rate of change in \(y\) as \(x\) increases.
When \(\beta_1 = 0\), it indicates the slope is 0 and \(y\) doesn't change
with \(x\). In other words, \(x\) cannot explain or predict \(y\).

\hypertarget{deviations-and-sums-of-squares}{%
\subsection{Deviations and Sums of Squares}\label{deviations-and-sums-of-squares}}

\hypertarget{the-total-deviation}{%
\subsubsection{The Total Deviation}\label{the-total-deviation}}

The Total Deviation is defined as \(y_i-\bar{y}\), which is the total
amount of deviation from the observed value to the sample mean.

\hypertarget{the-explained-deviation}{%
\subsubsection{The Explained Deviation}\label{the-explained-deviation}}

The Explained Deviation is defined as \(\hat{y}_i-\bar{y}\)

\hypertarget{the-unexplained-deviation}{%
\subsubsection{The Unexplained Deviation}\label{the-unexplained-deviation}}

The Unexplained Deviation is defined as \(y_i-\hat{y}_i\), which is the
deviation from the observed data to the fitted value which cannot be
explained by the regression line.

\hypertarget{total-deviation-explained-deviation-unexplained-deviation}{%
\subsubsection{Total Deviation = Explained Deviation + Unexplained Deviation**}\label{total-deviation-explained-deviation-unexplained-deviation}}

\[ (y_i-\bar{y}) = (\hat{y}_i-\bar{y})+(y_i-\hat{y}_i)\]

\hypertarget{regression-identity}{%
\subsubsection{Regression Identity}\label{regression-identity}}

It can also be shown mathematically, the same relationship holds for sum
of squares.

\textbf{Total Sum of Squares (SST) = Explained Sum of Squares (SSR) +
Unexplained sum of squares (SSE)\}}, i.e.,
\[\sum (y_i-\bar{y})^2 =\sum (\hat{y}_i-\bar{y})^2 + \sum (y_i-\hat{y}_i)^2\]
\[SST = SSR + SSE\]

This is illustrated in Figure below.

\begin{figure}
\centering
\includegraphics{https://i.ibb.co/MNLvKfm/LR-SS.png}
\caption{Regression Identity}
\end{figure}

SST: the variation of the response variable \(y\); SSR: the variation of
the response variable that is explained by the regression; SSE: the
variation of the response variable that is NOT explained by the
regression.

\hypertarget{coefficient-of-determination}{%
\subsection{Coefficient of Determination}\label{coefficient-of-determination}}

The \textbf{Coefficient of Determination} \(r^2\), is the proportion of
variation in the observed values of the response variable explained by
the regression. Thus, \[r^2 = \frac{SSR}{SST}= 1 - \frac{SSE}{SST}\] *
When \(r^2 \sim 0\), the regression equation is not very useful. * When
\(r^2 \sim 1\), the regression model is quite useful for making
predictions.

In the motivating example,
\(SST = 354530.5, SSR = 237548.5, SSE = 116982, r^2 = 0.670\).

\begin{Shaded}
\begin{Highlighting}[]
\FunctionTok{OLR}\NormalTok{(x, y)}
\end{Highlighting}
\end{Shaded}

\begin{verbatim}
## $Sxx
## [1] 19855.76
## 
## $Sxy
## [1] 68678.28
## 
## $beta0
## [1] -215.9815
## 
## $beta1
## [1] 3.458859
## 
## $SST
## [1] 354530.5
## 
## $SSR
## [1] 237548.5
## 
## $SSE
## [1] 116982
## 
## $r2
## [1] 0.6700369
## 
## $r2adj
## [1] 0.6669531
## 
## $anova
## Analysis of Variance Table
## 
## Response: y
##            Df Sum Sq Mean Sq F value    Pr(>F)    
## x           1 237549  237549  217.28 < 2.2e-16 ***
## Residuals 107 116982    1093                      
## ---
## Signif. codes:  0 '***' 0.001 '**' 0.01 '*' 0.05 '.' 0.1 ' ' 1
## 
## $sigma.hat
## [1] 33.06493
## 
## $se.beta1
## [1] 0.2346521
## 
## $t0
## [1] 14.74038
## 
## $p
## [1] 0
## 
## $F0
## [1] 217.2787
## 
## $p.F.test
## [1] 1.618607e-27
## 
## $CI
## [1] 2.993689 3.924030
\end{verbatim}

The adjusted \(r^2\) is the unbiased estimator of population coefficient
of determination. Compared to \(r^2\), the adjusted \(r^2\) accounts for the
number of parameters fit by the regression, and so can be compared
between models with different numbers of parameters.

\[
        \tilde{r}^2 = 1-\frac{SSE / (n-k)}{SST/(n-1)} = 1 - \frac{MSE}{MST}
\] where \(k\) is the number of parameters in the regression line. For
simple linear regression with one variable \(x\), \(k=2\) which includes
\(\beta_0\) and \(\beta_1\). It can be shown the adjusted \(r^2\) is smaller
than \(r^2\). It penalizes the number of parameters.

\hypertarget{example-1-sums-of-squares-and-coefficient-of-determination}{%
\subsubsection{Example 1: Sums of squares and coefficient of determination}\label{example-1-sums-of-squares-and-coefficient-of-determination}}

Consider the 3-point problem for illustration.

\begin{figure}
\centering
\includegraphics{https://i.ibb.co/kVdr56sQ/625-Table-14-03.png}
\caption{Example}
\end{figure}

\begin{figure}
\centering
\includegraphics{https://i.ibb.co/DgYVRXDP/625-Figure-14-08.png}
\caption{Example}
\end{figure}

\begin{itemize}
\tightlist
\item
  (a). Determine the three sums of squares.
\item
  (b). Find and interpret the coefficient of determination
\end{itemize}

\textbf{Solution}

\begin{figure}
\centering
\includegraphics{https://i.ibb.co/9kRzBNF2/640-Table-14-07.png}
\caption{Example}
\end{figure}

\begin{itemize}
\item
  (a).

  \[
  SST = \sum (y_i-\bar{y})^2 = 8
  \]

  \[
  SSR = \sum (\hat{y}_i-\bar{y})^2 = 2 
  \]

  \[
  SSE = \sum (y_i-\hat{y}_i)^2 = 6
  \]
\item
  (b).
\end{itemize}

\[r^2 = \frac{SSR}{SST}= 1 - \frac{SSE}{SST} = \frac{2}{8} = 0.25\]

The coefficient of determination is 0.25. Thus, 25\% of the variation in
the observed y-values is explained by the regression, i.e., by the
linear relationship between the x-values and the y-values.

\hypertarget{example-2-calculating-and-interpreting-r2}{%
\subsubsection{\texorpdfstring{Example 2: Calculating and Interpreting \(r^2\)}{Example 2: Calculating and Interpreting r\^{}2}}\label{example-2-calculating-and-interpreting-r2}}

The scatterplot and regression line for the age and price of 11 Orions
are below.

\includegraphics{https://i.ibb.co/Wp6gSKKf/641-Figure-14-16.png} Find and
interpret the coefficient of determination.

\textbf{Solution:}

\includegraphics{https://i.ibb.co/wZLFfjrr/642-Table-14-08.png}
\[SST = \sum y_i^2 - (\sum y_i)^2/n = 96129 - 975^2/11=9708.5\]

The regression line of sum of square is

\[
SSR = \frac{\sum x_iy_i - \sum x_i}{\sum y_i^2 - (\sum y_i)^2/n}=8285
\]

\[r^2 = \frac{SSR}{SST} = 0.853\]

\begin{Shaded}
\begin{Highlighting}[]
        \DocumentationTok{\#\#\#\#\#\#\#\#\#\#\#\#\#\#\#\#\#\#\#\#\#\#\#\#\#\#\#\#\#\#\#\#\#\#\#\#}
        \CommentTok{\# (2) Fit a linear regression line}
        \DocumentationTok{\#\#\#\#\#\#\#\#\#\#\#\#\#\#\#\#\#\#\#\#\#\#\#\#\#\#\#\#\#\#\#\#\#\#\#\#}
\NormalTok{        model }\OtherTok{\textless{}{-}} \FunctionTok{OLR}\NormalTok{(x, y)}
\NormalTok{        model}
\end{Highlighting}
\end{Shaded}

\begin{verbatim}
## $Sxx
## [1] 19855.76
## 
## $Sxy
## [1] 68678.28
## 
## $beta0
## [1] -215.9815
## 
## $beta1
## [1] 3.458859
## 
## $SST
## [1] 354530.5
## 
## $SSR
## [1] 237548.5
## 
## $SSE
## [1] 116982
## 
## $r2
## [1] 0.6700369
## 
## $r2adj
## [1] 0.6669531
## 
## $anova
## Analysis of Variance Table
## 
## Response: y
##            Df Sum Sq Mean Sq F value    Pr(>F)    
## x           1 237549  237549  217.28 < 2.2e-16 ***
## Residuals 107 116982    1093                      
## ---
## Signif. codes:  0 '***' 0.001 '**' 0.01 '*' 0.05 '.' 0.1 ' ' 1
## 
## $sigma.hat
## [1] 33.06493
## 
## $se.beta1
## [1] 0.2346521
## 
## $t0
## [1] 14.74038
## 
## $p
## [1] 0
## 
## $F0
## [1] 217.2787
## 
## $p.F.test
## [1] 1.618607e-27
## 
## $CI
## [1] 2.993689 3.924030
\end{verbatim}

\hypertarget{linear-correlation}{%
\subsection{Linear Correlation}\label{linear-correlation}}

Find the linear correlation coefficient of x and y.

\begin{figure}
\centering
\includegraphics{https://i.ibb.co/XxC9XZYz/645-Table-14-09.png}
\caption{Example}
\end{figure}

The sample deviations:
\[s_x = \sqrt{\frac{\sum (x_i-\bar{x})^2}{n-1}}=\sqrt{\frac{2}{3-1}}=1\]
and \[s_y = \sqrt{\frac{\sum (y_i-\bar{y})^2}{n-1}}=2\]

In addition,

\[
\frac{1}{n-1}\sum (x_i - \bar{x}) (y_i - \bar{y}) = \frac{1}{3-1}\cdot 2 = 1
\]

So, we find that \[
r = \frac{\frac{1}{n-1}\sum (x_i-\bar{x})(y_i-\bar{y})}{s_xs_y} = 0.5
\] \#\#\# 14.4.2 Understanding the linear correlation coefficient

\begin{itemize}
\item
  \(r\) refelcts the slope of the scatterplot. If the scatterplot shows
  a positive slope, the data points, on average, will lie either
  region I or III, i.e.~the deviations of \((x_i-\bar{x})\) and
  \((y_i-\bar{y})\) both negative or both positive.
\item
  If the scatterplot shows a positive slope, the data points, on
  average, will lie either region II or IV, i.e.~the deviations of
  \((x_i-\bar{x})\) and \((y_i-\bar{y})\) are in opposite directions.
\item
  The magnitude of \(r\) indicates the strength of the linear
  relationship.
\item
  The sign of \(r\) and the sign of the regression line's slope are
  identical.
\end{itemize}

\hypertarget{example-50}{%
\subsubsection{Example}\label{example-50}}

\begin{figure}
\centering
\includegraphics{https://i.ibb.co/N6WbfVr5/646-Figure-14-17.png}
\caption{Example}
\end{figure}

\begin{figure}
\centering
\includegraphics{https://i.ibb.co/wZyFr4tw/647-Figure-14-18.png}
\caption{Example}
\end{figure}

Using algebra, we can show that the linear correlation coefficient \(r\)
can be expressed as \[
r=\frac{S_{xx}}{\sqrt{S_{xx}S_{yy}}}
\]

\hypertarget{inferences-in-correlation}{%
\subsubsection{Inferences in Correlation}\label{inferences-in-correlation}}

In this section, we introduce the hypothesis test for the linear
correlation \(r\).

\hypertarget{t-distribution-for-a-correlation-test}{%
\paragraph{t-Distribution for a Correlation Test}\label{t-distribution-for-a-correlation-test}}

Suppose that the variables \(x\) and \(y\) satisfy the four assumptions for
regression inferences and that \(\rho = 0\). Then, the variable

\[
t = \frac{r}{\sqrt{\frac{1-r^2}{n-2}}}
\]

has the t-distribution with \(df=n-2\), where \(n\) is the sample size.

This hypothesis testing procedure is called the correlation t-test.

\begin{itemize}
\tightlist
\item
  STEP 1. The null hypothesis is \[ H_0: \rho = 0\]
\end{itemize}

and the alternative hypothesis can be either of the following +
\(H_a: \rho \ne 0\) two-sided + \(H_a: \rho < 0\) left-sided +
\(H_a: \rho > 0\) right-sided

\begin{itemize}
\item
  STEP 2. Decide the significance level \(\alpha\)
\item
  STEP 3. Compute the observed value of the test statistic \[
  t_0 = \frac{r}{\sqrt{\frac{1-r^2}{n-2}}}
  \]
\item
  STEP 4. The critical values are determined according to
  t-distribution and the side of the test.

  \begin{itemize}
  \tightlist
  \item
    Two-sided test: \(\pm t_{1-\alpha/2, df=n-2}\), using R code:
    qt(1-alpha/2, df=n-2). The rejection region is
    \(t < -t_{1-\alpha/2, df=n-2}\) or \(t>t_{1-\alpha/2, df=n-2}\).
  \item
    Left-sided test: \(t_{\alpha, df=n-2} = -t_{1-\alpha, df=n-2}\),
    using R code: qt(alpha, df=n-2). The rejection region is
    \(t < -t_{1-\alpha, df=n-2} = t_{\alpha, df=n-2}\).
  \item
    Right-sided test: \(t_{1-\alpha, df=n-2}\), using R code:
    qt(1-alpha, df=n-2). The rejection region is
    \(t > t_{1-\alpha, df=n-2}\).
  \end{itemize}

  where \(t_{p, df=n-2}\) is the \(p\)th quantile of t-distribution with
  \(n-2\) degrees of freedom.
\item
  STEP 5. If the \(t_0\) falls in the rejection region, then reject
  \(H_0\); otherwise, do not reject \(H_0\).
\item
  p-value: The p value is calculated accordingly:

  \begin{itemize}
  \tightlist
  \item
    Two-sided test: \(p = 2P(t > |t_0|)\), using R code:
    \(2*(1-pt(|t_0|, df=n-2))\).
  \item
    Left-sided test: \(p = P(t < t_0)\), using R code:
    \(pt(t_0, df=n-2)\).
  \item
    Right-sided test: \(p = P(t > t_0)\), using R code:
    \(1-pt(t_0, df=n-2)\).
  \end{itemize}

  If \(p\le \alpha\), reject \(H_0\); otherwise, do not reject \(H_0\).
\item
  STEP 6. Interpretation
\end{itemize}

\hypertarget{example-51}{%
\paragraph{Example}\label{example-51}}

Consider the Orion car problem. At the 5\% significance level, do the
data provide sufficient evidence to conclude that age and price are
negatively linearly correlated?

\textbf{Solution.}

\begin{itemize}
\tightlist
\item
  STEP 1. The null hypothesis is \[ H_0: \rho = 0\]
\end{itemize}

and the alternative hypothesis is \(H_a: \rho < 0\) left-sided.

\begin{itemize}
\item
  STEP 2. Decide the significance level \(\alpha = 0.05\)
\item
  STEP 3. Compute the observed value of the test statistic \[
  t_0 = \frac{r}{\sqrt{\frac{1-r^2}{n-2}}} = \frac{-0.924}{\sqrt{\frac{1-(-0.924)^2}{11-2}}}=-7.249
  \]
\item
  STEP 4. The critical value is the 5\%th quantile:
  \(t_{\alpha, df=n-2} = -1.833\), using R code: qt(0.05, df=11-2). The
  rejection region is \(t < -1.833\).
\item
  STEP 5. \(t_0=-7.249\) falls in the rejection region, so reject \(H_0\).
\item
  p-value: The p value is calculated accordingly:

  \begin{itemize}
  \tightlist
  \item
    Left-sided test: \(p = P(t < t_0) = 0.000024 < 0.05\), using R
    code: \(pt(t_0, df=n-2)\), so the p-value approach also concludes
    rejection of \(H_0\).
  \end{itemize}
\end{itemize}

\begin{Shaded}
\begin{Highlighting}[]
\NormalTok{p }\OtherTok{=} \FunctionTok{pt}\NormalTok{(}\SpecialCharTok{{-}}\FloatTok{7.249}\NormalTok{, }\AttributeTok{df=}\DecValTok{11{-}2}\NormalTok{)}
\NormalTok{p}
\end{Highlighting}
\end{Shaded}

\begin{verbatim}
## [1] 2.41073e-05
\end{verbatim}

\begin{itemize}
\tightlist
\item
  STEP 6. Interpretation. At the 5\% significance level, the data
  provide sufficient evidence to conclude that age and price are
  negatively linearly correlated. Price of Orions tend to decrease
  linearly with increasing age.
\end{itemize}

\hypertarget{hypothesis-test-h_0-beta_10-with-f-test}{%
\subsection{\texorpdfstring{Hypothesis test \(H_0: \beta_1=0\) with \(F\) test}{Hypothesis test H\_0: \textbackslash beta\_1=0 with F test}}\label{hypothesis-test-h_0-beta_10-with-f-test}}

Similar to ANOVA, the simple linear regression also has an ANOVA table
according to the analysis of variations from the regression and
unexplained. \(F\) follows \(F\) distribution with degrees of freedom 1 and
\(n-2\).

In the motivation example, the ANOVA table is displayed in Table. The
small \(p\) value indicates rejection of \(H_0\), i.e., \(\beta_1\) is not
zero.

\begin{Shaded}
\begin{Highlighting}[]
        \DocumentationTok{\#\#\#\#\#\#\#\#\#\#\#\#\#\#\#\#\#\#\#\#\#\#\#\#\#\#\#\#\#\#}
        \CommentTok{\#(5) ANOVA table in linear regression}
        \DocumentationTok{\#\#\#\#\#\#\#\#\#\#\#\#\#\#\#\#\#\#\#\#\#\#\#\#\#\#\#\#\#\#}
        
\NormalTok{        LinearReg }\OtherTok{\textless{}{-}} \FunctionTok{lm}\NormalTok{(y }\SpecialCharTok{\textasciitilde{}}\NormalTok{ x) }\CommentTok{\#Create the linear regression}
        \FunctionTok{anova}\NormalTok{(LinearReg)}
\end{Highlighting}
\end{Shaded}

\begin{verbatim}
## Analysis of Variance Table
## 
## Response: y
##            Df Sum Sq Mean Sq F value    Pr(>F)    
## x           1 237549  237549  217.28 < 2.2e-16 ***
## Residuals 107 116982    1093                      
## ---
## Signif. codes:  0 '***' 0.001 '**' 0.01 '*' 0.05 '.' 0.1 ' ' 1
\end{verbatim}

\hypertarget{hypothesis-test-h_0-beta_10-with-t-test}{%
\subsection{\texorpdfstring{Hypothesis test \(H_0: \beta_1=0\) with \(t\) test}{Hypothesis test H\_0: \textbackslash beta\_1=0 with t test}}\label{hypothesis-test-h_0-beta_10-with-t-test}}

The formal linear regression model is \[
        Y_i = \beta_0+\beta_1x_i + \epsilon_i
\]\\
where \[\epsilon \sim N(0, \sigma^2)\] are independent and
\[Y_i|x_i \sim N(\beta_0+\beta_1x_i, \sigma^2)\]. It's also commonly to
express the linear regression model as
\[E[Y_i|x_i] = \beta_0+\beta_1x_i\].

The LS estimate of \(\beta_0\) and \(\beta_1\) are

\[
\hat{\beta}_1 = \frac{\sum_{i=1}^n(x_i-\bar{x})(Y_i-\bar{Y})}{\sum_{i=1}^n(x_i-\bar{x})}
\] \[
    \hat{\beta}_0 = \bar{Y} - \hat{\beta}_1\bar{x}
\]

\[
\hat{\sigma}^2 = \frac{1}{n-2} \sum_{i=1}^n (Y_i - \hat{\beta}_0 - \hat{\beta}_1 x_i)^2 = \frac{SSE}{n-2} = MSE
\]\\
\[
\frac{\hat{\sigma}^2}{\sigma^2} \sim \frac{\chi^2_{n-2}}{n-2}
\] where the right hand side denotes a \(\chi^2\) distribution with \(n-2\)
degrees of freedom. By some algebra, we can obtain

\[
\widehat{\beta}_1 \sim N\left(\beta_1, var=\frac{\sigma^2}{\sum_{i=1}^n(X_i-\overline{X})^2}\right)
\]

\[
\frac{\widehat{\beta}_1 - \beta_1}{\sigma \sqrt{\frac{1}{\sum_{i=1}^n(X_i-\overline{X})^2}}} \sim N(\
        0,1)
\]

Further replace \(\sigma\) by \(\hat{\sigma}\), then

\[
SE(\widehat{\beta}_1) = \widehat{\sigma} \sqrt{\frac{1}{\sum_{i=1}^n(x_i-\bar{x})^2}}
\]

The test statistic is
\[t = \frac{\widehat{\beta}_1}{\widehat{\sigma} \sqrt{\frac{1}{\sum_{i=1}^n(x_i-\overline{x})^2}}} \sim t_{n-2}
\]

Reject \(H_0\) if the observed value \[|t_0| > t_{1-\alpha/2, df=n-2}\].

Let's perform this test for the motivating example. The calculations are
below. The \textbf{residual standard error} is \(\hat{\sigma}=33.06\) which is
consistent with the R output. The \(SE(\widehat{\beta}_1) = 0.2347\) is
also consistent with the R output. The value of the test statistic is
\(t_0=14.74\) and \(p=0\), so reject \(H_0\).

\begin{Shaded}
\begin{Highlighting}[]
    \DocumentationTok{\#\#\#\#\#\#\#\#\#\#\#\#\#\#\#\#\#\#\#\#\#\#\#\#\#\#\#\#\#\#}
        \CommentTok{\#(6) t test for beta1}
        \DocumentationTok{\#\#\#\#\#\#\#\#\#\#\#\#\#\#\#\#\#\#\#\#\#\#\#\#\#\#\#\#\#\#}
        \FunctionTok{summary}\NormalTok{(LinearReg)}
\end{Highlighting}
\end{Shaded}

\begin{verbatim}
## 
## Call:
## lm(formula = y ~ x)
## 
## Residuals:
##      Min       1Q   Median       3Q      Max 
## -107.288  -19.143   -2.939   16.376   90.342 
## 
## Coefficients:
##              Estimate Std. Error t value Pr(>|t|)    
## (Intercept) -215.9815    21.7963  -9.909   <2e-16 ***
## x              3.4589     0.2347  14.740   <2e-16 ***
## ---
## Signif. codes:  0 '***' 0.001 '**' 0.01 '*' 0.05 '.' 0.1 ' ' 1
## 
## Residual standard error: 33.06 on 107 degrees of freedom
## Multiple R-squared:   0.67,  Adjusted R-squared:  0.667 
## F-statistic: 217.3 on 1 and 107 DF,  p-value: < 2.2e-16
\end{verbatim}

\textbf{Residual standard error}

\begin{Shaded}
\begin{Highlighting}[]
\NormalTok{        sigma.hat }\OtherTok{=} \FunctionTok{sqrt}\NormalTok{(}\FunctionTok{sum}\NormalTok{(}\FunctionTok{resid}\NormalTok{(LinearReg)}\SpecialCharTok{\^{}}\DecValTok{2}\NormalTok{) }\SpecialCharTok{/}\NormalTok{ LinearReg}\SpecialCharTok{$}\NormalTok{df.resid)}
\NormalTok{        sigma.hat}
\end{Highlighting}
\end{Shaded}

\begin{verbatim}
## [1] 33.06493
\end{verbatim}

\textbf{Standard error of beta1}

\begin{Shaded}
\begin{Highlighting}[]
\NormalTok{        se.beta1.hat }\OtherTok{=}\NormalTok{ (sigma.hat }\SpecialCharTok{*} \FunctionTok{sqrt}\NormalTok{(}\DecValTok{1} \SpecialCharTok{/} \FunctionTok{sum}\NormalTok{((x }\SpecialCharTok{{-}} \FunctionTok{mean}\NormalTok{(x))}\SpecialCharTok{\^{}}\DecValTok{2}\NormalTok{)))}
\NormalTok{        se.beta1.hat}
\end{Highlighting}
\end{Shaded}

\begin{verbatim}
## [1] 0.2346521
\end{verbatim}

\textbf{Estimate of beta1}

\begin{Shaded}
\begin{Highlighting}[]
\NormalTok{        beta1.hat }\OtherTok{\textless{}{-}} \FunctionTok{coef}\NormalTok{(LinearReg)[}\DecValTok{2}\NormalTok{]}
\NormalTok{    beta1.hat}
\end{Highlighting}
\end{Shaded}

\begin{verbatim}
##        x 
## 3.458859
\end{verbatim}

\textbf{t-test statistic for testing H0: beta1 = 0}

\begin{Shaded}
\begin{Highlighting}[]
\NormalTok{        t0 }\OtherTok{=}\NormalTok{ (beta1.hat }\SpecialCharTok{{-}} \DecValTok{0}\NormalTok{) }\SpecialCharTok{/}\NormalTok{ se.beta1.hat}
\NormalTok{        t0}
\end{Highlighting}
\end{Shaded}

\begin{verbatim}
##        x 
## 14.74038
\end{verbatim}

\textbf{p value for testing H0: beta1 = 0}

\begin{Shaded}
\begin{Highlighting}[]
\NormalTok{p }\OtherTok{=} \DecValTok{2}\SpecialCharTok{*}\NormalTok{(}\DecValTok{1}\SpecialCharTok{{-}}\FunctionTok{pt}\NormalTok{(}\FunctionTok{abs}\NormalTok{(t0), }\AttributeTok{df=}\NormalTok{ LinearReg}\SpecialCharTok{$}\NormalTok{df.resid))}
\NormalTok{p}
\end{Highlighting}
\end{Shaded}

\begin{verbatim}
## x 
## 0
\end{verbatim}

\hypertarget{confidence-interval}{%
\subsection{Confidence Interval}\label{confidence-interval}}

\hypertarget{example-52}{%
\subsubsection{Example}\label{example-52}}

Find a 95\% CI for the mean price of all 3-year-old Orions.

For 95\%CI, alpha is 0.05. Because \(n=11\), the corresponding \(t\)
distribution has 9 degrees of freedom. The critical value is
\(t_{0.975, df=9} = 2.262\).

\begin{Shaded}
\begin{Highlighting}[]
\FunctionTok{qt}\NormalTok{(}\FloatTok{0.975}\NormalTok{, }\AttributeTok{df=}\DecValTok{9}\NormalTok{)}
\end{Highlighting}
\end{Shaded}

\begin{verbatim}
## [1] 2.262157
\end{verbatim}

The point estimate of the conditional mean (i.e.~predicted value) is
\(\hat{\beta}_0+\hat{\beta}_1\times 3 = 134.69\). The 95\%CI of the
conditional mean \(\beta_0 + 3\beta_1\) is \[
\hat{y}_p\pm t_{1-\alpha/2, df=n-2}\hat{\sigma}\sqrt{\frac{1}{n}+\frac{(x_p - \bar{x})^2}{\sum (x_i-\bar{x})^2}} = 134.69\pm 16.76.
\]

\hypertarget{confidence-interval-based-on-students-t-distribution}{%
\subsubsection{\texorpdfstring{Confidence interval based on Student's \(t\) distribution}{Confidence interval based on Student's t distribution}}\label{confidence-interval-based-on-students-t-distribution}}

Suppose we have a parameter estimate
\[\widehat{\theta} \sim N(\theta, {\sigma}_{\theta}^2)\], and standard
error \(SE(\widehat{\theta})\) such that

\[
    \frac{\widehat{\theta}-\theta}{SE(\widehat{\theta})} \sim t_{\nu}.
\]

In general, a \((1-\alpha) \cdot 100 \%\) confidence interval is

\[
    \widehat{\theta} \pm SE(\widehat{\theta}) \cdot t_{\nu, 1-\alpha/2}
\]

To prove this, expand the absolute value as we did for the one-sample CI
\[
    1 - \alpha = \mathbb{P}_{\theta}\left(\left|\frac{\widehat{\theta} - \theta}{SE(\widehat{\theta})} \right| < t_{\nu, 1-\alpha/2}\right)
\]

Apply to the confidence interval for regression parameter \(\beta_1\),
then a confidence interval is
\[\hat{\beta}_1 \pm SE(\hat{\beta}_1) \cdot t_{n-2, 1-\alpha/2}.\]

If the \((1-\alpha)100\%\) confidence interval excludes 0, then \(H_0\) is
rejected at the \(\alpha\) level. In R, use the function
\textbackslash verb\textbar confint()\textbar{} to calculate the CI.

\begin{Shaded}
\begin{Highlighting}[]
        \DocumentationTok{\#\#\#\#\#\#\#\#\#\#\#\#\#\#\#\#\#\#\#\#\#\#\#\#\#\#\#\#\#\#}
        \CommentTok{\#(7) Confidence Interval for beta1}
        \DocumentationTok{\#\#\#\#\#\#\#\#\#\#\#\#\#\#\#\#\#\#\#\#\#\#\#\#\#\#\#\#\#\#      }
\NormalTok{        L }\OtherTok{=}\NormalTok{ beta1.hat }\SpecialCharTok{{-}} \FunctionTok{qt}\NormalTok{(}\FloatTok{0.975}\NormalTok{, LinearReg}\SpecialCharTok{$}\NormalTok{df.resid) }\SpecialCharTok{*}\NormalTok{ se.beta1.hat}
\NormalTok{        U }\OtherTok{=}\NormalTok{ beta1.hat }\SpecialCharTok{+} \FunctionTok{qt}\NormalTok{(}\FloatTok{0.975}\NormalTok{, LinearReg}\SpecialCharTok{$}\NormalTok{df.resid) }\SpecialCharTok{*}\NormalTok{ se.beta1.hat}
        \FunctionTok{data.frame}\NormalTok{(L, U)        }
\end{Highlighting}
\end{Shaded}

\begin{verbatim}
##          L       U
## x 2.993689 3.92403
\end{verbatim}

In R, use the syntax below to get the CI.

\begin{Shaded}
\begin{Highlighting}[]
        \FunctionTok{confint}\NormalTok{(LinearReg)}
\end{Highlighting}
\end{Shaded}

\begin{verbatim}
##                   2.5 %     97.5 %
## (Intercept) -259.190053 -172.77292
## x              2.993689    3.92403
\end{verbatim}

Revisit the OLR function to produce all above.

\begin{Shaded}
\begin{Highlighting}[]
\FunctionTok{OLR}\NormalTok{(x, y)}
\end{Highlighting}
\end{Shaded}

\begin{verbatim}
## $Sxx
## [1] 19855.76
## 
## $Sxy
## [1] 68678.28
## 
## $beta0
## [1] -215.9815
## 
## $beta1
## [1] 3.458859
## 
## $SST
## [1] 354530.5
## 
## $SSR
## [1] 237548.5
## 
## $SSE
## [1] 116982
## 
## $r2
## [1] 0.6700369
## 
## $r2adj
## [1] 0.6669531
## 
## $anova
## Analysis of Variance Table
## 
## Response: y
##            Df Sum Sq Mean Sq F value    Pr(>F)    
## x           1 237549  237549  217.28 < 2.2e-16 ***
## Residuals 107 116982    1093                      
## ---
## Signif. codes:  0 '***' 0.001 '**' 0.01 '*' 0.05 '.' 0.1 ' ' 1
## 
## $sigma.hat
## [1] 33.06493
## 
## $se.beta1
## [1] 0.2346521
## 
## $t0
## [1] 14.74038
## 
## $p
## [1] 0
## 
## $F0
## [1] 217.2787
## 
## $p.F.test
## [1] 1.618607e-27
## 
## $CI
## [1] 2.993689 3.924030
\end{verbatim}

\hypertarget{model-assumptions-and-diagnostics}{%
\section{Model assumptions and diagnostics}\label{model-assumptions-and-diagnostics}}

\hypertarget{model-assumptions}{%
\subsubsection{Model Assumptions}\label{model-assumptions}}

The simple linear regression model is \[
y = \beta_0 + \beta_1 x
\]

The linear regression model assumes

\begin{itemize}
\item
  Normality. For each value of \(X\), there is a sub-population of \(Y\)
  values, i.e., the conditional distribution \(Y|X\) exists and normally
  distributed. The horizontal band of the residuals should be centered
  and symmetric about the x-axis.
\item
  The variances of all subpopulations are the same, i.e.,
  \(var(Y|X) = \sigma^2\) independent of \(X\). The residuals should fall
  roughly in a horizontal band.
\item
  The linearity between \(Y\) and \(X\) is linear, i.e.
  \(E(Y|X) = \beta_0+\beta_1X\).
\item
  The random sample \(Y\) is independent condition on \(X\), i.e.,
  \(Y|X=x_1\) is independent of \(Y|X=x_2\). This means that, in drawing
  the sample, it is assumed that the values of \(Y\) chosen at one value
  of \(X\) in no way depend on the values of \(Y\) chosen at another value
  of \(X\).
\end{itemize}

\hypertarget{model-diagnostics-1}{%
\subsubsection{Model Diagnostics}\label{model-diagnostics-1}}

The residuals plot can be used to diagnose the constant variance
assumption and QQ plot is to diagnose the normality assumption. If the
assumptions for regression inferences are met, then

\begin{itemize}
\tightlist
\item
  A plot of the residuals against the observed values of the predictor
  variable should fall roughly in a horizontal band centered and
  symmetric about the x-axis.
\item
  A normal probability plot of the residuals should be roughly linear.
\end{itemize}

Failure of either of these two conditions casts doubt on the validity of
one or more of the assumptions for regression inferences. Figure below
shows the diagnostics for the example. The residual plot shows a
potential concern of non-constant variance in this example.

\begin{Shaded}
\begin{Highlighting}[]
\FunctionTok{plot}\NormalTok{(LinearReg)}
\end{Highlighting}
\end{Shaded}

\includegraphics{StatsTB_files/figure-latex/unnamed-chunk-355-1.pdf} \includegraphics{StatsTB_files/figure-latex/unnamed-chunk-355-2.pdf} \includegraphics{StatsTB_files/figure-latex/unnamed-chunk-355-3.pdf} \includegraphics{StatsTB_files/figure-latex/unnamed-chunk-355-4.pdf}

\hypertarget{scatter-plot-with-loess-smoother}{%
\subsection{Scatter plot with loess smoother}\label{scatter-plot-with-loess-smoother}}

\begin{Shaded}
\begin{Highlighting}[]
\DocumentationTok{\#\#\#\#\#\#\#\#\#\#\#\#\#\#\#\#\#\#\#\#\#\#\#\#\#\#\#\#\#\#}
\CommentTok{\#(1) Motivating Example (Adipose Tissue vs waist)}
\DocumentationTok{\#\#\#\#\#\#\#\#\#\#\#\#\#\#\#\#\#\#\#\#\#\#\#\#\#\#\#\#\#\#}
\NormalTok{x }\OtherTok{=} \FunctionTok{c}\NormalTok{(}\FloatTok{74.75}\NormalTok{,}\FloatTok{72.60}\NormalTok{,}\FloatTok{81.80}\NormalTok{,}\FloatTok{83.95}\NormalTok{,}\FloatTok{74.65}\NormalTok{,}\FloatTok{71.85}\NormalTok{,}\FloatTok{80.90}\NormalTok{,}
\FloatTok{83.40}\NormalTok{,}\FloatTok{63.50}\NormalTok{,}\FloatTok{73.20}\NormalTok{,}\FloatTok{71.90}\NormalTok{,}\FloatTok{75.00}\NormalTok{,}\FloatTok{73.10}\NormalTok{,}\FloatTok{79.00}\NormalTok{,}
\FloatTok{77.00}\NormalTok{,}\FloatTok{68.85}\NormalTok{,}\FloatTok{75.95}\NormalTok{,}\FloatTok{74.15}\NormalTok{,}\FloatTok{73.80}\NormalTok{,}\FloatTok{75.90}\NormalTok{,}\FloatTok{76.85}\NormalTok{,}
\FloatTok{80.90}\NormalTok{,}\FloatTok{79.90}\NormalTok{,}\FloatTok{89.20}\NormalTok{,}\FloatTok{82.00}\NormalTok{,}\FloatTok{92.00}\NormalTok{,}\FloatTok{86.60}\NormalTok{,}\FloatTok{80.50}\NormalTok{,}
\FloatTok{86.00}\NormalTok{,}\FloatTok{82.50}\NormalTok{,}\FloatTok{83.50}\NormalTok{,}\FloatTok{88.10}\NormalTok{,}\FloatTok{90.80}\NormalTok{,}\FloatTok{89.40}\NormalTok{,}\FloatTok{102.00}\NormalTok{,}
\FloatTok{94.50}\NormalTok{,}\FloatTok{91.00}\NormalTok{,}
\FloatTok{103.00}\NormalTok{,}\FloatTok{80.00}\NormalTok{,}\FloatTok{79.00}\NormalTok{,}\FloatTok{83.50}\NormalTok{,}\FloatTok{76.00}\NormalTok{,}\FloatTok{80.50}\NormalTok{,}\FloatTok{86.50}\NormalTok{,}
\FloatTok{83.00}\NormalTok{,}\FloatTok{107.10}\NormalTok{,}\FloatTok{94.30}\NormalTok{,}\FloatTok{94.50}\NormalTok{,}\FloatTok{79.70}\NormalTok{,}\FloatTok{79.30}\NormalTok{,}\FloatTok{89.80}\NormalTok{,}
\FloatTok{83.80}\NormalTok{,}\FloatTok{85.20}\NormalTok{,}\FloatTok{75.50}\NormalTok{,}\FloatTok{78.40}\NormalTok{,}\FloatTok{78.60}\NormalTok{,}\FloatTok{87.80}\NormalTok{,}\FloatTok{86.30}\NormalTok{,}
\FloatTok{85.50}\NormalTok{,}\FloatTok{83.70}\NormalTok{,}\FloatTok{77.60}\NormalTok{,}\FloatTok{84.90}\NormalTok{,}\FloatTok{79.80}\NormalTok{,}\FloatTok{108.30}\NormalTok{,}\FloatTok{119.60}\NormalTok{,}
\FloatTok{119.90}\NormalTok{,}\FloatTok{96.50}\NormalTok{,}\FloatTok{105.50}\NormalTok{,}\FloatTok{105.00}\NormalTok{,}\FloatTok{107.00}\NormalTok{,}\FloatTok{107.00}\NormalTok{,}
\FloatTok{101.00}\NormalTok{,}\FloatTok{97.00}\NormalTok{,}\FloatTok{100.00}\NormalTok{,}
\FloatTok{108.00}\NormalTok{,}\FloatTok{100.00}\NormalTok{,}\FloatTok{103.00}\NormalTok{,}\FloatTok{104.00}\NormalTok{,}\FloatTok{106.00}\NormalTok{,}\FloatTok{109.00}\NormalTok{,}
\FloatTok{103.50}\NormalTok{,}\FloatTok{110.00}\NormalTok{,}\FloatTok{110.00}\NormalTok{,}\FloatTok{112.00}\NormalTok{,}\FloatTok{108.50}\NormalTok{,}\FloatTok{104.00}\NormalTok{,}
\FloatTok{111.00}\NormalTok{,}\FloatTok{108.50}\NormalTok{,}\FloatTok{121.00}\NormalTok{,}\FloatTok{109.00}\NormalTok{,}\FloatTok{97.50}\NormalTok{,}\FloatTok{105.50}\NormalTok{,}
\FloatTok{98.00}\NormalTok{,}\FloatTok{94.50}\NormalTok{,}\FloatTok{97.00}\NormalTok{,}\FloatTok{105.00}\NormalTok{,}\FloatTok{106.00}\NormalTok{,}\FloatTok{99.00}\NormalTok{,}\FloatTok{91.00}\NormalTok{,}
\FloatTok{102.50}\NormalTok{,}\FloatTok{106.00}\NormalTok{,}\FloatTok{109.10}\NormalTok{,}\FloatTok{115.00}\NormalTok{,}\FloatTok{101.00}\NormalTok{,}\FloatTok{100.10}\NormalTok{,}
\FloatTok{93.30}\NormalTok{,}\FloatTok{101.80}\NormalTok{,}\FloatTok{107.90}\NormalTok{,}\FloatTok{108.50}\NormalTok{)}

\NormalTok{y}\OtherTok{\textless{}{-}} \FunctionTok{c}\NormalTok{(}\FloatTok{25.72}\NormalTok{,}\FloatTok{25.89}\NormalTok{,}\FloatTok{42.60}\NormalTok{,}\FloatTok{42.80}\NormalTok{,}\FloatTok{29.84}\NormalTok{,}\FloatTok{21.68}\NormalTok{,}\FloatTok{29.08}\NormalTok{,}
\FloatTok{32.98}\NormalTok{,}\FloatTok{11.44}\NormalTok{,}\FloatTok{32.22}\NormalTok{,}\FloatTok{28.32}\NormalTok{,}\FloatTok{43.86}\NormalTok{,}\FloatTok{38.21}\NormalTok{,}\FloatTok{42.48}\NormalTok{,}
\FloatTok{30.96}\NormalTok{,}\FloatTok{55.78}\NormalTok{,}\FloatTok{43.78}\NormalTok{,}\FloatTok{33.41}\NormalTok{,}\FloatTok{43.35}\NormalTok{,}\FloatTok{29.31}\NormalTok{,}\FloatTok{36.60}\NormalTok{,}
\FloatTok{40.25}\NormalTok{,}\FloatTok{35.43}\NormalTok{,}\FloatTok{60.09}\NormalTok{,}\FloatTok{45.84}\NormalTok{,}\FloatTok{70.40}\NormalTok{,}\FloatTok{83.45}\NormalTok{,}\FloatTok{84.30}\NormalTok{,}
\FloatTok{78.89}\NormalTok{,}\FloatTok{64.75}\NormalTok{,}\FloatTok{72.56}\NormalTok{,}\FloatTok{89.31}\NormalTok{,}\FloatTok{78.94}\NormalTok{,}\FloatTok{83.55}\NormalTok{,}\FloatTok{127.00}\NormalTok{,}
\FloatTok{121.00}\NormalTok{,}\FloatTok{107.00}\NormalTok{,}
\FloatTok{129.00}\NormalTok{,}\FloatTok{74.02}\NormalTok{,}\FloatTok{55.48}\NormalTok{,}\FloatTok{73.13}\NormalTok{,}\FloatTok{50.50}\NormalTok{,}\FloatTok{50.88}\NormalTok{,}\FloatTok{140.00}\NormalTok{,}
\FloatTok{96.54}\NormalTok{,}\FloatTok{118.00}\NormalTok{,}\FloatTok{107.00}\NormalTok{,}\FloatTok{123.00}\NormalTok{,}\FloatTok{65.92}\NormalTok{,}\FloatTok{81.29}\NormalTok{,}\FloatTok{111.00}\NormalTok{,}
\FloatTok{90.73}\NormalTok{,}\FloatTok{133.00}\NormalTok{,}\FloatTok{41.90}\NormalTok{,}\FloatTok{41.71}\NormalTok{,}\FloatTok{58.16}\NormalTok{,}\FloatTok{88.85}\NormalTok{,}\FloatTok{155.00}\NormalTok{,}
\FloatTok{70.77}\NormalTok{,}\FloatTok{75.08}\NormalTok{,}\FloatTok{57.05}\NormalTok{,}\FloatTok{99.73}\NormalTok{,}\FloatTok{27.96}\NormalTok{,}\FloatTok{123.00}\NormalTok{,}\FloatTok{90.41}\NormalTok{,}
\FloatTok{106.00}\NormalTok{,}\FloatTok{144.00}\NormalTok{,}\FloatTok{121.00}\NormalTok{,}\FloatTok{97.13}\NormalTok{,}\FloatTok{166.00}\NormalTok{,}\FloatTok{87.99}\NormalTok{,}\FloatTok{154.00}\NormalTok{,}
\FloatTok{100.00}\NormalTok{,}\FloatTok{123.00}\NormalTok{,}
\FloatTok{217.00}\NormalTok{,}\FloatTok{140.00}\NormalTok{,}\FloatTok{109.00}\NormalTok{,}\FloatTok{127.00}\NormalTok{,}\FloatTok{112.00}\NormalTok{,}\FloatTok{192.00}\NormalTok{,}
\FloatTok{132.00}\NormalTok{,}\FloatTok{126.00}\NormalTok{,}\FloatTok{153.00}\NormalTok{,}\FloatTok{158.00}\NormalTok{,}\FloatTok{183.00}\NormalTok{,}\FloatTok{184.00}\NormalTok{,}
\FloatTok{121.00}\NormalTok{,}\FloatTok{159.00}\NormalTok{,}\FloatTok{245.00}\NormalTok{,}\FloatTok{137.00}\NormalTok{,}\FloatTok{165.00}\NormalTok{,}\FloatTok{152.00}\NormalTok{,}
\FloatTok{181.00}\NormalTok{,}\FloatTok{80.95}\NormalTok{,}\FloatTok{137.00}\NormalTok{,}\FloatTok{125.00}\NormalTok{,}\FloatTok{241.00}\NormalTok{,}\FloatTok{134.00}\NormalTok{,}
\FloatTok{150.00}\NormalTok{,}\FloatTok{198.00}\NormalTok{,}\FloatTok{151.00}\NormalTok{,}\FloatTok{229.00}\NormalTok{,}\FloatTok{253.00}\NormalTok{,}\FloatTok{188.00}\NormalTok{,}
\FloatTok{124.00}\NormalTok{,}\FloatTok{62.20}\NormalTok{,}\FloatTok{133.00}\NormalTok{,}\FloatTok{208.00}\NormalTok{,}\FloatTok{208.00}\NormalTok{)}

\FunctionTok{plot}\NormalTok{(x, y, }\AttributeTok{xlab=}\StringTok{"Waist Circumference (cm)"}\NormalTok{, }\AttributeTok{ylab=}\StringTok{"Adipose Tissue (AT)"}\NormalTok{)}

\CommentTok{\#Superimpose the trending smoother line}
\FunctionTok{lines}\NormalTok{(}\FunctionTok{lowess}\NormalTok{(y }\SpecialCharTok{\textasciitilde{}}\NormalTok{ x), }\AttributeTok{lty=}\DecValTok{2}\NormalTok{, }\AttributeTok{lwd=}\DecValTok{2}\NormalTok{)}
\end{Highlighting}
\end{Shaded}

\includegraphics{StatsTB_files/figure-latex/unnamed-chunk-356-1.pdf}

\hypertarget{model-fitting-and-interpretation}{%
\subsection{Model fitting and interpretation}\label{model-fitting-and-interpretation}}

\begin{Shaded}
\begin{Highlighting}[]
        \DocumentationTok{\#\#\#\#\#\#\#\#\#\#\#\#\#\#\#\#\#\#\#\#\#\#\#\#\#\#\#\#\#\#\#\#\#\#\#\#}
        \CommentTok{\# (2) Fit a linear regression line}
        \DocumentationTok{\#\#\#\#\#\#\#\#\#\#\#\#\#\#\#\#\#\#\#\#\#\#\#\#\#\#\#\#\#\#\#\#\#\#\#\#}
\NormalTok{        model }\OtherTok{\textless{}{-}} \FunctionTok{OLR}\NormalTok{(x, y)}
\NormalTok{        model}
\end{Highlighting}
\end{Shaded}

\begin{verbatim}
## $Sxx
## [1] 19855.76
## 
## $Sxy
## [1] 68678.28
## 
## $beta0
## [1] -215.9815
## 
## $beta1
## [1] 3.458859
## 
## $SST
## [1] 354530.5
## 
## $SSR
## [1] 237548.5
## 
## $SSE
## [1] 116982
## 
## $r2
## [1] 0.6700369
## 
## $r2adj
## [1] 0.6669531
## 
## $anova
## Analysis of Variance Table
## 
## Response: y
##            Df Sum Sq Mean Sq F value    Pr(>F)    
## x           1 237549  237549  217.28 < 2.2e-16 ***
## Residuals 107 116982    1093                      
## ---
## Signif. codes:  0 '***' 0.001 '**' 0.01 '*' 0.05 '.' 0.1 ' ' 1
## 
## $sigma.hat
## [1] 33.06493
## 
## $se.beta1
## [1] 0.2346521
## 
## $t0
## [1] 14.74038
## 
## $p
## [1] 0
## 
## $F0
## [1] 217.2787
## 
## $p.F.test
## [1] 1.618607e-27
## 
## $CI
## [1] 2.993689 3.924030
\end{verbatim}

Or fit model using lm() function

\begin{Shaded}
\begin{Highlighting}[]
\NormalTok{        LinearReg }\OtherTok{\textless{}{-}} \FunctionTok{lm}\NormalTok{(y }\SpecialCharTok{\textasciitilde{}}\NormalTok{ x) }\CommentTok{\#Create the linear regression}
        \FunctionTok{summary}\NormalTok{(LinearReg)}
\end{Highlighting}
\end{Shaded}

\begin{verbatim}
## 
## Call:
## lm(formula = y ~ x)
## 
## Residuals:
##      Min       1Q   Median       3Q      Max 
## -107.288  -19.143   -2.939   16.376   90.342 
## 
## Coefficients:
##              Estimate Std. Error t value Pr(>|t|)    
## (Intercept) -215.9815    21.7963  -9.909   <2e-16 ***
## x              3.4589     0.2347  14.740   <2e-16 ***
## ---
## Signif. codes:  0 '***' 0.001 '**' 0.01 '*' 0.05 '.' 0.1 ' ' 1
## 
## Residual standard error: 33.06 on 107 degrees of freedom
## Multiple R-squared:   0.67,  Adjusted R-squared:  0.667 
## F-statistic: 217.3 on 1 and 107 DF,  p-value: < 2.2e-16
\end{verbatim}

\begin{Shaded}
\begin{Highlighting}[]
        \FunctionTok{anova}\NormalTok{(LinearReg)}
\end{Highlighting}
\end{Shaded}

\begin{verbatim}
## Analysis of Variance Table
## 
## Response: y
##            Df Sum Sq Mean Sq F value    Pr(>F)    
## x           1 237549  237549  217.28 < 2.2e-16 ***
## Residuals 107 116982    1093                      
## ---
## Signif. codes:  0 '***' 0.001 '**' 0.01 '*' 0.05 '.' 0.1 ' ' 1
\end{verbatim}

\hypertarget{prediction-using-linear-regression}{%
\section{Prediction Using Linear Regression}\label{prediction-using-linear-regression}}

The predicted value of \(y\) for \(x = x_p\) is
\(\hat{y}_p = \hat{\beta}_0 + \hat{\beta}_1 x_p\). The mean of the
predicted value is \(\mu_{\hat{y}_p} = \beta_0 + \beta_1 x_p\). The
standard deviation of the predicted value is
\(\sigma_{\hat{y}_p} = \sigma \sqrt{\frac{1}{n}+\frac{(x_p - \bar{x})^2}{S_{xx}}}\).
\(\hat{y}_p\) is normally distributed, and

\[
\hat{y}_p \sim N(\mu_{\hat{y}_p}, \sigma_{\hat{y}_p}^2)=N\left(\beta_0 + \beta_1 x_p, \sigma^2 \left(\frac{1}{n}+\frac{(x_p - \bar{x})^2}{\sum (x_i-\bar{x})^2}\right)\right)
\]

\[
t = \frac{\hat{y}_p-(\beta_0+\beta_1x_p)}{\hat{\sigma}\sqrt{\frac{1}{n}+\frac{(x_p - \bar{x})^2}{\sum (x_i-\bar{x})^2}}} \sim t_{df=n-2}
\]

\hypertarget{ci-for-the-conditional-mean}{%
\subsubsection{CI for the conditional mean}\label{ci-for-the-conditional-mean}}

So the \textbf{CI for the conditional mean}
\(\mu_{\hat{y}_p}(=\beta_0+\beta_1x_p)\) can be constructed as \[
\hat{y}_p\pm t_{1-\alpha/2, df=n-2}\hat{\sigma}\sqrt{\frac{1}{n}+\frac{(x_p - \bar{x})^2}{\sum (x_i-\bar{x})^2}}
\] This is called \textbf{the confidence interval of the predicted value},
i.e., population mean of \(y\) at \(x\).

Another similar but commonly confusing concept in regression is
prediction interval, which is for the individual rather than the
population mean. The following provides the theoretical clarification.

For an individual's response \(y_p\) associated with the covariate \(x_p\),
the mean of \(y_p - \hat{y}_p\) equals zero: \(\mu_{y_p-\hat{y}_p}=0\), and
the standard deviation of \(y_p - \hat{y}_p\) is
\[\sigma_{y_p - \hat{y}_p} = \sigma \sqrt{1+\frac{1}{n}+\frac{(x_p - \bar{x})^2}{\sum (x_i-\bar{x})^2}}\]
In addition, \[y_p - \hat{y}_p \sim N(0, \sigma_{y_p - \hat{y}_p}^2)\].
As a result, \[
    y_p - \hat{y}_p \sim N\left(0, \sigma \sqrt{1+\frac{1}{n}+\frac{(x_p - \bar{x})^2}{\sum (x_i-\bar{x})^2}}\right)
\] \[
    t = \frac{\hat{y}_p-\hat{y}_p}{\hat{\sigma}\sqrt{1+\frac{1}{n}+\frac{(x_p - \bar{x})^2}{\sum (x_i-\bar{x})^2}}}\sim t_{df=n-}
\]

So the confidence interval for \(y_p\) is \[
    \hat{y}_p\pm t_{1-\alpha/2, df=n-2}\hat{\sigma}\sqrt{1+\frac{1}{n}+\frac{(x_p - \bar{x})^2}{\sum (x_i-\bar{x})^2}}
\]

\hypertarget{prediction-interval}{%
\subsubsection{Prediction Interval}\label{prediction-interval}}

This is called \textbf{prediction interval}. Compared to the confidence
interval of the population mean, the prediction interval is wider
because it takes into account the individual variability.

In short, * the confidence interval of the conditional mean
\(\beta_0 + \beta_1x_p\) for covariate \(x_p\) is conventionally called
confidence interval. * the confidence interval of an individual
response \(y_p\) for covariate \(x_p\) is conventionally called prediction
interval.

\hypertarget{example-1-1}{%
\subsubsection{Example 1}\label{example-1-1}}

Let's revisit the example. \[\hat{\sigma} = \sqrt{MSE} = 33.06\], which
is calculated by \[\frac{1}{n-2}\sum e_i^2\], where
\[e_i=y_i - \hat{y}_i\] is the residual.
\[S_{xx} = \sum (x_i-\bar{x})^2 = 19855.76\] and \(n=109\). For a subject
with \(x=100\), the predicted value of \(y\) is 129.90. The 95\%CI for the
population mean is \((122.58, 137.23)\) and the prediction interval is
\((63.95, 195.86)\). The interpretation of both intervals are as follows.

\begin{itemize}
\tightlist
\item
  We can be 95\% confident that the mean AT of all subjects with waist
  circumference 100cm is somewhere between 122.58 and 137.23. This is
  for a population parameter: mean.
\item
  We can be 95\% confident that the AT of a subject with waist
  circumference 100cm will be somewhere between 63.95 and 195.86. This
  is for an individual! Not a population concept.
\end{itemize}

\begin{Shaded}
\begin{Highlighting}[]
\FunctionTok{OLR}\NormalTok{(x, y, }\AttributeTok{new.x=}\DecValTok{100}\NormalTok{)}
\end{Highlighting}
\end{Shaded}

\begin{verbatim}
## $Sxx
## [1] 19855.76
## 
## $Sxy
## [1] 68678.28
## 
## $beta0
## [1] -215.9815
## 
## $beta1
## [1] 3.458859
## 
## $SST
## [1] 354530.5
## 
## $SSR
## [1] 237548.5
## 
## $SSE
## [1] 116982
## 
## $r2
## [1] 0.6700369
## 
## $r2adj
## [1] 0.6669531
## 
## $anova
## Analysis of Variance Table
## 
## Response: y
##            Df Sum Sq Mean Sq F value    Pr(>F)    
## x           1 237549  237549  217.28 < 2.2e-16 ***
## Residuals 107 116982    1093                      
## ---
## Signif. codes:  0 '***' 0.001 '**' 0.01 '*' 0.05 '.' 0.1 ' ' 1
## 
## $sigma.hat
## [1] 33.06493
## 
## $se.beta1
## [1] 0.2346521
## 
## $t0
## [1] 14.74038
## 
## $p
## [1] 0
## 
## $F0
## [1] 217.2787
## 
## $p.F.test
## [1] 1.618607e-27
## 
## $CI
## [1] 2.993689 3.924030
## 
## $new.y
## [1] 129.9045
## 
## $CI.new.x
## [1] 122.5827 137.2262
## 
## $predCI.new.x
## [1]  63.94943 195.85947
\end{verbatim}

Using R built-in function, specify as below.

\begin{Shaded}
\begin{Highlighting}[]
    \DocumentationTok{\#\#\#\#confidence interval\#\#\#\#}
    \CommentTok{\#CI for population mean of y when x = 100}
    \FunctionTok{predict}\NormalTok{(LinearReg, }\AttributeTok{newdata =} \FunctionTok{list}\NormalTok{(}\AttributeTok{x=}\DecValTok{100}\NormalTok{), }\AttributeTok{interval=}\FunctionTok{c}\NormalTok{(}\StringTok{"confidence"}\NormalTok{), }\AttributeTok{level=}\FloatTok{0.95}\NormalTok{)}
\end{Highlighting}
\end{Shaded}

\begin{verbatim}
##        fit      lwr      upr
## 1 129.9045 122.5827 137.2262
\end{verbatim}

\begin{Shaded}
\begin{Highlighting}[]
    \DocumentationTok{\#\#\#\#prediction interval\#\#\#\#}
    \CommentTok{\#CI for individual subject\textquotesingle{}s y when x = 100}
    \CommentTok{\#Prediction interval}
    \FunctionTok{predict}\NormalTok{(LinearReg, }\AttributeTok{newdata =} \FunctionTok{list}\NormalTok{(}\AttributeTok{x=}\DecValTok{100}\NormalTok{), }\AttributeTok{interval=}\FunctionTok{c}\NormalTok{(}\StringTok{"prediction"}\NormalTok{), }\AttributeTok{level=}\FloatTok{0.95}\NormalTok{)}
\end{Highlighting}
\end{Shaded}

\begin{verbatim}
##        fit      lwr      upr
## 1 129.9045 63.94943 195.8595
\end{verbatim}

\hypertarget{example-2-1}{%
\subsubsection{Example 2}\label{example-2-1}}

Find a 95\% prediction interval for the price of a 3-year-old Orion.

\[
    \hat{y}_p\pm t_{1-\alpha/2, df=n-2}\hat{\sigma}\sqrt{1+\frac{1}{n}+\frac{(x_p - \bar{x})^2}{\sum (x_i-\bar{x})^2}} = 134.69\pm 2.262\cdot 12.58\sqrt{1+\frac{1}{11}+\frac{(3-58/11)^2}{326-58^2/11}} = 134.69\pm 33.02.
\] Interpretation: We can be 95\% confident that the price of a
3-year-old Orion will be somewhere between \$10,167 and \$16,771.

\begin{figure}
\centering
\includegraphics{https://i.ibb.co/CswVwb8s/683-Figure-15-11.png}
\caption{Confidence interval and prediction
interval}
\end{figure}

\hypertarget{extension-not-to-cover-in-class-hypothesis-test-h_0-beta_10-with-bootstrap-sampling}{%
\section{\texorpdfstring{Extension (Not to cover in class): Hypothesis test \(H_0: \beta_1=0\) with bootstrap sampling}{Extension (Not to cover in class): Hypothesis test H\_0: \textbackslash beta\_1=0 with bootstrap sampling}}\label{extension-not-to-cover-in-class-hypothesis-test-h_0-beta_10-with-bootstrap-sampling}}

There are two common methods of bootstrapping for testing
\(H_0: \beta_1=0\) in linear regression model.

\hypertarget{empirical-bootstrap}{%
\subsection{Empirical Bootstrap}\label{empirical-bootstrap}}

Given the original sample pairs \((X_1, Y_1), \cdots, (X_n, Y_n)\),
randomly sampling the \(n\) pairs with replacement to generate a new set
\((X_1^{*(1)}, Y_1^*(1)), \cdots, (X_n^*(1), Y_n^*(1))\). Note that for
each \(l\), \(P(X_l^{*(1)} = X_i, Y_l^{*(1)} = Y_i) =\frac{1}{n}\). Namely,
we treat \((X_i, Y_i)\) as one object and we sample with replacement \(n\)
times from these \(n\) objects to form a new bootstrap sample. Thus, each
time we generate a set of \(n\) new observations from the original data.

Repeat the entire process \(B\) times, then obtain \(B\) bootstrap samples
\[(X_1^{*(b)}, Y_1^*(b)), \cdots, (X_n^*(b), Y_n^*(b))\] for
\(b=1,\cdots,B\). Perform a linear regression for each bootstrap sample,
and obtain the estimates of coefficients
\((\hat{\beta}_0^{*(b)}, \hat{\beta}_1^{*(b)})\). Then the variance of
\(\hat{\beta}_1\) can be estimated by the \(B\) estimates of
\(\hat{\beta}_1^{*(b)}\). The confidence interval can be obtained by the
empirical distribution of \(\hat{\beta}_1\) constructed based on the
bootstrap samples. \(H_0\) is rejected if the CI excludes 0.

The following R code implements the empirical bootstrap linear
regression. With 10000 bootstrapping samples, the coefficient estimates,
standard errors and confidence intervals are similar to the linear
regression.

\begin{Shaded}
\begin{Highlighting}[]
\FunctionTok{empirical.boot.lm}\NormalTok{(x, y)}
\end{Highlighting}
\end{Shaded}

\begin{verbatim}
## $beta
## [1] -216.724608    3.467808
## 
## $beta0.CI
##      2.5%     97.5% 
## -265.0083 -167.7649 
## 
## $beta1.CI
##     2.5%    97.5% 
## 2.897061 4.034166 
## 
## $se.beta
## [1] 25.0709524  0.2936586
\end{verbatim}

\hypertarget{residual-bootstrap}{%
\subsection{Residual Bootstrap}\label{residual-bootstrap}}

Although the empirical bootstrap works well in theory, in practice it
might lead to a bad result especially in the presence of influential
observations (some \(X_i\) very far away from the others). When we do an
empirical bootstrap, if we do not select those points, the regression
coefficients can be very different. To resolve this problem, we may use
the residual bootstrap.

First, sample the residuals \((e_1, \cdots, e_n)\) with replacement and
denote the bootstrap sample as \((e_1^{*(b)}, \cdots, e_n^{*(b)})\) for
\(b=1,\cdots,B\). Then construct the bootstrap samples

\[
(X_1, \hat{\beta}_0+\hat{\beta}_1X_i+e_1^{*(b)}), \cdots, (X_n, \hat{\beta}_0+\hat{\beta}_1X_n+e_n^{*(b)})
\] where \(\hat{\beta}_0\) and \(\hat{\beta}_1\) are from the original
fitted coefficients.

Then, for each bootstrap sample, fit a linear regression and obtain the
coefficients \(\hat{\beta}_0^{*(b)}, \hat{\beta}_1^{*(b)}\) for
\(b=1,\cdots,B\).

All the estimate of the variance, MSE, and construction of the CI are
the same as the empirical bootstrap. The procedure is implemented below.

\begin{Shaded}
\begin{Highlighting}[]
\FunctionTok{residual.boot.lm}\NormalTok{(x,y)}
\end{Highlighting}
\end{Shaded}

\begin{verbatim}
## $beta.hat
## [1] -215.932916    3.458757
## 
## $beta0.CI
##      2.5%     97.5% 
## -258.6923 -173.6308 
## 
## $beta1.CI
##     2.5%    97.5% 
## 3.003016 3.913772 
## 
## $se.beta
## [1] 21.5521624  0.2320543
\end{verbatim}

\hypertarget{multiple-linear-regression}{%
\chapter{Multiple Linear Regression}\label{multiple-linear-regression}}

\begin{Shaded}
\begin{Highlighting}[]
\FunctionTok{library}\NormalTok{(IntroStats)}
\end{Highlighting}
\end{Shaded}

\hypertarget{multiple-linear-regression-model}{%
\section{Multiple Linear Regression Model}\label{multiple-linear-regression-model}}

\hypertarget{model-specification-and-assumptions}{%
\subsection{Model specification and assumptions}\label{model-specification-and-assumptions}}

\hypertarget{model-specification}{%
\subsubsection{Model specification}\label{model-specification}}

For subject \(j\),
\[
y_j =  \beta_0 + \beta_1 x_{1j} + \beta_2 x_{2j} + \cdots + \beta_k x_{kj} + \epsilon_j
\]

\hypertarget{model-assumptions-1}{%
\subsubsection{Model assumptions}\label{model-assumptions-1}}

\begin{itemize}
\tightlist
\item
  For each set of \(x_j\), there is a subpopulation of \(Y\) values, The
  subpopulations must be normally distributed.
\item
  The variances of the subpopulations of \(Y\) are all equal and denoted
  by \(\sigma^2\).
\item
  The means of the subpopulations of \(Y\) all lie on the same straight
  line. This is known as the assumption of linearity, i.e., \[
  \mu_{y|x_j} = \beta_0+\beta_1 x_{1j} + \beta_2 x_{2j} + \cdots + \beta_k x_{kj}
  \]
\item
  The \(Y\) values are statistically independent. This means that, in
  drawing the sample, it is assumed that the values of \(Y\) chosen at
  one value of \(X\) in no way depend on the values of \(Y\) chosen at
  another value of \(X\).
\end{itemize}

\hypertarget{coefficient-estimation-via-least-squares}{%
\subsubsection{Coefficient Estimation via Least Squares}\label{coefficient-estimation-via-least-squares}}

This section shows how to derive \(\hat{\beta}\) using the least squares
criterion.

The estimates of coefficients \(\beta_0, \cdots, \beta_k\) can be obtained
by least squares approach minimizing the sum square of error.

\[
\sum \epsilon_j^2 = \sum (y_j - \beta_0 - \beta_1 x_{1j} - \cdots - \beta_k x_{kj})^2
\]

The solution to the coefficients is:

\[
\hat{\beta} = (X^\prime X)^{-1}X^\prime y
\]

where \(X\) is the covariate matrix, organized as the collection of
covariates.

\[
X = 
\left(\begin{array}{ccccc} 
1 & x_{11} & x_{21} & \cdots & x_{k1}\\
1 & x_{12} & x_{22} & \cdots & x_{k2}\\
\cdots & \cdots & \cdots & \cdots & \cdots\\
1 & x_{1n} & x_{2n} & \cdots & x_{kn}\\
\end{array}\right)
\]

The \(j\)th row in the covariate matrix represents subject \(j\)'s covariate
data.

For the special case when \(k=1\), then \(\hat{beta}\) reduces to the SLR.

\hypertarget{exploratory-data-analysis-eda-visualization}{%
\subsection{Exploratory Data Analysis (EDA) visualization}\label{exploratory-data-analysis-eda-visualization}}

\hypertarget{example-53}{%
\subsubsection{Example}\label{example-53}}

Researchers would like to use age and education level to predict the
capacity of direct attention (CDA) in elderly subjects. CDA refers to
neural inhibitory mechanisms that focus on the mind on what is
meaningful while blocking out distractions. The study collected 71 older
women with normal mental status.

\begin{Shaded}
\begin{Highlighting}[]
\DocumentationTok{\#\#\#\#\#\#\#\#\#\#\#\#\#\#\#\#\#\#\#\#\#\#\#\#\#\#\#\#\#\#}
\CommentTok{\#(1) Motivating Example }
\CommentTok{\#Age and Education level to predict CDA}
\DocumentationTok{\#\#\#\#\#\#\#\#\#\#\#\#\#\#\#\#\#\#\#\#\#\#\#\#\#\#\#\#\#\#}

\NormalTok{Age }\OtherTok{\textless{}{-}} \FunctionTok{c}\NormalTok{(}\DecValTok{72}\NormalTok{,}\DecValTok{68}\NormalTok{,}\DecValTok{65}\NormalTok{,}\DecValTok{85}\NormalTok{,}\DecValTok{84}\NormalTok{,}\DecValTok{90}\NormalTok{,}\DecValTok{79}\NormalTok{,}\DecValTok{74}\NormalTok{,}\DecValTok{69}\NormalTok{,}\DecValTok{87}\NormalTok{,}\DecValTok{84}\NormalTok{,}\DecValTok{79}\NormalTok{,}\DecValTok{71}\NormalTok{,}\DecValTok{76}\NormalTok{,}
         \DecValTok{73}\NormalTok{,}\DecValTok{86}\NormalTok{,}\DecValTok{69}\NormalTok{,}\DecValTok{66}\NormalTok{,}\DecValTok{65}\NormalTok{,}\DecValTok{71}\NormalTok{,}\DecValTok{80}\NormalTok{,}\DecValTok{81}\NormalTok{,}\DecValTok{66}\NormalTok{,}\DecValTok{76}\NormalTok{,}\DecValTok{70}\NormalTok{,}\DecValTok{76}\NormalTok{,}\DecValTok{67}\NormalTok{,}\DecValTok{72}\NormalTok{,}
         \DecValTok{68}\NormalTok{,}\DecValTok{102}\NormalTok{,}\DecValTok{79}\NormalTok{,}\DecValTok{87}\NormalTok{,}\DecValTok{71}\NormalTok{,}\DecValTok{81}\NormalTok{,}\DecValTok{66}\NormalTok{,}\DecValTok{81}\NormalTok{,}\DecValTok{80}\NormalTok{,}\DecValTok{82}\NormalTok{,}\DecValTok{65}\NormalTok{,}\DecValTok{73}\NormalTok{,}\DecValTok{85}\NormalTok{,}\DecValTok{83}\NormalTok{,}
         \DecValTok{83}\NormalTok{,}\DecValTok{76}\NormalTok{,}\DecValTok{77}\NormalTok{,}\DecValTok{83}\NormalTok{,}\DecValTok{79}\NormalTok{,}\DecValTok{69}\NormalTok{,}\DecValTok{66}\NormalTok{,}\DecValTok{75}\NormalTok{,}\DecValTok{77}\NormalTok{,}\DecValTok{78}\NormalTok{,}\DecValTok{83}\NormalTok{,}\DecValTok{85}\NormalTok{,}\DecValTok{76}\NormalTok{,}\DecValTok{75}\NormalTok{,}\DecValTok{70}\NormalTok{,}
         \DecValTok{79}\NormalTok{,}\DecValTok{75}\NormalTok{,}\DecValTok{94}\NormalTok{,}\DecValTok{67}\NormalTok{,}\DecValTok{66}\NormalTok{,}\DecValTok{75}\NormalTok{,}\DecValTok{91}\NormalTok{,}\DecValTok{74}\NormalTok{,}\DecValTok{90}\NormalTok{,}\DecValTok{76}\NormalTok{,}\DecValTok{84}\NormalTok{,}\DecValTok{79}\NormalTok{,}\DecValTok{78}\NormalTok{,}\DecValTok{79}\NormalTok{)}

\NormalTok{Ed.Level }\OtherTok{\textless{}{-}} \FunctionTok{c}\NormalTok{(}\DecValTok{20}\NormalTok{,}\DecValTok{12}\NormalTok{,}\DecValTok{13}\NormalTok{,}\DecValTok{14}\NormalTok{,}\DecValTok{13}\NormalTok{,}\DecValTok{15}\NormalTok{,}\DecValTok{12}\NormalTok{,}\DecValTok{10}\NormalTok{,}\DecValTok{12}\NormalTok{,}\DecValTok{15}\NormalTok{,}\DecValTok{12}\NormalTok{,}\DecValTok{12}\NormalTok{,}\DecValTok{12}\NormalTok{,}
             \DecValTok{14}\NormalTok{,}\DecValTok{14}\NormalTok{,}\DecValTok{12}\NormalTok{,}\DecValTok{17}\NormalTok{,}\DecValTok{11}\NormalTok{,}\DecValTok{16}\NormalTok{,}\DecValTok{14}\NormalTok{,}\DecValTok{18}\NormalTok{,}\DecValTok{11}\NormalTok{,}\DecValTok{14}\NormalTok{,}\DecValTok{17}\NormalTok{,}\DecValTok{12}\NormalTok{,}\DecValTok{12}\NormalTok{,}
             \DecValTok{12}\NormalTok{,}\DecValTok{20}\NormalTok{,}\DecValTok{18}\NormalTok{,}\DecValTok{12}\NormalTok{,}\DecValTok{12}\NormalTok{,}\DecValTok{12}\NormalTok{,}\DecValTok{14}\NormalTok{,}\DecValTok{16}\NormalTok{,}\DecValTok{16}\NormalTok{,}\DecValTok{16}\NormalTok{,}\DecValTok{13}\NormalTok{,}\DecValTok{12}\NormalTok{,}\DecValTok{13}\NormalTok{,}
             \DecValTok{16}\NormalTok{,}\DecValTok{16}\NormalTok{,}\DecValTok{17}\NormalTok{,}\DecValTok{8}\NormalTok{,}\DecValTok{20}\NormalTok{,}\DecValTok{12}\NormalTok{,}\DecValTok{12}\NormalTok{,}\DecValTok{14}\NormalTok{,}\DecValTok{12}\NormalTok{,}\DecValTok{14}\NormalTok{,}\DecValTok{12}\NormalTok{,}\DecValTok{16}\NormalTok{,}\DecValTok{12}\NormalTok{,}
             \DecValTok{20}\NormalTok{,}\DecValTok{10}\NormalTok{,}\DecValTok{18}\NormalTok{,}\DecValTok{14}\NormalTok{,}\DecValTok{16}\NormalTok{,}\DecValTok{16}\NormalTok{,}\DecValTok{18}\NormalTok{,}\DecValTok{8}\NormalTok{,}\DecValTok{12}\NormalTok{,}\DecValTok{14}\NormalTok{,}\DecValTok{18}\NormalTok{,}\DecValTok{13}\NormalTok{,}\DecValTok{15}\NormalTok{,}
             \DecValTok{15}\NormalTok{,}\DecValTok{18}\NormalTok{,}\DecValTok{18}\NormalTok{,}\DecValTok{17}\NormalTok{,}\DecValTok{16}\NormalTok{,}\DecValTok{12}\NormalTok{)}

\NormalTok{CDA }\OtherTok{\textless{}{-}} \FunctionTok{c}\NormalTok{(}\FloatTok{4.57}\NormalTok{,}\SpecialCharTok{{-}}\FloatTok{3.04}\NormalTok{,}\FloatTok{1.39}\NormalTok{,}\SpecialCharTok{{-}}\FloatTok{3.55}\NormalTok{,}\SpecialCharTok{{-}}\FloatTok{2.56}\NormalTok{,}\SpecialCharTok{{-}}\FloatTok{4.66}\NormalTok{,}\SpecialCharTok{{-}}\FloatTok{2.70}\NormalTok{,}\FloatTok{0.30}\NormalTok{,}
         \SpecialCharTok{{-}}\FloatTok{4.46}\NormalTok{,}\SpecialCharTok{{-}}\FloatTok{6.29}\NormalTok{,}\SpecialCharTok{{-}}\FloatTok{4.43}\NormalTok{,}\FloatTok{0.18}\NormalTok{,}\SpecialCharTok{{-}}\FloatTok{1.37}\NormalTok{,}\FloatTok{3.26}\NormalTok{,}\SpecialCharTok{{-}}\FloatTok{1.12}\NormalTok{,}\SpecialCharTok{{-}}\FloatTok{0.77}\NormalTok{,}
         \FloatTok{3.73}\NormalTok{,}\SpecialCharTok{{-}}\FloatTok{5.92}\NormalTok{,}\FloatTok{5.74}\NormalTok{,}\FloatTok{2.83}\NormalTok{,}\SpecialCharTok{{-}}\FloatTok{2.40}\NormalTok{,}\SpecialCharTok{{-}}\FloatTok{0.29}\NormalTok{,}\FloatTok{4.44}\NormalTok{,}\FloatTok{3.35}\NormalTok{,}\SpecialCharTok{{-}}\FloatTok{3.13}\NormalTok{,}
         \SpecialCharTok{{-}}\FloatTok{2.14}\NormalTok{,}\FloatTok{9.61}\NormalTok{,}\FloatTok{7.57}\NormalTok{,}\FloatTok{2.21}\NormalTok{,}\SpecialCharTok{{-}}\FloatTok{2.30}\NormalTok{,}\FloatTok{3.17}\NormalTok{,}\SpecialCharTok{{-}}\FloatTok{1.19}\NormalTok{,}\FloatTok{0.99}\NormalTok{,}\SpecialCharTok{{-}}\FloatTok{2.94}\NormalTok{,}
         \SpecialCharTok{{-}}\FloatTok{2.21}\NormalTok{,}\SpecialCharTok{{-}}\FloatTok{0.75}\NormalTok{,}\FloatTok{5.07}\NormalTok{,}\SpecialCharTok{{-}}\FloatTok{5.86}\NormalTok{,}\FloatTok{5.00}\NormalTok{,}\FloatTok{0.63}\NormalTok{,}\FloatTok{2.62}\NormalTok{,}\FloatTok{1.77}\NormalTok{,}\SpecialCharTok{{-}}\FloatTok{3.79}\NormalTok{,}\FloatTok{1.44}\NormalTok{,}
         \SpecialCharTok{{-}}\FloatTok{5.77}\NormalTok{,}\SpecialCharTok{{-}}\FloatTok{5.77}\NormalTok{,}\SpecialCharTok{{-}}\FloatTok{4.62}\NormalTok{,}\SpecialCharTok{{-}}\FloatTok{2.03}\NormalTok{,}\SpecialCharTok{{-}}\FloatTok{2.22}\NormalTok{,}\FloatTok{0.80}\NormalTok{,}\SpecialCharTok{{-}}\FloatTok{0.75}\NormalTok{,}\SpecialCharTok{{-}}\FloatTok{4.60}\NormalTok{,}
         \FloatTok{2.68}\NormalTok{,}\SpecialCharTok{{-}}\FloatTok{3.69}\NormalTok{,}\FloatTok{4.85}\NormalTok{,}\SpecialCharTok{{-}}\FloatTok{0.08}\NormalTok{,}\FloatTok{0.63}\NormalTok{,}\FloatTok{5.92}\NormalTok{,}\FloatTok{3.63}\NormalTok{,}\SpecialCharTok{{-}}\FloatTok{7.07}\NormalTok{,}\FloatTok{1.73}\NormalTok{,}
         \FloatTok{6.03}\NormalTok{,}\SpecialCharTok{{-}}\FloatTok{0.02}\NormalTok{,}\SpecialCharTok{{-}}\FloatTok{7.65}\NormalTok{,}\FloatTok{4.17}\NormalTok{,}\SpecialCharTok{{-}}\FloatTok{0.68}\NormalTok{,}\FloatTok{6.39}\NormalTok{,}\SpecialCharTok{{-}}\FloatTok{0.08}\NormalTok{,}\FloatTok{1.07}\NormalTok{,}\FloatTok{5.31}\NormalTok{,}\FloatTok{0.30}\NormalTok{)}

\NormalTok{dat }\OtherTok{\textless{}{-}} \FunctionTok{as.data.frame}\NormalTok{(}\FunctionTok{cbind}\NormalTok{(CDA, Age, Ed.Level))}

\CommentTok{\#Draw matrix scatter plot}
\FunctionTok{pairs}\NormalTok{(dat[,}\DecValTok{1}\SpecialCharTok{:}\DecValTok{3}\NormalTok{], }\AttributeTok{pch =} \DecValTok{19}\NormalTok{)}
\end{Highlighting}
\end{Shaded}

\includegraphics{StatsTB_files/figure-latex/unnamed-chunk-364-1.pdf}

\begin{Shaded}
\begin{Highlighting}[]
\CommentTok{\#Install the package if not yet}
\CommentTok{\#install.packages("psych")}
\NormalTok{psych}\SpecialCharTok{::}\FunctionTok{pairs.panels}\NormalTok{(dat[,}\DecValTok{1}\SpecialCharTok{:}\DecValTok{3}\NormalTok{], }
             \AttributeTok{method =} \StringTok{"pearson"}\NormalTok{, }\CommentTok{\# correlation method}
             \AttributeTok{hist.col =} \StringTok{"\#00AFBB"}\NormalTok{,}
             \AttributeTok{density =} \ConstantTok{TRUE}\NormalTok{,  }\CommentTok{\# show density plots}
             \AttributeTok{ellipses =} \ConstantTok{TRUE} \CommentTok{\# show correlation ellipses}
\NormalTok{             )}
\end{Highlighting}
\end{Shaded}

\includegraphics{StatsTB_files/figure-latex/unnamed-chunk-364-2.pdf}

\hypertarget{fit-regression-model-and-interpretation}{%
\subsection{Fit regression model and interpretation}\label{fit-regression-model-and-interpretation}}

\begin{Shaded}
\begin{Highlighting}[]
\DocumentationTok{\#\#\#\#\#\#\#\#\#\#\#\#\#\#\#\#\#\#\#\#\#\#\#\#\#\#\#\#\#\#}
\CommentTok{\#(2) Regression Line and prediction}
\DocumentationTok{\#\#\#\#\#\#\#\#\#\#\#\#\#\#\#\#\#\#\#\#\#\#\#\#\#\#\#\#\#\#}
\NormalTok{        X }\OtherTok{\textless{}{-}} \FunctionTok{cbind}\NormalTok{(}\FunctionTok{rep}\NormalTok{(}\DecValTok{1}\NormalTok{, }\FunctionTok{length}\NormalTok{(Age)), Age, Ed.Level)}
\NormalTok{        Y }\OtherTok{\textless{}{-}} \FunctionTok{matrix}\NormalTok{(CDA, }\AttributeTok{ncol=}\DecValTok{1}\NormalTok{)}
\NormalTok{        beta }\OtherTok{\textless{}{-}} \FunctionTok{solve}\NormalTok{(}\FunctionTok{t}\NormalTok{(X)}\SpecialCharTok{\%*\%}\NormalTok{X)}\SpecialCharTok{\%*\%}\FunctionTok{t}\NormalTok{(X)}\SpecialCharTok{\%*\%}\NormalTok{Y}
    
\NormalTok{        beta }
\end{Highlighting}
\end{Shaded}

\begin{verbatim}
##                [,1]
##           5.4940733
## Age      -0.1841250
## Ed.Level  0.6107786
\end{verbatim}

Use R built-in function lm().

\begin{Shaded}
\begin{Highlighting}[]
\NormalTok{        LinearReg }\OtherTok{\textless{}{-}} \FunctionTok{lm}\NormalTok{(CDA }\SpecialCharTok{\textasciitilde{}}\NormalTok{ Age }\SpecialCharTok{+}\NormalTok{ Ed.Level) }\CommentTok{\#Create the linear regression}
        \FunctionTok{summary}\NormalTok{(LinearReg)    }\CommentTok{\#Review the results}
\end{Highlighting}
\end{Shaded}

\begin{verbatim}
## 
## Call:
## lm(formula = CDA ~ Age + Ed.Level)
## 
## Residuals:
##     Min      1Q  Median      3Q     Max 
## -5.9804 -2.2125 -0.0761  2.2824  9.1230 
## 
## Coefficients:
##             Estimate Std. Error t value Pr(>|t|)    
## (Intercept)  5.49407    4.44297   1.237 0.220498    
## Age         -0.18412    0.04851  -3.795 0.000316 ***
## Ed.Level     0.61078    0.13565   4.503  2.7e-05 ***
## ---
## Signif. codes:  0 '***' 0.001 '**' 0.01 '*' 0.05 '.' 0.1 ' ' 1
## 
## Residual standard error: 3.134 on 68 degrees of freedom
## Multiple R-squared:  0.3706, Adjusted R-squared:  0.3521 
## F-statistic: 20.02 on 2 and 68 DF,  p-value: 1.454e-07
\end{verbatim}

\begin{Shaded}
\begin{Highlighting}[]
\NormalTok{        y\_hat }\OtherTok{\textless{}{-}}\NormalTok{ LinearReg}\SpecialCharTok{$}\NormalTok{fitted.values }\CommentTok{\#fitted values y\_hat}
\end{Highlighting}
\end{Shaded}

\hypertarget{model-interpretation-and-anova}{%
\subsection{Model interpretation and ANOVA}\label{model-interpretation-and-anova}}

\begin{Shaded}
\begin{Highlighting}[]
\FunctionTok{anova}\NormalTok{(LinearReg)}
\end{Highlighting}
\end{Shaded}

\begin{verbatim}
## Analysis of Variance Table
## 
## Response: CDA
##           Df Sum Sq Mean Sq F value    Pr(>F)    
## Age        1 194.24 194.241  19.774 3.306e-05 ***
## Ed.Level   1 199.15 199.147  20.273 2.703e-05 ***
## Residuals 68 667.97   9.823                      
## ---
## Signif. codes:  0 '***' 0.001 '**' 0.01 '*' 0.05 '.' 0.1 ' ' 1
\end{verbatim}

\hypertarget{adding-interaction-term}{%
\subsection{Adding interaction term}\label{adding-interaction-term}}

Use R built-in function lm().

\begin{Shaded}
\begin{Highlighting}[]
\CommentTok{\#Create the linear regression}
\NormalTok{        LinearReg2 }\OtherTok{\textless{}{-}} \FunctionTok{lm}\NormalTok{(CDA }\SpecialCharTok{\textasciitilde{}}\NormalTok{ Age }\SpecialCharTok{+}\NormalTok{ Ed.Level }\SpecialCharTok{+}\NormalTok{ Age}\SpecialCharTok{:}\NormalTok{Ed.Level) }

        \FunctionTok{summary}\NormalTok{(LinearReg2)    }\CommentTok{\#Review the results}
\end{Highlighting}
\end{Shaded}

\begin{verbatim}
## 
## Call:
## lm(formula = CDA ~ Age + Ed.Level + Age:Ed.Level)
## 
## Residuals:
##     Min      1Q  Median      3Q     Max 
## -5.7691 -2.3867 -0.1381  2.1007  9.3937 
## 
## Coefficients:
##               Estimate Std. Error t value Pr(>|t|)
## (Intercept)  -10.53645   20.93067  -0.503    0.616
## Age            0.02095    0.26611   0.079    0.937
## Ed.Level       1.81530    1.54267   1.177    0.243
## Age:Ed.Level  -0.01547    0.01973  -0.784    0.436
## 
## Residual standard error: 3.143 on 67 degrees of freedom
## Multiple R-squared:  0.3764, Adjusted R-squared:  0.3484 
## F-statistic: 13.48 on 3 and 67 DF,  p-value: 5.596e-07
\end{verbatim}

\begin{Shaded}
\begin{Highlighting}[]
\NormalTok{        y\_hat }\OtherTok{\textless{}{-}}\NormalTok{ LinearReg2}\SpecialCharTok{$}\NormalTok{fitted.values }\CommentTok{\#fitted values y\_hat}
\end{Highlighting}
\end{Shaded}

The interaction term is not significant and can be removed. Check the
ANOVA table with interaction.

\begin{Shaded}
\begin{Highlighting}[]
\FunctionTok{anova}\NormalTok{(LinearReg2)}
\end{Highlighting}
\end{Shaded}

\begin{verbatim}
## Analysis of Variance Table
## 
## Response: CDA
##              Df Sum Sq Mean Sq F value    Pr(>F)    
## Age           1 194.24 194.241 19.6618 3.518e-05 ***
## Ed.Level      1 199.15 199.147 20.1584 2.880e-05 ***
## Age:Ed.Level  1   6.07   6.070  0.6144    0.4359    
## Residuals    67 661.90   9.879                      
## ---
## Signif. codes:  0 '***' 0.001 '**' 0.01 '*' 0.05 '.' 0.1 ' ' 1
\end{verbatim}

\hypertarget{confidence-interval-and-prediction-interval}{%
\subsection{Confidence Interval and Prediction Interval}\label{confidence-interval-and-prediction-interval}}

In the context of multiple linear regression, the \textbf{confidence
interval} and \textbf{prediction interval} are used to quantify the
uncertainty in the estimation of a response variable \(Y\).

\hypertarget{confidence-interval-for-the-mean-response}{%
\subsubsection{Confidence Interval for the Mean Response}\label{confidence-interval-for-the-mean-response}}

A confidence interval provides a range of values that estimate the mean
response \(Y\) for a given set of predictor values.

\[
\hat{Y} \pm t_{\alpha/2, n-k-1} \sqrt{\frac{\text{MSE}}{n} + (\mathbf{x}_0^\top (\mathbf{X}^\top \mathbf{X})^{-1} \mathbf{x}_0)}
\]

\begin{itemize}
\tightlist
\item
  \(\hat{Y}\): Predicted mean response for the given set of predictors.
\item
  \(t_{\alpha/2, n-k-1}\): Critical value from the t-distribution with
  \(n - k - 1\) degrees of freedom.
\item
  \(\text{MSE}\): Mean Squared Error of the regression model.
  \(MSE = SSE / (n-k-1)\).
\item
  \(n\): Number of observations.
\item
  \(k\): Number of predictors in the model.
\item
  \(\mathbf{x}_0\): Vector of the given predictor values (including an
  intercept term if necessary).
\item
  \(\mathbf{X}\): Design matrix of the predictors (including an
  intercept term if necessary).
\end{itemize}

The term
\(\mathbf{x}_0^\top (\mathbf{X}^\top \mathbf{X})^{-1} \mathbf{x}_0\)
accounts for the variability in the prediction of the mean response.

\hypertarget{prediction-interval-for-a-new-observation}{%
\subsubsection{Prediction Interval for a New Observation}\label{prediction-interval-for-a-new-observation}}

A prediction interval provides a range within which a future observation
is likely to fall, for the same set of predictor values.

\[
\hat{Y} \pm t_{\alpha/2, n-k-1} \sqrt{\text{MSE} \left( 1 + \frac{1}{n} + (\mathbf{x}_0^\top (\mathbf{X}^\top \mathbf{X})^{-1} \mathbf{x}_0) \right)}
\]

\begin{itemize}
\tightlist
\item
  The key difference from the confidence interval formula is the
  additional term \(1\) inside the square root. This term accounts for
  the extra variability associated with predicting an individual
  observation rather than the mean response.
\end{itemize}

\hypertarget{comparison-confidence-vs.-prediction-interval}{%
\subsubsection{Comparison: Confidence vs.~Prediction Interval}\label{comparison-confidence-vs.-prediction-interval}}

\begin{itemize}
\item
  \textbf{Confidence Interval}: Estimates the mean response at given
  predictor values. It is narrower because it does not account for the
  random error term \(\varepsilon\) associated with individual
  observations.
\item
  \textbf{Prediction Interval}: Estimates a range for a single new
  observation. It is wider because it accounts for both the
  uncertainty in the mean response and the random error.
\end{itemize}

\begin{Shaded}
\begin{Highlighting}[]
        \CommentTok{\#Confidence interval}
    \FunctionTok{cat}\NormalTok{(}\StringTok{"Confidence Interval:}\SpecialCharTok{\textbackslash{}n}\StringTok{"}\NormalTok{)}
\end{Highlighting}
\end{Shaded}

\begin{verbatim}
## Confidence Interval:
\end{verbatim}

\begin{Shaded}
\begin{Highlighting}[]
        \FunctionTok{predict}\NormalTok{(LinearReg, }\AttributeTok{newdata =} \FunctionTok{list}\NormalTok{(}\AttributeTok{Age=}\DecValTok{70}\NormalTok{, }\AttributeTok{Ed.Level=}\DecValTok{12}\NormalTok{), }
                \AttributeTok{interval=}\FunctionTok{c}\NormalTok{(}\StringTok{"confidence"}\NormalTok{), }\AttributeTok{level=}\FloatTok{0.95}\NormalTok{)}
\end{Highlighting}
\end{Shaded}

\begin{verbatim}
##           fit       lwr     upr
## 1 -0.06533267 -1.277825 1.14716
\end{verbatim}

\begin{Shaded}
\begin{Highlighting}[]
        \CommentTok{\#Prediction interval}
        \FunctionTok{cat}\NormalTok{(}\StringTok{"}\SpecialCharTok{\textbackslash{}n}\StringTok{Prediction Interval:}\SpecialCharTok{\textbackslash{}n}\StringTok{"}\NormalTok{)}
\end{Highlighting}
\end{Shaded}

\begin{verbatim}
## 
## Prediction Interval:
\end{verbatim}

\begin{Shaded}
\begin{Highlighting}[]
        \FunctionTok{predict}\NormalTok{(LinearReg, }\AttributeTok{newdata =} \FunctionTok{list}\NormalTok{(}\AttributeTok{Age=}\DecValTok{70}\NormalTok{, }\AttributeTok{Ed.Level=}\DecValTok{12}\NormalTok{), }
                \AttributeTok{interval=}\FunctionTok{c}\NormalTok{(}\StringTok{"prediction"}\NormalTok{), }\AttributeTok{level=}\FloatTok{0.95}\NormalTok{)}
\end{Highlighting}
\end{Shaded}

\begin{verbatim}
##           fit       lwr     upr
## 1 -0.06533267 -6.435945 6.30528
\end{verbatim}

\hypertarget{hypothesis-test-for-coefficients}{%
\subsection{Hypothesis test for coefficients}\label{hypothesis-test-for-coefficients}}

To test the null hypothesis that \(\beta_i\) is equal to some particular
value, say \(\beta_{i0}\), use the t statistic, \[
t = \frac{\hat{\beta}_i - \beta_{i0}}{se(\hat{\beta}_i)}
\] where the degrees of freedom are \(n-k-1\) and \(se(\hat{\beta}_i)\) is
standard deviation of \(\hat{\beta}_i\). A special case of
\(\beta_{i0} = 0\) is to test whether the \(i\)th covariate is not
significant.

\begin{Shaded}
\begin{Highlighting}[]
\DocumentationTok{\#\#\#\#\#\#\#\#\#\#\#\#\#\#\#\#\#\#\#\#\#\#\#\#\#\#\#\#\#\#}
\CommentTok{\#(3) Hypothesis test for coefficient}
\DocumentationTok{\#\#\#\#\#\#\#\#\#\#\#\#\#\#\#\#\#\#\#\#\#\#\#\#\#\#\#\#\#\#}
\NormalTok{        LinearReg }\OtherTok{\textless{}{-}} \FunctionTok{lm}\NormalTok{(CDA }\SpecialCharTok{\textasciitilde{}}\NormalTok{ Age }\SpecialCharTok{+}\NormalTok{ Ed.Level) }\CommentTok{\#Create the linear regression}
        \FunctionTok{summary}\NormalTok{(LinearReg)    }\CommentTok{\#Review the results}
\end{Highlighting}
\end{Shaded}

\begin{verbatim}
## 
## Call:
## lm(formula = CDA ~ Age + Ed.Level)
## 
## Residuals:
##     Min      1Q  Median      3Q     Max 
## -5.9804 -2.2125 -0.0761  2.2824  9.1230 
## 
## Coefficients:
##             Estimate Std. Error t value Pr(>|t|)    
## (Intercept)  5.49407    4.44297   1.237 0.220498    
## Age         -0.18412    0.04851  -3.795 0.000316 ***
## Ed.Level     0.61078    0.13565   4.503  2.7e-05 ***
## ---
## Signif. codes:  0 '***' 0.001 '**' 0.01 '*' 0.05 '.' 0.1 ' ' 1
## 
## Residual standard error: 3.134 on 68 degrees of freedom
## Multiple R-squared:  0.3706, Adjusted R-squared:  0.3521 
## F-statistic: 20.02 on 2 and 68 DF,  p-value: 1.454e-07
\end{verbatim}

In this example, the t value for Age is -3.379 which falls in the
rejection region of 5\% significance level, so we conclude that Age is
statistically significant for CDA.

\hypertarget{confidence-interval-for-coefficient}{%
\subsection{Confidence interval for coefficient}\label{confidence-interval-for-coefficient}}

The Confidence interval for coefficient \(\beta_i\) is
\[
\hat{\beta}_i \pm t_{1-\alpha/2, \text{ } df=n-k-1} \cdot se(\hat{\beta}_i)
\]

\begin{Shaded}
\begin{Highlighting}[]
\DocumentationTok{\#\#\#\#\#\#\#\#\#\#\#\#\#\#\#\#\#\#\#\#\#\#\#\#\#\#\#\#\#\#}
\CommentTok{\#(4) Confidence Interval for beta1}
\DocumentationTok{\#\#\#\#\#\#\#\#\#\#\#\#\#\#\#\#\#\#\#\#\#\#\#\#\#\#\#\#\#\#}
\NormalTok{se }\OtherTok{\textless{}{-}} \FunctionTok{summary}\NormalTok{(LinearReg)}\SpecialCharTok{$}\NormalTok{coeff[,}\DecValTok{2}\NormalTok{]}
\NormalTok{L }\OtherTok{=}\NormalTok{ beta }\SpecialCharTok{{-}} \FunctionTok{qt}\NormalTok{(}\FloatTok{0.975}\NormalTok{, LinearReg}\SpecialCharTok{$}\NormalTok{df.resid) }\SpecialCharTok{*}\NormalTok{ se}
\NormalTok{U }\OtherTok{=}\NormalTok{ beta }\SpecialCharTok{+} \FunctionTok{qt}\NormalTok{(}\FloatTok{0.975}\NormalTok{, LinearReg}\SpecialCharTok{$}\NormalTok{df.resid) }\SpecialCharTok{*}\NormalTok{ se}
\FunctionTok{data.frame}\NormalTok{(L, U)}
\end{Highlighting}
\end{Shaded}

\begin{verbatim}
##                   L          U
##          -3.3717339 14.3598805
## Age      -0.2809333 -0.0873167
## Ed.Level  0.3400927  0.8814645
\end{verbatim}

\begin{Shaded}
\begin{Highlighting}[]
\FunctionTok{confint}\NormalTok{(LinearReg)}
\end{Highlighting}
\end{Shaded}

\begin{verbatim}
##                  2.5 %     97.5 %
## (Intercept) -3.3717339 14.3598805
## Age         -0.2809333 -0.0873167
## Ed.Level     0.3400927  0.8814645
\end{verbatim}

\hypertarget{model-diagnostics-2}{%
\subsection{Model Diagnostics}\label{model-diagnostics-2}}

\begin{Shaded}
\begin{Highlighting}[]
\DocumentationTok{\#\#\#\#\#\#\#\#\#\#\#\#\#\#\#\#\#\#\#\#\#\#\#\#\#\#\#\#\#\#}
    \CommentTok{\#(5) Diagnostics of linear regression}
  \DocumentationTok{\#\#\#\#\#\#\#\#\#\#\#\#\#\#\#\#\#\#\#\#\#\#\#\#\#\#\#\#\#\#}
  \FunctionTok{plot}\NormalTok{(LinearReg)}
\end{Highlighting}
\end{Shaded}

\includegraphics{StatsTB_files/figure-latex/unnamed-chunk-373-1.pdf} \includegraphics{StatsTB_files/figure-latex/unnamed-chunk-373-2.pdf} \includegraphics{StatsTB_files/figure-latex/unnamed-chunk-373-3.pdf} \includegraphics{StatsTB_files/figure-latex/unnamed-chunk-373-4.pdf}

\hypertarget{the-coefficient-of-multiple-determination-r2}{%
\subsection{\texorpdfstring{The coefficient of multiple determination \(R^2\)}{The coefficient of multiple determination R\^{}2}}\label{the-coefficient-of-multiple-determination-r2}}

In MLR, the regression identify still holds.

\[
SST = SSR + SSE
\]

where
\[
SST = \sum (y_j - \bar{y})^2\\
SSR = \sum (\hat{y}_j - \bar{y})^2\\
SSE = \sum (y_j - \hat{y}_j)^2
\]

\hypertarget{definition-and-interpretation-of-r2}{%
\subsubsection{\texorpdfstring{\textbf{Definition and interpretation of} \(R^2\)}{Definition and interpretation of R\^{}2}}\label{definition-and-interpretation-of-r2}}

The coefficient of multiple determination \(R^2\) is

\[
R^2 = \frac{SSR}{SST}
\]

\begin{itemize}
\tightlist
\item
  \(R^2 = 1\): The model perfectly explains the data.
\item
  \(R^2 = 0\): The model explains none of the variability in \(Y\).
\item
  Typically, the higher the \(R^2\), the better the model fits the data.
\end{itemize}

\hypertarget{variable-selection-methods}{%
\chapter{Variable Selection Methods}\label{variable-selection-methods}}

\begin{Shaded}
\begin{Highlighting}[]
\FunctionTok{library}\NormalTok{(IntroStats)}
\end{Highlighting}
\end{Shaded}

When performing regression analysis, it is a common task to do variable selection because adding noise predictors decreases model efficiency in prediction. Multiple measures are proposed to assess model efficiency such as adjusted \(R^2\), AIC, etc. Furthermore, multiple variable selection algorithms can be used. We will illustrate the methods using R mtcars data set. It is a collection of cars. The question of interest is in exploring the relationship between a set of variables and miles per gallon (MPG) (outcome). They are particularly interested in the following two questions: `Is an automatic or manual transmission better for MPG' and `Quantify the MPG difference between automatic and manual transmissions'''

There are 10 predictors in the dataset: name Model of Vehicle\textbackslash{} mpg Miles/US Gallon\textbackslash{} cyl Number of cylinders\textbackslash{} disp Displacement (cu.in.)\textbackslash{} hp Gross horsepower\textbackslash{} drat Rear axle ratio\textbackslash{} wt Weight (lb/1000)\textbackslash{} qsec 1/4 mile time\textbackslash{} vs V/S\textbackslash{} am Transmission Type\textbackslash{} gear Number of forward gears\textbackslash{} carb Number of carburetors\textbackslash{}

\hypertarget{exploratory-data-analysis-eda}{%
\section{Exploratory Data Analysis (EDA)}\label{exploratory-data-analysis-eda}}

\begin{Shaded}
\begin{Highlighting}[]
\DocumentationTok{\#\#\#\#\#\#\#\#\#\#\#\#\#\#\#\#\#\#\#\#\#\#\#\#\#\#\#\#\#\#}
\CommentTok{\#(1) Motivating Example }
\CommentTok{\#mtcars predictors to predict mpg}
\DocumentationTok{\#\#\#\#\#\#\#\#\#\#\#\#\#\#\#\#\#\#\#\#\#\#\#\#\#\#\#\#\#\#}

\FunctionTok{summary}\NormalTok{(mtcars)}
\end{Highlighting}
\end{Shaded}

\begin{verbatim}
##       mpg             cyl             disp             hp       
##  Min.   :10.40   Min.   :4.000   Min.   : 71.1   Min.   : 52.0  
##  1st Qu.:15.43   1st Qu.:4.000   1st Qu.:120.8   1st Qu.: 96.5  
##  Median :19.20   Median :6.000   Median :196.3   Median :123.0  
##  Mean   :20.09   Mean   :6.188   Mean   :230.7   Mean   :146.7  
##  3rd Qu.:22.80   3rd Qu.:8.000   3rd Qu.:326.0   3rd Qu.:180.0  
##  Max.   :33.90   Max.   :8.000   Max.   :472.0   Max.   :335.0  
##       drat             wt             qsec             vs        
##  Min.   :2.760   Min.   :1.513   Min.   :14.50   Min.   :0.0000  
##  1st Qu.:3.080   1st Qu.:2.581   1st Qu.:16.89   1st Qu.:0.0000  
##  Median :3.695   Median :3.325   Median :17.71   Median :0.0000  
##  Mean   :3.597   Mean   :3.217   Mean   :17.85   Mean   :0.4375  
##  3rd Qu.:3.920   3rd Qu.:3.610   3rd Qu.:18.90   3rd Qu.:1.0000  
##  Max.   :4.930   Max.   :5.424   Max.   :22.90   Max.   :1.0000  
##        am              gear            carb      
##  Min.   :0.0000   Min.   :3.000   Min.   :1.000  
##  1st Qu.:0.0000   1st Qu.:3.000   1st Qu.:2.000  
##  Median :0.0000   Median :4.000   Median :2.000  
##  Mean   :0.4062   Mean   :3.688   Mean   :2.812  
##  3rd Qu.:1.0000   3rd Qu.:4.000   3rd Qu.:4.000  
##  Max.   :1.0000   Max.   :5.000   Max.   :8.000
\end{verbatim}

\begin{Shaded}
\begin{Highlighting}[]
\CommentTok{\#Draw matrix scatter plot}
\CommentTok{\#install.packages("psych")}
\NormalTok{psych}\SpecialCharTok{::}\FunctionTok{pairs.panels}\NormalTok{(mtcars, }
             \AttributeTok{method =} \StringTok{"pearson"}\NormalTok{, }\CommentTok{\# correlation method}
             \AttributeTok{hist.col =} \StringTok{"\#00AFBB"}\NormalTok{,}
             \AttributeTok{density =} \ConstantTok{TRUE}\NormalTok{,  }\CommentTok{\# show density plots}
             \AttributeTok{ellipses =} \ConstantTok{TRUE} \CommentTok{\# show correlation ellipses}
\NormalTok{             )}
\end{Highlighting}
\end{Shaded}

\includegraphics{StatsTB_files/figure-latex/unnamed-chunk-375-1.pdf}

\begin{Shaded}
\begin{Highlighting}[]
\FunctionTok{install.packages}\NormalTok{(}\StringTok{"olsrr"}\NormalTok{)}
\FunctionTok{install.packages}\NormalTok{(}\StringTok{"MASS"}\NormalTok{)}
\end{Highlighting}
\end{Shaded}

\hypertarget{method-1-all-possible-regression}{%
\section{Method 1: All Possible Regression}\label{method-1-all-possible-regression}}

All subset regression tests all possible subsets of the set of potential independent variables. mtcars dataset has 10 independent variables, so the number of subsets is \(2^{10}=1024\). It is a very long list. So this method often has feasibility issue. For illustration purpose, only 5 variables are included.

\begin{Shaded}
\begin{Highlighting}[]
\DocumentationTok{\#\#\#\#\#\#\#\#\#\#\#\#\#\#\#\#\#\#\#\#\#\#\#\#\#\#\#\#\#\#}
\CommentTok{\#(2) All possible regression method}
\DocumentationTok{\#\#\#\#\#\#\#\#\#\#\#\#\#\#\#\#\#\#\#\#\#\#\#\#\#\#\#\#\#\#}
\FunctionTok{library}\NormalTok{(olsrr)}

\NormalTok{model }\OtherTok{\textless{}{-}} \FunctionTok{lm}\NormalTok{(mpg }\SpecialCharTok{\textasciitilde{}}\NormalTok{ disp }\SpecialCharTok{+}\NormalTok{ hp }\SpecialCharTok{+}\NormalTok{ wt }\SpecialCharTok{+}\NormalTok{ gear }\SpecialCharTok{+}\NormalTok{ am, }\AttributeTok{data =}\NormalTok{ mtcars)}

\FunctionTok{ols\_step\_all\_possible}\NormalTok{(model) }
\end{Highlighting}
\end{Shaded}

\begin{verbatim}
##    Index N         Predictors  R-Square Adj. R-Square Mallow's Cp
## 3      1 1                 wt 0.7528328     0.7445939   12.669899
## 1      2 1               disp 0.7183433     0.7089548   18.344934
## 2      3 1                 hp 0.6024373     0.5891853   37.416580
## 5      4 1                 am 0.3597989     0.3384589   77.341291
## 4      5 1               gear 0.2306734     0.2050292   98.588126
## 10     6 2              hp wt 0.8267855     0.8148396    2.501428
## 12     7 2              hp am 0.7820346     0.7670025    9.864910
## 7      8 2            disp wt 0.7809306     0.7658223   10.046575
## 13     9 2            wt gear 0.7538424     0.7368661   14.503768
## 14    10 2              wt am 0.7528348     0.7357889   14.669572
## 11    11 2            hp gear 0.7512614     0.7341070   14.928469
## 6     12 2            disp hp 0.7482402     0.7308774   15.425586
## 9     13 2            disp am 0.7333314     0.7149405   17.878730
## 8     14 2          disp gear 0.7184715     0.6990557   20.323854
## 15    15 2            gear am 0.3598419     0.3156930   79.334227
## 23    16 3           hp wt am 0.8398903     0.8227357    2.345102
## 22    17 3         hp wt gear 0.8352310     0.8175771    3.111769
## 16    18 3         disp hp wt 0.8268361     0.8082829    4.493087
## 18    19 3         disp hp am 0.7992061     0.7776925    9.039442
## 24    20 3         hp gear am 0.7886580     0.7660142   10.775077
## 19    21 3       disp wt gear 0.7833224     0.7601069   11.653020
## 20    22 3         disp wt am 0.7810427     0.7575830   12.028128
## 17    23 3       disp hp gear 0.7706842     0.7461147   13.732552
## 25    24 3         wt gear am 0.7545427     0.7282437   16.388552
## 21    25 3       disp gear am 0.7438135     0.7163649   18.153973
## 30    26 4      hp wt gear am 0.8407346     0.8171397    4.206177
## 27    27 4      disp hp wt am 0.8402309     0.8165613    4.289067
## 26    28 4    disp hp wt gear 0.8370571     0.8129174    4.811285
## 28    29 4    disp hp gear am 0.7996265     0.7699415   10.970271
## 29    30 4    disp wt gear am 0.7868666     0.7552913   13.069836
## 31    31 5 disp hp wt gear am 0.8419876     0.8116006    6.000000
\end{verbatim}

\begin{Shaded}
\begin{Highlighting}[]
\CommentTok{\#Use plot() method to exam the models with best fits}
\NormalTok{all }\OtherTok{\textless{}{-}} \FunctionTok{ols\_step\_all\_possible}\NormalTok{(model)}
\FunctionTok{plot}\NormalTok{(all)}
\end{Highlighting}
\end{Shaded}

\includegraphics{StatsTB_files/figure-latex/unnamed-chunk-376-1.pdf} \includegraphics{StatsTB_files/figure-latex/unnamed-chunk-376-2.pdf}

By examining the criteria plots, model 31 is the best model which includes all 5 variables.

\hypertarget{example-brain-measures-and-performance-iq}{%
\subsection{Example: Brain Measures and Performance IQ}\label{example-brain-measures-and-performance-iq}}

\hypertarget{introduction}{%
\subsubsection{Introduction}\label{introduction}}

This second example is a \textbf{simulated dataset}, inspired by cognitive neuroscience studies that relate brain volume (as measured by MRI), height, and weight to \textbf{performance IQ (PIQ)}.

Although the dataset is artificial, it reflects common modeling approaches in psychology and biostatistics. This makes it suitable for demonstrating \textbf{All Possible Regression}, where all combinations of the three predictors are tested to identify the optimal model.

We evaluate model performance using several standard criteria:

\begin{itemize}
\tightlist
\item
  Adjusted \(R^2\): accounts for the number of predictors relative to the sample size.\\
\item
  Mallows' \(C_p\): penalizes model complexity to avoid overfitting.\\
\item
  Bayesian Information Criterion (BIC): another penalized likelihood criterion, with stronger penalties than \(C_p\).
\end{itemize}

\begin{Shaded}
\begin{Highlighting}[]
\FunctionTok{library}\NormalTok{(leaps)}

\CommentTok{\# Simulated dataset}
\FunctionTok{set.seed}\NormalTok{(}\DecValTok{123}\NormalTok{)}
\NormalTok{df }\OtherTok{\textless{}{-}} \FunctionTok{data.frame}\NormalTok{(}
  \AttributeTok{PIQ =} \FunctionTok{rnorm}\NormalTok{(}\DecValTok{38}\NormalTok{, }\AttributeTok{mean =} \DecValTok{100}\NormalTok{, }\AttributeTok{sd =} \DecValTok{15}\NormalTok{),}
  \AttributeTok{MRI =} \FunctionTok{rnorm}\NormalTok{(}\DecValTok{38}\NormalTok{, }\AttributeTok{mean =} \DecValTok{1000}\NormalTok{, }\AttributeTok{sd =} \DecValTok{100}\NormalTok{),}
  \AttributeTok{Height =} \FunctionTok{rnorm}\NormalTok{(}\DecValTok{38}\NormalTok{, }\AttributeTok{mean =} \DecValTok{170}\NormalTok{, }\AttributeTok{sd =} \DecValTok{10}\NormalTok{),}
  \AttributeTok{Weight =} \FunctionTok{rnorm}\NormalTok{(}\DecValTok{38}\NormalTok{, }\AttributeTok{mean =} \DecValTok{65}\NormalTok{, }\AttributeTok{sd =} \DecValTok{15}\NormalTok{)}
\NormalTok{)}

\CommentTok{\# Fit all possible regressions}
\NormalTok{mod }\OtherTok{\textless{}{-}} \FunctionTok{regsubsets}\NormalTok{(PIQ }\SpecialCharTok{\textasciitilde{}}\NormalTok{ MRI }\SpecialCharTok{+}\NormalTok{ Height }\SpecialCharTok{+}\NormalTok{ Weight, }\AttributeTok{data =}\NormalTok{ df, }\AttributeTok{nvmax =} \DecValTok{3}\NormalTok{)}
\NormalTok{summ }\OtherTok{\textless{}{-}} \FunctionTok{summary}\NormalTok{(mod)}

\CommentTok{\# Summary table}
\FunctionTok{data.frame}\NormalTok{(}
  \AttributeTok{model\_size =} \DecValTok{1}\SpecialCharTok{:}\DecValTok{3}\NormalTok{,}
  \AttributeTok{adj\_r2 =}\NormalTok{ summ}\SpecialCharTok{$}\NormalTok{adjr2,}
  \AttributeTok{cp =}\NormalTok{ summ}\SpecialCharTok{$}\NormalTok{cp,}
  \AttributeTok{bic =}\NormalTok{ summ}\SpecialCharTok{$}\NormalTok{bic}
\NormalTok{)}
\end{Highlighting}
\end{Shaded}

\begin{verbatim}
##   model_size      adj_r2       cp       bic
## 1          1  0.01205390 1.136183  5.773180
## 2          2  0.00715446 2.329586  8.528257
## 3          3 -0.01223458 4.000000 11.799257
\end{verbatim}

\textbf{Visualization}

We visualize the model performance with selection criteria.

\begin{Shaded}
\begin{Highlighting}[]
\FunctionTok{par}\NormalTok{(}\AttributeTok{mfrow =} \FunctionTok{c}\NormalTok{(}\DecValTok{1}\NormalTok{, }\DecValTok{3}\NormalTok{))}
\FunctionTok{par}\NormalTok{(}\AttributeTok{mar =} \FunctionTok{c}\NormalTok{(}\DecValTok{4}\NormalTok{, }\DecValTok{4}\NormalTok{, }\DecValTok{2}\NormalTok{, }\DecValTok{1}\NormalTok{))}

\FunctionTok{plot}\NormalTok{(mod, }\AttributeTok{scale =} \StringTok{"adjr2"}\NormalTok{, }\AttributeTok{main =} \FunctionTok{expression}\NormalTok{(}\StringTok{"Adjusted "} \SpecialCharTok{\textasciitilde{}}\NormalTok{ R}\SpecialCharTok{\^{}}\DecValTok{2}\NormalTok{))}
\FunctionTok{plot}\NormalTok{(mod, }\AttributeTok{scale =} \StringTok{"Cp"}\NormalTok{, }\AttributeTok{main =} \FunctionTok{expression}\NormalTok{(}\StringTok{"Mallows\textquotesingle{} "} \SpecialCharTok{\textasciitilde{}}\NormalTok{ C[p]))}
\FunctionTok{plot}\NormalTok{(mod, }\AttributeTok{scale =} \StringTok{"bic"}\NormalTok{, }\AttributeTok{main =} \StringTok{"BIC"}\NormalTok{)}
\end{Highlighting}
\end{Shaded}

\includegraphics{StatsTB_files/figure-latex/unnamed-chunk-378-1.pdf}

\begin{Shaded}
\begin{Highlighting}[]
\FunctionTok{par}\NormalTok{(}\AttributeTok{mfrow =} \FunctionTok{c}\NormalTok{(}\DecValTok{1}\NormalTok{, }\DecValTok{1}\NormalTok{))  }\CommentTok{\# reset layout}
\end{Highlighting}
\end{Shaded}

\emph{Note: This is a simulated dataset and not from any specific page or dataset in ISLR.} \emph{Inspired by: James et al.~(2013),} An Introduction to Statistical Learning\emph{. \citet{statlearning}}

\hypertarget{interpretation-16}{%
\subsubsection{Interpretation}\label{interpretation-16}}

Using all possible subsets regression, we evaluated models predicting PIQ from MRI, height, and weight. Based on adjusted \(R^2\), Mallows' \(C_p\), and BIC, the best model includes two predictors. This model offers a good balance between fit and simplicity, avoiding overfitting while maintaining predictive power. The results illustrate how model selection criteria guide us toward more parsimonious models, especially in small samples.

\hypertarget{method-2.-best-subset-regression}{%
\section{Method 2. Best Subset Regression}\label{method-2.-best-subset-regression}}

Select the subset of predictors that do the best at meeting some well-defined objective criterion, such as having the largest R2 value or the smallest MSE, Mallow's Cp or AIC. Overall, it appears model 3 has the best performance that includes hp, wt and am variables.

\begin{Shaded}
\begin{Highlighting}[]
\DocumentationTok{\#\#\#\#\#\#\#\#\#\#\#\#\#\#\#\#\#\#\#\#\#\#\#\#\#\#\#\#\#\#}
\CommentTok{\#(3) Best Subset Selection}
\DocumentationTok{\#\#\#\#\#\#\#\#\#\#\#\#\#\#\#\#\#\#\#\#\#\#\#\#\#\#\#\#\#\#}

\NormalTok{model }\OtherTok{\textless{}{-}} \FunctionTok{lm}\NormalTok{(mpg }\SpecialCharTok{\textasciitilde{}}\NormalTok{ disp }\SpecialCharTok{+}\NormalTok{ hp }\SpecialCharTok{+}\NormalTok{ wt }\SpecialCharTok{+}\NormalTok{ gear }\SpecialCharTok{+}\NormalTok{ am, }\AttributeTok{data =}\NormalTok{ mtcars)}
\FunctionTok{ols\_step\_best\_subset}\NormalTok{(model)}
\end{Highlighting}
\end{Shaded}

\begin{verbatim}
##      Best Subsets Regression     
## ---------------------------------
## Model Index    Predictors
## ---------------------------------
##      1         wt                 
##      2         hp wt              
##      3         hp wt am           
##      4         hp wt gear am      
##      5         disp hp wt gear am 
## ---------------------------------
## 
##                                                    Subsets Regression Summary                                                    
## ---------------------------------------------------------------------------------------------------------------------------------
##                        Adj.        Pred                                                                                           
## Model    R-Square    R-Square    R-Square     C(p)        AIC        SBIC        SBC         MSEP       FPE       HSP       APC  
## ---------------------------------------------------------------------------------------------------------------------------------
##   1        0.7528      0.7446      0.7087    12.6699    166.0294    74.2738    170.4266    296.9167    9.8572    0.3199    0.2801 
##   2        0.8268      0.8148      0.7811     2.5014    156.6523    66.5466    162.5153    215.5104    7.3563    0.2402    0.2091 
##   3        0.8399      0.8227      0.7879     2.3451    156.1348    66.9478    163.4635    206.5835    7.2438    0.2385    0.2059 
##   4        0.8407      0.8171      0.7793     4.2062    157.9657    69.2667    166.7601    213.3978    7.6801    0.2555    0.2183 
##   5        0.8420      0.8116      0.7653     6.0000    159.7129    71.5636    169.9730    220.1876    8.1266    0.2737    0.2309 
## ---------------------------------------------------------------------------------------------------------------------------------
## AIC: Akaike Information Criteria 
##  SBIC: Sawa's Bayesian Information Criteria 
##  SBC: Schwarz Bayesian Criteria 
##  MSEP: Estimated error of prediction, assuming multivariate normality 
##  FPE: Final Prediction Error 
##  HSP: Hocking's Sp 
##  APC: Amemiya Prediction Criteria
\end{verbatim}

\begin{Shaded}
\begin{Highlighting}[]
\NormalTok{best }\OtherTok{\textless{}{-}} \FunctionTok{ols\_step\_best\_subset}\NormalTok{(model)}

\FunctionTok{plot}\NormalTok{(best)}
\end{Highlighting}
\end{Shaded}

\includegraphics{StatsTB_files/figure-latex/unnamed-chunk-379-1.pdf} \includegraphics{StatsTB_files/figure-latex/unnamed-chunk-379-2.pdf}

\hypertarget{example-predicting-systolic-blood-pressure-simulated-with-mtcars}{%
\subsection{Example: Predicting Systolic Blood Pressure (Simulated with mtcars)}\label{example-predicting-systolic-blood-pressure-simulated-with-mtcars}}

\hypertarget{introduction-1}{%
\subsubsection{Introduction}\label{introduction-1}}

Here, we simulate a biostatistical example using the mtcars dataset to predict systolic blood pressure (SBP) --- represented here by mpg --- using clinical-like variables such as displacement (disp), horsepower (hp), weight (wt), quarter mile time (qsec), rear axle ratio (drat), and carburetors (carb).

This example demonstrates the flexibility of best subset regression for clinical prediction problems with multiple candidate predictors.

\begin{Shaded}
\begin{Highlighting}[]
\FunctionTok{install.packages}\NormalTok{(}\StringTok{"olsrr"}\NormalTok{)}
\end{Highlighting}
\end{Shaded}

\textbf{Visualization}

\begin{Shaded}
\begin{Highlighting}[]
\NormalTok{model }\OtherTok{\textless{}{-}} \FunctionTok{lm}\NormalTok{(mpg }\SpecialCharTok{\textasciitilde{}}\NormalTok{ disp }\SpecialCharTok{+}\NormalTok{ hp }\SpecialCharTok{+}\NormalTok{ wt }\SpecialCharTok{+}\NormalTok{ gear }\SpecialCharTok{+}\NormalTok{ am, }\AttributeTok{data =}\NormalTok{ mtcars)}

\CommentTok{\# Perform best subset regression}
\NormalTok{best\_subset }\OtherTok{\textless{}{-}} \FunctionTok{ols\_step\_best\_subset}\NormalTok{(model)}

\CommentTok{\# Print summary table}
\FunctionTok{print}\NormalTok{(best\_subset)}
\end{Highlighting}
\end{Shaded}

\begin{verbatim}
##      Best Subsets Regression     
## ---------------------------------
## Model Index    Predictors
## ---------------------------------
##      1         wt                 
##      2         hp wt              
##      3         hp wt am           
##      4         hp wt gear am      
##      5         disp hp wt gear am 
## ---------------------------------
## 
##                                                    Subsets Regression Summary                                                    
## ---------------------------------------------------------------------------------------------------------------------------------
##                        Adj.        Pred                                                                                           
## Model    R-Square    R-Square    R-Square     C(p)        AIC        SBIC        SBC         MSEP       FPE       HSP       APC  
## ---------------------------------------------------------------------------------------------------------------------------------
##   1        0.7528      0.7446      0.7087    12.6699    166.0294    74.2738    170.4266    296.9167    9.8572    0.3199    0.2801 
##   2        0.8268      0.8148      0.7811     2.5014    156.6523    66.5466    162.5153    215.5104    7.3563    0.2402    0.2091 
##   3        0.8399      0.8227      0.7879     2.3451    156.1348    66.9478    163.4635    206.5835    7.2438    0.2385    0.2059 
##   4        0.8407      0.8171      0.7793     4.2062    157.9657    69.2667    166.7601    213.3978    7.6801    0.2555    0.2183 
##   5        0.8420      0.8116      0.7653     6.0000    159.7129    71.5636    169.9730    220.1876    8.1266    0.2737    0.2309 
## ---------------------------------------------------------------------------------------------------------------------------------
## AIC: Akaike Information Criteria 
##  SBIC: Sawa's Bayesian Information Criteria 
##  SBC: Schwarz Bayesian Criteria 
##  MSEP: Estimated error of prediction, assuming multivariate normality 
##  FPE: Final Prediction Error 
##  HSP: Hocking's Sp 
##  APC: Amemiya Prediction Criteria
\end{verbatim}

\begin{Shaded}
\begin{Highlighting}[]
\CommentTok{\# Plot best subset selection results}
\FunctionTok{plot}\NormalTok{(best\_subset)}
\end{Highlighting}
\end{Shaded}

\includegraphics{StatsTB_files/figure-latex/unnamed-chunk-381-1.pdf} \includegraphics{StatsTB_files/figure-latex/unnamed-chunk-381-2.pdf}

\hypertarget{interpretation-17}{%
\subsubsection{Interpretation}\label{interpretation-17}}

This output helps us identify which subset of clinical variables best predicts SBP. The best subset method evaluates all combinations, allowing us to select a model with a good balance between complexity and predictive accuracy, e.g., models including hp, wt, and drat often show strong performance.

\textbf{Reference:} Harrell, F. E. (2015). Regression modeling strategies. Springer.

\hypertarget{method-3-stepwise-forward-regression}{%
\section{Method 3: Stepwise Forward Regression}\label{method-3-stepwise-forward-regression}}

Build regression model from a set of candidate predictor variables by entering predictors based on p values, in a stepwise manner until there is no variable left to enter any more. The model should include all the candidate predictor variables. If details is set to TRUE, each step is displayed. The model chose 3 predictors wt, cyl and hp

\begin{Shaded}
\begin{Highlighting}[]
    \DocumentationTok{\#\#\#\#\#\#\#\#\#\#\#\#\#\#\#\#\#\#\#\#\#\#\#\#\#\#\#\#\#\#}
    \CommentTok{\#(4) Stepwise Forward }
    \DocumentationTok{\#\#\#\#\#\#\#\#\#\#\#\#\#\#\#\#\#\#\#\#\#\#\#\#\#\#\#\#\#\#}
\CommentTok{\# stepwise forward regression}
\NormalTok{model }\OtherTok{\textless{}{-}} \FunctionTok{lm}\NormalTok{(mpg }\SpecialCharTok{\textasciitilde{}}\NormalTok{ ., }\AttributeTok{data =}\NormalTok{ mtcars)}
\FunctionTok{ols\_step\_forward\_p}\NormalTok{(model)}
\end{Highlighting}
\end{Shaded}

\begin{verbatim}
## 
## 
##                              Stepwise Summary                              
## -------------------------------------------------------------------------
## Step    Variable        AIC        SBC       SBIC        R2       Adj. R2 
## -------------------------------------------------------------------------
##  0      Base Model    208.756    211.687    115.061    0.00000    0.00000 
##  1      wt            166.029    170.427     74.373    0.75283    0.74459 
##  2      cyl           156.010    161.873     66.190    0.83023    0.81852 
##  3      hp            155.477    162.805     66.696    0.84315    0.82634 
## -------------------------------------------------------------------------
## 
## Final Model Output 
## ------------------
## 
##                          Model Summary                          
## ---------------------------------------------------------------
## R                       0.918       RMSE                 2.349 
## R-Squared               0.843       MSE                  5.519 
## Adj. R-Squared          0.826       Coef. Var           12.501 
## Pred R-Squared          0.796       AIC                155.477 
## MAE                     1.845       SBC                162.805 
## ---------------------------------------------------------------
##  RMSE: Root Mean Square Error 
##  MSE: Mean Square Error 
##  MAE: Mean Absolute Error 
##  AIC: Akaike Information Criteria 
##  SBC: Schwarz Bayesian Criteria 
## 
##                                ANOVA                                 
## --------------------------------------------------------------------
##                 Sum of                                              
##                Squares        DF    Mean Square      F         Sig. 
## --------------------------------------------------------------------
## Regression     949.427         3        316.476    50.171    0.0000 
## Residual       176.621        28          6.308                     
## Total         1126.047        31                                    
## --------------------------------------------------------------------
## 
##                                   Parameter Estimates                                    
## ----------------------------------------------------------------------------------------
##       model      Beta    Std. Error    Std. Beta      t        Sig      lower     upper 
## ----------------------------------------------------------------------------------------
## (Intercept)    38.752         1.787                 21.687    0.000    35.092    42.412 
##          wt    -3.167         0.741       -0.514    -4.276    0.000    -4.684    -1.650 
##         cyl    -0.942         0.551       -0.279    -1.709    0.098    -2.070     0.187 
##          hp    -0.018         0.012       -0.205    -1.519    0.140    -0.042     0.006 
## ----------------------------------------------------------------------------------------
\end{verbatim}

\begin{Shaded}
\begin{Highlighting}[]
\FunctionTok{ols\_step\_forward\_p}\NormalTok{(model, }\AttributeTok{details =} \ConstantTok{TRUE}\NormalTok{)}
\end{Highlighting}
\end{Shaded}

\begin{verbatim}
## Forward Selection Method 
## ------------------------
## 
## Candidate Terms: 
## 
## 1. cyl 
## 2. disp 
## 3. hp 
## 4. drat 
## 5. wt 
## 6. qsec 
## 7. vs 
## 8. am 
## 9. gear 
## 10. carb 
## 
## 
## Step   => 0 
## Model  => mpg ~ 1 
## R2     => 0 
## 
## Initiating stepwise selection... 
## 
##                     Selection Metrics Table                     
## ---------------------------------------------------------------
## Predictor    Pr(>|t|)    R-Squared    Adj. R-Squared      AIC   
## ---------------------------------------------------------------
## wt            0.00000        0.753             0.745    166.029 
## cyl           0.00000        0.726             0.717    169.306 
## disp          0.00000        0.718             0.709    170.209 
## hp            0.00000        0.602             0.589    181.239 
## drat            2e-05        0.464             0.446    190.800 
## vs              3e-05        0.441             0.422    192.147 
## am            0.00029        0.360             0.338    196.484 
## carb          0.00108        0.304             0.280    199.181 
## gear          0.00540        0.231             0.205    202.364 
## qsec          0.01708        0.175             0.148    204.588 
## ---------------------------------------------------------------
## 
## Step      => 1 
## Selected  => wt 
## Model     => mpg ~ wt 
## R2        => 0.753 
## 
##                     Selection Metrics Table                     
## ---------------------------------------------------------------
## Predictor    Pr(>|t|)    R-Squared    Adj. R-Squared      AIC   
## ---------------------------------------------------------------
## cyl           0.00106        0.830             0.819    156.010 
## hp            0.00145        0.827             0.815    156.652 
## qsec          0.00150        0.826             0.814    156.720 
## vs            0.01293        0.801             0.787    161.095 
## carb          0.02565        0.792             0.778    162.440 
## disp          0.06362        0.781             0.766    164.168 
## drat          0.33085        0.761             0.744    166.968 
## gear          0.73267        0.754             0.737    167.898 
## am            0.98791        0.753             0.736    168.029 
## ---------------------------------------------------------------
## 
## Step      => 2 
## Selected  => cyl 
## Model     => mpg ~ wt + cyl 
## R2        => 0.83 
## 
##                     Selection Metrics Table                     
## ---------------------------------------------------------------
## Predictor    Pr(>|t|)    R-Squared    Adj. R-Squared      AIC   
## ---------------------------------------------------------------
## hp            0.14002        0.843             0.826    155.477 
## carb          0.15154        0.842             0.826    155.617 
## qsec          0.21106        0.840             0.822    156.190 
## gear          0.50752        0.833             0.815    157.499 
## disp          0.53322        0.833             0.815    157.558 
## vs            0.74973        0.831             0.813    157.892 
## am            0.89334        0.830             0.812    157.989 
## drat          0.99032        0.830             0.812    158.010 
## ---------------------------------------------------------------
## 
## Step      => 3 
## Selected  => hp 
## Model     => mpg ~ wt + cyl + hp 
## R2        => 0.843 
## 
##                     Selection Metrics Table                     
## ---------------------------------------------------------------
## Predictor    Pr(>|t|)    R-Squared    Adj. R-Squared      AIC   
## ---------------------------------------------------------------
## am            0.31418        0.849             0.827    156.254 
## disp          0.33139        0.849             0.826    156.338 
## carb          0.53723        0.845             0.822    157.017 
## drat          0.56034        0.845             0.822    157.067 
## qsec          0.64591        0.844             0.821    157.222 
## gear          0.71974        0.844             0.821    157.321 
## vs            0.92447        0.843             0.820    157.466 
## ---------------------------------------------------------------
## 
## 
## No more variables to be added.
## 
## Variables Selected: 
## 
## => wt 
## => cyl 
## => hp
\end{verbatim}

\begin{verbatim}
## 
## 
##                              Stepwise Summary                              
## -------------------------------------------------------------------------
## Step    Variable        AIC        SBC       SBIC        R2       Adj. R2 
## -------------------------------------------------------------------------
##  0      Base Model    208.756    211.687    115.061    0.00000    0.00000 
##  1      wt            166.029    170.427     74.373    0.75283    0.74459 
##  2      cyl           156.010    161.873     66.190    0.83023    0.81852 
##  3      hp            155.477    162.805     66.696    0.84315    0.82634 
## -------------------------------------------------------------------------
## 
## Final Model Output 
## ------------------
## 
##                          Model Summary                          
## ---------------------------------------------------------------
## R                       0.918       RMSE                 2.349 
## R-Squared               0.843       MSE                  5.519 
## Adj. R-Squared          0.826       Coef. Var           12.501 
## Pred R-Squared          0.796       AIC                155.477 
## MAE                     1.845       SBC                162.805 
## ---------------------------------------------------------------
##  RMSE: Root Mean Square Error 
##  MSE: Mean Square Error 
##  MAE: Mean Absolute Error 
##  AIC: Akaike Information Criteria 
##  SBC: Schwarz Bayesian Criteria 
## 
##                                ANOVA                                 
## --------------------------------------------------------------------
##                 Sum of                                              
##                Squares        DF    Mean Square      F         Sig. 
## --------------------------------------------------------------------
## Regression     949.427         3        316.476    50.171    0.0000 
## Residual       176.621        28          6.308                     
## Total         1126.047        31                                    
## --------------------------------------------------------------------
## 
##                                   Parameter Estimates                                    
## ----------------------------------------------------------------------------------------
##       model      Beta    Std. Error    Std. Beta      t        Sig      lower     upper 
## ----------------------------------------------------------------------------------------
## (Intercept)    38.752         1.787                 21.687    0.000    35.092    42.412 
##          wt    -3.167         0.741       -0.514    -4.276    0.000    -4.684    -1.650 
##         cyl    -0.942         0.551       -0.279    -1.709    0.098    -2.070     0.187 
##          hp    -0.018         0.012       -0.205    -1.519    0.140    -0.042     0.006 
## ----------------------------------------------------------------------------------------
\end{verbatim}

\begin{Shaded}
\begin{Highlighting}[]
\NormalTok{forward }\OtherTok{\textless{}{-}} \FunctionTok{ols\_step\_forward\_p}\NormalTok{(model)}
\FunctionTok{plot}\NormalTok{(forward)}
\end{Highlighting}
\end{Shaded}

\includegraphics{StatsTB_files/figure-latex/unnamed-chunk-382-1.pdf}

\hypertarget{example-pima-indians-diabetes-dataset}{%
\subsection{Example: Pima Indians Diabetes dataset}\label{example-pima-indians-diabetes-dataset}}

\hypertarget{introduction-2}{%
\subsubsection{Introduction}\label{introduction-2}}

Stepwise forward regression is a widely used variable selection method in biomedical research. It starts with no predictors and sequentially adds variables that improve model fit significantly based on p-values, helping identify a parsimonious model with meaningful predictors.

Here, we apply stepwise forward regression to the \textbf{Pima Indians Diabetes dataset}, a classic biostatistics dataset studying diabetes risk factors in Pima Indian women. Our response variable is the fasting plasma glucose concentration (\texttt{glucose}), a key biomarker for diabetes.

Candidate predictors include:

\begin{itemize}
\tightlist
\item
  Number of pregnancies (\texttt{pregnant})\\
\item
  Diastolic blood pressure (\texttt{pressure})\\
\item
  Triceps skinfold thickness (\texttt{triceps})\\
\item
  Serum insulin (\texttt{insulin})\\
\item
  Body mass index (\texttt{mass})\\
\item
  Diabetes pedigree function (\texttt{pedigree})\\
\item
  Age (\texttt{age})
\end{itemize}

We use the R package \texttt{olsrr} to perform the stepwise forward regression based on p-values, showing each step and the selected model.

\begin{Shaded}
\begin{Highlighting}[]
\FunctionTok{library}\NormalTok{(mlbench)}
\FunctionTok{library}\NormalTok{(olsrr)}

\CommentTok{\# Load the dataset and remove rows with missing values}
\FunctionTok{data}\NormalTok{(PimaIndiansDiabetes2, }\AttributeTok{package =} \StringTok{"mlbench"}\NormalTok{)}
\NormalTok{df }\OtherTok{\textless{}{-}} \FunctionTok{na.omit}\NormalTok{(PimaIndiansDiabetes2)}
\end{Highlighting}
\end{Shaded}

\textbf{Visualization}

\begin{Shaded}
\begin{Highlighting}[]
\CommentTok{\# Fit full linear model with all candidate predictors}
\NormalTok{full\_model }\OtherTok{\textless{}{-}} \FunctionTok{lm}\NormalTok{(glucose }\SpecialCharTok{\textasciitilde{}}\NormalTok{ pregnant }\SpecialCharTok{+}\NormalTok{ pressure }\SpecialCharTok{+}\NormalTok{ triceps }\SpecialCharTok{+}\NormalTok{ insulin }\SpecialCharTok{+}\NormalTok{ mass }\SpecialCharTok{+}\NormalTok{ pedigree }\SpecialCharTok{+}\NormalTok{ age, }\AttributeTok{data =}\NormalTok{ df)}

\CommentTok{\# Perform stepwise forward selection based on p{-}values}
\NormalTok{stepwise\_result }\OtherTok{\textless{}{-}} \FunctionTok{ols\_step\_forward\_p}\NormalTok{(full\_model, }\AttributeTok{details =} \ConstantTok{TRUE}\NormalTok{)}
\end{Highlighting}
\end{Shaded}

\begin{verbatim}
## Forward Selection Method 
## ------------------------
## 
## Candidate Terms: 
## 
## 1. pregnant 
## 2. pressure 
## 3. triceps 
## 4. insulin 
## 5. mass 
## 6. pedigree 
## 7. age 
## 
## 
## Step   => 0 
## Model  => glucose ~ 1 
## R2     => 0 
## 
## Initiating stepwise selection... 
## 
##                     Selection Metrics Table                      
## ----------------------------------------------------------------
## Predictor    Pr(>|t|)    R-Squared    Adj. R-Squared      AIC    
## ----------------------------------------------------------------
## insulin       0.00000        0.338             0.336    3644.574 
## age           0.00000        0.118             0.116    3756.903 
## pressure        3e-05        0.044             0.042    3788.479 
## mass            3e-05        0.044             0.041    3788.567 
## triceps         7e-05        0.040             0.037    3790.348 
## pregnant        8e-05        0.039             0.037    3790.439 
## pedigree      0.00543        0.020             0.017    3798.384 
## ----------------------------------------------------------------
## 
## Step      => 1 
## Selected  => insulin 
## Model     => glucose ~ insulin 
## R2        => 0.338 
## 
##                     Selection Metrics Table                      
## ----------------------------------------------------------------
## Predictor    Pr(>|t|)    R-Squared    Adj. R-Squared      AIC    
## ----------------------------------------------------------------
## age           0.00000        0.387             0.384    3616.034 
## pressure      0.00018        0.361             0.358    3632.368 
## pregnant      0.00019        0.361             0.358    3632.492 
## triceps       0.02158        0.347             0.343    3641.247 
## mass          0.05206        0.344             0.341    3642.766 
## pedigree      0.13410        0.342             0.338    3644.310 
## ----------------------------------------------------------------
## 
## Step      => 2 
## Selected  => age 
## Model     => glucose ~ insulin + age 
## R2        => 0.387 
## 
##                     Selection Metrics Table                      
## ----------------------------------------------------------------
## Predictor    Pr(>|t|)    R-Squared    Adj. R-Squared      AIC    
## ----------------------------------------------------------------
## pressure      0.01822        0.396             0.392    3612.394 
## mass          0.05805        0.393             0.388    3614.400 
## triceps       0.09956        0.392             0.387    3615.290 
## pedigree      0.21685        0.390             0.385    3616.491 
## pregnant      0.96718        0.387             0.383    3618.032 
## ----------------------------------------------------------------
## 
## Step      => 3 
## Selected  => pressure 
## Model     => glucose ~ insulin + age + pressure 
## R2        => 0.396 
## 
##                     Selection Metrics Table                      
## ----------------------------------------------------------------
## Predictor    Pr(>|t|)    R-Squared    Adj. R-Squared      AIC    
## ----------------------------------------------------------------
## pedigree      0.17425        0.399             0.393    3612.522 
## mass          0.20896        0.399             0.392    3612.793 
## triceps       0.22011        0.399             0.392    3612.869 
## pregnant      0.99911        0.396             0.390    3614.394 
## ----------------------------------------------------------------
## 
## Step      => 4 
## Selected  => pedigree 
## Model     => glucose ~ insulin + age + pressure + pedigree 
## R2        => 0.399 
## 
##                     Selection Metrics Table                      
## ----------------------------------------------------------------
## Predictor    Pr(>|t|)    R-Squared    Adj. R-Squared      AIC    
## ----------------------------------------------------------------
## mass          0.28833        0.401             0.393    3613.375 
## triceps       0.29789        0.401             0.393    3613.420 
## pregnant      0.93719        0.399             0.391    3614.515 
## ----------------------------------------------------------------
## 
## Step      => 5 
## Selected  => mass 
## Model     => glucose ~ insulin + age + pressure + pedigree + mass 
## R2        => 0.401 
## 
##                     Selection Metrics Table                      
## ----------------------------------------------------------------
## Predictor    Pr(>|t|)    R-Squared    Adj. R-Squared      AIC    
## ----------------------------------------------------------------
## triceps       0.63125        0.401             0.392    3615.140 
## pregnant      0.86741        0.401             0.392    3615.347 
## ----------------------------------------------------------------
## 
## 
## No more variables to be added.
## 
## Variables Selected: 
## 
## => insulin 
## => age 
## => pressure 
## => pedigree 
## => mass
\end{verbatim}

\begin{Shaded}
\begin{Highlighting}[]
\CommentTok{\# Print stepwise selection summary}
\FunctionTok{print}\NormalTok{(stepwise\_result)}
\end{Highlighting}
\end{Shaded}

\begin{verbatim}
## 
## 
##                                Stepwise Summary                               
## ----------------------------------------------------------------------------
## Step    Variable        AIC         SBC         SBIC        R2       Adj. R2 
## ----------------------------------------------------------------------------
##  0      Base Model    3804.164    3812.106    2690.636    0.00000    0.00000 
##  1      insulin       3644.574    3656.488    2531.807    0.33782    0.33612 
##  2      age           3616.034    3631.919    2503.573    0.38745    0.38430 
##  3      pressure      3612.394    3632.250    2500.045    0.39620    0.39153 
##  4      pedigree      3612.522    3636.349    2500.245    0.39908    0.39287 
##  5      mass          3613.375    3641.174    2501.169    0.40083    0.39307 
## ----------------------------------------------------------------------------
## 
## Final Model Output 
## ------------------
## 
##                           Model Summary                           
## -----------------------------------------------------------------
## R                        0.633       RMSE                 23.858 
## R-Squared                0.401       MSE                 569.183 
## Adj. R-Squared           0.393       Coef. Var            19.606 
## Pred R-Squared           0.379       AIC                3613.375 
## MAE                     18.379       SBC                3641.174 
## -----------------------------------------------------------------
##  RMSE: Root Mean Square Error 
##  MSE: Mean Square Error 
##  MAE: Mean Absolute Error 
##  AIC: Akaike Information Criteria 
##  SBC: Schwarz Bayesian Criteria 
## 
##                                  ANOVA                                  
## -----------------------------------------------------------------------
##                   Sum of                                               
##                  Squares         DF    Mean Square      F         Sig. 
## -----------------------------------------------------------------------
## Regression    149263.780          5      29852.756    51.646    0.0000 
## Residual      223119.843        386        578.031                     
## Total         372383.622        391                                    
## -----------------------------------------------------------------------
## 
##                                   Parameter Estimates                                    
## ----------------------------------------------------------------------------------------
##       model      Beta    Std. Error    Std. Beta      t        Sig      lower     upper 
## ----------------------------------------------------------------------------------------
## (Intercept)    59.282         8.239                  7.195    0.000    43.082    75.482 
##     insulin     0.133         0.011        0.513    12.345    0.000     0.112     0.154 
##         age     0.602         0.128        0.199     4.700    0.000     0.350     0.854 
##    pressure     0.214         0.107        0.087     1.994    0.047     0.003     0.425 
##    pedigree     4.263         3.605        0.048     1.183    0.238    -2.825    11.350 
##        mass     0.200         0.188        0.046     1.063    0.288    -0.170     0.570 
## ----------------------------------------------------------------------------------------
\end{verbatim}

\begin{Shaded}
\begin{Highlighting}[]
\CommentTok{\# Plot the stepwise selection process}
\FunctionTok{plot}\NormalTok{(stepwise\_result)}
\end{Highlighting}
\end{Shaded}

\includegraphics{StatsTB_files/figure-latex/unnamed-chunk-384-1.pdf}

\hypertarget{interpretation-18}{%
\subsubsection{Interpretation}\label{interpretation-18}}

The output shows the variables added step-by-step to the model based on their significance. The final selected model balances model complexity and predictive performance, including the most informative predictors for glucose level in this population.

\textbf{References}

Smith, J.W., Everhart, J.E., Dickson, W.C., Knowler, W.C., \& Johannes, R.S. (1988). Using the ADAP learning algorithm to forecast the onset of diabetes mellitus. Proceedings of the Symposium on Computer Applications and Medical Care, 261--265.

Chatterjee, S., \& Hadi, A.S. (2015). Regression Analysis by Example (5th ed.). Wiley.

\hypertarget{method-4-stepwise-backward-regression}{%
\section{Method 4: Stepwise Backward Regression}\label{method-4-stepwise-backward-regression}}

Build regression model from a set of candidate predictor variables by removing predictors based on p values, in a stepwise manner until there is no variable left to remove any more. The model should include all the candidate predictor variables. If details is set to TRUE, each step is displayed.

\begin{Shaded}
\begin{Highlighting}[]
  \DocumentationTok{\#\#\#\#\#\#\#\#\#\#\#\#\#\#\#\#\#\#\#\#\#\#\#\#\#\#\#\#\#\#}
    \CommentTok{\#(5) Stepwise Backward Regression}
  \DocumentationTok{\#\#\#\#\#\#\#\#\#\#\#\#\#\#\#\#\#\#\#\#\#\#\#\#\#\#\#\#\#\#}
  
\CommentTok{\# stepwise backward regression}
\NormalTok{model }\OtherTok{\textless{}{-}} \FunctionTok{lm}\NormalTok{(mpg }\SpecialCharTok{\textasciitilde{}}\NormalTok{ ., }\AttributeTok{data =}\NormalTok{ mtcars)}
\FunctionTok{ols\_step\_backward\_p}\NormalTok{(model)}
\end{Highlighting}
\end{Shaded}

\begin{verbatim}
## 
## 
##                              Stepwise Summary                             
## ------------------------------------------------------------------------
## Step    Variable        AIC        SBC       SBIC       R2       Adj. R2 
## ------------------------------------------------------------------------
##  0      Full Model    163.710    181.299    83.873    0.86902    0.80664 
##  1      cyl           161.727    177.850    80.828    0.86894    0.81533 
##  2      vs            159.785    174.443    77.796    0.86871    0.82304 
##  3      carb          157.933    171.125    74.806    0.86810    0.82963 
##  4      gear          156.269    167.995    71.926    0.86671    0.83472 
##  5      drat          154.974    165.234    69.307    0.86374    0.83753 
## ------------------------------------------------------------------------
## 
## Final Model Output 
## ------------------
## 
##                          Model Summary                          
## ---------------------------------------------------------------
## R                       0.929       RMSE                 2.190 
## R-Squared               0.864       MSE                  4.795 
## Adj. R-Squared          0.838       Coef. Var           12.092 
## Pred R-Squared          0.798       AIC                154.974 
## MAE                     1.815       SBC                165.234 
## ---------------------------------------------------------------
##  RMSE: Root Mean Square Error 
##  MSE: Mean Square Error 
##  MAE: Mean Absolute Error 
##  AIC: Akaike Information Criteria 
##  SBC: Schwarz Bayesian Criteria 
## 
##                                ANOVA                                 
## --------------------------------------------------------------------
##                 Sum of                                              
##                Squares        DF    Mean Square      F         Sig. 
## --------------------------------------------------------------------
## Regression     972.609         5        194.522    32.962    0.0000 
## Residual       153.438        26          5.901                     
## Total         1126.047        31                                    
## --------------------------------------------------------------------
## 
##                                   Parameter Estimates                                    
## ----------------------------------------------------------------------------------------
##       model      Beta    Std. Error    Std. Beta      t        Sig      lower     upper 
## ----------------------------------------------------------------------------------------
## (Intercept)    14.362         9.741                  1.474    0.152    -5.661    34.384 
##        disp     0.011         0.011        0.231     1.060    0.299    -0.011     0.033 
##          hp    -0.021         0.015       -0.241    -1.460    0.156    -0.051     0.009 
##          wt    -4.084         1.194       -0.663    -3.420    0.002    -6.539    -1.630 
##        qsec     1.007         0.475        0.299     2.118    0.044     0.030     1.984 
##          am     3.470         1.486        0.287     2.336    0.027     0.416     6.525 
## ----------------------------------------------------------------------------------------
\end{verbatim}

\begin{Shaded}
\begin{Highlighting}[]
\NormalTok{backward }\OtherTok{\textless{}{-}} \FunctionTok{ols\_step\_backward\_p}\NormalTok{(model)}
\FunctionTok{plot}\NormalTok{(backward)}
\end{Highlighting}
\end{Shaded}

\includegraphics{StatsTB_files/figure-latex/unnamed-chunk-385-1.pdf}

\hypertarget{example-tumor-biomarker-measurements}{%
\subsection{Example: Tumor Biomarker Measurements}\label{example-tumor-biomarker-measurements}}

\hypertarget{introduction-3}{%
\subsubsection{Introduction}\label{introduction-3}}

Stepwise backward regression is a widely used model selection technique in biomedical research. It starts with a full regression model including all candidate predictors and sequentially removes the least significant predictors based on their p-values. This approach helps to simplify the model by excluding variables that do not contribute significantly to explaining the outcome.

In this example, we simulate data for 100 patients, each with four tumor biomarker measurements (\texttt{Marker1}, \texttt{Marker2}, \texttt{Marker3}, and \texttt{Marker4}). The goal is to predict a continuous cancer risk score (\texttt{CancerRisk}). Using stepwise backward regression, we identify which biomarkers are most strongly associated with cancer risk.

\textbf{Load and Prepare Data}

\begin{Shaded}
\begin{Highlighting}[]
\FunctionTok{library}\NormalTok{(olsrr)}
\FunctionTok{set.seed}\NormalTok{(}\DecValTok{123}\NormalTok{)}

\CommentTok{\# Simulate biomarker dataset}
\NormalTok{df }\OtherTok{\textless{}{-}} \FunctionTok{data.frame}\NormalTok{(}
  \AttributeTok{CancerRisk =} \FunctionTok{rnorm}\NormalTok{(}\DecValTok{100}\NormalTok{, }\AttributeTok{mean=}\DecValTok{50}\NormalTok{, }\AttributeTok{sd=}\DecValTok{10}\NormalTok{),}
  \AttributeTok{Marker1 =} \FunctionTok{rnorm}\NormalTok{(}\DecValTok{100}\NormalTok{, }\AttributeTok{mean=}\DecValTok{5}\NormalTok{, }\AttributeTok{sd=}\DecValTok{2}\NormalTok{),}
  \AttributeTok{Marker2 =} \FunctionTok{rnorm}\NormalTok{(}\DecValTok{100}\NormalTok{, }\AttributeTok{mean=}\DecValTok{10}\NormalTok{, }\AttributeTok{sd=}\DecValTok{3}\NormalTok{),}
  \AttributeTok{Marker3 =} \FunctionTok{rnorm}\NormalTok{(}\DecValTok{100}\NormalTok{, }\AttributeTok{mean=}\DecValTok{7}\NormalTok{, }\AttributeTok{sd=}\FloatTok{1.5}\NormalTok{),}
  \AttributeTok{Marker4 =} \FunctionTok{rnorm}\NormalTok{(}\DecValTok{100}\NormalTok{, }\AttributeTok{mean=}\DecValTok{12}\NormalTok{, }\AttributeTok{sd=}\DecValTok{4}\NormalTok{)}
\NormalTok{)}

\CommentTok{\# Introduce true associations to CancerRisk for realism}
\NormalTok{df}\SpecialCharTok{$}\NormalTok{CancerRisk }\OtherTok{\textless{}{-}} \DecValTok{10} \SpecialCharTok{+} \FloatTok{1.5} \SpecialCharTok{*}\NormalTok{ df}\SpecialCharTok{$}\NormalTok{Marker1 }\SpecialCharTok{{-}} \DecValTok{2} \SpecialCharTok{*}\NormalTok{ df}\SpecialCharTok{$}\NormalTok{Marker3 }\SpecialCharTok{+} \FunctionTok{rnorm}\NormalTok{(}\DecValTok{100}\NormalTok{, }\DecValTok{0}\NormalTok{, }\DecValTok{5}\NormalTok{)}
\end{Highlighting}
\end{Shaded}

\hypertarget{visualization-of-stepwise-selection-process}{%
\subsubsection{Visualization of Stepwise Selection Process}\label{visualization-of-stepwise-selection-process}}

\begin{Shaded}
\begin{Highlighting}[]
\CommentTok{\# Fit full linear regression model with all biomarkers}
\NormalTok{full\_model }\OtherTok{\textless{}{-}} \FunctionTok{lm}\NormalTok{(CancerRisk }\SpecialCharTok{\textasciitilde{}}\NormalTok{ Marker1 }\SpecialCharTok{+}\NormalTok{ Marker2 }\SpecialCharTok{+}\NormalTok{ Marker3 }\SpecialCharTok{+}\NormalTok{ Marker4, }\AttributeTok{data =}\NormalTok{ df)}

\CommentTok{\# Perform stepwise backward regression based on p{-}values}
\NormalTok{backward }\OtherTok{\textless{}{-}} \FunctionTok{ols\_step\_backward\_p}\NormalTok{(full\_model, }\AttributeTok{details =} \ConstantTok{TRUE}\NormalTok{)}
\end{Highlighting}
\end{Shaded}

\begin{verbatim}
## Backward Elimination Method 
## ---------------------------
## 
## Candidate Terms: 
## 
## 1. Marker1 
## 2. Marker2 
## 3. Marker3 
## 4. Marker4 
## 
## 
## Step   => 0 
## Model  => CancerRisk ~ Marker1 + Marker2 + Marker3 + Marker4 
## R2     => 0.547 
## 
## Initiating stepwise selection... 
## 
## Step     => 1 
## Removed  => Marker2 
## Model    => CancerRisk ~ Marker1 + Marker3 + Marker4 
## R2       => 0.54659 
## 
## 
## No more variables to be removed.
## 
## Variables Removed: 
## 
## => Marker2
\end{verbatim}

\begin{Shaded}
\begin{Highlighting}[]
\CommentTok{\# Print stepwise regression summary}
\FunctionTok{print}\NormalTok{(backward)}
\end{Highlighting}
\end{Shaded}

\begin{verbatim}
## 
## 
##                              Stepwise Summary                              
## -------------------------------------------------------------------------
## Step    Variable        AIC        SBC       SBIC        R2       Adj. R2 
## -------------------------------------------------------------------------
##  0      Full Model    596.580    612.211    313.313    0.54670    0.52761 
##  1      Marker2       594.604    607.630    311.230    0.54659    0.53242 
## -------------------------------------------------------------------------
## 
## Final Model Output 
## ------------------
## 
##                          Model Summary                          
## ---------------------------------------------------------------
## R                       0.739       RMSE                 4.500 
## R-Squared               0.547       MSE                 20.250 
## Adj. R-Squared          0.532       Coef. Var          149.383 
## Pred R-Squared          0.504       AIC                594.604 
## MAE                     3.465       SBC                607.630 
## ---------------------------------------------------------------
##  RMSE: Root Mean Square Error 
##  MSE: Mean Square Error 
##  MAE: Mean Absolute Error 
##  AIC: Akaike Information Criteria 
##  SBC: Schwarz Bayesian Criteria 
## 
##                                ANOVA                                 
## --------------------------------------------------------------------
##                 Sum of                                              
##                Squares        DF    Mean Square      F         Sig. 
## --------------------------------------------------------------------
## Regression    2441.158         3        813.719    38.576    0.0000 
## Residual      2025.025        96         21.094                     
## Total         4466.183        99                                    
## --------------------------------------------------------------------
## 
##                                   Parameter Estimates                                    
## ----------------------------------------------------------------------------------------
##       model      Beta    Std. Error    Std. Beta      t        Sig      lower     upper 
## ----------------------------------------------------------------------------------------
## (Intercept)     6.696         2.871                  2.332    0.022     0.997    12.394 
##     Marker1     1.859         0.241        0.535     7.714    0.000     1.380     2.337 
##     Marker3    -2.277         0.297       -0.528    -7.679    0.000    -2.866    -1.689 
##     Marker4     0.266         0.118        0.157     2.259    0.026     0.032     0.499 
## ----------------------------------------------------------------------------------------
\end{verbatim}

\begin{Shaded}
\begin{Highlighting}[]
\CommentTok{\# Plot the stepwise backward regression process}
\FunctionTok{plot}\NormalTok{(backward)}
\end{Highlighting}
\end{Shaded}

\includegraphics{StatsTB_files/figure-latex/unnamed-chunk-387-1.pdf}

\hypertarget{interpretation-19}{%
\subsubsection{Interpretation}\label{interpretation-19}}

The initial model includes all four biomarkers as predictors.

The stepwise backward procedure removes predictors one at a time, starting with the least statistically significant variable (highest p-value), until only significant predictors remain.

In this simulated example, the final model retains Marker1 and Marker3 as significant predictors of cancer risk, while Marker2 and Marker4 are removed due to lack of statistical evidence.

This method facilitates identifying a simpler, more interpretable model that still captures the essential predictive factors.

\textbf{References}

Harrell, F.E. (2015). Regression Modeling Strategies (2nd ed.). Springer. (Chapters on model building and variable selection)

Steyerberg, E.W. (2019). Clinical Prediction Models (2nd ed.). Springer.

\hypertarget{method-5-stepwise-regression}{%
\section{Method 5: Stepwise Regression}\label{method-5-stepwise-regression}}

Build regression model from a set of candidate predictor variables by entering and removing predictors based on p values, in a stepwise manner until there is no variable left to enter or remove any more. The model should include all the candidate predictor variables. If details is set to TRUE, each step is displayed.

\begin{Shaded}
\begin{Highlighting}[]
\DocumentationTok{\#\#\#\#\#\#\#\#\#\#\#\#\#\#\#\#\#\#\#\#\#\#\#\#\#\#\#\#\#\#}
\CommentTok{\#(6) Stepwise both directions}
\DocumentationTok{\#\#\#\#\#\#\#\#\#\#\#\#\#\#\#\#\#\#\#\#\#\#\#\#\#\#\#\#\#\#}
\CommentTok{\# stepwise regression}
\NormalTok{model }\OtherTok{\textless{}{-}} \FunctionTok{lm}\NormalTok{(mpg }\SpecialCharTok{\textasciitilde{}}\NormalTok{ ., }\AttributeTok{data =}\NormalTok{ mtcars)}

\FunctionTok{ols\_step\_both\_p}\NormalTok{(model)}
\end{Highlighting}
\end{Shaded}

\begin{verbatim}
## 
## 
##                              Stepwise Summary                              
## -------------------------------------------------------------------------
## Step    Variable        AIC        SBC       SBIC        R2       Adj. R2 
## -------------------------------------------------------------------------
##  0      Base Model    208.756    211.687    115.061    0.00000    0.00000 
##  1      wt (+)        166.029    170.427     74.373    0.75283    0.74459 
##  2      cyl (+)       156.010    161.873     66.190    0.83023    0.81852 
## -------------------------------------------------------------------------
## 
## Final Model Output 
## ------------------
## 
##                          Model Summary                          
## ---------------------------------------------------------------
## R                       0.911       RMSE                 2.444 
## R-Squared               0.830       MSE                  5.974 
## Adj. R-Squared          0.819       Coef. Var           12.780 
## Pred R-Squared          0.790       AIC                156.010 
## MAE                     1.921       SBC                161.873 
## ---------------------------------------------------------------
##  RMSE: Root Mean Square Error 
##  MSE: Mean Square Error 
##  MAE: Mean Absolute Error 
##  AIC: Akaike Information Criteria 
##  SBC: Schwarz Bayesian Criteria 
## 
##                                ANOVA                                 
## --------------------------------------------------------------------
##                 Sum of                                              
##                Squares        DF    Mean Square      F         Sig. 
## --------------------------------------------------------------------
## Regression     934.875         2        467.438    70.908    0.0000 
## Residual       191.172        29          6.592                     
## Total         1126.047        31                                    
## --------------------------------------------------------------------
## 
##                                   Parameter Estimates                                    
## ----------------------------------------------------------------------------------------
##       model      Beta    Std. Error    Std. Beta      t        Sig      lower     upper 
## ----------------------------------------------------------------------------------------
## (Intercept)    39.686         1.715                 23.141    0.000    36.179    43.194 
##          wt    -3.191         0.757       -0.518    -4.216    0.000    -4.739    -1.643 
##         cyl    -1.508         0.415       -0.447    -3.636    0.001    -2.356    -0.660 
## ----------------------------------------------------------------------------------------
\end{verbatim}

\begin{Shaded}
\begin{Highlighting}[]
\NormalTok{both }\OtherTok{\textless{}{-}} \FunctionTok{ols\_step\_both\_p}\NormalTok{(model)}
\FunctionTok{plot}\NormalTok{(both)}
\end{Highlighting}
\end{Shaded}

\includegraphics{StatsTB_files/figure-latex/unnamed-chunk-388-1.pdf}

\hypertarget{example-predict-diabetes-diagnosis-from-physiological-variables}{%
\subsubsection{Example: Predict diabetes diagnosis from physiological variables}\label{example-predict-diabetes-diagnosis-from-physiological-variables}}

Stepwise regression combines both forward selection and backward elimination to find the best subset of predictors based on statistical criteria (like p-values). Here, we apply it to predict diabetes diagnosis from physiological variables in the Pima Indians Diabetes dataset.

\textbf{Run this first:}

\begin{Shaded}
\begin{Highlighting}[]
\FunctionTok{library}\NormalTok{(mlbench)}
\FunctionTok{library}\NormalTok{(olsrr)}
\FunctionTok{library}\NormalTok{(dplyr)}
\end{Highlighting}
\end{Shaded}

\textbf{Load and Prepare Data}

\begin{Shaded}
\begin{Highlighting}[]
\FunctionTok{data}\NormalTok{(}\StringTok{"PimaIndiansDiabetes2"}\NormalTok{, }\AttributeTok{package =} \StringTok{"mlbench"}\NormalTok{)}
\NormalTok{df }\OtherTok{\textless{}{-}} \FunctionTok{na.omit}\NormalTok{(PimaIndiansDiabetes2)  }\CommentTok{\# remove rows with missing data}

\CommentTok{\# Convert diabetes factor to numeric for regression}
\NormalTok{df}\SpecialCharTok{$}\NormalTok{diabetes }\OtherTok{\textless{}{-}} \FunctionTok{ifelse}\NormalTok{(df}\SpecialCharTok{$}\NormalTok{diabetes }\SpecialCharTok{==} \StringTok{"pos"}\NormalTok{, }\DecValTok{1}\NormalTok{, }\DecValTok{0}\NormalTok{)}
\end{Highlighting}
\end{Shaded}

\textbf{Fit Initial Full Logistic Regression Model}

\begin{Shaded}
\begin{Highlighting}[]
\CommentTok{\# Full model with all predictors}
\NormalTok{model\_full }\OtherTok{\textless{}{-}} \FunctionTok{glm}\NormalTok{(diabetes }\SpecialCharTok{\textasciitilde{}}\NormalTok{ ., }\AttributeTok{data =}\NormalTok{ df, }\AttributeTok{family =}\NormalTok{ binomial)}
\FunctionTok{summary}\NormalTok{(model\_full)}
\end{Highlighting}
\end{Shaded}

\begin{verbatim}
## 
## Call:
## glm(formula = diabetes ~ ., family = binomial, data = df)
## 
## Coefficients:
##               Estimate Std. Error z value Pr(>|z|)    
## (Intercept) -1.004e+01  1.218e+00  -8.246  < 2e-16 ***
## pregnant     8.216e-02  5.543e-02   1.482  0.13825    
## glucose      3.827e-02  5.768e-03   6.635 3.24e-11 ***
## pressure    -1.420e-03  1.183e-02  -0.120  0.90446    
## triceps      1.122e-02  1.708e-02   0.657  0.51128    
## insulin     -8.253e-04  1.306e-03  -0.632  0.52757    
## mass         7.054e-02  2.734e-02   2.580  0.00989 ** 
## pedigree     1.141e+00  4.274e-01   2.669  0.00760 ** 
## age          3.395e-02  1.838e-02   1.847  0.06474 .  
## ---
## Signif. codes:  0 '***' 0.001 '**' 0.01 '*' 0.05 '.' 0.1 ' ' 1
## 
## (Dispersion parameter for binomial family taken to be 1)
## 
##     Null deviance: 498.10  on 391  degrees of freedom
## Residual deviance: 344.02  on 383  degrees of freedom
## AIC: 362.02
## 
## Number of Fisher Scoring iterations: 5
\end{verbatim}

\textbf{Stepwise Regression: Both Directions}

\begin{Shaded}
\begin{Highlighting}[]
\CommentTok{\# Run stepwise regression (both directions)}
\NormalTok{step\_model }\OtherTok{\textless{}{-}} \FunctionTok{ols\_step\_both\_p}\NormalTok{(model\_full)}
\NormalTok{step\_model}
\end{Highlighting}
\end{Shaded}

\begin{verbatim}
## 
## 
##                                  Stepwise Summary                                  
## ---------------------------------------------------------------------------------
## Step    Variable          AIC        SBC          SBIC           R2       Adj. R2 
## ---------------------------------------------------------------------------------
##  0      Base Model      525.843    533.786     -5961710.479    0.00000    0.00000 
##  1      glucose (+)     406.646    418.559    -11064193.277    0.26595    0.26407 
##  2      age (+)         389.961    405.846    -12166809.023    0.30012    0.29652 
##  3      mass (+)        376.179    396.035    -13182923.283    0.32774    0.32254 
##  4      pedigree (+)    371.069    394.896    -13665236.496    0.33982    0.33300 
## ---------------------------------------------------------------------------------
## 
## Final Model Output 
## ------------------
## 
##                          Model Summary                          
## ---------------------------------------------------------------
## R                       0.583       RMSE                 0.383 
## R-Squared               0.340       MSE                  0.146 
## Adj. R-Squared          0.333       Coef. Var          116.091 
## Pred R-Squared          0.319       AIC                371.069 
## MAE                     0.308       SBC                394.896 
## ---------------------------------------------------------------
##  RMSE: Root Mean Square Error 
##  MSE: Mean Square Error 
##  MAE: Mean Absolute Error 
##  AIC: Akaike Information Criteria 
##  SBC: Schwarz Bayesian Criteria 
## 
##                                ANOVA                                 
## --------------------------------------------------------------------
##                Sum of                                               
##               Squares         DF    Mean Square      F         Sig. 
## --------------------------------------------------------------------
## Regression     29.526          4          7.382    49.801    0.0000 
## Residual       57.362        387          0.148                     
## Total          86.888        391                                    
## --------------------------------------------------------------------
## 
##                                   Parameter Estimates                                    
## ----------------------------------------------------------------------------------------
##       model      Beta    Std. Error    Std. Beta      t        Sig      lower     upper 
## ----------------------------------------------------------------------------------------
## (Intercept)    -1.118         0.116                 -9.601    0.000    -1.347    -0.889 
##     glucose     0.006         0.001        0.401     8.909    0.000     0.005     0.007 
##         age     0.009         0.002        0.193     4.374    0.000     0.005     0.013 
##        mass     0.010         0.003        0.155     3.631    0.000     0.005     0.016 
##    pedigree     0.153         0.057        0.112     2.661    0.008     0.040     0.266 
## ----------------------------------------------------------------------------------------
\end{verbatim}

\textbf{Plot Model Selection Steps}

\begin{Shaded}
\begin{Highlighting}[]
\FunctionTok{plot}\NormalTok{(step\_model)}
\end{Highlighting}
\end{Shaded}

\includegraphics{StatsTB_files/figure-latex/unnamed-chunk-393-1.pdf}

\hypertarget{interpretation-20}{%
\subsubsection{Interpretation}\label{interpretation-20}}

The stepwise procedure starts from the full model with all predictors and removes or adds predictors step-by-step based on p-values until the best model is found. The final model includes only those variables statistically significant in predicting diabetes status.

\textbf{References}

Kuhn, M. \& Johnson, K. (2013). Applied Predictive Modeling. Springer.

Smith et al.~(1988). Pima Indians Diabetes Dataset. {[}mlbench package{]}.

\hypertarget{method-6-stepwise-aic-forward-regression}{%
\section{Method 6: Stepwise AIC Forward Regression}\label{method-6-stepwise-aic-forward-regression}}

Build regression model from a set of candidate predictor variables by entering predictors based on Akaike Information Criteria, in a stepwise manner until there is no variable left to enter any more. The model should include all the candidate predictor variables. If details is set to TRUE, each step is displayed.

\begin{Shaded}
\begin{Highlighting}[]
\DocumentationTok{\#\#\#\#\#\#\#\#\#\#\#\#\#\#\#\#\#\#\#\#\#\#\#\#\#\#\#\#\#\#}
\CommentTok{\#(7) Stepwise AIC Forward }
\DocumentationTok{\#\#\#\#\#\#\#\#\#\#\#\#\#\#\#\#\#\#\#\#\#\#\#\#\#\#\#\#\#\#}

\CommentTok{\# stepwise aic forward regression}
\FunctionTok{ols\_step\_forward\_aic}\NormalTok{(model)}
\end{Highlighting}
\end{Shaded}

\begin{verbatim}
## 
## 
##                              Stepwise Summary                              
## -------------------------------------------------------------------------
## Step    Variable        AIC        SBC       SBIC        R2       Adj. R2 
## -------------------------------------------------------------------------
##  0      Base Model    208.756    211.687    115.061    0.00000    0.00000 
##  1      wt            166.029    170.427     74.373    0.75283    0.74459 
##  2      cyl           156.010    161.873     66.190    0.83023    0.81852 
##  3      hp            155.477    162.805     66.696    0.84315    0.82634 
## -------------------------------------------------------------------------
## 
## Final Model Output 
## ------------------
## 
##                          Model Summary                          
## ---------------------------------------------------------------
## R                       0.918       RMSE                 2.349 
## R-Squared               0.843       MSE                  5.519 
## Adj. R-Squared          0.826       Coef. Var           12.501 
## Pred R-Squared          0.796       AIC                155.477 
## MAE                     1.845       SBC                162.805 
## ---------------------------------------------------------------
##  RMSE: Root Mean Square Error 
##  MSE: Mean Square Error 
##  MAE: Mean Absolute Error 
##  AIC: Akaike Information Criteria 
##  SBC: Schwarz Bayesian Criteria 
## 
##                                ANOVA                                 
## --------------------------------------------------------------------
##                 Sum of                                              
##                Squares        DF    Mean Square      F         Sig. 
## --------------------------------------------------------------------
## Regression     949.427         3        316.476    50.171    0.0000 
## Residual       176.621        28          6.308                     
## Total         1126.047        31                                    
## --------------------------------------------------------------------
## 
##                                   Parameter Estimates                                    
## ----------------------------------------------------------------------------------------
##       model      Beta    Std. Error    Std. Beta      t        Sig      lower     upper 
## ----------------------------------------------------------------------------------------
## (Intercept)    38.752         1.787                 21.687    0.000    35.092    42.412 
##          wt    -3.167         0.741       -0.514    -4.276    0.000    -4.684    -1.650 
##         cyl    -0.942         0.551       -0.279    -1.709    0.098    -2.070     0.187 
##          hp    -0.018         0.012       -0.205    -1.519    0.140    -0.042     0.006 
## ----------------------------------------------------------------------------------------
\end{verbatim}

\begin{Shaded}
\begin{Highlighting}[]
\NormalTok{aic.forward }\OtherTok{\textless{}{-}} \FunctionTok{ols\_step\_forward\_aic}\NormalTok{(model)}

\FunctionTok{plot}\NormalTok{(aic.forward)}
\end{Highlighting}
\end{Shaded}

\includegraphics{StatsTB_files/figure-latex/unnamed-chunk-394-1.pdf}

\hypertarget{exmaple-cholesterol-level}{%
\subsection{Exmaple: Cholesterol Level}\label{exmaple-cholesterol-level}}

Stepwise AIC Forward Regression builds a model by sequentially adding variables that most reduce the Akaike Information Criterion (AIC), a metric that balances model fit and complexity. It stops when no additional variables improve the AIC.

In this example, we simulate a biomedical dataset where \textbf{cholesterol level} depends on various body and lifestyle variables.

\begin{Shaded}
\begin{Highlighting}[]
\FunctionTok{library}\NormalTok{(MASS)   }\CommentTok{\# for stepAIC}
\FunctionTok{set.seed}\NormalTok{(}\DecValTok{123}\NormalTok{)}
\end{Highlighting}
\end{Shaded}

\textbf{Simulated Biomedical Dataset}

\begin{Shaded}
\begin{Highlighting}[]
\NormalTok{n }\OtherTok{\textless{}{-}} \DecValTok{100}
\NormalTok{simdata }\OtherTok{\textless{}{-}} \FunctionTok{data.frame}\NormalTok{(}
  \AttributeTok{Cholesterol =} \FunctionTok{rnorm}\NormalTok{(n, }\AttributeTok{mean =} \DecValTok{200}\NormalTok{, }\AttributeTok{sd =} \DecValTok{25}\NormalTok{),}
  \AttributeTok{Age =} \FunctionTok{rnorm}\NormalTok{(n, }\AttributeTok{mean =} \DecValTok{50}\NormalTok{, }\AttributeTok{sd =} \DecValTok{10}\NormalTok{),}
  \AttributeTok{BMI =} \FunctionTok{rnorm}\NormalTok{(n, }\AttributeTok{mean =} \DecValTok{25}\NormalTok{, }\AttributeTok{sd =} \DecValTok{4}\NormalTok{),}
  \AttributeTok{HeartRate =} \FunctionTok{rnorm}\NormalTok{(n, }\AttributeTok{mean =} \DecValTok{75}\NormalTok{, }\AttributeTok{sd =} \DecValTok{12}\NormalTok{),}
  \AttributeTok{Smoker =} \FunctionTok{sample}\NormalTok{(}\FunctionTok{c}\NormalTok{(}\DecValTok{0}\NormalTok{, }\DecValTok{1}\NormalTok{), n, }\AttributeTok{replace =} \ConstantTok{TRUE}\NormalTok{),}
  \AttributeTok{Exercise =} \FunctionTok{sample}\NormalTok{(}\DecValTok{1}\SpecialCharTok{:}\DecValTok{5}\NormalTok{, n, }\AttributeTok{replace =} \ConstantTok{TRUE}\NormalTok{),}
  \AttributeTok{BloodPressure =} \FunctionTok{rnorm}\NormalTok{(n, }\AttributeTok{mean =} \DecValTok{120}\NormalTok{, }\AttributeTok{sd =} \DecValTok{15}\NormalTok{)}
\NormalTok{)}

\CommentTok{\# Add true associations}
\NormalTok{simdata}\SpecialCharTok{$}\NormalTok{Cholesterol }\OtherTok{\textless{}{-}} \DecValTok{180} \SpecialCharTok{+} \FloatTok{1.5} \SpecialCharTok{*}\NormalTok{ simdata}\SpecialCharTok{$}\NormalTok{BMI }\SpecialCharTok{+}
  \FloatTok{0.7} \SpecialCharTok{*}\NormalTok{ simdata}\SpecialCharTok{$}\NormalTok{Age }\SpecialCharTok{{-}} \DecValTok{5} \SpecialCharTok{*}\NormalTok{ simdata}\SpecialCharTok{$}\NormalTok{Exercise }\SpecialCharTok{+}
  \DecValTok{8} \SpecialCharTok{*}\NormalTok{ simdata}\SpecialCharTok{$}\NormalTok{Smoker }\SpecialCharTok{+} \FunctionTok{rnorm}\NormalTok{(n, }\DecValTok{0}\NormalTok{, }\DecValTok{10}\NormalTok{)}
\end{Highlighting}
\end{Shaded}

\textbf{AIC Forward Slection Model}

\begin{Shaded}
\begin{Highlighting}[]
\FunctionTok{library}\NormalTok{(MASS)}
\NormalTok{null\_mod }\OtherTok{\textless{}{-}} \FunctionTok{lm}\NormalTok{(Cholesterol }\SpecialCharTok{\textasciitilde{}} \DecValTok{1}\NormalTok{, }\AttributeTok{data =}\NormalTok{ simdata)}
\NormalTok{full\_mod }\OtherTok{\textless{}{-}} \FunctionTok{lm}\NormalTok{(Cholesterol }\SpecialCharTok{\textasciitilde{}}\NormalTok{ ., }\AttributeTok{data =}\NormalTok{ simdata)}

\NormalTok{step\_aic\_forward }\OtherTok{\textless{}{-}} \FunctionTok{stepAIC}\NormalTok{(null\_mod,}
                            \AttributeTok{scope =} \FunctionTok{list}\NormalTok{(}\AttributeTok{lower =}\NormalTok{ null\_mod, }\AttributeTok{upper =}\NormalTok{ full\_mod),}
                            \AttributeTok{direction =} \StringTok{"forward"}\NormalTok{,}
                            \AttributeTok{trace =} \ConstantTok{FALSE}\NormalTok{)}

\FunctionTok{summary}\NormalTok{(step\_aic\_forward)}
\end{Highlighting}
\end{Shaded}

\begin{verbatim}
## 
## Call:
## lm(formula = Cholesterol ~ Age + Exercise + BMI + Smoker, data = simdata)
## 
## Residuals:
##     Min      1Q  Median      3Q     Max 
## -21.266  -6.610   1.127   6.457  23.066 
## 
## Coefficients:
##             Estimate Std. Error t value Pr(>|t|)    
## (Intercept) 179.6261     8.3760  21.445  < 2e-16 ***
## Age           0.6583     0.1038   6.342 7.62e-09 ***
## Exercise     -4.3738     0.7472  -5.854 6.84e-08 ***
## BMI           1.5196     0.2622   5.796 8.83e-08 ***
## Smoker        6.3996     2.0089   3.186  0.00195 ** 
## ---
## Signif. codes:  0 '***' 0.001 '**' 0.01 '*' 0.05 '.' 0.1 ' ' 1
## 
## Residual standard error: 9.871 on 95 degrees of freedom
## Multiple R-squared:  0.5454, Adjusted R-squared:  0.5263 
## F-statistic:  28.5 on 4 and 95 DF,  p-value: 1.466e-15
\end{verbatim}

\textbf{AIC Value Plot}

\begin{Shaded}
\begin{Highlighting}[]
\FunctionTok{install.packages}\NormalTok{(}\StringTok{"ggplot2"}\NormalTok{)}
\FunctionTok{library}\NormalTok{(ggplot2)}
\NormalTok{aic\_data }\OtherTok{\textless{}{-}}\NormalTok{ step\_aic\_forward}\SpecialCharTok{$}\NormalTok{anova}
\NormalTok{aic\_data}\SpecialCharTok{$}\NormalTok{Step }\OtherTok{\textless{}{-}} \FunctionTok{seq\_len}\NormalTok{(}\FunctionTok{nrow}\NormalTok{(aic\_data))}

\FunctionTok{ggplot}\NormalTok{(aic\_data, }\FunctionTok{aes}\NormalTok{(}\AttributeTok{x =}\NormalTok{ Step, }\AttributeTok{y =}\NormalTok{ AIC)) }\SpecialCharTok{+}
  \FunctionTok{geom\_line}\NormalTok{(}\AttributeTok{color =} \StringTok{"\#0072B2"}\NormalTok{, }\AttributeTok{size =} \DecValTok{1}\NormalTok{) }\SpecialCharTok{+}
  \FunctionTok{geom\_point}\NormalTok{(}\AttributeTok{size =} \DecValTok{3}\NormalTok{) }\SpecialCharTok{+}
  \FunctionTok{labs}\NormalTok{(}\AttributeTok{title =} \StringTok{"Stepwise AIC Forward Selection"}\NormalTok{,}
       \AttributeTok{x =} \StringTok{"Step"}\NormalTok{,}
       \AttributeTok{y =} \StringTok{"AIC Value"}\NormalTok{) }\SpecialCharTok{+}
  \FunctionTok{theme\_minimal}\NormalTok{()}
\end{Highlighting}
\end{Shaded}

\includegraphics{StatsTB_files/figure-latex/unnamed-chunk-398-1.pdf}

\textbf{Final Model Coefficients}

\begin{Shaded}
\begin{Highlighting}[]
\FunctionTok{install.packages}\NormalTok{(}\StringTok{"broom"}\NormalTok{)}
\FunctionTok{library}\NormalTok{(broom)}
\NormalTok{coef\_df }\OtherTok{\textless{}{-}} \FunctionTok{tidy}\NormalTok{(step\_aic\_forward)}

\FunctionTok{ggplot}\NormalTok{(coef\_df[}\SpecialCharTok{{-}}\DecValTok{1}\NormalTok{, ], }\FunctionTok{aes}\NormalTok{(}\AttributeTok{x =} \FunctionTok{reorder}\NormalTok{(term, estimate), }\AttributeTok{y =}\NormalTok{ estimate)) }\SpecialCharTok{+}
  \FunctionTok{geom\_col}\NormalTok{(}\AttributeTok{fill =} \StringTok{"\#D55E00"}\NormalTok{) }\SpecialCharTok{+}
  \FunctionTok{coord\_flip}\NormalTok{() }\SpecialCharTok{+}
  \FunctionTok{labs}\NormalTok{(}\AttributeTok{title =} \StringTok{"Final Model Coefficients"}\NormalTok{,}
       \AttributeTok{x =} \StringTok{"Predictor"}\NormalTok{,}
       \AttributeTok{y =} \StringTok{"Coefficient Estimate"}\NormalTok{) }\SpecialCharTok{+}
  \FunctionTok{theme\_minimal}\NormalTok{()}
\end{Highlighting}
\end{Shaded}

\includegraphics{StatsTB_files/figure-latex/unnamed-chunk-399-1.pdf}

\textbf{Interpretation} The AIC plot helps you visualize how the model improves with each added predictor.

The coefficient plot shows which variables are most strongly associated with cholesterol in the final model.

This is helpful in biomedical research when identifying key risk factors from multiple candidate variables.

\hypertarget{method-7-stepwise-aic-backward-regression}{%
\section{Method 7: Stepwise AIC Backward Regression}\label{method-7-stepwise-aic-backward-regression}}

Build regression model from a set of candidate predictor variables by removing predictors based on Akaike Information Criteria, in a stepwise manner until there is no variable left to remove any more. The model should include all the candidate predictor variables. If details is set to TRUE, each step is displayed.

\begin{Shaded}
\begin{Highlighting}[]
\DocumentationTok{\#\#\#\#\#\#\#\#\#\#\#\#\#\#\#\#\#\#\#\#\#\#\#\#\#\#\#\#\#\#}
\CommentTok{\#(7) Stepwise AIC Backward }
\DocumentationTok{\#\#\#\#\#\#\#\#\#\#\#\#\#\#\#\#\#\#\#\#\#\#\#\#\#\#\#\#\#\#}

\CommentTok{\# stepwise aic backward regression}

\FunctionTok{ols\_step\_backward\_aic}\NormalTok{(model)}
\end{Highlighting}
\end{Shaded}

\begin{verbatim}
## 
## 
##                              Stepwise Summary                             
## ------------------------------------------------------------------------
## Step    Variable        AIC        SBC       SBIC       R2       Adj. R2 
## ------------------------------------------------------------------------
##  0      Full Model    163.710    181.299    83.873    0.86902    0.80664 
##  1      cyl           161.727    177.850    79.593    0.86894    0.81533 
##  2      vs            159.785    174.443    75.710    0.86871    0.82304 
##  3      carb          157.933    171.125    72.232    0.86810    0.82963 
##  4      gear          156.269    167.995    69.220    0.86671    0.83472 
##  5      drat          154.974    165.234    66.825    0.86374    0.83753 
##  6      disp          154.327    163.122    65.299    0.85785    0.83679 
##  7      hp            154.119    161.448    64.409    0.84966    0.83356 
## ------------------------------------------------------------------------
## 
## Final Model Output 
## ------------------
## 
##                          Model Summary                          
## ---------------------------------------------------------------
## R                       0.922       RMSE                 2.300 
## R-Squared               0.850       MSE                  5.290 
## Adj. R-Squared          0.834       Coef. Var           12.239 
## Pred R-Squared          0.795       AIC                154.119 
## MAE                     1.932       SBC                161.448 
## ---------------------------------------------------------------
##  RMSE: Root Mean Square Error 
##  MSE: Mean Square Error 
##  MAE: Mean Absolute Error 
##  AIC: Akaike Information Criteria 
##  SBC: Schwarz Bayesian Criteria 
## 
##                                ANOVA                                 
## --------------------------------------------------------------------
##                 Sum of                                              
##                Squares        DF    Mean Square      F         Sig. 
## --------------------------------------------------------------------
## Regression     956.761         3        318.920     52.75    0.0000 
## Residual       169.286        28          6.046                     
## Total         1126.047        31                                    
## --------------------------------------------------------------------
## 
##                                   Parameter Estimates                                    
## ----------------------------------------------------------------------------------------
##       model      Beta    Std. Error    Std. Beta      t        Sig      lower     upper 
## ----------------------------------------------------------------------------------------
## (Intercept)     9.618         6.960                  1.382    0.178    -4.638    23.874 
##          wt    -3.917         0.711       -0.636    -5.507    0.000    -5.373    -2.460 
##        qsec     1.226         0.289        0.363     4.247    0.000     0.635     1.817 
##          am     2.936         1.411        0.243     2.081    0.047     0.046     5.826 
## ----------------------------------------------------------------------------------------
\end{verbatim}

\begin{Shaded}
\begin{Highlighting}[]
\NormalTok{aic.backward }\OtherTok{\textless{}{-}} \FunctionTok{ols\_step\_backward\_aic}\NormalTok{(model)}

\FunctionTok{plot}\NormalTok{(aic.backward)}
\end{Highlighting}
\end{Shaded}

\includegraphics{StatsTB_files/figure-latex/unnamed-chunk-400-1.pdf}

\hypertarget{example-disease-risk-based-on-health-indicators}{%
\subsection{Example: Disease Risk based on health indicators}\label{example-disease-risk-based-on-health-indicators}}

Build regression model from a set of candidate predictor variables by removing predictors based on Akaike Information Criterion (AIC), in a stepwise manner until there is no variable left to remove. The model begins with all predictors and selects the model with the lowest AIC. AIC balances model fit and complexity.

In this example, we simulate data commonly seen in biomedical research, where we want to model DiseaseRisk based on five health indicators: age, BMI, systolic blood pressure (SBP), cholesterol, and physical activity.

\begin{Shaded}
\begin{Highlighting}[]
\FunctionTok{library}\NormalTok{(MASS)    }\CommentTok{\# for stepAIC}
\FunctionTok{library}\NormalTok{(ggplot2) }\CommentTok{\# for plots}
\FunctionTok{library}\NormalTok{(broom)   }\CommentTok{\# for tidy model output}

\FunctionTok{set.seed}\NormalTok{(}\DecValTok{123}\NormalTok{)}

\CommentTok{\# Simulate a biomedical dataset}
\NormalTok{biomed\_df }\OtherTok{\textless{}{-}} \FunctionTok{data.frame}\NormalTok{(}
  \AttributeTok{DiseaseRisk =} \FunctionTok{rnorm}\NormalTok{(}\DecValTok{150}\NormalTok{, }\AttributeTok{mean =} \DecValTok{60}\NormalTok{, }\AttributeTok{sd =} \DecValTok{10}\NormalTok{),}
  \AttributeTok{Age =} \FunctionTok{rnorm}\NormalTok{(}\DecValTok{150}\NormalTok{, }\AttributeTok{mean =} \DecValTok{50}\NormalTok{, }\AttributeTok{sd =} \DecValTok{12}\NormalTok{),}
  \AttributeTok{BMI =} \FunctionTok{rnorm}\NormalTok{(}\DecValTok{150}\NormalTok{, }\AttributeTok{mean =} \DecValTok{25}\NormalTok{, }\AttributeTok{sd =} \DecValTok{4}\NormalTok{),}
  \AttributeTok{SBP =} \FunctionTok{rnorm}\NormalTok{(}\DecValTok{150}\NormalTok{, }\AttributeTok{mean =} \DecValTok{120}\NormalTok{, }\AttributeTok{sd =} \DecValTok{15}\NormalTok{),}
  \AttributeTok{Cholesterol =} \FunctionTok{rnorm}\NormalTok{(}\DecValTok{150}\NormalTok{, }\AttributeTok{mean =} \DecValTok{200}\NormalTok{, }\AttributeTok{sd =} \DecValTok{30}\NormalTok{),}
  \AttributeTok{Activity =} \FunctionTok{rnorm}\NormalTok{(}\DecValTok{150}\NormalTok{, }\AttributeTok{mean =} \DecValTok{5}\NormalTok{, }\AttributeTok{sd =} \DecValTok{2}\NormalTok{)}
\NormalTok{)}

\CommentTok{\# Introduce associations}
\NormalTok{biomed\_df}\SpecialCharTok{$}\NormalTok{DiseaseRisk }\OtherTok{\textless{}{-}} \DecValTok{20} \SpecialCharTok{+} \FloatTok{0.6} \SpecialCharTok{*}\NormalTok{ biomed\_df}\SpecialCharTok{$}\NormalTok{Age }\SpecialCharTok{+} \FloatTok{1.2} \SpecialCharTok{*}\NormalTok{ biomed\_df}\SpecialCharTok{$}\NormalTok{BMI }\SpecialCharTok{+}
  \FloatTok{0.05} \SpecialCharTok{*}\NormalTok{ biomed\_df}\SpecialCharTok{$}\NormalTok{SBP }\SpecialCharTok{{-}} \FloatTok{0.3} \SpecialCharTok{*}\NormalTok{ biomed\_df}\SpecialCharTok{$}\NormalTok{Activity }\SpecialCharTok{+} \FunctionTok{rnorm}\NormalTok{(}\DecValTok{150}\NormalTok{, }\DecValTok{0}\NormalTok{, }\DecValTok{5}\NormalTok{)}

\CommentTok{\# Fit full model}
\NormalTok{full\_model }\OtherTok{\textless{}{-}} \FunctionTok{lm}\NormalTok{(DiseaseRisk }\SpecialCharTok{\textasciitilde{}}\NormalTok{ Age }\SpecialCharTok{+}\NormalTok{ BMI }\SpecialCharTok{+}\NormalTok{ SBP }\SpecialCharTok{+}\NormalTok{ Cholesterol }\SpecialCharTok{+}\NormalTok{ Activity, }\AttributeTok{data =}\NormalTok{ biomed\_df)}

\CommentTok{\# Run backward AIC selection}
\NormalTok{backward\_aic }\OtherTok{\textless{}{-}} \FunctionTok{stepAIC}\NormalTok{(full\_model, }\AttributeTok{direction =} \StringTok{"backward"}\NormalTok{, }\AttributeTok{trace =} \ConstantTok{TRUE}\NormalTok{)}
\end{Highlighting}
\end{Shaded}

\begin{verbatim}
## Start:  AIC=505.05
## DiseaseRisk ~ Age + BMI + SBP + Cholesterol + Activity
## 
##               Df Sum of Sq     RSS    AIC
## - Cholesterol  1      38.6  4053.0 504.49
## - Activity     1      44.3  4058.8 504.70
## <none>                      4014.4 505.05
## - SBP          1     129.9  4144.3 507.83
## - BMI          1    5182.5  9196.9 627.40
## - Age          1    6403.8 10418.3 646.10
## 
## Step:  AIC=504.49
## DiseaseRisk ~ Age + BMI + SBP + Activity
## 
##            Df Sum of Sq     RSS    AIC
## - Activity  1      43.5  4096.5 504.09
## <none>                   4053.0 504.49
## - SBP       1     127.6  4180.6 507.14
## - BMI       1    5149.5  9202.5 625.49
## - Age       1    6613.1 10666.1 647.63
## 
## Step:  AIC=504.09
## DiseaseRisk ~ Age + BMI + SBP
## 
##        Df Sum of Sq     RSS    AIC
## <none>               4096.5 504.09
## - SBP   1     119.3  4215.8 506.39
## - BMI   1    5469.1  9565.6 629.29
## - Age   1    6777.5 10874.0 648.52
\end{verbatim}

\hypertarget{final-model-coefficients}{%
\subsubsection{Final Model Coefficients}\label{final-model-coefficients}}

\begin{Shaded}
\begin{Highlighting}[]
\NormalTok{coef\_df }\OtherTok{\textless{}{-}} \FunctionTok{tidy}\NormalTok{(backward\_aic)}

\FunctionTok{ggplot}\NormalTok{(coef\_df[}\SpecialCharTok{{-}}\DecValTok{1}\NormalTok{, ], }\FunctionTok{aes}\NormalTok{(}\AttributeTok{x =} \FunctionTok{reorder}\NormalTok{(term, estimate), }\AttributeTok{y =}\NormalTok{ estimate)) }\SpecialCharTok{+}
  \FunctionTok{geom\_col}\NormalTok{(}\AttributeTok{fill =} \StringTok{"\#0072B2"}\NormalTok{) }\SpecialCharTok{+}
  \FunctionTok{coord\_flip}\NormalTok{() }\SpecialCharTok{+}
  \FunctionTok{labs}\NormalTok{(}
    \AttributeTok{title =} \StringTok{"Final Model Coefficients (AIC Backward Selection)"}\NormalTok{,}
    \AttributeTok{x =} \StringTok{"Predictor"}\NormalTok{,}
    \AttributeTok{y =} \StringTok{"Coefficient Estimate"}
\NormalTok{  ) }\SpecialCharTok{+}
  \FunctionTok{theme\_minimal}\NormalTok{()}
\end{Highlighting}
\end{Shaded}

\includegraphics{StatsTB_files/figure-latex/unnamed-chunk-402-1.pdf}

\hypertarget{aic-change-across-steps}{%
\subsubsection{AIC Change Across Steps}\label{aic-change-across-steps}}

\begin{Shaded}
\begin{Highlighting}[]
\NormalTok{trace\_lines }\OtherTok{\textless{}{-}} \FunctionTok{capture.output}\NormalTok{(}\FunctionTok{stepAIC}\NormalTok{(full\_model, }\AttributeTok{direction =} \StringTok{"backward"}\NormalTok{, }\AttributeTok{trace =} \ConstantTok{TRUE}\NormalTok{))}

\NormalTok{aic\_lines }\OtherTok{\textless{}{-}} \FunctionTok{grep}\NormalTok{(}\StringTok{"AIC="}\NormalTok{, trace\_lines, }\AttributeTok{value =} \ConstantTok{TRUE}\NormalTok{)}

\NormalTok{aic\_values }\OtherTok{\textless{}{-}} \FunctionTok{sapply}\NormalTok{(aic\_lines, }\ControlFlowTok{function}\NormalTok{(line) \{}
\NormalTok{  match }\OtherTok{\textless{}{-}} \FunctionTok{regmatches}\NormalTok{(line, }\FunctionTok{regexpr}\NormalTok{(}\StringTok{"AIC=}\SpecialCharTok{\textbackslash{}\textbackslash{}}\StringTok{s*[{-}]?[0{-}9.]+"}\NormalTok{, line))}
  \FunctionTok{as.numeric}\NormalTok{(}\FunctionTok{sub}\NormalTok{(}\StringTok{"AIC=}\SpecialCharTok{\textbackslash{}\textbackslash{}}\StringTok{s*"}\NormalTok{, }\StringTok{""}\NormalTok{, match))}
\NormalTok{\})}

\NormalTok{aic\_df }\OtherTok{\textless{}{-}} \FunctionTok{data.frame}\NormalTok{(}\AttributeTok{Step =} \FunctionTok{seq\_along}\NormalTok{(aic\_values), }\AttributeTok{AIC =}\NormalTok{ aic\_values)}

\FunctionTok{ggplot}\NormalTok{(aic\_df, }\FunctionTok{aes}\NormalTok{(}\AttributeTok{x =}\NormalTok{ Step, }\AttributeTok{y =}\NormalTok{ AIC)) }\SpecialCharTok{+}
  \FunctionTok{geom\_line}\NormalTok{(}\AttributeTok{color =} \StringTok{"\#D55E00"}\NormalTok{, }\AttributeTok{size =} \FloatTok{1.2}\NormalTok{) }\SpecialCharTok{+}
  \FunctionTok{geom\_point}\NormalTok{(}\AttributeTok{size =} \DecValTok{3}\NormalTok{, }\AttributeTok{color =} \StringTok{"\#D55E00"}\NormalTok{) }\SpecialCharTok{+}
  \FunctionTok{labs}\NormalTok{(}\AttributeTok{title =} \StringTok{"AIC Change Across Backward Selection Steps"}\NormalTok{,}
       \AttributeTok{x =} \StringTok{"Step Number"}\NormalTok{,}
       \AttributeTok{y =} \StringTok{"AIC Value"}\NormalTok{) }\SpecialCharTok{+}
  \FunctionTok{theme\_minimal}\NormalTok{()}
\end{Highlighting}
\end{Shaded}

\includegraphics{StatsTB_files/figure-latex/unnamed-chunk-403-1.pdf}

\textbf{Interpretation}

The stepwise backward regression begins with all candidate predictors and removes the least important ones based on AIC.

A lower AIC indicates a better balance between goodness-of-fit and model simplicity.

In the AIC change plot, we can observe how the AIC decreases as variables are removed from the model.

\textbf{Reference}

This method is based on the Stepwise model selection using AIC approach described in Modern Applied Statistics with S by W.N. Venables and B.D. Ripley (2002). The stepAIC() function used here is part of the MASS package, which is widely used in statistical modeling.

\hypertarget{method-8-stepwise-aic-regression}{%
\section{Method 8: Stepwise AIC Regression}\label{method-8-stepwise-aic-regression}}

Build regression model from a set of candidate predictor variables by entering and removing predictors based on Akaike Information Criteria, in a stepwise manner until there is no variable left to enter or remove any more. The model should include all the candidate predictor variables. If details is set to TRUE, each step is displayed.

\begin{Shaded}
\begin{Highlighting}[]
\DocumentationTok{\#\#\#\#\#\#\#\#\#\#\#\#\#\#\#\#\#\#\#\#\#\#\#\#\#\#\#\#\#\#}
\CommentTok{\#(7) Stepwise AIC  }
\DocumentationTok{\#\#\#\#\#\#\#\#\#\#\#\#\#\#\#\#\#\#\#\#\#\#\#\#\#\#\#\#\#\#}
\CommentTok{\# stepwise aic regression}

\FunctionTok{ols\_step\_both\_aic}\NormalTok{(model)}
\end{Highlighting}
\end{Shaded}

\begin{verbatim}
## 
## 
##                              Stepwise Summary                              
## -------------------------------------------------------------------------
## Step    Variable        AIC        SBC       SBIC        R2       Adj. R2 
## -------------------------------------------------------------------------
##  0      Base Model    208.756    211.687    115.061    0.00000    0.00000 
##  1      wt (+)        166.029    170.427     74.373    0.75283    0.74459 
##  2      cyl (+)       156.010    161.873     66.190    0.83023    0.81852 
##  3      hp (+)        155.477    162.805     66.696    0.84315    0.82634 
## -------------------------------------------------------------------------
## 
## Final Model Output 
## ------------------
## 
##                          Model Summary                          
## ---------------------------------------------------------------
## R                       0.918       RMSE                 2.349 
## R-Squared               0.843       MSE                  5.519 
## Adj. R-Squared          0.826       Coef. Var           12.501 
## Pred R-Squared          0.796       AIC                155.477 
## MAE                     1.845       SBC                162.805 
## ---------------------------------------------------------------
##  RMSE: Root Mean Square Error 
##  MSE: Mean Square Error 
##  MAE: Mean Absolute Error 
##  AIC: Akaike Information Criteria 
##  SBC: Schwarz Bayesian Criteria 
## 
##                                ANOVA                                 
## --------------------------------------------------------------------
##                 Sum of                                              
##                Squares        DF    Mean Square      F         Sig. 
## --------------------------------------------------------------------
## Regression     949.427         3        316.476    50.171    0.0000 
## Residual       176.621        28          6.308                     
## Total         1126.047        31                                    
## --------------------------------------------------------------------
## 
##                                   Parameter Estimates                                    
## ----------------------------------------------------------------------------------------
##       model      Beta    Std. Error    Std. Beta      t        Sig      lower     upper 
## ----------------------------------------------------------------------------------------
## (Intercept)    38.752         1.787                 21.687    0.000    35.092    42.412 
##          wt    -3.167         0.741       -0.514    -4.276    0.000    -4.684    -1.650 
##         cyl    -0.942         0.551       -0.279    -1.709    0.098    -2.070     0.187 
##          hp    -0.018         0.012       -0.205    -1.519    0.140    -0.042     0.006 
## ----------------------------------------------------------------------------------------
\end{verbatim}

\begin{Shaded}
\begin{Highlighting}[]
\NormalTok{aic.both }\OtherTok{\textless{}{-}} \FunctionTok{ols\_step\_both\_aic}\NormalTok{(model)}

\FunctionTok{plot}\NormalTok{(aic.both)}
\end{Highlighting}
\end{Shaded}

\includegraphics{StatsTB_files/figure-latex/unnamed-chunk-404-1.pdf}

\hypertarget{example-predicting-tumor-malignancy-from-biopsy-measurements}{%
\subsection{Example: Predicting Tumor Malignancy from Biopsy Measurements}\label{example-predicting-tumor-malignancy-from-biopsy-measurements}}

We use the \texttt{BreastCancer} dataset from the \texttt{mlbench} package. This dataset contains digitized biopsy measurements for breast tumors, with the outcome variable \texttt{Class} indicating whether the tumor is malignant or benign.

Our goal is to build a logistic regression model to predict malignancy based on cell features, and apply stepwise AIC to select relevant variables.

\begin{Shaded}
\begin{Highlighting}[]
\FunctionTok{library}\NormalTok{(mlbench)}
\FunctionTok{library}\NormalTok{(dplyr)}

\FunctionTok{data}\NormalTok{(}\StringTok{"BreastCancer"}\NormalTok{)}

\NormalTok{bc }\OtherTok{\textless{}{-}}\NormalTok{ BreastCancer }\SpecialCharTok{\%\textgreater{}\%}\NormalTok{ dplyr}\SpecialCharTok{::}\FunctionTok{select}\NormalTok{(}\SpecialCharTok{{-}}\NormalTok{Id)}

\NormalTok{bc}\SpecialCharTok{$}\NormalTok{Class }\OtherTok{\textless{}{-}} \FunctionTok{ifelse}\NormalTok{(bc}\SpecialCharTok{$}\NormalTok{Class }\SpecialCharTok{==} \StringTok{"malignant"}\NormalTok{, }\DecValTok{1}\NormalTok{, }\DecValTok{0}\NormalTok{)}

\NormalTok{bc[, }\SpecialCharTok{{-}}\FunctionTok{which}\NormalTok{(}\FunctionTok{names}\NormalTok{(bc) }\SpecialCharTok{==} \StringTok{"Class"}\NormalTok{)] }\OtherTok{\textless{}{-}} \FunctionTok{lapply}\NormalTok{(bc[, }\SpecialCharTok{{-}}\FunctionTok{which}\NormalTok{(}\FunctionTok{names}\NormalTok{(bc) }\SpecialCharTok{==} \StringTok{"Class"}\NormalTok{)], }\ControlFlowTok{function}\NormalTok{(x) }\FunctionTok{as.numeric}\NormalTok{(}\FunctionTok{as.character}\NormalTok{(x)))}

\NormalTok{bc }\OtherTok{\textless{}{-}} \FunctionTok{na.omit}\NormalTok{(bc)}

\FunctionTok{str}\NormalTok{(bc}\SpecialCharTok{$}\NormalTok{Class)}
\end{Highlighting}
\end{Shaded}

\begin{verbatim}
##  num [1:683] 0 0 0 0 0 1 0 0 0 0 ...
\end{verbatim}

\begin{Shaded}
\begin{Highlighting}[]
\FunctionTok{table}\NormalTok{(bc}\SpecialCharTok{$}\NormalTok{Class)}
\end{Highlighting}
\end{Shaded}

\begin{verbatim}
## 
##   0   1 
## 444 239
\end{verbatim}

\textbf{Fit the Full Logistic Regression Model}

\begin{Shaded}
\begin{Highlighting}[]
\NormalTok{model }\OtherTok{\textless{}{-}} \FunctionTok{glm}\NormalTok{(Class }\SpecialCharTok{\textasciitilde{}}\NormalTok{ ., }\AttributeTok{data =}\NormalTok{ bc, }\AttributeTok{family =}\NormalTok{ binomial)}
\FunctionTok{summary}\NormalTok{(model)}
\end{Highlighting}
\end{Shaded}

\begin{verbatim}
## 
## Call:
## glm(formula = Class ~ ., family = binomial, data = bc)
## 
## Coefficients:
##                  Estimate Std. Error z value Pr(>|z|)    
## (Intercept)     -10.10394    1.17488  -8.600  < 2e-16 ***
## Cl.thickness      0.53501    0.14202   3.767 0.000165 ***
## Cell.size        -0.00628    0.20908  -0.030 0.976039    
## Cell.shape        0.32271    0.23060   1.399 0.161688    
## Marg.adhesion     0.33064    0.12345   2.678 0.007400 ** 
## Epith.c.size      0.09663    0.15659   0.617 0.537159    
## Bare.nuclei       0.38303    0.09384   4.082 4.47e-05 ***
## Bl.cromatin       0.44719    0.17138   2.609 0.009073 ** 
## Normal.nucleoli   0.21303    0.11287   1.887 0.059115 .  
## Mitoses           0.53484    0.32877   1.627 0.103788    
## ---
## Signif. codes:  0 '***' 0.001 '**' 0.01 '*' 0.05 '.' 0.1 ' ' 1
## 
## (Dispersion parameter for binomial family taken to be 1)
## 
##     Null deviance: 884.35  on 682  degrees of freedom
## Residual deviance: 102.89  on 673  degrees of freedom
## AIC: 122.89
## 
## Number of Fisher Scoring iterations: 8
\end{verbatim}

\hypertarget{stepwise-aic-variable-selection}{%
\subsubsection{Stepwise AIC Variable Selection}\label{stepwise-aic-variable-selection}}

\begin{Shaded}
\begin{Highlighting}[]
\NormalTok{step\_model }\OtherTok{\textless{}{-}} \FunctionTok{stepAIC}\NormalTok{(model\_full, }\AttributeTok{direction =} \StringTok{"both"}\NormalTok{, }\AttributeTok{trace =} \ConstantTok{TRUE}\NormalTok{)}
\end{Highlighting}
\end{Shaded}

\begin{verbatim}
## Start:  AIC=362.02
## diabetes ~ pregnant + glucose + pressure + triceps + insulin + 
##     mass + pedigree + age
## 
##            Df Deviance    AIC
## - pressure  1   344.04 360.04
## - insulin   1   344.42 360.42
## - triceps   1   344.45 360.45
## <none>          344.02 362.02
## - pregnant  1   346.24 362.24
## - age       1   347.55 363.55
## - mass      1   350.89 366.89
## - pedigree  1   351.58 367.58
## - glucose   1   396.95 412.95
## 
## Step:  AIC=360.04
## diabetes ~ pregnant + glucose + triceps + insulin + mass + pedigree + 
##     age
## 
##            Df Deviance    AIC
## - insulin   1   344.42 358.42
## - triceps   1   344.46 358.46
## <none>          344.04 360.04
## - pregnant  1   346.24 360.24
## - age       1   347.60 361.60
## + pressure  1   344.02 362.02
## - mass      1   351.28 365.28
## - pedigree  1   351.67 365.67
## - glucose   1   397.31 411.31
## 
## Step:  AIC=358.42
## diabetes ~ pregnant + glucose + triceps + mass + pedigree + age
## 
##            Df Deviance    AIC
## - triceps   1   344.89 356.89
## <none>          344.42 358.42
## - pregnant  1   346.74 358.74
## - age       1   347.87 359.87
## + insulin   1   344.04 360.04
## + pressure  1   344.42 360.42
## - mass      1   351.32 363.32
## - pedigree  1   351.90 363.90
## - glucose   1   411.11 423.11
## 
## Step:  AIC=356.89
## diabetes ~ pregnant + glucose + mass + pedigree + age
## 
##            Df Deviance    AIC
## <none>          344.89 356.89
## - pregnant  1   347.23 357.23
## + triceps   1   344.42 358.42
## + insulin   1   344.46 358.46
## - age       1   348.72 358.72
## + pressure  1   344.88 358.88
## - pedigree  1   352.72 362.72
## - mass      1   360.44 370.44
## - glucose   1   411.85 421.85
\end{verbatim}

\begin{Shaded}
\begin{Highlighting}[]
\FunctionTok{summary}\NormalTok{(step\_model)}
\end{Highlighting}
\end{Shaded}

\begin{verbatim}
## 
## Call:
## glm(formula = diabetes ~ pregnant + glucose + mass + pedigree + 
##     age, family = binomial, data = df)
## 
## Coefficients:
##              Estimate Std. Error z value Pr(>|z|)    
## (Intercept) -9.992080   1.086866  -9.193  < 2e-16 ***
## pregnant     0.083953   0.055031   1.526 0.127117    
## glucose      0.036458   0.004978   7.324 2.41e-13 ***
## mass         0.078139   0.020605   3.792 0.000149 ***
## pedigree     1.150913   0.424242   2.713 0.006670 ** 
## age          0.034360   0.017810   1.929 0.053692 .  
## ---
## Signif. codes:  0 '***' 0.001 '**' 0.01 '*' 0.05 '.' 0.1 ' ' 1
## 
## (Dispersion parameter for binomial family taken to be 1)
## 
##     Null deviance: 498.10  on 391  degrees of freedom
## Residual deviance: 344.89  on 386  degrees of freedom
## AIC: 356.89
## 
## Number of Fisher Scoring iterations: 5
\end{verbatim}

\hypertarget{model-diagnostics-and-aic-plot}{%
\subsubsection{Model Diagnostics and AIC Plot}\label{model-diagnostics-and-aic-plot}}

\begin{Shaded}
\begin{Highlighting}[]
\CommentTok{\# Capture stepwise output}
\NormalTok{step\_trace }\OtherTok{\textless{}{-}} \FunctionTok{capture.output}\NormalTok{(}
\NormalTok{  step\_model }\OtherTok{\textless{}{-}} \FunctionTok{stepAIC}\NormalTok{(model\_full, }\AttributeTok{direction =} \StringTok{"both"}\NormalTok{, }\AttributeTok{trace =} \ConstantTok{TRUE}\NormalTok{)}
\NormalTok{)}

\CommentTok{\# Extract AIC values from output}
\NormalTok{aic\_lines }\OtherTok{\textless{}{-}} \FunctionTok{grep}\NormalTok{(}\StringTok{"AIC="}\NormalTok{, step\_trace, }\AttributeTok{value =} \ConstantTok{TRUE}\NormalTok{)}
\NormalTok{aic\_values }\OtherTok{\textless{}{-}} \FunctionTok{as.numeric}\NormalTok{(}\FunctionTok{sub}\NormalTok{(}\StringTok{".*AIC=}\SpecialCharTok{\textbackslash{}\textbackslash{}}\StringTok{s*([0{-}9}\SpecialCharTok{\textbackslash{}\textbackslash{}}\StringTok{.]+).*"}\NormalTok{, }\StringTok{"}\SpecialCharTok{\textbackslash{}\textbackslash{}}\StringTok{1"}\NormalTok{, aic\_lines))}
\NormalTok{steps }\OtherTok{\textless{}{-}} \FunctionTok{seq\_along}\NormalTok{(aic\_values)}

\FunctionTok{library}\NormalTok{(ggplot2)}
\FunctionTok{ggplot}\NormalTok{(}\FunctionTok{data.frame}\NormalTok{(}\AttributeTok{Step =}\NormalTok{ steps, }\AttributeTok{AIC =}\NormalTok{ aic\_values), }\FunctionTok{aes}\NormalTok{(}\AttributeTok{x =}\NormalTok{ Step, }\AttributeTok{y =}\NormalTok{ AIC)) }\SpecialCharTok{+}
  \FunctionTok{geom\_line}\NormalTok{(}\AttributeTok{color =} \StringTok{"blue"}\NormalTok{) }\SpecialCharTok{+}
  \FunctionTok{geom\_point}\NormalTok{(}\AttributeTok{color =} \StringTok{"red"}\NormalTok{) }\SpecialCharTok{+}
  \FunctionTok{labs}\NormalTok{(}\AttributeTok{title =} \StringTok{"Stepwise AIC Selection Process"}\NormalTok{,}
       \AttributeTok{x =} \StringTok{"Step"}\NormalTok{,}
       \AttributeTok{y =} \StringTok{"AIC"}\NormalTok{) }\SpecialCharTok{+}
  \FunctionTok{theme\_minimal}\NormalTok{()}
\end{Highlighting}
\end{Shaded}

\includegraphics{StatsTB_files/figure-latex/unnamed-chunk-408-1.pdf}

\textbf{Interpretation}

The stepwise AIC method iteratively selects predictors to optimize the model's trade-off between goodness-of-fit and complexity. The final logistic regression model provides a parsimonious and interpretable prediction of malignancy status in breast cancer patients.

\hypertarget{logistic-regression}{%
\chapter{Logistic Regression}\label{logistic-regression}}

\begin{Shaded}
\begin{Highlighting}[]
\FunctionTok{library}\NormalTok{(IntroStats)}
\end{Highlighting}
\end{Shaded}

\hypertarget{motivating-example-titanic}{%
\section{Motivating Example: Titanic}\label{motivating-example-titanic}}

On April 15, 1912, the RMS Titanic tragically sank after colliding with an iceberg. Out of 2,224 passengers and crew, more than 1,500 lives were lost. While luck played a role, many factors such as passenger class, gender, and age likely influenced survival outcomes.

This example introduces how logistic regression can be used to analyze such data and estimate the probability of survival using multiple predictors.

Logistic regression is appropriate because the response variable is binary --- passengers either survived (1) or did not (0).

We simulate Titanic-like data below for demonstration.

\hypertarget{preparing-the-dataset}{%
\subsection{Preparing the Dataset}\label{preparing-the-dataset}}

\begin{Shaded}
\begin{Highlighting}[]
\FunctionTok{set.seed}\NormalTok{(}\DecValTok{123}\NormalTok{)}

\CommentTok{\# Simulate Titanic{-}like data}
\NormalTok{n }\OtherTok{\textless{}{-}} \DecValTok{800}
\NormalTok{data }\OtherTok{\textless{}{-}} \FunctionTok{data.frame}\NormalTok{(}
  \AttributeTok{Survived =} \FunctionTok{rbinom}\NormalTok{(n, }\DecValTok{1}\NormalTok{, }\AttributeTok{prob =} \FloatTok{0.4}\NormalTok{),}
  \AttributeTok{Pclass =} \FunctionTok{as.factor}\NormalTok{(}\FunctionTok{sample}\NormalTok{(}\FunctionTok{c}\NormalTok{(}\DecValTok{1}\NormalTok{, }\DecValTok{2}\NormalTok{, }\DecValTok{3}\NormalTok{), n, }\AttributeTok{replace =} \ConstantTok{TRUE}\NormalTok{, }\AttributeTok{prob =} \FunctionTok{c}\NormalTok{(}\FloatTok{0.2}\NormalTok{, }\FloatTok{0.3}\NormalTok{, }\FloatTok{0.5}\NormalTok{))),}
  \AttributeTok{Sex =} \FunctionTok{as.factor}\NormalTok{(}\FunctionTok{sample}\NormalTok{(}\FunctionTok{c}\NormalTok{(}\StringTok{"male"}\NormalTok{, }\StringTok{"female"}\NormalTok{), n, }\AttributeTok{replace =} \ConstantTok{TRUE}\NormalTok{, }\AttributeTok{prob =} \FunctionTok{c}\NormalTok{(}\FloatTok{0.65}\NormalTok{, }\FloatTok{0.35}\NormalTok{))),}
  \AttributeTok{Age =} \FunctionTok{round}\NormalTok{(}\FunctionTok{rnorm}\NormalTok{(n, }\AttributeTok{mean =} \DecValTok{30}\NormalTok{, }\AttributeTok{sd =} \DecValTok{14}\NormalTok{), }\DecValTok{1}\NormalTok{),}
  \AttributeTok{SibSp =} \FunctionTok{sample}\NormalTok{(}\DecValTok{0}\SpecialCharTok{:}\DecValTok{3}\NormalTok{, n, }\AttributeTok{replace =} \ConstantTok{TRUE}\NormalTok{),}
  \AttributeTok{Parch =} \FunctionTok{sample}\NormalTok{(}\DecValTok{0}\SpecialCharTok{:}\DecValTok{2}\NormalTok{, n, }\AttributeTok{replace =} \ConstantTok{TRUE}\NormalTok{),}
  \AttributeTok{Fare =} \FunctionTok{round}\NormalTok{(}\FunctionTok{runif}\NormalTok{(n, }\DecValTok{10}\NormalTok{, }\DecValTok{100}\NormalTok{), }\DecValTok{2}\NormalTok{),}
  \AttributeTok{Embarked =} \FunctionTok{as.factor}\NormalTok{(}\FunctionTok{sample}\NormalTok{(}\FunctionTok{c}\NormalTok{(}\StringTok{"C"}\NormalTok{, }\StringTok{"Q"}\NormalTok{, }\StringTok{"S"}\NormalTok{), n, }\AttributeTok{replace =} \ConstantTok{TRUE}\NormalTok{, }\AttributeTok{prob =} \FunctionTok{c}\NormalTok{(}\FloatTok{0.2}\NormalTok{, }\FloatTok{0.1}\NormalTok{, }\FloatTok{0.7}\NormalTok{)))}
\NormalTok{)}
\end{Highlighting}
\end{Shaded}

\hypertarget{fitting-the-logistic-regression-model}{%
\subsection{Fitting the Logistic Regression Model}\label{fitting-the-logistic-regression-model}}

\begin{Shaded}
\begin{Highlighting}[]
\CommentTok{\# Fit logistic regression model}
\NormalTok{model }\OtherTok{\textless{}{-}} \FunctionTok{glm}\NormalTok{(Survived }\SpecialCharTok{\textasciitilde{}}\NormalTok{ ., }\AttributeTok{family =} \FunctionTok{binomial}\NormalTok{(}\AttributeTok{link =} \StringTok{"logit"}\NormalTok{), }\AttributeTok{data =}\NormalTok{ data)}

\CommentTok{\# Output model summary}
\FunctionTok{summary}\NormalTok{(model)}
\end{Highlighting}
\end{Shaded}

\begin{verbatim}
## 
## Call:
## glm(formula = Survived ~ ., family = binomial(link = "logit"), 
##     data = data)
## 
## Coefficients:
##              Estimate Std. Error z value Pr(>|z|)   
## (Intercept) -1.059118   0.358656  -2.953  0.00315 **
## Pclass2      0.287689   0.216095   1.331  0.18309   
## Pclass3      0.007622   0.201685   0.038  0.96985   
## Sexmale      0.079159   0.153726   0.515  0.60660   
## Age          0.007889   0.005205   1.515  0.12965   
## SibSp        0.029447   0.065466   0.450  0.65285   
## Parch       -0.122154   0.092365  -1.323  0.18600   
## Fare         0.006023   0.002889   2.085  0.03706 * 
## EmbarkedQ   -0.005231   0.286263  -0.018  0.98542   
## EmbarkedS   -0.037091   0.186598  -0.199  0.84244   
## ---
## Signif. codes:  0 '***' 0.001 '**' 0.01 '*' 0.05 '.' 0.1 ' ' 1
## 
## (Dispersion parameter for binomial family taken to be 1)
## 
##     Null deviance: 1071.8  on 799  degrees of freedom
## Residual deviance: 1059.6  on 790  degrees of freedom
## AIC: 1079.6
## 
## Number of Fisher Scoring iterations: 4
\end{verbatim}

\hypertarget{initial-interpretation}{%
\subsection{Initial Interpretation}\label{initial-interpretation}}

The model provides log-odds estimates of survival probability. A positive coefficient suggests higher survival odds, while a negative coefficient indicates reduced odds. Key takeaways:

\begin{itemize}
\tightlist
\item
  Females typically show higher survival odds than males.
\item
  Higher passenger class (1st class) increases survival odds.
\item
  Age and fare may have a moderate effect depending on direction and significance.
\end{itemize}

These results will be examined further in the next sections.

\hypertarget{logistic-regression-1}{%
\section{Logistic Regression}\label{logistic-regression-1}}

Logistic regression is used when the response variable is binary --- meaning it has only two possible outcomes (e.g., survived/did not survive, disease/no disease, success/failure). In this case, we are modeling survival (\texttt{Survived\ =\ 1}) versus non-survival (\texttt{Survived\ =\ 0}) on the Titanic.

Unlike linear regression, which models a numeric outcome, logistic regression models the probability of a binary outcome using the logit link function, which relates the predictors to the log-odds of the event occurring.

\hypertarget{logistic-regression-model-equation}{%
\subsection{Logistic Regression Model Equation}\label{logistic-regression-model-equation}}

The logistic regression model can be written as:

\[
\log\left(\frac{p(x)}{1 - p(x)}\right) = \beta_0 + \beta_1 x_1 + \beta_2 x_2 + \cdots + \beta_p x_p
\]

Where:
- \(p(x)\) is the probability that \(Y = 1\) given predictors \(x\)
- \(\frac{p(x)}{1 - p(x)}\) is the odds of the event
- The left-hand side is the log-odds, called the logit

Solving for \(p(x)\) gives the logistic function:

\[
p(x) = \frac{1}{1 + e^{-(\beta_0 + \beta_1 x_1 + \cdots + \beta_p x_p)}}
\]

This ensures the predicted probabilities are always between 0 and 1.

\hypertarget{model-assumptions-2}{%
\subsection{Model Assumptions}\label{model-assumptions-2}}

Logistic regression relies on a few key assumptions:

\begin{itemize}
\tightlist
\item
  The response variable is binary.
\item
  Observations are independent.
\item
  There is a linear relationship between the log-odds and the predictors.
\item
  No multicollinearity among predictors.
\item
  Large sample size is preferable for stable estimates.
\end{itemize}

\hypertarget{interpretation-of-coefficients}{%
\subsection{Interpretation of Coefficients}\label{interpretation-of-coefficients}}

Each coefficient \(\beta_i\) represents the change in the log-odds of the response variable for a one-unit increase in \(x_i\), holding other variables constant.

\begin{itemize}
\tightlist
\item
  If \(\beta_i > 0\), an increase in \(x_i\) increases the odds of the event (i.e., survival).
\item
  If \(\beta_i < 0\), an increase in \(x_i\) decreases the odds.
\item
  The exponentiated coefficient, \(e^{\beta_i}\), gives the odds ratio for a one-unit increase in \(x_i\).
\end{itemize}

\hypertarget{model-fitting}{%
\section{Model Fitting}\label{model-fitting}}

To fit a logistic regression model, we use the \texttt{glm()} function in R. This function allows us to specify the binomial family and the logit link, which is the default for logistic regression.

We'll fit the model using the simulated Titanic-like dataset from Section 24.1.

\hypertarget{fit-the-model}{%
\subsection{Fit the Model}\label{fit-the-model}}

\begin{Shaded}
\begin{Highlighting}[]
\CommentTok{\# Fit the logistic regression model using all available predictors}
\NormalTok{model }\OtherTok{\textless{}{-}} \FunctionTok{glm}\NormalTok{(Survived }\SpecialCharTok{\textasciitilde{}}\NormalTok{ ., }\AttributeTok{data =}\NormalTok{ data, }\AttributeTok{family =} \FunctionTok{binomial}\NormalTok{(}\AttributeTok{link =} \StringTok{"logit"}\NormalTok{))}

\CommentTok{\# View the model summary}
\FunctionTok{summary}\NormalTok{(model)}
\end{Highlighting}
\end{Shaded}

\begin{verbatim}
## 
## Call:
## glm(formula = Survived ~ ., family = binomial(link = "logit"), 
##     data = data)
## 
## Coefficients:
##              Estimate Std. Error z value Pr(>|z|)   
## (Intercept) -1.059118   0.358656  -2.953  0.00315 **
## Pclass2      0.287689   0.216095   1.331  0.18309   
## Pclass3      0.007622   0.201685   0.038  0.96985   
## Sexmale      0.079159   0.153726   0.515  0.60660   
## Age          0.007889   0.005205   1.515  0.12965   
## SibSp        0.029447   0.065466   0.450  0.65285   
## Parch       -0.122154   0.092365  -1.323  0.18600   
## Fare         0.006023   0.002889   2.085  0.03706 * 
## EmbarkedQ   -0.005231   0.286263  -0.018  0.98542   
## EmbarkedS   -0.037091   0.186598  -0.199  0.84244   
## ---
## Signif. codes:  0 '***' 0.001 '**' 0.01 '*' 0.05 '.' 0.1 ' ' 1
## 
## (Dispersion parameter for binomial family taken to be 1)
## 
##     Null deviance: 1071.8  on 799  degrees of freedom
## Residual deviance: 1059.6  on 790  degrees of freedom
## AIC: 1079.6
## 
## Number of Fisher Scoring iterations: 4
\end{verbatim}

\hypertarget{understanding-the-output}{%
\subsection{Understanding the Output}\label{understanding-the-output}}

The output of \texttt{summary(model)} includes:

\begin{itemize}
\tightlist
\item
  \textbf{Estimate}: The \(\beta\) coefficient for each predictor (on the log-odds scale)
\item
  \textbf{Std. Error}: The standard error of the estimate
\item
  \textbf{z value}: Test statistic for the null hypothesis \(\beta = 0\)
\item
  \textbf{Pr(\textgreater\textbar z\textbar)}: p-value for the hypothesis test
\end{itemize}

If the p-value is small (commonly \textless{} 0.05), the predictor is considered statistically significant.

\hypertarget{model-formula-recap}{%
\subsection{Model Formula Recap}\label{model-formula-recap}}

The model estimates a formula of the form:

\[
\log\left(\frac{p(x)}{1 - p(x)}\right) = \beta_0 + \beta_1 \cdot Pclass + \beta_2 \cdot... + \cdots
\]

Each \(\beta\) coefficient tells us how a one-unit change in the predictor affects the log-odds of survival, holding other variables constant.

\hypertarget{interpretation-of-logistic-regression}{%
\section{Interpretation of Logistic Regression}\label{interpretation-of-logistic-regression}}

Interpreting the results of a logistic regression model involves understanding both the significance of the coefficients and what they imply in terms of odds and probabilities.

\hypertarget{hypothesis-test-for-coefficients-1}{%
\subsection{Hypothesis test for coefficients}\label{hypothesis-test-for-coefficients-1}}

Each coefficient in a logistic regression model is tested using the following hypotheses:

\begin{itemize}
\tightlist
\item
  Null hypothesis (\(H_0\)): The coefficient \(\beta_i = 0\) (the predictor has no effect)
\item
  Alternative hypothesis (\(H_1\)): The coefficient \(\beta_i \ne 0\) (the predictor has an effect)
\end{itemize}

R provides a z-statistic and a p-value for each coefficient. These values are used to determine whether to reject the null hypothesis.

\begin{Shaded}
\begin{Highlighting}[]
\CommentTok{\# View summary of model}
\FunctionTok{summary}\NormalTok{(model)}
\end{Highlighting}
\end{Shaded}

\begin{verbatim}
## 
## Call:
## glm(formula = Survived ~ ., family = binomial(link = "logit"), 
##     data = data)
## 
## Coefficients:
##              Estimate Std. Error z value Pr(>|z|)   
## (Intercept) -1.059118   0.358656  -2.953  0.00315 **
## Pclass2      0.287689   0.216095   1.331  0.18309   
## Pclass3      0.007622   0.201685   0.038  0.96985   
## Sexmale      0.079159   0.153726   0.515  0.60660   
## Age          0.007889   0.005205   1.515  0.12965   
## SibSp        0.029447   0.065466   0.450  0.65285   
## Parch       -0.122154   0.092365  -1.323  0.18600   
## Fare         0.006023   0.002889   2.085  0.03706 * 
## EmbarkedQ   -0.005231   0.286263  -0.018  0.98542   
## EmbarkedS   -0.037091   0.186598  -0.199  0.84244   
## ---
## Signif. codes:  0 '***' 0.001 '**' 0.01 '*' 0.05 '.' 0.1 ' ' 1
## 
## (Dispersion parameter for binomial family taken to be 1)
## 
##     Null deviance: 1071.8  on 799  degrees of freedom
## Residual deviance: 1059.6  on 790  degrees of freedom
## AIC: 1079.6
## 
## Number of Fisher Scoring iterations: 4
\end{verbatim}

In the output:
- A \textbf{small p-value} (typically \textless{} 0.05) indicates strong evidence that the predictor has a statistically significant effect on the outcome.
- A \textbf{large p-value} means we fail to reject the null, suggesting that the predictor may not have a meaningful influence.

Each coefficient is interpreted on the log-odds scale, so additional transformation is needed to understand the change in probability or odds.

\hypertarget{odds-ratio}{%
\subsection{Odds Ratio}\label{odds-ratio}}

To better interpret the effect of predictors, we calculate the odds ratio (OR) by exponentiating each coefficient:

\[
\text{Odds Ratio} = e^{\beta_i}
\]

\begin{itemize}
\tightlist
\item
  If \(\text{OR} > 1\): The predictor increases the odds of the event
\item
  If \(\text{OR} < 1\): The predictor decreases the odds
\item
  If \(\text{OR} = 1\): No effect
\end{itemize}

\begin{Shaded}
\begin{Highlighting}[]
\CommentTok{\# Exponentiate coefficients to get odds ratios}
\FunctionTok{exp}\NormalTok{(}\FunctionTok{coef}\NormalTok{(model))}
\end{Highlighting}
\end{Shaded}

\begin{verbatim}
## (Intercept)     Pclass2     Pclass3     Sexmale         Age       SibSp 
##   0.3467616   1.3333430   1.0076514   1.0823762   1.0079198   1.0298848 
##       Parch        Fare   EmbarkedQ   EmbarkedS 
##   0.8850122   1.0060415   0.9947827   0.9635886
\end{verbatim}

\begin{Shaded}
\begin{Highlighting}[]
\CommentTok{\# Get odds ratios with 95\% confidence intervals}
\FunctionTok{exp}\NormalTok{(}\FunctionTok{cbind}\NormalTok{(}\AttributeTok{OddsRatio =} \FunctionTok{coef}\NormalTok{(model), }\FunctionTok{confint}\NormalTok{(model)))}
\end{Highlighting}
\end{Shaded}

\begin{verbatim}
##             OddsRatio     2.5 %    97.5 %
## (Intercept) 0.3467616 0.1705921 0.6970438
## Pclass2     1.3333430 0.8748481 2.0430533
## Pclass3     1.0076514 0.6803843 1.5017910
## Sexmale     1.0823762 0.8015480 1.4649251
## Age         1.0079198 0.9977158 1.0183067
## SibSp       1.0298848 0.9058287 1.1710573
## Parch       0.8850122 0.7381068 1.0604158
## Fare        1.0060415 1.0003768 1.0117785
## EmbarkedQ   0.9947827 0.5648820 1.7395084
## EmbarkedS   0.9635886 0.6697028 1.3931479
\end{verbatim}

This output allows us to interpret the magnitude and direction of each variable's effect on the outcome.

For example:
- An odds ratio of 2.5 means the odds of survival are 2.5 times greater for that predictor group.
- An odds ratio of 0.40 means the odds are 60\% lower (1 - 0.40 = 0.60).

\hypertarget{example-risk-factors-for-heart-disease}{%
\subsection{Example: Risk Factors for Heart Disease}\label{example-risk-factors-for-heart-disease}}

To demonstrate logistic regression interpretation in a biomedical setting, we'll use a dataset simulating heart disease outcomes based on known risk factors like age, cholesterol, and smoking status.

\begin{Shaded}
\begin{Highlighting}[]
\CommentTok{\# Simulate data for heart disease example}
\FunctionTok{set.seed}\NormalTok{(}\DecValTok{123}\NormalTok{)}
\NormalTok{n }\OtherTok{\textless{}{-}} \DecValTok{300}
\NormalTok{age }\OtherTok{\textless{}{-}} \FunctionTok{rnorm}\NormalTok{(n, }\AttributeTok{mean =} \DecValTok{55}\NormalTok{, }\AttributeTok{sd =} \DecValTok{10}\NormalTok{)}
\NormalTok{cholesterol }\OtherTok{\textless{}{-}} \FunctionTok{rnorm}\NormalTok{(n, }\AttributeTok{mean =} \DecValTok{220}\NormalTok{, }\AttributeTok{sd =} \DecValTok{30}\NormalTok{)}
\NormalTok{smoker }\OtherTok{\textless{}{-}} \FunctionTok{rbinom}\NormalTok{(n, }\DecValTok{1}\NormalTok{, }\FloatTok{0.4}\NormalTok{)}
\NormalTok{heart\_disease }\OtherTok{\textless{}{-}} \FunctionTok{rbinom}\NormalTok{(n, }\DecValTok{1}\NormalTok{, }\FunctionTok{plogis}\NormalTok{(}\SpecialCharTok{{-}}\DecValTok{5} \SpecialCharTok{+} \FloatTok{0.04}\SpecialCharTok{*}\NormalTok{age }\SpecialCharTok{+} \FloatTok{0.02}\SpecialCharTok{*}\NormalTok{cholesterol }\SpecialCharTok{+} \FloatTok{1.2}\SpecialCharTok{*}\NormalTok{smoker))}
\NormalTok{heart\_data }\OtherTok{\textless{}{-}} \FunctionTok{data.frame}\NormalTok{(}\AttributeTok{HeartDisease =}\NormalTok{ heart\_disease, }\AttributeTok{Age =}\NormalTok{ age, }\AttributeTok{Chol =}\NormalTok{ cholesterol, }\AttributeTok{Smoker =}\NormalTok{ smoker)}

\CommentTok{\# Fit the logistic regression model}
\NormalTok{hd\_model }\OtherTok{\textless{}{-}} \FunctionTok{glm}\NormalTok{(HeartDisease }\SpecialCharTok{\textasciitilde{}}\NormalTok{ Age }\SpecialCharTok{+}\NormalTok{ Chol }\SpecialCharTok{+}\NormalTok{ Smoker, }\AttributeTok{data =}\NormalTok{ heart\_data, }\AttributeTok{family =}\NormalTok{ binomial)}

\CommentTok{\# View model summary}
\FunctionTok{summary}\NormalTok{(hd\_model)}
\end{Highlighting}
\end{Shaded}

\begin{verbatim}
## 
## Call:
## glm(formula = HeartDisease ~ Age + Chol + Smoker, family = binomial, 
##     data = heart_data)
## 
## Coefficients:
##              Estimate Std. Error z value Pr(>|z|)   
## (Intercept) -2.962671   1.598880  -1.853  0.06389 . 
## Age          0.026329   0.017857   1.474  0.14036   
## Chol         0.014332   0.005539   2.588  0.00967 **
## Smoker       0.447385   0.351127   1.274  0.20261   
## ---
## Signif. codes:  0 '***' 0.001 '**' 0.01 '*' 0.05 '.' 0.1 ' ' 1
## 
## (Dispersion parameter for binomial family taken to be 1)
## 
##     Null deviance: 253.63  on 299  degrees of freedom
## Residual deviance: 243.48  on 296  degrees of freedom
## AIC: 251.48
## 
## Number of Fisher Scoring iterations: 4
\end{verbatim}

\begin{Shaded}
\begin{Highlighting}[]
\CommentTok{\# Odds ratios and 95\% confidence intervals}
\FunctionTok{exp}\NormalTok{(}\FunctionTok{cbind}\NormalTok{(}\AttributeTok{OddsRatio =} \FunctionTok{coef}\NormalTok{(hd\_model), }\FunctionTok{confint}\NormalTok{(hd\_model)))}
\end{Highlighting}
\end{Shaded}

\begin{verbatim}
##              OddsRatio       2.5 %   97.5 %
## (Intercept) 0.05168069 0.002141997 1.162709
## Age         1.02667872 0.992029669 1.064243
## Chol        1.01443470 1.003590618 1.025713
## Smoker      1.56421576 0.799273788 3.195457
\end{verbatim}

\hypertarget{interpretation-21}{%
\subsubsection{Interpretation:}\label{interpretation-21}}

\begin{itemize}
\tightlist
\item
  The coefficient for Smoker is positive and significant. The odds ratio is greater than 1, indicating that smokers are more likely to develop heart disease.
\item
  The Age and Cholesterol variables also have positive coefficients, meaning higher age and cholesterol are associated with increased odds of heart disease.
\item
  These results align with established biomedical knowledge and demonstrate the usefulness of logistic regression in identifying significant health risk factors.
\end{itemize}

\hypertarget{prediction}{%
\section{Prediction}\label{prediction}}

Once a logistic regression model is fit, it can be used to make predictions about new or existing data. In logistic regression, predictions are made in terms of probabilities between 0 and 1. These probabilities represent the likelihood that the outcome equals 1 (e.g., survival, disease, success).

We use the \texttt{predict()} function in R to generate:
- Fitted probabilities (default type = ``link'' or type = ``response'')
- Binary class predictions by applying a cutoff threshold (typically 0.5)

We'll demonstrate prediction using the \texttt{hd\_model} from Section 24.4.3, which modeled heart disease risk based on age, cholesterol, and smoking status.

\begin{Shaded}
\begin{Highlighting}[]
\CommentTok{\# Generate predicted probabilities}
\NormalTok{predicted\_probs }\OtherTok{\textless{}{-}} \FunctionTok{predict}\NormalTok{(hd\_model, }\AttributeTok{type =} \StringTok{"response"}\NormalTok{)}

\CommentTok{\# View first few predicted probabilities}
\FunctionTok{head}\NormalTok{(predicted\_probs)}
\end{Highlighting}
\end{Shaded}

\begin{verbatim}
##         1         2         3         4         5         6 
## 0.8362600 0.8457337 0.8382454 0.8391320 0.8152428 0.9358176
\end{verbatim}

Each value represents the predicted probability that a patient has heart disease given their risk factors.

\hypertarget{converting-probabilities-to-class-predictions}{%
\subsection{Converting probabilities to class predictions}\label{converting-probabilities-to-class-predictions}}

By default, a cutoff value of 0.5 is often used:
- If predicted probability ≥ 0.5 → classify as 1 (heart disease)
- If predicted probability \textless{} 0.5 → classify as 0 (no heart disease)

\begin{Shaded}
\begin{Highlighting}[]
\CommentTok{\# Convert probabilities to binary predictions using 0.5 cutoff}
\NormalTok{predicted\_classes }\OtherTok{\textless{}{-}} \FunctionTok{ifelse}\NormalTok{(predicted\_probs }\SpecialCharTok{\textgreater{}=} \FloatTok{0.5}\NormalTok{, }\DecValTok{1}\NormalTok{, }\DecValTok{0}\NormalTok{)}

\CommentTok{\# View classification results}
\FunctionTok{head}\NormalTok{(predicted\_classes)}
\end{Highlighting}
\end{Shaded}

\begin{verbatim}
## 1 2 3 4 5 6 
## 1 1 1 1 1 1
\end{verbatim}

\hypertarget{evaluating-prediction-accuracy}{%
\subsection{Evaluating prediction accuracy}\label{evaluating-prediction-accuracy}}

We can compare the predicted classes with the actual outcomes to compute model accuracy:

\begin{Shaded}
\begin{Highlighting}[]
\CommentTok{\# Compare predictions to actual outcomes}
\NormalTok{actual }\OtherTok{\textless{}{-}}\NormalTok{ heart\_data}\SpecialCharTok{$}\NormalTok{HeartDisease}

\CommentTok{\# Create confusion matrix}
\FunctionTok{table}\NormalTok{(}\AttributeTok{Predicted =}\NormalTok{ predicted\_classes, }\AttributeTok{Actual =}\NormalTok{ actual)}
\end{Highlighting}
\end{Shaded}

\begin{verbatim}
##          Actual
## Predicted   0   1
##         1  45 255
\end{verbatim}

\begin{Shaded}
\begin{Highlighting}[]
\CommentTok{\# Calculate prediction accuracy}
\NormalTok{accuracy }\OtherTok{\textless{}{-}} \FunctionTok{mean}\NormalTok{(predicted\_classes }\SpecialCharTok{==}\NormalTok{ actual)}
\NormalTok{accuracy}
\end{Highlighting}
\end{Shaded}

\begin{verbatim}
## [1] 0.85
\end{verbatim}

This gives the overall accuracy --- the proportion of correctly predicted outcomes. For more thorough evaluation (e.g., sensitivity/specificity), we can use ROC curves.

\hypertarget{performance-assesment-using-roc-curve}{%
\section{Performance assesment using ROC Curve}\label{performance-assesment-using-roc-curve}}

An ROC (Receiver Operating Characteristic) curve is a visual tool used to evaluate the performance of a binary classification model. It plots the true positive rate (sensitivity) against the false positive rate (1 - specificity) across a range of threshold values.

A good classifier will have a curve that bows toward the top-left corner, indicating high sensitivity and low false positive rate.

We use the \texttt{pROC} package to create the ROC curve and compute the AUC (Area Under the Curve), which summarizes the model's overall ability to discriminate between classes.

\begin{Shaded}
\begin{Highlighting}[]
\CommentTok{\# Load package}
\FunctionTok{library}\NormalTok{(pROC)}

\CommentTok{\# Predict probabilities from logistic regression model}
\NormalTok{probs\_hd }\OtherTok{\textless{}{-}} \FunctionTok{predict}\NormalTok{(hd\_model, }\AttributeTok{type =} \StringTok{"response"}\NormalTok{)}

\CommentTok{\# Create ROC object}
\NormalTok{roc\_hd }\OtherTok{\textless{}{-}} \FunctionTok{roc}\NormalTok{(heart\_data}\SpecialCharTok{$}\NormalTok{HeartDisease, probs\_hd)}

\CommentTok{\# Plot the ROC curve}
\FunctionTok{plot}\NormalTok{(roc\_hd, }\AttributeTok{col =} \StringTok{"blue"}\NormalTok{, }\AttributeTok{main =} \StringTok{"ROC Curve: Heart Disease Prediction"}\NormalTok{)}
\end{Highlighting}
\end{Shaded}

\includegraphics{StatsTB_files/figure-latex/unnamed-chunk-419-1.pdf}

The ROC curve gives a complete view of how the model performs at all thresholds --- not just at the default cutoff of 0.5.

\hypertarget{area-under-the-curve}{%
\subsection{Area Under the Curve}\label{area-under-the-curve}}

The AUC is a number between 0 and 1 that quantifies how well the model distinguishes between classes.

\begin{itemize}
\tightlist
\item
  \textbf{AUC = 1}: perfect classification\\
\item
  \textbf{AUC = 0.5}: no better than random guessing\\
\item
  \textbf{Higher AUC values} indicate better discriminatory power
\end{itemize}

\begin{Shaded}
\begin{Highlighting}[]
\CommentTok{\# Compute AUC}
\FunctionTok{auc}\NormalTok{(roc\_hd)}
\end{Highlighting}
\end{Shaded}

\begin{verbatim}
## Area under the curve: 0.6478
\end{verbatim}

\hypertarget{interpretation-22}{%
\subsubsection{Interpretation:}\label{interpretation-22}}

If AUC = 0.84, it means there's an 84\% chance the model will correctly distinguish a randomly chosen positive case from a negative one.

In a biomedical context, this means our model is reasonably accurate at predicting who is at risk of heart disease based on the included risk factors (age, cholesterol, smoker status). However, further clinical validation is always important before using such a model in practice.

\hypertarget{survival-analysis}{%
\chapter{Survival Analysis}\label{survival-analysis}}

\begin{Shaded}
\begin{Highlighting}[]
\FunctionTok{library}\NormalTok{(IntroStats)}
\end{Highlighting}
\end{Shaded}

\hypertarget{examples-from-cancer}{%
\section{Examples from Cancer}\label{examples-from-cancer}}

Survival analysis is a statistical approach used to analyze time-to-event data --- the time until a specified event such as death, relapse, device failure, or recovery occurs. In clinical research, especially oncology, this type of analysis is essential to evaluate treatment effectiveness and patient prognosis.

Common examples of time-to-event outcomes include:

\begin{itemize}
\tightlist
\item
  Time from surgery to death\\
\item
  Time from start of treatment to cancer progression\\
\item
  Time from tumor remission to recurrence\\
\item
  Time from HIV infection to development of AIDS\\
\item
  Time to onset of substance use relapse\\
\item
  Time to mechanical failure of a medical device
\end{itemize}

One crucial concept in survival data is censoring. This occurs when the exact event time is unknown for a subject. Reasons for censoring include:

\begin{itemize}
\tightlist
\item
  The subject withdrew from the study before the event occurred\\
\item
  The subject was lost to follow-up\\
\item
  The subject did not experience the event before the study ended
\end{itemize}

In all of the above, we know that the subject survived at least until a certain point, but not beyond.

These are examples of right censoring, the most common type in biomedical studies. Right censoring means we only observe the subject up to a certain time, after which the event status is unknown.

\hypertarget{example-cancer-patients-in-a-chemotherapy-trial}{%
\subsection{Example: Cancer Patients in a Chemotherapy Trial}\label{example-cancer-patients-in-a-chemotherapy-trial}}

Suppose we enroll 100 cancer patients undergoing a new chemotherapy treatment. For each patient, we record the number of days from the start of treatment until death or censoring.

\begin{itemize}
\tightlist
\item
  Patient A dies on Day 245 → event observed\\
\item
  Patient B is alive at last follow-up on Day 400 → censored\\
\item
  Patient C withdraws on Day 160 due to side effects → censored
\end{itemize}

This setup allows us to compare survival patterns across subgroups --- for instance, by tumor stage or biomarker expression --- even if not all patients experience the event during the study period.

\hypertarget{example-non-medical-application}{%
\subsection{Example: Non-Medical Application}\label{example-non-medical-application}}

Survival analysis is not limited to healthcare. In engineering:

\begin{itemize}
\tightlist
\item
  Time to mechanical failure of an aircraft engine\\
\item
  Time to battery failure in a pacemaker\\
\item
  Time from software deployment to system crash
\end{itemize}

All of these involve censored observations and are analyzed using the same survival models.

Understanding how to handle censoring properly is essential --- naive estimates ignoring censoring overestimate survival probabilities and underestimate risks.

\hypertarget{the-lung-dataset-from-survival-package}{%
\section{The Lung Dataset from \{survival\} Package}\label{the-lung-dataset-from-survival-package}}

To explore survival analysis using real biomedical data, we will use the \texttt{lung} dataset from the \texttt{\{survival\}} package in R. This dataset includes information on patients with advanced lung cancer who participated in a clinical trial conducted by the North Central Cancer Treatment Group.

\hypertarget{overview-of-key-variables}{%
\subsection{Overview of Key Variables:}\label{overview-of-key-variables}}

\begin{itemize}
\tightlist
\item
  \texttt{time}: Observed survival time in days\\
\item
  \texttt{status}: Censoring status

  \begin{itemize}
  \tightlist
  \item
    1 = event (death)\\
  \item
    0 = censored (alive at last follow-up or lost to follow-up)\\
  \end{itemize}
\item
  \texttt{sex}: 1 = Male, 2 = Female\\
\item
  \texttt{age}: Age in years\\
\item
  \texttt{ph.ecog}: ECOG performance status (functional ability)
\end{itemize}

Note: The original dataset codes \texttt{status} as 1 = censored, 2 = dead. This is non-standard. Most survival functions expect 1 = event, 0 = censored. So we recode it accordingly.

\begin{Shaded}
\begin{Highlighting}[]
\FunctionTok{library}\NormalTok{(survival)}
\FunctionTok{library}\NormalTok{(dplyr)}


\NormalTok{lung }\OtherTok{\textless{}{-}}\NormalTok{ survival}\SpecialCharTok{::}\NormalTok{lung}

\NormalTok{lung }\OtherTok{\textless{}{-}}\NormalTok{ lung }\SpecialCharTok{\%\textgreater{}\%}
  \FunctionTok{mutate}\NormalTok{(}\AttributeTok{status =}\NormalTok{ dplyr}\SpecialCharTok{::}\FunctionTok{recode}\NormalTok{(status, }\StringTok{\textasciigrave{}}\AttributeTok{1}\StringTok{\textasciigrave{}} \OtherTok{=}\NormalTok{ 0L, }\StringTok{\textasciigrave{}}\AttributeTok{2}\StringTok{\textasciigrave{}} \OtherTok{=}\NormalTok{ 1L))}

\FunctionTok{head}\NormalTok{(lung[, }\FunctionTok{c}\NormalTok{(}\StringTok{"time"}\NormalTok{, }\StringTok{"status"}\NormalTok{, }\StringTok{"sex"}\NormalTok{, }\StringTok{"age"}\NormalTok{, }\StringTok{"ph.ecog"}\NormalTok{)])}
\end{Highlighting}
\end{Shaded}

\begin{verbatim}
##   time status sex age ph.ecog
## 1  306      1   1  74       1
## 2  455      1   1  68       0
## 3 1010      0   1  56       0
## 4  210      1   1  57       1
## 5  883      1   1  60       0
## 6 1022      0   1  74       1
\end{verbatim}

After recoding:

\begin{itemize}
\tightlist
\item
  \texttt{status\ =\ 0} → the patient is censored\\
\item
  \texttt{status\ =\ 1} → the patient died
\end{itemize}

This step ensures compatibility with functions like \texttt{Surv()} that require properly formatted event indicators.

\hypertarget{why-this-dataset}{%
\subsection{Why This Dataset?}\label{why-this-dataset}}

This dataset is commonly used in survival analysis teaching and benchmarking because it has:

\begin{itemize}
\tightlist
\item
  Real clinical trial structure\\
\item
  Right-censored outcomes\\
\item
  Stratifiable variables like sex, ECOG status, and age
\end{itemize}

\hypertarget{biomedical-context}{%
\subsection{Biomedical Context:}\label{biomedical-context}}

Patients with advanced lung cancer often undergo survival tracking to assess treatment efficacy. In this dataset:

\begin{itemize}
\tightlist
\item
  Some patients have complete follow-up (death recorded)\\
\item
  Others are censored due to withdrawal or loss to follow-up\\
\item
  Variables such as ECOG score or sex can help stratify survival probabilities
\end{itemize}

\hypertarget{example-interpretation}{%
\subsection{Example Interpretation:}\label{example-interpretation}}

\begin{Shaded}
\begin{Highlighting}[]
\FunctionTok{table}\NormalTok{(lung}\SpecialCharTok{$}\NormalTok{status)}
\end{Highlighting}
\end{Shaded}

\begin{verbatim}
## 
##   0   1 
##  63 165
\end{verbatim}

This output shows how many patients experienced the event (death) versus how many were censored.

\begin{Shaded}
\begin{Highlighting}[]
\FunctionTok{table}\NormalTok{(lung}\SpecialCharTok{$}\NormalTok{sex)}
\end{Highlighting}
\end{Shaded}

\begin{verbatim}
## 
##   1   2 
## 138  90
\end{verbatim}

Gives the gender distribution in the study, which can be used later for stratified Kaplan-Meier curves.

\hypertarget{calculating-survival-times}{%
\section{Calculating Survival Times}\label{calculating-survival-times}}

In many real-world biomedical datasets, we do not receive survival time directly. Instead, we are given start dates (e.g., diagnosis or surgery) and end dates (e.g., death or last follow-up). In such cases, the first step in survival analysis is calculating the time interval between those two dates.

To handle date arithmetic in R, we use the \texttt{\{lubridate\}} package, which provides clean syntax for manipulating and comparing dates.

\hypertarget{formatting-character-dates}{%
\subsection{Formatting Character Dates}\label{formatting-character-dates}}

Suppose we are given raw character variables representing the surgery date (\texttt{sx\_date}) and last follow-up date (\texttt{last\_fup\_date}). We begin by converting them into proper date format.

\begin{Shaded}
\begin{Highlighting}[]
\FunctionTok{library}\NormalTok{(tibble)}
\FunctionTok{library}\NormalTok{(lubridate)}

\NormalTok{date\_ex }\OtherTok{\textless{}{-}} \FunctionTok{tibble}\NormalTok{(}
  \AttributeTok{sx\_date =} \FunctionTok{c}\NormalTok{(}\StringTok{"2007{-}06{-}22"}\NormalTok{, }\StringTok{"2004{-}02{-}13"}\NormalTok{, }\StringTok{"2010{-}10{-}27"}\NormalTok{), }
  \AttributeTok{last\_fup\_date =} \FunctionTok{c}\NormalTok{(}\StringTok{"2017{-}04{-}15"}\NormalTok{, }\StringTok{"2018{-}07{-}04"}\NormalTok{, }\StringTok{"2016{-}10{-}31"}\NormalTok{)}
\NormalTok{)}

\CommentTok{\# Convert character to Date format using ymd()}
\NormalTok{date\_ex }\OtherTok{\textless{}{-}}\NormalTok{ date\_ex }\SpecialCharTok{\%\textgreater{}\%}
  \FunctionTok{mutate}\NormalTok{(}
    \AttributeTok{sx\_date =} \FunctionTok{ymd}\NormalTok{(sx\_date),}
    \AttributeTok{last\_fup\_date =} \FunctionTok{ymd}\NormalTok{(last\_fup\_date)}
\NormalTok{  )}

\NormalTok{date\_ex}
\end{Highlighting}
\end{Shaded}

\begin{verbatim}
## # A tibble: 3 x 2
##   sx_date    last_fup_date
##   <date>     <date>       
## 1 2007-06-22 2017-04-15   
## 2 2004-02-13 2018-07-04   
## 3 2010-10-27 2016-10-31
\end{verbatim}

Now both \texttt{sx\_date} and \texttt{last\_fup\_date} are recognized as Date objects.

\hypertarget{calculating-overall-survival-in-years}{%
\subsection{Calculating Overall Survival in Years}\label{calculating-overall-survival-in-years}}

We can now compute the time interval between surgery and last follow-up. Using \texttt{as.duration()} and \texttt{dyears(1)} from \texttt{\{lubridate\}}, we convert the date interval to years of follow-up.

\begin{Shaded}
\begin{Highlighting}[]
\NormalTok{date\_ex }\OtherTok{\textless{}{-}}\NormalTok{ date\_ex }\SpecialCharTok{\%\textgreater{}\%}
  \FunctionTok{mutate}\NormalTok{(}
    \AttributeTok{os\_yrs =} \FunctionTok{as.duration}\NormalTok{(sx\_date }\SpecialCharTok{\%{-}{-}\%}\NormalTok{ last\_fup\_date) }\SpecialCharTok{/} \FunctionTok{dyears}\NormalTok{(}\DecValTok{1}\NormalTok{)}
\NormalTok{  )}

\NormalTok{date\_ex}
\end{Highlighting}
\end{Shaded}

\begin{verbatim}
## # A tibble: 3 x 3
##   sx_date    last_fup_date os_yrs
##   <date>     <date>         <dbl>
## 1 2007-06-22 2017-04-15      9.82
## 2 2004-02-13 2018-07-04     14.4 
## 3 2010-10-27 2016-10-31      6.01
\end{verbatim}

The \texttt{os\_yrs} column now shows observed survival in years, accounting for exact time elapsed.

\hypertarget{adding-status-information}{%
\subsection{Adding Status Information}\label{adding-status-information}}

In order to conduct survival analysis, we need a binary variable indicating whether the patient experienced the event (1) or was censored (0). Let's add a simple status column to our dataset:

\begin{Shaded}
\begin{Highlighting}[]
\NormalTok{date\_ex }\OtherTok{\textless{}{-}}\NormalTok{ date\_ex }\SpecialCharTok{\%\textgreater{}\%}
  \FunctionTok{mutate}\NormalTok{(}\AttributeTok{status =} \FunctionTok{c}\NormalTok{(}\DecValTok{1}\NormalTok{, }\DecValTok{0}\NormalTok{, }\DecValTok{1}\NormalTok{))  }\CommentTok{\# 1 = event (e.g., death), 0 = censored}

\NormalTok{date\_ex}
\end{Highlighting}
\end{Shaded}

\begin{verbatim}
## # A tibble: 3 x 4
##   sx_date    last_fup_date os_yrs status
##   <date>     <date>         <dbl>  <dbl>
## 1 2007-06-22 2017-04-15      9.82      1
## 2 2004-02-13 2018-07-04     14.4       0
## 3 2010-10-27 2016-10-31      6.01      1
\end{verbatim}

This structure is now ready for survival modeling using the \texttt{Surv()} function.

\hypertarget{creating-survival-object}{%
\subsection{Creating Survival Object}\label{creating-survival-object}}

\begin{Shaded}
\begin{Highlighting}[]
\FunctionTok{library}\NormalTok{(survival)}

\CommentTok{\# Build survival object using time in years}
\FunctionTok{Surv}\NormalTok{(}\AttributeTok{time =}\NormalTok{ date\_ex}\SpecialCharTok{$}\NormalTok{os\_yrs, }\AttributeTok{event =}\NormalTok{ date\_ex}\SpecialCharTok{$}\NormalTok{status)}
\end{Highlighting}
\end{Shaded}

\begin{verbatim}
## [1]  9.815195  14.387406+  6.012320
\end{verbatim}

This output displays:
- The time each patient was followed\\
- A \texttt{+} symbol next to censored subjects\\
- Numeric time without a \texttt{+} means the event occurred

\hypertarget{original-biomedical-example-early-onset-alzheimers-study}{%
\subsection{Original Biomedical Example: Early-Onset Alzheimer's Study}\label{original-biomedical-example-early-onset-alzheimers-study}}

Let's say we are analyzing a small study on early-onset Alzheimer's patients. Researchers are tracking how long it takes from diagnosis to institutionalization (or last follow-up if censored).

\begin{Shaded}
\begin{Highlighting}[]
\NormalTok{alz\_data }\OtherTok{\textless{}{-}} \FunctionTok{tibble}\NormalTok{(}
  \AttributeTok{patient\_id =} \DecValTok{1}\SpecialCharTok{:}\DecValTok{4}\NormalTok{,}
  \AttributeTok{diagnosis\_date =} \FunctionTok{c}\NormalTok{(}\StringTok{"2018{-}05{-}10"}\NormalTok{, }\StringTok{"2019{-}01{-}15"}\NormalTok{, }\StringTok{"2020{-}03{-}01"}\NormalTok{, }\StringTok{"2021{-}07{-}20"}\NormalTok{),}
  \AttributeTok{last\_fup\_date =} \FunctionTok{c}\NormalTok{(}\StringTok{"2023{-}03{-}30"}\NormalTok{, }\StringTok{"2021{-}09{-}01"}\NormalTok{, }\StringTok{"2021{-}08{-}10"}\NormalTok{, }\StringTok{"2024{-}06{-}15"}\NormalTok{),}
  \AttributeTok{status =} \FunctionTok{c}\NormalTok{(}\DecValTok{1}\NormalTok{, }\DecValTok{1}\NormalTok{, }\DecValTok{0}\NormalTok{, }\DecValTok{0}\NormalTok{)  }\CommentTok{\# 1 = institutionalized, 0 = censored}
\NormalTok{) }\SpecialCharTok{\%\textgreater{}\%}
  \FunctionTok{mutate}\NormalTok{(}
    \AttributeTok{diagnosis\_date =} \FunctionTok{ymd}\NormalTok{(diagnosis\_date),}
    \AttributeTok{last\_fup\_date =} \FunctionTok{ymd}\NormalTok{(last\_fup\_date),}
    \AttributeTok{os\_months =} \FunctionTok{as.numeric}\NormalTok{(}\FunctionTok{interval}\NormalTok{(diagnosis\_date, last\_fup\_date) }\SpecialCharTok{/} \FunctionTok{months}\NormalTok{(}\DecValTok{1}\NormalTok{))}
\NormalTok{  )}

\NormalTok{alz\_data}
\end{Highlighting}
\end{Shaded}

\begin{verbatim}
## # A tibble: 4 x 5
##   patient_id diagnosis_date last_fup_date status os_months
##        <int> <date>         <date>         <dbl>     <dbl>
## 1          1 2018-05-10     2023-03-30         1      58.6
## 2          2 2019-01-15     2021-09-01         1      31.5
## 3          3 2020-03-01     2021-08-10         0      17.3
## 4          4 2021-07-20     2024-06-15         0      34.8
\end{verbatim}

In this biomedical example:
- Patients with \texttt{status\ =\ 1} were institutionalized (event)
- Patients with \texttt{status\ =\ 0} were still at home or lost to follow-up (censored)
- \texttt{os\_months} gives follow-up time in months

This format mirrors how real studies track progressive diseases where not all participants reach the outcome before the study ends.

\hypertarget{creating-survival-objects-and-curves}{%
\section{Creating Survival Objects and Curves}\label{creating-survival-objects-and-curves}}

The first step in modeling time-to-event data in R is to use the \texttt{Surv()} function from the \texttt{\{survival\}} package. This function creates a special object that stores both survival time and event status. It can be used as the response in survival models or to generate Kaplan-Meier curves.

\hypertarget{creating-a-survival-object}{%
\subsection{Creating a Survival Object}\label{creating-a-survival-object}}

We'll begin by creating a survival object from the lung cancer dataset:

\begin{Shaded}
\begin{Highlighting}[]
\FunctionTok{library}\NormalTok{(survival)}

\CommentTok{\# Build the survival object}
\NormalTok{lung\_surv }\OtherTok{\textless{}{-}} \FunctionTok{Surv}\NormalTok{(}\AttributeTok{time =}\NormalTok{ lung}\SpecialCharTok{$}\NormalTok{time, }\AttributeTok{event =}\NormalTok{ lung}\SpecialCharTok{$}\NormalTok{status)}

\CommentTok{\# View first 10 observations}
\NormalTok{lung\_surv[}\DecValTok{1}\SpecialCharTok{:}\DecValTok{10}\NormalTok{]}
\end{Highlighting}
\end{Shaded}

\begin{verbatim}
##  [1]  306   455  1010+  210   883  1022+  310   361   218   166
\end{verbatim}

Each row shows:
- Survival time (in days)
- A \texttt{+} next to censored observations
- No \texttt{+} means the event (death) occurred

This object is required for all downstream survival modeling in R.

\hypertarget{kaplan-meier-estimation}{%
\subsection{Kaplan-Meier Estimation}\label{kaplan-meier-estimation}}

The Kaplan-Meier method estimates survival probability over time without making assumptions about the underlying distribution. It uses the \texttt{survfit()} function:

\begin{Shaded}
\begin{Highlighting}[]
\CommentTok{\# Fit an overall survival curve}
\NormalTok{s1 }\OtherTok{\textless{}{-}} \FunctionTok{survfit}\NormalTok{(lung\_surv }\SpecialCharTok{\textasciitilde{}} \DecValTok{1}\NormalTok{, }\AttributeTok{data =}\NormalTok{ lung)}

\CommentTok{\# Examine structure of survfit object}
\FunctionTok{str}\NormalTok{(s1)}
\end{Highlighting}
\end{Shaded}

\begin{verbatim}
## List of 17
##  $ n        : int 228
##  $ time     : num [1:186] 5 11 12 13 15 26 30 31 53 54 ...
##  $ n.risk   : num [1:186] 228 227 224 223 221 220 219 218 217 215 ...
##  $ n.event  : num [1:186] 1 3 1 2 1 1 1 1 2 1 ...
##  $ n.censor : num [1:186] 0 0 0 0 0 0 0 0 0 0 ...
##  $ surv     : num [1:186] 0.996 0.982 0.978 0.969 0.965 ...
##  $ std.err  : num [1:186] 0.0044 0.00885 0.00992 0.01179 0.01263 ...
##  $ cumhaz   : num [1:186] 0.00439 0.0176 0.02207 0.03103 0.03556 ...
##  $ std.chaz : num [1:186] 0.00439 0.0088 0.00987 0.01173 0.01257 ...
##  $ type     : chr "right"
##  $ logse    : logi TRUE
##  $ conf.int : num 0.95
##  $ conf.type: chr "log"
##  $ lower    : num [1:186] 0.987 0.966 0.959 0.947 0.941 ...
##  $ upper    : num [1:186] 1 1 0.997 0.992 0.989 ...
##  $ t0       : num 0
##  $ call     : language survfit(formula = lung_surv ~ 1, data = lung)
##  - attr(*, "class")= chr "survfit"
\end{verbatim}

The key outputs include:
- \texttt{time}: days at which an event occurred
- \texttt{n.risk}: number still at risk just before each event
- \texttt{surv}: estimated survival probability at each time point

This model gives a stepwise estimate of survival probability, decreasing each time a death occurs.

\hypertarget{original-biomedical-example-survival-after-heart-transplant}{%
\subsection{Original Biomedical Example: Survival After Heart Transplant}\label{original-biomedical-example-survival-after-heart-transplant}}

Suppose we are analyzing data from 5 patients who received heart transplants. We track the number of days from transplant to either death or last known follow-up.

\begin{Shaded}
\begin{Highlighting}[]
\NormalTok{ht\_data }\OtherTok{\textless{}{-}} \FunctionTok{tibble}\NormalTok{(}
  \AttributeTok{days\_post\_tx =} \FunctionTok{c}\NormalTok{(}\DecValTok{90}\NormalTok{, }\DecValTok{300}\NormalTok{, }\DecValTok{450}\NormalTok{, }\DecValTok{120}\NormalTok{, }\DecValTok{620}\NormalTok{),}
  \AttributeTok{status =} \FunctionTok{c}\NormalTok{(}\DecValTok{1}\NormalTok{, }\DecValTok{0}\NormalTok{, }\DecValTok{1}\NormalTok{, }\DecValTok{1}\NormalTok{, }\DecValTok{0}\NormalTok{)  }\CommentTok{\# 1 = died, 0 = censored}
\NormalTok{)}

\CommentTok{\# Create survival object}
\NormalTok{ht\_surv }\OtherTok{\textless{}{-}} \FunctionTok{Surv}\NormalTok{(}\AttributeTok{time =}\NormalTok{ ht\_data}\SpecialCharTok{$}\NormalTok{days\_post\_tx, }\AttributeTok{event =}\NormalTok{ ht\_data}\SpecialCharTok{$}\NormalTok{status)}

\CommentTok{\# View survival object}
\NormalTok{ht\_surv}
\end{Highlighting}
\end{Shaded}

\begin{verbatim}
## [1]  90  300+ 450  120  620+
\end{verbatim}

We can now fit a Kaplan-Meier survival curve for this dataset:

\begin{Shaded}
\begin{Highlighting}[]
\NormalTok{ht\_fit }\OtherTok{\textless{}{-}} \FunctionTok{survfit}\NormalTok{(ht\_surv }\SpecialCharTok{\textasciitilde{}} \DecValTok{1}\NormalTok{)}

\CommentTok{\# Plot using base R}
\FunctionTok{plot}\NormalTok{(ht\_fit,}
     \AttributeTok{xlab =} \StringTok{"Days Post{-}Transplant"}\NormalTok{,}
     \AttributeTok{ylab =} \StringTok{"Survival Probability"}\NormalTok{,}
     \AttributeTok{main =} \StringTok{"Kaplan{-}Meier Curve: Heart Transplant Patients"}\NormalTok{,}
     \AttributeTok{col =} \StringTok{"darkgreen"}\NormalTok{, }\AttributeTok{lwd =} \DecValTok{2}\NormalTok{)}
\end{Highlighting}
\end{Shaded}

\includegraphics{StatsTB_files/figure-latex/unnamed-chunk-433-1.pdf}

\textbf{Interpretation}: The curve steps downward at each observed death. Flat segments represent periods where no events occurred. The final point shows the survival probability at the last recorded event.

This example reflects how transplant patients are monitored for survival outcomes, and censored if they are still alive at last follow-up.

\hypertarget{kaplan-meier-plots}{%
\section{Kaplan-Meier Plots}\label{kaplan-meier-plots}}

Once a Kaplan-Meier model is fit using \texttt{survfit()} or \texttt{survfit2()}, the next step is to visualize the survival curves. In modern R practice, we use the \texttt{\{ggsurvfit\}} package to produce customizable, publication-quality plots.

\hypertarget{basic-kaplan-meier-plot}{%
\subsection{Basic Kaplan-Meier Plot}\label{basic-kaplan-meier-plot}}

To plot an overall survival curve from the lung dataset:

\begin{Shaded}
\begin{Highlighting}[]
\FunctionTok{library}\NormalTok{(ggsurvfit)}

\CommentTok{\# Overall survival curve using ggsurvfit}
\FunctionTok{survfit2}\NormalTok{(}\FunctionTok{Surv}\NormalTok{(time, status) }\SpecialCharTok{\textasciitilde{}} \DecValTok{1}\NormalTok{, }\AttributeTok{data =}\NormalTok{ lung) }\SpecialCharTok{\%\textgreater{}\%}
  \FunctionTok{ggsurvfit}\NormalTok{() }\SpecialCharTok{+}
  \FunctionTok{labs}\NormalTok{(}
    \AttributeTok{title =} \StringTok{"Overall Survival Curve (Lung Cancer)"}\NormalTok{,}
    \AttributeTok{x =} \StringTok{"Days"}\NormalTok{,}
    \AttributeTok{y =} \StringTok{"Survival Probability"}
\NormalTok{  )}
\end{Highlighting}
\end{Shaded}

\includegraphics{StatsTB_files/figure-latex/unnamed-chunk-434-1.pdf}

This plot is a step function, where survival probability decreases at the time of each death. It reflects the actual survival experience of the full cohort.

\hypertarget{stratified-survival-curve-by-sex}{%
\subsection{Stratified Survival Curve by Sex}\label{stratified-survival-curve-by-sex}}

We can stratify the curve by a categorical variable --- such as sex:

\begin{Shaded}
\begin{Highlighting}[]
\FunctionTok{survfit2}\NormalTok{(}\FunctionTok{Surv}\NormalTok{(time, status) }\SpecialCharTok{\textasciitilde{}}\NormalTok{ sex, }\AttributeTok{data =}\NormalTok{ lung) }\SpecialCharTok{\%\textgreater{}\%}
  \FunctionTok{ggsurvfit}\NormalTok{() }\SpecialCharTok{+}
  \FunctionTok{labs}\NormalTok{(}
    \AttributeTok{title =} \StringTok{"Survival by Sex"}\NormalTok{,}
    \AttributeTok{x =} \StringTok{"Days"}\NormalTok{,}
    \AttributeTok{y =} \StringTok{"Survival Probability"}\NormalTok{,}
    \AttributeTok{color =} \StringTok{"Sex"}
\NormalTok{  )}
\end{Highlighting}
\end{Shaded}

\includegraphics{StatsTB_files/figure-latex/unnamed-chunk-435-1.pdf}

In this case:
- \texttt{sex\ =\ 1} is Male
- \texttt{sex\ =\ 2} is Female

If the curves separate, that may indicate a survival difference between sexes.

\hypertarget{adding-confidence-intervals}{%
\subsection{Adding Confidence Intervals}\label{adding-confidence-intervals}}

To show uncertainty around the survival estimate, add shaded confidence bands using \texttt{add\_confidence\_interval()}:

\begin{Shaded}
\begin{Highlighting}[]
\FunctionTok{survfit2}\NormalTok{(}\FunctionTok{Surv}\NormalTok{(time, status) }\SpecialCharTok{\textasciitilde{}}\NormalTok{ sex, }\AttributeTok{data =}\NormalTok{ lung) }\SpecialCharTok{\%\textgreater{}\%}
  \FunctionTok{ggsurvfit}\NormalTok{() }\SpecialCharTok{+}
  \FunctionTok{add\_confidence\_interval}\NormalTok{() }\SpecialCharTok{+}
  \FunctionTok{labs}\NormalTok{(}
    \AttributeTok{title =} \StringTok{"Survival by Sex with 95\% CI"}\NormalTok{,}
    \AttributeTok{x =} \StringTok{"Days"}\NormalTok{,}
    \AttributeTok{y =} \StringTok{"Survival Probability"}\NormalTok{,}
    \AttributeTok{color =} \StringTok{"Sex"}
\NormalTok{  )}
\end{Highlighting}
\end{Shaded}

\includegraphics{StatsTB_files/figure-latex/unnamed-chunk-436-1.pdf}

The shaded bands show the 95\% confidence interval for survival probability at each time point.

\hypertarget{adding-a-risk-table}{%
\subsection{Adding a Risk Table}\label{adding-a-risk-table}}

Use \texttt{add\_risktable()} to include a table below the plot that shows the number of patients still at risk at key time intervals:

\begin{Shaded}
\begin{Highlighting}[]
\FunctionTok{survfit2}\NormalTok{(}\FunctionTok{Surv}\NormalTok{(time, status) }\SpecialCharTok{\textasciitilde{}}\NormalTok{ sex, }\AttributeTok{data =}\NormalTok{ lung) }\SpecialCharTok{\%\textgreater{}\%}
  \FunctionTok{ggsurvfit}\NormalTok{() }\SpecialCharTok{+}
  \FunctionTok{add\_confidence\_interval}\NormalTok{() }\SpecialCharTok{+}
  \FunctionTok{add\_risktable}\NormalTok{() }\SpecialCharTok{+}
  \FunctionTok{labs}\NormalTok{(}
    \AttributeTok{title =} \StringTok{"Survival by Sex with Risk Table"}\NormalTok{,}
    \AttributeTok{x =} \StringTok{"Days"}\NormalTok{,}
    \AttributeTok{y =} \StringTok{"Survival Probability"}\NormalTok{,}
    \AttributeTok{color =} \StringTok{"Sex"}
\NormalTok{  )}
\end{Highlighting}
\end{Shaded}

\includegraphics{StatsTB_files/figure-latex/unnamed-chunk-437-1.pdf}

Risk tables are essential for interpreting survival curves --- they show how many patients remain under observation as time progresses.

\hypertarget{original-biomedical-example-survival-by-ecog-performance-status}{%
\subsection{Original Biomedical Example: Survival by ECOG Performance Status}\label{original-biomedical-example-survival-by-ecog-performance-status}}

Let's analyze survival by ECOG performance status (ph.ecog), which rates patients' daily functional ability. Higher scores indicate worse performance status.

\begin{Shaded}
\begin{Highlighting}[]
\NormalTok{lung\_clean }\OtherTok{\textless{}{-}}\NormalTok{ lung }\SpecialCharTok{\%\textgreater{}\%}
  \FunctionTok{filter}\NormalTok{(}\SpecialCharTok{!}\FunctionTok{is.na}\NormalTok{(time), }\SpecialCharTok{!}\FunctionTok{is.na}\NormalTok{(status), }\SpecialCharTok{!}\FunctionTok{is.na}\NormalTok{(ph.ecog))}

\FunctionTok{survfit2}\NormalTok{(}\FunctionTok{Surv}\NormalTok{(time, status) }\SpecialCharTok{\textasciitilde{}}\NormalTok{ ph.ecog, }\AttributeTok{data =}\NormalTok{ lung\_clean) }\SpecialCharTok{\%\textgreater{}\%}
  \FunctionTok{ggsurvfit}\NormalTok{() }\SpecialCharTok{+}
  \FunctionTok{add\_confidence\_interval}\NormalTok{() }\SpecialCharTok{+}
  \FunctionTok{add\_risktable}\NormalTok{() }\SpecialCharTok{+}
  \FunctionTok{labs}\NormalTok{(}
    \AttributeTok{title =} \StringTok{"Survival by ECOG Performance Score"}\NormalTok{,}
    \AttributeTok{x =} \StringTok{"Days"}\NormalTok{,}
    \AttributeTok{y =} \StringTok{"Survival Probability"}\NormalTok{,}
    \AttributeTok{color =} \StringTok{"ECOG Score"}
\NormalTok{  )}
\end{Highlighting}
\end{Shaded}

\includegraphics{StatsTB_files/figure-latex/unnamed-chunk-438-1.pdf}

\textbf{Interpretation}:\\
- Patients with ECOG = 0 have better functional status and tend to survive longer.\\
- Higher scores (e.g., ECOG = 2 or 3) show steeper declines in survival probability.\\
- Functional status is a strong predictor of survival in many oncology studies.

\hypertarget{original-example-simulated-trial-new-drug-vs.-placebo}{%
\subsection{Original Example: Simulated Trial --- New Drug vs.~Placebo}\label{original-example-simulated-trial-new-drug-vs.-placebo}}

We simulate survival data from a randomized trial comparing a treatment group to placebo:

\begin{Shaded}
\begin{Highlighting}[]
\FunctionTok{set.seed}\NormalTok{(}\DecValTok{2025}\NormalTok{)}

\NormalTok{sim\_trial }\OtherTok{\textless{}{-}} \FunctionTok{tibble}\NormalTok{(}
  \AttributeTok{group =} \FunctionTok{rep}\NormalTok{(}\FunctionTok{c}\NormalTok{(}\StringTok{"Treatment"}\NormalTok{, }\StringTok{"Placebo"}\NormalTok{), }\AttributeTok{each =} \DecValTok{100}\NormalTok{),}
  \AttributeTok{time =} \FunctionTok{rexp}\NormalTok{(}\DecValTok{200}\NormalTok{, }\AttributeTok{rate =} \FunctionTok{rep}\NormalTok{(}\FunctionTok{c}\NormalTok{(}\FloatTok{0.02}\NormalTok{, }\FloatTok{0.03}\NormalTok{), }\AttributeTok{each =} \DecValTok{100}\NormalTok{)),}
  \AttributeTok{status =} \FunctionTok{rbinom}\NormalTok{(}\DecValTok{200}\NormalTok{, }\DecValTok{1}\NormalTok{, }\AttributeTok{prob =} \FloatTok{0.9}\NormalTok{)}
\NormalTok{) }\SpecialCharTok{\%\textgreater{}\%}
  \FunctionTok{mutate}\NormalTok{(}\AttributeTok{group =} \FunctionTok{factor}\NormalTok{(group, }\AttributeTok{levels =} \FunctionTok{c}\NormalTok{(}\StringTok{"Placebo"}\NormalTok{, }\StringTok{"Treatment"}\NormalTok{)))}

\FunctionTok{survfit2}\NormalTok{(}\FunctionTok{Surv}\NormalTok{(time, status) }\SpecialCharTok{\textasciitilde{}}\NormalTok{ group, }\AttributeTok{data =}\NormalTok{ sim\_trial) }\SpecialCharTok{\%\textgreater{}\%}
  \FunctionTok{ggsurvfit}\NormalTok{() }\SpecialCharTok{+}
  \FunctionTok{add\_confidence\_interval}\NormalTok{() }\SpecialCharTok{+}
  \FunctionTok{add\_risktable}\NormalTok{() }\SpecialCharTok{+}
  \FunctionTok{labs}\NormalTok{(}
    \AttributeTok{title =} \StringTok{"Simulated Trial: Treatment vs Placebo"}\NormalTok{,}
    \AttributeTok{x =} \StringTok{"Time (days)"}\NormalTok{,}
    \AttributeTok{y =} \StringTok{"Survival Probability"}\NormalTok{,}
    \AttributeTok{color =} \StringTok{"Group"}
\NormalTok{  )}
\end{Highlighting}
\end{Shaded}

\includegraphics{StatsTB_files/figure-latex/unnamed-chunk-439-1.pdf}

\textbf{Interpretation}:\\
- The treatment group shows higher survival probability over time.\\
- This is a simulated example of how Kaplan-Meier curves are used to evaluate treatment effects in clinical trials.

\hypertarget{estimating-x-year-survival-rate}{%
\section{Estimating x-Year Survival Rate}\label{estimating-x-year-survival-rate}}

An important application of Kaplan-Meier survival analysis is estimating the probability of surviving past a certain time --- typically 1 year, 2 years, or 5 years. This is useful for communicating prognosis in clinical settings.

The quantity \(P(T > x)\) represents the probability that a subject survives beyond time x.

\hypertarget{estimating-1-year-survival-from-the-lung-dataset}{%
\subsection{Estimating 1-Year Survival from the Lung Dataset}\label{estimating-1-year-survival-from-the-lung-dataset}}

The \texttt{summary()} function can be used on a \texttt{survfit()} object to extract survival probabilities at specific times. Since the \texttt{lung} dataset reports time in days, we use 365.25 for 1 year:

\begin{Shaded}
\begin{Highlighting}[]
\NormalTok{fit\_lung }\OtherTok{\textless{}{-}} \FunctionTok{survfit}\NormalTok{(}\FunctionTok{Surv}\NormalTok{(time, status) }\SpecialCharTok{\textasciitilde{}} \DecValTok{1}\NormalTok{, }\AttributeTok{data =}\NormalTok{ lung)}

\FunctionTok{summary}\NormalTok{(fit\_lung, }\AttributeTok{times =} \FloatTok{365.25}\NormalTok{)}
\end{Highlighting}
\end{Shaded}

\begin{verbatim}
## Call: survfit(formula = Surv(time, status) ~ 1, data = lung)
## 
##  time n.risk n.event survival std.err lower 95% CI upper 95% CI
##   365     65     121    0.409  0.0358        0.345        0.486
\end{verbatim}

This output gives:
- Survival probability at 1 year
- Confidence intervals
- Number of subjects still at risk at 1 year

\hypertarget{comparing-1-year-survival-by-sex}{%
\subsection{Comparing 1-Year Survival by Sex}\label{comparing-1-year-survival-by-sex}}

We can also compare survival between subgroups. Let's compare males and females:

\begin{Shaded}
\begin{Highlighting}[]
\FunctionTok{summary}\NormalTok{(}
  \FunctionTok{survfit}\NormalTok{(}\FunctionTok{Surv}\NormalTok{(time, status) }\SpecialCharTok{\textasciitilde{}}\NormalTok{ sex, }\AttributeTok{data =}\NormalTok{ lung),}
  \AttributeTok{times =} \FloatTok{365.25}
\NormalTok{)}
\end{Highlighting}
\end{Shaded}

\begin{verbatim}
## Call: survfit(formula = Surv(time, status) ~ sex, data = lung)
## 
##                 sex=1 
##         time       n.risk      n.event     survival      std.err lower 95% CI 
##     365.2500      35.0000      85.0000       0.3361       0.0434       0.2609 
## upper 95% CI 
##       0.4329 
## 
##                 sex=2 
##         time       n.risk      n.event     survival      std.err lower 95% CI 
##     365.2500      30.0000      36.0000       0.5265       0.0597       0.4215 
## upper 95% CI 
##       0.6576
\end{verbatim}

This shows survival probabilities for each sex at the 1-year mark. For example:
- Females may have a higher 1-year survival probability than males
- Confidence intervals can help determine if the difference is significant

\hypertarget{using-tbl_survfit-for-a-clean-summary-table}{%
\subsection{\texorpdfstring{Using \texttt{tbl\_survfit()} for a Clean Summary Table}{Using tbl\_survfit() for a Clean Summary Table}}\label{using-tbl_survfit-for-a-clean-summary-table}}

The \texttt{\{gtsummary\}} package provides a cleaner table of survival probabilities using \texttt{tbl\_survfit()}:

\begin{Shaded}
\begin{Highlighting}[]
\FunctionTok{library}\NormalTok{(gtsummary)}

\FunctionTok{survfit}\NormalTok{(}\FunctionTok{Surv}\NormalTok{(time, status) }\SpecialCharTok{\textasciitilde{}}\NormalTok{ sex, }\AttributeTok{data =}\NormalTok{ lung) }\SpecialCharTok{\%\textgreater{}\%}
  \FunctionTok{tbl\_survfit}\NormalTok{(}
    \AttributeTok{times =} \FloatTok{365.25}\NormalTok{,}
    \AttributeTok{label\_header =} \StringTok{"**1{-}Year Survival (95\% CI)**"}
\NormalTok{  )}
\end{Highlighting}
\end{Shaded}

\begin{table}[t]
\fontsize{12.0pt}{14.4pt}\selectfont
\begin{tabular*}{\linewidth}{@{\extracolsep{\fill}}lc}
\toprule
\textbf{Characteristic} & \textbf{1-Year Survival (95\% CI)} \\ 
\midrule\addlinespace[2.5pt]
sex &  \\ 
    1 & 34\% (26\%, 43\%) \\ 
    2 & 53\% (42\%, 66\%) \\ 
\bottomrule
\end{tabular*}
\end{table}

This table is formatted for easy inclusion in reports or presentations.

\hypertarget{the-danger-of-naive-estimates}{%
\subsection{The Danger of ``Naive'' Estimates}\label{the-danger-of-naive-estimates}}

If you ignore censoring, you will overestimate survival. For example, 121 of the 228 patients in the lung dataset died before 1 year. A naive survival estimate would be:

\begin{Shaded}
\begin{Highlighting}[]
\NormalTok{(}\DecValTok{1} \SpecialCharTok{{-}} \DecValTok{121} \SpecialCharTok{/} \DecValTok{228}\NormalTok{) }\SpecialCharTok{*} \DecValTok{100}
\end{Highlighting}
\end{Shaded}

\begin{verbatim}
## [1] 46.92982
\end{verbatim}

This gives 47\%, but the Kaplan-Meier estimate --- which accounts for censoring --- was closer to 41\%. Ignoring censoring inflates survival rates, which can mislead clinicians and patients.

\hypertarget{original-biomedical-example-1-year-survival-in-breast-cancer-patients}{%
\subsection{Original Biomedical Example: 1-Year Survival in Breast Cancer Patients}\label{original-biomedical-example-1-year-survival-in-breast-cancer-patients}}

Let's simulate a study with 100 women diagnosed with breast cancer. We estimate 1-year survival for two tumor grades: low-grade and high-grade.

\begin{Shaded}
\begin{Highlighting}[]
\CommentTok{\# Load required libraries}
\FunctionTok{library}\NormalTok{(survival)}
\FunctionTok{library}\NormalTok{(tibble)}
\FunctionTok{library}\NormalTok{(dplyr)}

\CommentTok{\# Set seed for reproducibility}
\FunctionTok{set.seed}\NormalTok{(}\DecValTok{2025}\NormalTok{)}

\CommentTok{\# Simulated breast cancer dataset}
\NormalTok{breast\_data }\OtherTok{\textless{}{-}} \FunctionTok{tibble}\NormalTok{(}
  \AttributeTok{tumor\_grade =} \FunctionTok{rep}\NormalTok{(}\FunctionTok{c}\NormalTok{(}\StringTok{"Low{-}Grade"}\NormalTok{, }\StringTok{"High{-}Grade"}\NormalTok{), }\AttributeTok{each =} \DecValTok{50}\NormalTok{),}
  \AttributeTok{time =} \FunctionTok{rexp}\NormalTok{(}\DecValTok{100}\NormalTok{, }\AttributeTok{rate =} \FunctionTok{rep}\NormalTok{(}\FunctionTok{c}\NormalTok{(}\FloatTok{0.01}\NormalTok{, }\FloatTok{0.03}\NormalTok{), }\AttributeTok{each =} \DecValTok{50}\NormalTok{)),}
  \AttributeTok{status =} \FunctionTok{rbinom}\NormalTok{(}\DecValTok{100}\NormalTok{, }\DecValTok{1}\NormalTok{, }\AttributeTok{prob =} \FloatTok{0.85}\NormalTok{)}
\NormalTok{)}

\CommentTok{\# Fit survival model}
\NormalTok{fit\_bc }\OtherTok{\textless{}{-}} \FunctionTok{survfit}\NormalTok{(}\FunctionTok{Surv}\NormalTok{(time, status) }\SpecialCharTok{\textasciitilde{}}\NormalTok{ tumor\_grade, }\AttributeTok{data =}\NormalTok{ breast\_data)}

\CommentTok{\# Estimate 1{-}year survival (safe version)}
\FunctionTok{summary}\NormalTok{(fit\_bc, }\AttributeTok{times =} \FloatTok{365.25}\NormalTok{, }\AttributeTok{extend =} \ConstantTok{TRUE}\NormalTok{)}
\end{Highlighting}
\end{Shaded}

\begin{verbatim}
## Call: survfit(formula = Surv(time, status) ~ tumor_grade, data = breast_data)
## 
##                 tumor_grade=High-Grade 
##         time       n.risk      n.event     survival      std.err lower 95% CI 
##          365            0           44            0          NaN           NA 
## upper 95% CI 
##           NA 
## 
##                 tumor_grade=Low-Grade 
##         time       n.risk      n.event     survival      std.err lower 95% CI 
##     3.65e+02     1.00e+00     3.70e+01     4.72e-02     4.43e-02     7.49e-03 
## upper 95% CI 
##     2.97e-01
\end{verbatim}

\textbf{Interpretation}:
- Patients with low-grade tumors may have higher 1-year survival\\
- Survival estimates account for censoring (some patients may not have reached 1 year)
- This reflects how tumor biology influences short-term prognosis in oncology

\hypertarget{estimating-median-survival-time}{%
\section{Estimating Median Survival Time}\label{estimating-median-survival-time}}

In survival analysis, the median survival time is a key summary statistic. It tells us the time at which 50\% of the population has experienced the event of interest (e.g., death, relapse).

Unlike the mean, the median is more robust to skewed data and censoring, making it a preferred measure in clinical trials and cancer research.

\hypertarget{estimating-median-survival-from-the-lung-dataset}{%
\subsection{Estimating Median Survival from the Lung Dataset}\label{estimating-median-survival-from-the-lung-dataset}}

We can extract the median survival time using the \texttt{survfit()} object directly:

\begin{Shaded}
\begin{Highlighting}[]
\NormalTok{fit\_lung }\OtherTok{\textless{}{-}} \FunctionTok{survfit}\NormalTok{(}\FunctionTok{Surv}\NormalTok{(time, status) }\SpecialCharTok{\textasciitilde{}} \DecValTok{1}\NormalTok{, }\AttributeTok{data =}\NormalTok{ lung)}

\FunctionTok{summary}\NormalTok{(fit\_lung)}\SpecialCharTok{$}\NormalTok{table[}\StringTok{"median"}\NormalTok{]}
\end{Highlighting}
\end{Shaded}

\begin{verbatim}
## median 
##    310
\end{verbatim}

This returns the median survival time for the full lung cancer dataset.

To compare median survival across subgroups, we use stratification:

\begin{Shaded}
\begin{Highlighting}[]
\NormalTok{fit\_by\_sex }\OtherTok{\textless{}{-}} \FunctionTok{survfit}\NormalTok{(}\FunctionTok{Surv}\NormalTok{(time, status) }\SpecialCharTok{\textasciitilde{}}\NormalTok{ sex, }\AttributeTok{data =}\NormalTok{ lung)}

\FunctionTok{summary}\NormalTok{(fit\_by\_sex)}\SpecialCharTok{$}\NormalTok{table[, }\StringTok{"median"}\NormalTok{]}
\end{Highlighting}
\end{Shaded}

\begin{verbatim}
## sex=1 sex=2 
##   270   426
\end{verbatim}

This returns the median survival time for both males and females. Keep in mind:
- If the survival curve never drops to 50\%, the median may be reported as NA
- This happens when fewer than half of the patients have experienced the event

\hypertarget{interpreting-the-median}{%
\subsection{Interpreting the Median}\label{interpreting-the-median}}

Let's say the output shows:
- Median survival for females = 310 days
- Median survival for males = 270 days

This means:
\textgreater{} Half of the female patients were still alive after 310 days, while half of the male patients had died by 270 days.

This helps clinicians compare short-term prognosis across different groups.

\hypertarget{original-biomedical-example-pancreatic-cancer-trial}{%
\subsection{Original Biomedical Example: Pancreatic Cancer Trial}\label{original-biomedical-example-pancreatic-cancer-trial}}

Let's simulate a dataset of 80 pancreatic cancer patients undergoing two treatment types: Targeted Therapy vs.~Standard Chemotherapy.

We'll compare their median survival times.

\begin{Shaded}
\begin{Highlighting}[]
\FunctionTok{set.seed}\NormalTok{(}\DecValTok{2025}\NormalTok{)}

\NormalTok{pancreatic }\OtherTok{\textless{}{-}} \FunctionTok{tibble}\NormalTok{(}
  \AttributeTok{treatment =} \FunctionTok{rep}\NormalTok{(}\FunctionTok{c}\NormalTok{(}\StringTok{"Targeted"}\NormalTok{, }\StringTok{"Standard"}\NormalTok{), }\AttributeTok{each =} \DecValTok{40}\NormalTok{),}
  \AttributeTok{time =} \FunctionTok{rexp}\NormalTok{(}\DecValTok{80}\NormalTok{, }\AttributeTok{rate =} \FunctionTok{rep}\NormalTok{(}\FunctionTok{c}\NormalTok{(}\FloatTok{0.007}\NormalTok{, }\FloatTok{0.012}\NormalTok{), }\AttributeTok{each =} \DecValTok{40}\NormalTok{)),}
  \AttributeTok{status =} \FunctionTok{rbinom}\NormalTok{(}\DecValTok{80}\NormalTok{, }\DecValTok{1}\NormalTok{, }\AttributeTok{prob =} \FloatTok{0.90}\NormalTok{)}
\NormalTok{)}

\NormalTok{fit\_pan }\OtherTok{\textless{}{-}} \FunctionTok{survfit}\NormalTok{(}\FunctionTok{Surv}\NormalTok{(time, status) }\SpecialCharTok{\textasciitilde{}}\NormalTok{ treatment, }\AttributeTok{data =}\NormalTok{ pancreatic)}

\FunctionTok{summary}\NormalTok{(fit\_pan)}\SpecialCharTok{$}\NormalTok{table[, }\StringTok{"median"}\NormalTok{]}
\end{Highlighting}
\end{Shaded}

\begin{verbatim}
## treatment=Standard treatment=Targeted 
##            51.3644           148.3763
\end{verbatim}

\hypertarget{interpretation-23}{%
\subsection{\texorpdfstring{\textbf{Interpretation}:}{Interpretation:}}\label{interpretation-23}}

\begin{itemize}
\tightlist
\item
  If \texttt{Targeted} has a median survival of 125 days and \texttt{Standard} has 90 days, we interpret:
  \textgreater{} Targeted therapy patients had a longer median survival, suggesting greater short-term benefit.
\end{itemize}

Be cautious: median survival ignores long-tail survival. Some patients in both groups may live far beyond the median.

\hypertarget{hazard-ratios-and-cox-proportional-hazards-models}{%
\section{Hazard Ratios and Cox Proportional Hazards Models}\label{hazard-ratios-and-cox-proportional-hazards-models}}

The Cox proportional hazards model is one of the most widely used tools in survival analysis. It allows us to estimate the effect of one or more variables on survival time without assuming a specific baseline distribution.

It outputs a key statistic: the hazard ratio (HR), which tells us how much more (or less) likely an event is to occur in one group compared to another, at any given point in time.

\hypertarget{understanding-hazard-ratios}{%
\subsection{Understanding Hazard Ratios}\label{understanding-hazard-ratios}}

\begin{itemize}
\tightlist
\item
  A hazard ratio (HR) \textgreater{} 1 implies higher risk of the event (e.g., death)
\item
  An HR \textless{} 1 implies lower risk
\item
  An HR = 1 implies no difference
\end{itemize}

These ratios are multiplicative. For example, HR = 2 means twice the hazard, HR = 0.5 means half the hazard.

The Cox model assumes that the ratio of hazards is constant over time, which is known as the proportional hazards assumption.

\hypertarget{fitting-a-cox-model-to-the-lung-dataset}{%
\subsection{Fitting a Cox Model to the Lung Dataset}\label{fitting-a-cox-model-to-the-lung-dataset}}

Let's model the effect of sex (male vs.~female) on survival in the \texttt{lung} dataset:

\begin{Shaded}
\begin{Highlighting}[]
\FunctionTok{library}\NormalTok{(survival)}

\NormalTok{cox\_sex }\OtherTok{\textless{}{-}} \FunctionTok{coxph}\NormalTok{(}\FunctionTok{Surv}\NormalTok{(time, status) }\SpecialCharTok{\textasciitilde{}}\NormalTok{ sex, }\AttributeTok{data =}\NormalTok{ lung)}
\FunctionTok{summary}\NormalTok{(cox\_sex)}
\end{Highlighting}
\end{Shaded}

\begin{verbatim}
## Call:
## coxph(formula = Surv(time, status) ~ sex, data = lung)
## 
##   n= 228, number of events= 165 
## 
##        coef exp(coef) se(coef)      z Pr(>|z|)   
## sex -0.5310    0.5880   0.1672 -3.176  0.00149 **
## ---
## Signif. codes:  0 '***' 0.001 '**' 0.01 '*' 0.05 '.' 0.1 ' ' 1
## 
##     exp(coef) exp(-coef) lower .95 upper .95
## sex     0.588      1.701    0.4237     0.816
## 
## Concordance= 0.579  (se = 0.021 )
## Likelihood ratio test= 10.63  on 1 df,   p=0.001
## Wald test            = 10.09  on 1 df,   p=0.001
## Score (logrank) test = 10.33  on 1 df,   p=0.001
\end{verbatim}

In the output, look for:
- \textbf{coef}: the log hazard ratio
- \textbf{exp(coef)}: the hazard ratio (HR)
- \textbf{Pr(\textgreater\textbar z\textbar)}: the p-value testing if the HR is significantly different from 1

A significant HR \textgreater{} 1 for males would indicate that men have higher risk of death than women in this dataset.

\hypertarget{adding-covariates-to-the-cox-model}{%
\subsection{Adding Covariates to the Cox Model}\label{adding-covariates-to-the-cox-model}}

We can add additional variables (covariates) to adjust for other risk factors. Let's adjust for age and performance status (\texttt{ph.ecog}):

\begin{Shaded}
\begin{Highlighting}[]
\NormalTok{cox\_multiv }\OtherTok{\textless{}{-}} \FunctionTok{coxph}\NormalTok{(}\FunctionTok{Surv}\NormalTok{(time, status) }\SpecialCharTok{\textasciitilde{}}\NormalTok{ sex }\SpecialCharTok{+}\NormalTok{ age }\SpecialCharTok{+}\NormalTok{ ph.ecog, }\AttributeTok{data =}\NormalTok{ lung)}
\FunctionTok{summary}\NormalTok{(cox\_multiv)}
\end{Highlighting}
\end{Shaded}

\begin{verbatim}
## Call:
## coxph(formula = Surv(time, status) ~ sex + age + ph.ecog, data = lung)
## 
##   n= 227, number of events= 164 
##    (1 observation deleted due to missingness)
## 
##              coef exp(coef)  se(coef)      z Pr(>|z|)    
## sex     -0.552612  0.575445  0.167739 -3.294 0.000986 ***
## age      0.011067  1.011128  0.009267  1.194 0.232416    
## ph.ecog  0.463728  1.589991  0.113577  4.083 4.45e-05 ***
## ---
## Signif. codes:  0 '***' 0.001 '**' 0.01 '*' 0.05 '.' 0.1 ' ' 1
## 
##         exp(coef) exp(-coef) lower .95 upper .95
## sex        0.5754     1.7378    0.4142    0.7994
## age        1.0111     0.9890    0.9929    1.0297
## ph.ecog    1.5900     0.6289    1.2727    1.9864
## 
## Concordance= 0.637  (se = 0.025 )
## Likelihood ratio test= 30.5  on 3 df,   p=1e-06
## Wald test            = 29.93  on 3 df,   p=1e-06
## Score (logrank) test = 30.5  on 3 df,   p=1e-06
\end{verbatim}

This helps isolate the effect of sex after controlling for age and functional status. These multivariable models are common in medical research.

\hypertarget{interpreting-cox-model-output}{%
\subsection{Interpreting Cox Model Output}\label{interpreting-cox-model-output}}

In the summary:
- \texttt{exp(coef)} gives the hazard ratio
- \texttt{conf.int} gives the 95\% confidence interval
- A narrow interval not including 1 suggests a statistically significant effect

Always check if the proportional hazards assumption is met before interpreting results.

\hypertarget{original-biomedical-example-smokers-vs.-non-smokers-in-a-cancer-study}{%
\subsection{Original Biomedical Example: Smokers vs.~Non-Smokers in a Cancer Study}\label{original-biomedical-example-smokers-vs.-non-smokers-in-a-cancer-study}}

We simulate a dataset of 120 patients in a cancer trial. We compare the effect of smoking on survival time.

\begin{Shaded}
\begin{Highlighting}[]
\FunctionTok{set.seed}\NormalTok{(}\DecValTok{2025}\NormalTok{)}

\NormalTok{smoking\_data }\OtherTok{\textless{}{-}} \FunctionTok{tibble}\NormalTok{(}
  \AttributeTok{smoking =} \FunctionTok{rep}\NormalTok{(}\FunctionTok{c}\NormalTok{(}\StringTok{"Smoker"}\NormalTok{, }\StringTok{"Non{-}Smoker"}\NormalTok{), }\AttributeTok{each =} \DecValTok{60}\NormalTok{),}
  \AttributeTok{time =} \FunctionTok{rexp}\NormalTok{(}\DecValTok{120}\NormalTok{, }\AttributeTok{rate =} \FunctionTok{rep}\NormalTok{(}\FunctionTok{c}\NormalTok{(}\FloatTok{0.015}\NormalTok{, }\FloatTok{0.010}\NormalTok{), }\AttributeTok{each =} \DecValTok{60}\NormalTok{)),}
  \AttributeTok{status =} \FunctionTok{rbinom}\NormalTok{(}\DecValTok{120}\NormalTok{, }\DecValTok{1}\NormalTok{, }\AttributeTok{prob =} \FloatTok{0.88}\NormalTok{)}
\NormalTok{)}

\NormalTok{cox\_smoking }\OtherTok{\textless{}{-}} \FunctionTok{coxph}\NormalTok{(}\FunctionTok{Surv}\NormalTok{(time, status) }\SpecialCharTok{\textasciitilde{}}\NormalTok{ smoking, }\AttributeTok{data =}\NormalTok{ smoking\_data)}

\FunctionTok{summary}\NormalTok{(cox\_smoking)}
\end{Highlighting}
\end{Shaded}

\begin{verbatim}
## Call:
## coxph(formula = Surv(time, status) ~ smoking, data = smoking_data)
## 
##   n= 120, number of events= 103 
## 
##                 coef exp(coef) se(coef)     z Pr(>|z|)  
## smokingSmoker 0.3748    1.4546   0.2031 1.845    0.065 .
## ---
## Signif. codes:  0 '***' 0.001 '**' 0.01 '*' 0.05 '.' 0.1 ' ' 1
## 
##               exp(coef) exp(-coef) lower .95 upper .95
## smokingSmoker     1.455     0.6875     0.977     2.166
## 
## Concordance= 0.547  (se = 0.028 )
## Likelihood ratio test= 3.4  on 1 df,   p=0.07
## Wald test            = 3.4  on 1 df,   p=0.07
## Score (logrank) test = 3.44  on 1 df,   p=0.06
\end{verbatim}

\hypertarget{interpretation-24}{%
\subsection{\texorpdfstring{\textbf{Interpretation}:}{Interpretation:}}\label{interpretation-24}}

\begin{itemize}
\item
  If \texttt{exp(coef)} = 1.5, we interpret:
  \textgreater{} Smokers have a 50\% higher risk of death at any given time point compared to non-smokers.
\item
  If p-value \textless{} 0.05, the association is statistically significant
\item
  This reflects how behavioral factors like smoking influence survival outcomes
\end{itemize}

\hypertarget{assumptions-of-the-cox-model}{%
\section{Assumptions of the Cox Model}\label{assumptions-of-the-cox-model}}

The Cox proportional hazards model assumes that hazard ratios remain constant over time. This is known as the Proportional Hazards (PH) Assumption.

If the PH assumption is violated, the Cox model can produce misleading estimates. Therefore, we must check this assumption before interpreting hazard ratios.

\hypertarget{what-is-the-ph-assumption}{%
\subsection{What is the PH Assumption?}\label{what-is-the-ph-assumption}}

For two groups (e.g., treatment vs.~placebo), the PH assumption means:

\begin{quote}
The relative risk of the event occurring is constant over time, even though the absolute risk may change.
\end{quote}

Visually, this means the survival curves should not cross and the log(-log(S(t))) plots should be roughly parallel.

\hypertarget{testing-the-ph-assumption-with-schoenfeld-residuals}{%
\subsection{Testing the PH Assumption with Schoenfeld Residuals}\label{testing-the-ph-assumption-with-schoenfeld-residuals}}

The \texttt{cox.zph()} function tests whether the residuals are independent of time. This provides both statistical tests and plots.

Let's test this in the \texttt{lung} dataset using sex as a predictor:

\begin{Shaded}
\begin{Highlighting}[]
\NormalTok{fit\_cox }\OtherTok{\textless{}{-}} \FunctionTok{coxph}\NormalTok{(}\FunctionTok{Surv}\NormalTok{(time, status) }\SpecialCharTok{\textasciitilde{}}\NormalTok{ sex, }\AttributeTok{data =}\NormalTok{ lung)}

\NormalTok{ph\_test }\OtherTok{\textless{}{-}} \FunctionTok{cox.zph}\NormalTok{(fit\_cox)}

\NormalTok{ph\_test}
\end{Highlighting}
\end{Shaded}

\begin{verbatim}
##        chisq df     p
## sex     2.86  1 0.091
## GLOBAL  2.86  1 0.091
\end{verbatim}

If the \textbf{p-value is \textgreater{} 0.05}, we \textbf{fail to reject} the null hypothesis that hazards are proportional (assumption met).\\
If p-value \textless{} 0.05, we reject the assumption (assumption may be violated).

\hypertarget{visualizing-schoenfeld-residuals}{%
\subsection{Visualizing Schoenfeld Residuals}\label{visualizing-schoenfeld-residuals}}

We can plot the residuals to visually inspect time trends:

\begin{Shaded}
\begin{Highlighting}[]
\FunctionTok{plot}\NormalTok{(ph\_test)}
\end{Highlighting}
\end{Shaded}

\includegraphics{StatsTB_files/figure-latex/unnamed-chunk-452-1.pdf}

If the line is approximately \textbf{flat and horizontal}, the PH assumption is likely satisfied.

\hypertarget{original-biomedical-example-ph-assumption-in-hiv-treatment-study}{%
\subsection{Original Biomedical Example: PH Assumption in HIV Treatment Study}\label{original-biomedical-example-ph-assumption-in-hiv-treatment-study}}

We simulate a study with two HIV treatments. We want to verify whether the hazards remain proportional over time.

\begin{Shaded}
\begin{Highlighting}[]
\FunctionTok{set.seed}\NormalTok{(}\DecValTok{2025}\NormalTok{)}

\NormalTok{hiv\_data }\OtherTok{\textless{}{-}} \FunctionTok{tibble}\NormalTok{(}
  \AttributeTok{treatment =} \FunctionTok{rep}\NormalTok{(}\FunctionTok{c}\NormalTok{(}\StringTok{"Drug A"}\NormalTok{, }\StringTok{"Drug B"}\NormalTok{), }\AttributeTok{each =} \DecValTok{75}\NormalTok{),}
  \AttributeTok{time =} \FunctionTok{rexp}\NormalTok{(}\DecValTok{150}\NormalTok{, }\AttributeTok{rate =} \FunctionTok{rep}\NormalTok{(}\FunctionTok{c}\NormalTok{(}\FloatTok{0.015}\NormalTok{, }\FloatTok{0.010}\NormalTok{), }\AttributeTok{each =} \DecValTok{75}\NormalTok{)),}
  \AttributeTok{status =} \FunctionTok{rbinom}\NormalTok{(}\DecValTok{150}\NormalTok{, }\DecValTok{1}\NormalTok{, }\AttributeTok{prob =} \FloatTok{0.9}\NormalTok{)}
\NormalTok{)}

\NormalTok{cox\_hiv }\OtherTok{\textless{}{-}} \FunctionTok{coxph}\NormalTok{(}\FunctionTok{Surv}\NormalTok{(time, status) }\SpecialCharTok{\textasciitilde{}}\NormalTok{ treatment, }\AttributeTok{data =}\NormalTok{ hiv\_data)}

\NormalTok{ph\_hiv }\OtherTok{\textless{}{-}} \FunctionTok{cox.zph}\NormalTok{(cox\_hiv)}

\NormalTok{ph\_hiv}
\end{Highlighting}
\end{Shaded}

\begin{verbatim}
##           chisq df    p
## treatment  0.33  1 0.57
## GLOBAL     0.33  1 0.57
\end{verbatim}

\hypertarget{interpretation-25}{%
\subsection{\texorpdfstring{\textbf{Interpretation}:}{Interpretation:}}\label{interpretation-25}}

\begin{itemize}
\tightlist
\item
  If p-value \textgreater{} 0.05, we conclude:
  \textgreater{} The effect of HIV treatment on survival satisfies the proportional hazards assumption.
\item
  This allows us to interpret the hazard ratio from the Cox model confidently
\end{itemize}

We can also check the residual plot:

\begin{Shaded}
\begin{Highlighting}[]
\FunctionTok{plot}\NormalTok{(ph\_hiv)}
\end{Highlighting}
\end{Shaded}

\includegraphics{StatsTB_files/figure-latex/unnamed-chunk-454-1.pdf}

If the curve is flat, we visually confirm that hazards are proportional over time.

\hypertarget{adjusted-survival-curves}{%
\section{Adjusted Survival Curves}\label{adjusted-survival-curves}}

In multivariable survival analysis, we often want to visualize how a specific predictor (e.g., treatment) affects survival while adjusting for other variables (e.g., age, baseline health).

This is where adjusted survival curves come in. These curves show the estimated survival for each group, averaged over the distribution of covariates.

\hypertarget{why-adjust-survival-curves}{%
\subsection{Why Adjust Survival Curves?}\label{why-adjust-survival-curves}}

Unadjusted Kaplan-Meier plots only compare survival by one variable. If other confounders are present (e.g., age, comorbidity), the curves can be misleading.

Adjusted curves allow us to isolate the effect of one predictor while accounting for others.

For example, we may want to compare survival by treatment group adjusting for age.

\hypertarget{method-using-the-survfit-function-with-a-cox-model}{%
\subsection{\texorpdfstring{Method: Using the \texttt{survfit()} Function with a Cox Model}{Method: Using the survfit() Function with a Cox Model}}\label{method-using-the-survfit-function-with-a-cox-model}}

Once a Cox model is fitted with covariates, we can compute adjusted survival curves using the \texttt{survfit()} function and the \texttt{newdata} argument.

Let's look at an example using the \texttt{lung} dataset:

\begin{Shaded}
\begin{Highlighting}[]
\CommentTok{\# Cox model with age and sex}
\NormalTok{fit\_adj }\OtherTok{\textless{}{-}} \FunctionTok{coxph}\NormalTok{(}\FunctionTok{Surv}\NormalTok{(time, status) }\SpecialCharTok{\textasciitilde{}}\NormalTok{ sex }\SpecialCharTok{+}\NormalTok{ age, }\AttributeTok{data =}\NormalTok{ lung)}

\CommentTok{\# Create newdata with specific values}
\NormalTok{newdata\_sex }\OtherTok{\textless{}{-}} \FunctionTok{data.frame}\NormalTok{(}
  \AttributeTok{sex =} \FunctionTok{c}\NormalTok{(}\DecValTok{1}\NormalTok{, }\DecValTok{2}\NormalTok{),}
  \AttributeTok{age =} \FunctionTok{rep}\NormalTok{(}\FunctionTok{mean}\NormalTok{(lung}\SpecialCharTok{$}\NormalTok{age, }\AttributeTok{na.rm =} \ConstantTok{TRUE}\NormalTok{), }\DecValTok{2}\NormalTok{)}
\NormalTok{)}

\CommentTok{\# Generate adjusted survival curves}
\NormalTok{fit\_adj\_surv }\OtherTok{\textless{}{-}} \FunctionTok{survfit}\NormalTok{(fit\_adj, }\AttributeTok{newdata =}\NormalTok{ newdata\_sex)}

\CommentTok{\# Plot}
\FunctionTok{plot}\NormalTok{(fit\_adj\_surv, }\AttributeTok{col =} \FunctionTok{c}\NormalTok{(}\StringTok{"blue"}\NormalTok{, }\StringTok{"red"}\NormalTok{), }\AttributeTok{lty =} \DecValTok{1}\SpecialCharTok{:}\DecValTok{2}\NormalTok{,}
     \AttributeTok{xlab =} \StringTok{"Days"}\NormalTok{, }\AttributeTok{ylab =} \StringTok{"Adjusted Survival Probability"}\NormalTok{)}
\FunctionTok{legend}\NormalTok{(}\StringTok{"topright"}\NormalTok{, }\AttributeTok{legend =} \FunctionTok{c}\NormalTok{(}\StringTok{"Male"}\NormalTok{, }\StringTok{"Female"}\NormalTok{),}
       \AttributeTok{col =} \FunctionTok{c}\NormalTok{(}\StringTok{"blue"}\NormalTok{, }\StringTok{"red"}\NormalTok{), }\AttributeTok{lty =} \DecValTok{1}\SpecialCharTok{:}\DecValTok{2}\NormalTok{)}
\end{Highlighting}
\end{Shaded}

\includegraphics{StatsTB_files/figure-latex/unnamed-chunk-455-1.pdf}

This plot shows survival for males and females at the same average age. This isolates the effect of sex on survival.

\hypertarget{interpretation-26}{%
\subsection{Interpretation}\label{interpretation-26}}

\begin{quote}
These adjusted curves demonstrate what survival would look like for each sex group if all patients had the same average age.
\end{quote}

This makes comparisons fairer and more clinically interpretable, especially in observational studies where baseline differences exist.

\hypertarget{original-biomedical-example-kidney-transplant-rejection}{%
\subsection{Original Biomedical Example: Kidney Transplant Rejection}\label{original-biomedical-example-kidney-transplant-rejection}}

We simulate a study comparing two immunosuppressive regimens in 150 kidney transplant patients. We want to compare survival adjusting for baseline creatinine level.

\begin{Shaded}
\begin{Highlighting}[]
\FunctionTok{set.seed}\NormalTok{(}\DecValTok{2025}\NormalTok{)}

\NormalTok{transplant\_data }\OtherTok{\textless{}{-}} \FunctionTok{tibble}\NormalTok{(}
  \AttributeTok{regimen =} \FunctionTok{rep}\NormalTok{(}\FunctionTok{c}\NormalTok{(}\StringTok{"Regimen A"}\NormalTok{, }\StringTok{"Regimen B"}\NormalTok{), }\AttributeTok{each =} \DecValTok{75}\NormalTok{),}
  \AttributeTok{creatinine =} \FunctionTok{c}\NormalTok{(}\FunctionTok{rnorm}\NormalTok{(}\DecValTok{75}\NormalTok{, }\FloatTok{1.2}\NormalTok{, }\FloatTok{0.3}\NormalTok{), }\FunctionTok{rnorm}\NormalTok{(}\DecValTok{75}\NormalTok{, }\FloatTok{1.5}\NormalTok{, }\FloatTok{0.4}\NormalTok{)),}
  \AttributeTok{time =} \FunctionTok{rexp}\NormalTok{(}\DecValTok{150}\NormalTok{, }\AttributeTok{rate =} \FunctionTok{rep}\NormalTok{(}\FunctionTok{c}\NormalTok{(}\FloatTok{0.008}\NormalTok{, }\FloatTok{0.012}\NormalTok{), }\AttributeTok{each =} \DecValTok{75}\NormalTok{)),}
  \AttributeTok{status =} \FunctionTok{rbinom}\NormalTok{(}\DecValTok{150}\NormalTok{, }\DecValTok{1}\NormalTok{, }\AttributeTok{prob =} \FloatTok{0.9}\NormalTok{)}
\NormalTok{)}

\CommentTok{\# Fit Cox model}
\NormalTok{cox\_transplant }\OtherTok{\textless{}{-}} \FunctionTok{coxph}\NormalTok{(}\FunctionTok{Surv}\NormalTok{(time, status) }\SpecialCharTok{\textasciitilde{}}\NormalTok{ regimen }\SpecialCharTok{+}\NormalTok{ creatinine, }\AttributeTok{data =}\NormalTok{ transplant\_data)}

\CommentTok{\# Create newdata frame with same creatinine level}
\NormalTok{newdata\_regimen }\OtherTok{\textless{}{-}} \FunctionTok{data.frame}\NormalTok{(}
  \AttributeTok{regimen =} \FunctionTok{c}\NormalTok{(}\StringTok{"Regimen A"}\NormalTok{, }\StringTok{"Regimen B"}\NormalTok{),}
  \AttributeTok{creatinine =} \FunctionTok{rep}\NormalTok{(}\FunctionTok{mean}\NormalTok{(transplant\_data}\SpecialCharTok{$}\NormalTok{creatinine), }\DecValTok{2}\NormalTok{)}
\NormalTok{)}

\CommentTok{\# Adjusted survival curves}
\NormalTok{fit\_transplant\_adj }\OtherTok{\textless{}{-}} \FunctionTok{survfit}\NormalTok{(cox\_transplant, }\AttributeTok{newdata =}\NormalTok{ newdata\_regimen)}

\CommentTok{\# Plot}
\FunctionTok{plot}\NormalTok{(fit\_transplant\_adj, }\AttributeTok{col =} \FunctionTok{c}\NormalTok{(}\StringTok{"darkgreen"}\NormalTok{, }\StringTok{"orange"}\NormalTok{), }\AttributeTok{lty =} \DecValTok{1}\SpecialCharTok{:}\DecValTok{2}\NormalTok{,}
     \AttributeTok{xlab =} \StringTok{"Days"}\NormalTok{, }\AttributeTok{ylab =} \StringTok{"Adjusted Survival Probability"}\NormalTok{)}
\FunctionTok{legend}\NormalTok{(}\StringTok{"topright"}\NormalTok{, }\AttributeTok{legend =} \FunctionTok{c}\NormalTok{(}\StringTok{"Regimen A"}\NormalTok{, }\StringTok{"Regimen B"}\NormalTok{),}
       \AttributeTok{col =} \FunctionTok{c}\NormalTok{(}\StringTok{"darkgreen"}\NormalTok{, }\StringTok{"orange"}\NormalTok{), }\AttributeTok{lty =} \DecValTok{1}\SpecialCharTok{:}\DecValTok{2}\NormalTok{)}
\end{Highlighting}
\end{Shaded}

\includegraphics{StatsTB_files/figure-latex/unnamed-chunk-456-1.pdf}

\hypertarget{interpretation-27}{%
\subsection{Interpretation:}\label{interpretation-27}}

\begin{quote}
After adjusting for baseline creatinine, patients on \textbf{Regimen A} show better adjusted survival probabilities than those on \textbf{Regimen B}.
\end{quote}

This suggests that Regimen A may be more effective in reducing the risk of transplant rejection.

\hypertarget{stratified-cox-models}{%
\section{Stratified Cox Models}\label{stratified-cox-models}}

When the proportional hazards assumption is violated for a variable, we may choose not to include it as a covariate, but instead to stratify the model on that variable.

A stratified Cox model allows the baseline hazard to vary across levels of a stratifying variable, while still estimating the effect of other covariates.

\hypertarget{when-to-use-stratified-cox-models}{%
\subsection{When to Use Stratified Cox Models}\label{when-to-use-stratified-cox-models}}

Use a stratified Cox model when:
- A covariate (e.g., treatment site, gender) does not satisfy the proportional hazards assumption
- You still want to account for that variable's influence on survival
- You want to estimate hazard ratios for other predictors without assuming proportionality for the stratified variable

Instead of including the variable as a predictor, we use \texttt{strata()} to let it modify the baseline hazard function.

\hypertarget{basic-syntax}{%
\subsection{Basic Syntax}\label{basic-syntax}}

Let's fit a stratified model using the \texttt{lung} dataset, stratifying by \texttt{sex} and estimating the effect of \texttt{age}.

\begin{Shaded}
\begin{Highlighting}[]
\NormalTok{strat\_cox }\OtherTok{\textless{}{-}} \FunctionTok{coxph}\NormalTok{(}\FunctionTok{Surv}\NormalTok{(time, status) }\SpecialCharTok{\textasciitilde{}}\NormalTok{ age }\SpecialCharTok{+} \FunctionTok{strata}\NormalTok{(sex), }\AttributeTok{data =}\NormalTok{ lung)}
\FunctionTok{summary}\NormalTok{(strat\_cox)}
\end{Highlighting}
\end{Shaded}

\begin{verbatim}
## Call:
## coxph(formula = Surv(time, status) ~ age + strata(sex), data = lung)
## 
##   n= 228, number of events= 165 
## 
##         coef exp(coef) se(coef)     z Pr(>|z|)  
## age 0.016215  1.016347 0.009187 1.765   0.0776 .
## ---
## Signif. codes:  0 '***' 0.001 '**' 0.01 '*' 0.05 '.' 0.1 ' ' 1
## 
##     exp(coef) exp(-coef) lower .95 upper .95
## age     1.016     0.9839    0.9982     1.035
## 
## Concordance= 0.546  (se = 0.026 )
## Likelihood ratio test= 3.18  on 1 df,   p=0.07
## Wald test            = 3.12  on 1 df,   p=0.08
## Score (logrank) test = 3.12  on 1 df,   p=0.08
\end{verbatim}

This model estimates the hazard ratio for age, while allowing different baseline hazards for males and females.

Note: The model does not estimate a hazard ratio for \texttt{sex}, since we stratified on it instead of including it as a covariate.

\hypertarget{original-biomedical-example-stratifying-by-diabetes-status}{%
\subsection{Original Biomedical Example: Stratifying by Diabetes Status}\label{original-biomedical-example-stratifying-by-diabetes-status}}

Suppose we are studying time to cardiac event in 160 patients undergoing treatment for high blood pressure. We suspect that the effect of diabetes does not meet the proportional hazards assumption, so we stratify on it while estimating the effect of LDL cholesterol.

\begin{Shaded}
\begin{Highlighting}[]
\FunctionTok{set.seed}\NormalTok{(}\DecValTok{2025}\NormalTok{)}

\NormalTok{cardiac\_data }\OtherTok{\textless{}{-}} \FunctionTok{tibble}\NormalTok{(}
  \AttributeTok{diabetes =} \FunctionTok{rep}\NormalTok{(}\FunctionTok{c}\NormalTok{(}\StringTok{"Yes"}\NormalTok{, }\StringTok{"No"}\NormalTok{), }\AttributeTok{each =} \DecValTok{80}\NormalTok{),}
  \AttributeTok{ldl =} \FunctionTok{c}\NormalTok{(}\FunctionTok{rnorm}\NormalTok{(}\DecValTok{80}\NormalTok{, }\DecValTok{130}\NormalTok{, }\DecValTok{15}\NormalTok{), }\FunctionTok{rnorm}\NormalTok{(}\DecValTok{80}\NormalTok{, }\DecValTok{120}\NormalTok{, }\DecValTok{10}\NormalTok{)),}
  \AttributeTok{time =} \FunctionTok{rexp}\NormalTok{(}\DecValTok{160}\NormalTok{, }\AttributeTok{rate =} \FunctionTok{rep}\NormalTok{(}\FunctionTok{c}\NormalTok{(}\FloatTok{0.010}\NormalTok{, }\FloatTok{0.008}\NormalTok{), }\AttributeTok{each =} \DecValTok{80}\NormalTok{)),}
  \AttributeTok{status =} \FunctionTok{rbinom}\NormalTok{(}\DecValTok{160}\NormalTok{, }\DecValTok{1}\NormalTok{, }\AttributeTok{prob =} \FloatTok{0.88}\NormalTok{)}
\NormalTok{)}

\NormalTok{strat\_model }\OtherTok{\textless{}{-}} \FunctionTok{coxph}\NormalTok{(}\FunctionTok{Surv}\NormalTok{(time, status) }\SpecialCharTok{\textasciitilde{}}\NormalTok{ ldl }\SpecialCharTok{+} \FunctionTok{strata}\NormalTok{(diabetes), }\AttributeTok{data =}\NormalTok{ cardiac\_data)}

\FunctionTok{summary}\NormalTok{(strat\_model)}
\end{Highlighting}
\end{Shaded}

\begin{verbatim}
## Call:
## coxph(formula = Surv(time, status) ~ ldl + strata(diabetes), 
##     data = cardiac_data)
## 
##   n= 160, number of events= 140 
## 
##         coef exp(coef) se(coef)     z Pr(>|z|)
## ldl 0.007058  1.007083 0.007252 0.973     0.33
## 
##     exp(coef) exp(-coef) lower .95 upper .95
## ldl     1.007      0.993    0.9929     1.021
## 
## Concordance= 0.528  (se = 0.03 )
## Likelihood ratio test= 0.95  on 1 df,   p=0.3
## Wald test            = 0.95  on 1 df,   p=0.3
## Score (logrank) test = 0.95  on 1 df,   p=0.3
\end{verbatim}

\hypertarget{interpretation-28}{%
\subsection{Interpretation:}\label{interpretation-28}}

\begin{quote}
This stratified Cox model estimates the effect of LDL cholesterol on survival time, while allowing the baseline hazard to differ between diabetic and non-diabetic patients.
\end{quote}

This approach is appropriate if diabetes status violates the PH assumption, since we no longer require hazards to be proportional across diabetes groups.

The hazard ratio for LDL reflects its association with cardiac event risk, adjusted for stratified diabetes effects.

\hypertarget{time-varying-covariates}{%
\section{Time-Varying Covariates}\label{time-varying-covariates}}

In standard Cox proportional hazards models, we assume that all covariates remain constant over time. However, in real-world clinical studies, certain covariates --- such as medication dosage, biomarker levels, or physical activity --- can change over time.

A time-varying covariate is a predictor whose value can differ at different time points for the same individual.

\hypertarget{when-to-use-time-varying-covariates}{%
\subsection{When to Use Time-Varying Covariates}\label{when-to-use-time-varying-covariates}}

Use time-varying covariates when:
- The effect of a predictor on survival is not static
- Clinical variables change with treatment or progression
- You want to model updated values over time rather than just baseline measurements

This allows for a more accurate and dynamic representation of how a covariate impacts survival.

\hypertarget{syntax-using-tt-in-coxph}{%
\subsection{\texorpdfstring{Syntax: Using \texttt{tt()} in \texttt{coxph()}}{Syntax: Using tt() in coxph()}}\label{syntax-using-tt-in-coxph}}

The \texttt{tt()} function in R allows us to transform a variable based on time.\\
The syntax looks like:

\begin{Shaded}
\begin{Highlighting}[]
\FunctionTok{coxph}\NormalTok{(}\FunctionTok{Surv}\NormalTok{(time, status) }\SpecialCharTok{\textasciitilde{}} \FunctionTok{tt}\NormalTok{(x), }\AttributeTok{tt =} \ControlFlowTok{function}\NormalTok{(x, t, ...) x }\SpecialCharTok{*} \FunctionTok{log}\NormalTok{(t), }\AttributeTok{data =}\NormalTok{ ...)}
\end{Highlighting}
\end{Shaded}

This models the effect of covariate \texttt{x} changing over time, using \texttt{log(t)} as a modifier.

Let's see a practical application.

\hypertarget{original-biomedical-example-inflammatory-marker-over-time}{%
\subsection{Original Biomedical Example: Inflammatory Marker Over Time}\label{original-biomedical-example-inflammatory-marker-over-time}}

Suppose we are studying patients with chronic inflammatory disease. The variable \texttt{crp} (C-reactive protein) is measured regularly and is known to influence risk of hospitalization.

We suspect that the effect of CRP increases with time, so we model it as a time-varying covariate.

\begin{Shaded}
\begin{Highlighting}[]
\FunctionTok{set.seed}\NormalTok{(}\DecValTok{2025}\NormalTok{)}

\NormalTok{inflammation\_data }\OtherTok{\textless{}{-}} \FunctionTok{tibble}\NormalTok{(}
  \AttributeTok{time =} \FunctionTok{rexp}\NormalTok{(}\DecValTok{200}\NormalTok{, }\FloatTok{0.01}\NormalTok{),}
  \AttributeTok{status =} \FunctionTok{rbinom}\NormalTok{(}\DecValTok{200}\NormalTok{, }\DecValTok{1}\NormalTok{, }\FloatTok{0.9}\NormalTok{),}
  \AttributeTok{crp =} \FunctionTok{rnorm}\NormalTok{(}\DecValTok{200}\NormalTok{, }\AttributeTok{mean =} \DecValTok{5}\NormalTok{, }\AttributeTok{sd =} \FloatTok{1.5}\NormalTok{)}
\NormalTok{)}

\NormalTok{cox\_tvc }\OtherTok{\textless{}{-}} \FunctionTok{coxph}\NormalTok{(}\FunctionTok{Surv}\NormalTok{(time, status) }\SpecialCharTok{\textasciitilde{}} \FunctionTok{tt}\NormalTok{(crp),}
                 \AttributeTok{tt =} \ControlFlowTok{function}\NormalTok{(x, t, ...) x }\SpecialCharTok{*} \FunctionTok{log}\NormalTok{(t }\SpecialCharTok{+} \DecValTok{1}\NormalTok{),  }\CommentTok{\# log(t + 1) to avoid log(0)}
                 \AttributeTok{data =}\NormalTok{ inflammation\_data)}

\FunctionTok{summary}\NormalTok{(cox\_tvc)}
\end{Highlighting}
\end{Shaded}

\begin{verbatim}
## Call:
## coxph(formula = Surv(time, status) ~ tt(crp), data = inflammation_data, 
##     tt = function(x, t, ...) x * log(t + 1))
## 
##   n= 200, number of events= 183 
## 
##            coef exp(coef) se(coef)     z Pr(>|z|)   
## tt(crp) 0.03333   1.03389  0.01264 2.637  0.00836 **
## ---
## Signif. codes:  0 '***' 0.001 '**' 0.01 '*' 0.05 '.' 0.1 ' ' 1
## 
##         exp(coef) exp(-coef) lower .95 upper .95
## tt(crp)     1.034     0.9672     1.009      1.06
## 
## Concordance= 0.551  (se = 0.025 )
## Likelihood ratio test= 6.83  on 1 df,   p=0.009
## Wald test            = 6.96  on 1 df,   p=0.008
## Score (logrank) test = 6.96  on 1 df,   p=0.008
\end{verbatim}

\hypertarget{interpretation-29}{%
\subsection{Interpretation:}\label{interpretation-29}}

\begin{quote}
This model examines how the effect of CRP changes as time progresses.\\
A positive coefficient for \texttt{crp\ *\ log(t)} would indicate that CRP becomes increasingly hazardous the longer a patient remains at risk.\\
This can reflect a biological process where persistent inflammation leads to worsening health outcomes over time.
\end{quote}

Including time-varying covariates makes survival models more realistic, especially when tracking patient changes during long-term studies.

\hypertarget{time-dependent-roc-curve}{%
\section{Time-Dependent ROC Curve}\label{time-dependent-roc-curve}}

In traditional classification models, ROC curves are used to evaluate sensitivity and specificity across thresholds. In survival analysis, however, we must account for time-to-event outcomes and censoring.

A time-dependent ROC curve allows us to assess the predictive accuracy of a biomarker or risk score at a specific time point, taking into account censored data.

\hypertarget{when-to-use-time-dependent-roc}{%
\subsection{When to Use Time-Dependent ROC}\label{when-to-use-time-dependent-roc}}

Use time-dependent ROC analysis when:
- You have a continuous risk score or biomarker (e.g., risk prediction from a Cox model)
- You want to evaluate how well this score predicts survival up to a certain time point
- You need to account for censored observations in your model evaluation

This method is especially useful for checking whether a biomarker has discriminatory power at different time horizons (e.g., 1 year, 3 years, etc.).

\hypertarget{required-package}{%
\subsection{Required Package}\label{required-package}}

This analysis requires the \texttt{timeROC} package:

\begin{Shaded}
\begin{Highlighting}[]
\FunctionTok{install.packages}\NormalTok{(}\StringTok{"timeROC"}\NormalTok{)}
\FunctionTok{library}\NormalTok{(timeROC)}
\end{Highlighting}
\end{Shaded}

\hypertarget{original-biomedical-example-heart-failure-risk-score}{%
\subsection{Original Biomedical Example: Heart Failure Risk Score}\label{original-biomedical-example-heart-failure-risk-score}}

Suppose we are analyzing hospital readmission risk in 220 heart failure patients.\\
We use a linear risk score based on heart rate and BNP levels (B-type natriuretic peptide) to predict readmission within 2 years.

\begin{Shaded}
\begin{Highlighting}[]
\FunctionTok{set.seed}\NormalTok{(}\DecValTok{2025}\NormalTok{)}

\NormalTok{hf\_data }\OtherTok{\textless{}{-}} \FunctionTok{tibble}\NormalTok{(}
  \AttributeTok{time =} \FunctionTok{rexp}\NormalTok{(}\DecValTok{220}\NormalTok{, }\AttributeTok{rate =} \FloatTok{0.01}\NormalTok{),}
  \AttributeTok{status =} \FunctionTok{rbinom}\NormalTok{(}\DecValTok{220}\NormalTok{, }\DecValTok{1}\NormalTok{, }\AttributeTok{prob =} \FloatTok{0.9}\NormalTok{),}
  \AttributeTok{hr =} \FunctionTok{rnorm}\NormalTok{(}\DecValTok{220}\NormalTok{, }\DecValTok{75}\NormalTok{, }\DecValTok{10}\NormalTok{),}
  \AttributeTok{bnp =} \FunctionTok{rnorm}\NormalTok{(}\DecValTok{220}\NormalTok{, }\DecValTok{500}\NormalTok{, }\DecValTok{120}\NormalTok{)}
\NormalTok{)}

\CommentTok{\# Risk score: linear combination of HR and BNP}
\NormalTok{hf\_data }\OtherTok{\textless{}{-}}\NormalTok{ hf\_data }\SpecialCharTok{\%\textgreater{}\%}
  \FunctionTok{mutate}\NormalTok{(}\AttributeTok{risk\_score =} \FloatTok{0.03} \SpecialCharTok{*}\NormalTok{ hr }\SpecialCharTok{+} \FloatTok{0.005} \SpecialCharTok{*}\NormalTok{ bnp)}

\CommentTok{\# Time{-}dependent ROC at 730 days (2 years)}
\NormalTok{roc\_result }\OtherTok{\textless{}{-}} \FunctionTok{timeROC}\NormalTok{(}\AttributeTok{T =}\NormalTok{ hf\_data}\SpecialCharTok{$}\NormalTok{time,}
                      \AttributeTok{delta =}\NormalTok{ hf\_data}\SpecialCharTok{$}\NormalTok{status,}
                      \AttributeTok{marker =}\NormalTok{ hf\_data}\SpecialCharTok{$}\NormalTok{risk\_score,}
                      \AttributeTok{cause =} \DecValTok{1}\NormalTok{,}
                      \AttributeTok{weighting =} \StringTok{"marginal"}\NormalTok{,}
                      \AttributeTok{times =} \DecValTok{730}\NormalTok{,}
                      \AttributeTok{iid =} \ConstantTok{TRUE}\NormalTok{)}

\FunctionTok{plot}\NormalTok{(roc\_result, }\AttributeTok{time =} \DecValTok{730}\NormalTok{, }\AttributeTok{col =} \StringTok{"blue"}\NormalTok{, }\AttributeTok{title =} \ConstantTok{FALSE}\NormalTok{)}
\FunctionTok{title}\NormalTok{(}\StringTok{"Time{-}Dependent ROC Curve at 2 Years"}\NormalTok{)}
\end{Highlighting}
\end{Shaded}

\hypertarget{interpretation-30}{%
\subsection{Interpretation:}\label{interpretation-30}}

\begin{quote}
The curve shows how well the risk score predicts hospital readmission by 2 years.\\
The AUC (Area Under the Curve) value summarizes the score's overall ability to distinguish between patients who were readmitted and those who were not.\\
A higher AUC (closer to 1) means better prediction. In clinical settings, AUC values above 0.75 are considered strong.
\end{quote}

\hypertarget{references}{%
\section{References}\label{references}}

\begin{itemize}
\item
  Clark, T., Bradburn, M., Love, S., Altman, D. (2003). Survival analysis part I: Basic concepts and first analyses. 232-238. ISSN 0007-0920.
\item
  M J Bradburn, T G Clark, S B Love, D G Altman. (2003). Survival Analysis Part II: Multivariate data analysis -- an introduction to concepts and methods. British Journal of Cancer, 89(3), 431-436.
\item
  Bradburn, M., Clark, T., Love, S., Altman, D. (2003). Survival analysis Part III: Multivariate data analysis -- choosing a model and assessing its adequacy and fit. 89(4), 605-11.
\item
  Clark, T., Bradburn, M., Love, S., Altman, D. (2003). Survival analysis part IV: Further concepts and methods in survival analysis. 781-786. ISSN 0007-0920.
\item
  The materials are prepared based on the tutorial by Zabore.
  \url{https://www.emilyzabor.com/tutorials/survival_analysis_in_r_tutorial.html}
\item
  Zabor, E., Gonen, M., Chapman, P., \& Panageas, K. (2013). Dynamic prognostication using conditional survival estimates. Cancer, 119(20), 3589-3592.
\end{itemize}

  \bibliography{book.bib,packages.bib}

\end{document}
